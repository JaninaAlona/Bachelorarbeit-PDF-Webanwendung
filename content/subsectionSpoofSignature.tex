\subsection{PDF Signature Spoofing}
Die PDF-Spezifikation ist sehr ungenau formuliert im Bezug auf Signaturen und auf welche Art und Weise sie validiert werden müssen. PDF-Viewer haben eine hohe Toleranz beim Öffnen, Validieren und Anzeigen von beschädigten PDF-Dateien. \\
Im Jahr 2019 wurden 3 Attacken zum Vortäuschen von validen PDF Signaturen gefunden: \gls{isa}, \gls{swa} und \gls{usf}. Das Attack Scenario sieht wie folgt aus: Der Angreifer besitzt das signierte PDF mit einer gültigen Signatur und manipuliert es. Das manipulierte  signierte PDF wird an das Opfer geschickt und die Signatur bleibt gültig, obwohl der Inhalt geändert wurde. \cite{ccc-break-pdf}
\par
Die \gls{isa} nutzt das Incremental Update-Feature von PDF aus, um bösartigen Inhalt ins PDF zu schleusen. Im Prozess des Signierens wird ein Incremental Update verwendet, um die Signatur zu speichern. Am Ende des Original-Trailers wird ein neuer Katalog und ein neues Signatur-Objekt als Body Updates angehängt, was den Signaturwert und Informationen über den Ersteller der Signatur enthält. Danach kommt eine updated Xref-Section und ein Updated Trailer. Beim Inremental Update können Objekte mit anderem Inhalt neu definiert werden. Der Updated Body wird hinter dem Originaltrailer angehängt, darunter zeigt die Updated Xref-Section auf das neue Objekt und der Updated Trailer wird ganz am Ende angefügt. Zuerst hat man geprüft, ob die PDF-Reader gegen eine Attacke anfällig waren, bei der hinter dem Incremental Update, was die Signatur enthält, eine weitere Sektion mit Body Updates, Xref Table und Trailer angefügt wurde. Ausschließlich LibreOffice wurde durch diese Vorgehensweise getäuscht. Alle weiteren Strategien produzieren kaputte PDF-Dateien, die keine Standardkonformität aufweisen, jedoch PDF-Viewer sind fehlertolerant. Als Nächstes haben die Forscher hinter dem Updated Trailer des Incremental Updates der Signatur Body Updates hinzugefügt, was eigentlich eine beschädigte PDF-Datei darstellt. Einige getestete PDF-Viewer haben lediglich überprüft, ob eine neue Xref-Sektion und Trailer vorhanden sind. Da sie nicht vorhanden waren, blieb die Signatur gültig und die zusätzlichen Body Updates wurden ausgeführt und die Modifikation blieb ohne Warnung unbemerkt. Andere PDF-Viewer benötigten neben zusätzlichen Body-Updates noch einen Trailer ohne Xref dazwischen, damit keine Warnung geworfen wurde. Die komplexeste Vorgehensweise, bei der neue Body Updates mit einer Kopie des Signatur-Objekts am Dateiende angebracht wurde, hat einige PDF-Viewer wie Foxit gezwungen die Signatur 2 Mal zu validieren und die Body Updates wurden eingefügt. \cite{ccc-break-pdf}
\par
Bei der \gls{swa} werden die Werte der signierten ByteRange modifiziert und Platz wird für das Einschleusen von bösartigen Inhalt geschaffen. Das Signatur-Objekt besteht aus einem /Contents-Eintrag mit dem Signaturwert und einem /ByteRange-Eintrag mit 4 Werten, was den signierten Teil im Dokument anzeigt. Die ersten 2 Einträge der ByteRange beziehen sich auf den Beginn des Dokuments bis zum Anfang des Signaturwerts. Hingegen definieren die letzten beiden ByteRange-Einträge den Bereich nach dem Signaturwert bis zu \%\%EOF. Diese beiden Bereiche wurden nicht angerührt. Die erfolgreiche Idee war eine zweite ByteRange hinter dem Signaturwert mit einem angepassten dritten Wert, der Platz für bösartige Objekte, etwas Padding und eine bösartige Xref, die auf die bösartigen Objekte zeigt, zu setzen. Lediglich die bösartige Xref-Position ist im Trailer vorgegeben, der nicht verändert werden kann. \cite{ccc-break-pdf}
\par
Mittels \gls{usf} wird die Signaturvalidierung außer Kraft gesetzt. Dennoch wird die Meldung „PDF is validly signed" dem Benutzer angezeigt. Vor allem die Adobe-Produkte waren anfällig für diese Attacke. Die Vorgehensweise gestaltete sich wie folgt: Die Forscher versuchten /Contents oder /ByteRange der Signatur entweder wegzulassen oder unübliche Werte wie kein Wert oder null zuzuweisen. Zwei PDF-Viewer versagten, wenn man /Contents auf 0 Bytes setzte. Das Weglassen der /ByteRange oder das Gleichsetzen mit null brach die Sicherheitsmechanismen von Adobe. \cite{ccc-break-pdf}
\par
Mittels dieser Attacken ist es den Forschern gelungen zu zeigen, dass 21 von 22 PDF-Viewer inklusive den Viewern von Adobe und 6 von 8 Online-Anbietern anfällig waren. Der einzige PDF-Viewer, der gegen alle Attacken immun war, war die veraltete Version von Adobe Reader 9, die jedoch Remote Code Execution enthält. \cite{ccc-break-pdf}