\chapter{Diskussion und Kritik}
Es gibt einige Aspekte meiner PDF Web App, die zur weiteren Diskussion anregen. Die Erweiterung der PDF Web App um eine Druckfunktion kann zu einem späteren Zeitpunkt noch erfolgen. Eine Druckfunktionalität wird durch die same-orgin policy des Browsers beeinflusst. Diese same-origin policy blockiert den Lesezugriff von Ressourcen, die von einer anderen Herkunft geladen werden. Diese Sicherheitsfunktionalität regelt, wie Dokumente und Scripte mit Ressourcen anderen Ursprungs interagieren \cite{same-origin}. 
\par
Aktuell kann der Creator keine exakte Größe des Seitenformats erstellen. Im Nachkommabereich ist die eingestellte Größe der PDF-Datei um einen Bruchteil kleiner. In visueller Hinsicht macht diese Eigenschaft keinen Unterschied. Ursächlich ist, dass die PDF-Spezifikation keine expliziten Breiten und Höhen von Dokumenten implementiert. Stattdessen wird die Seitengröße durch die MediaBox, CropBox, BeleedBox, TrimBox und ArtBox definiert. Daher lassen sich die Dimensionen der Seite nicht direkt umgestalten \cite{pdf-lib-pagesize}. Es gibt einige Optimierungsmöglichkeiten von bestehenden Funktionalitäten. Im Falle der input fields für die Seitenliste könnte die gültige Benutzereingabe durch die Möglichkeit eines Bindestrichs ergänzt werden, z.B. Seiten x-y. 
\par
Kopierte Text- und Bildelemente sind minimal, aber kaum wahrnehmbar, verschoben. Ihre controlBoxes sind jedoch kongruent. Dieser Effekt verschwindet, sobald eine Elementoperation ausführt wird. Komplett wegradierte Layers im Drawer könnten automatisch entfernt werden. Im aktuellen Release müssen vollständig radierte Layers manuell gelöscht werden. Eventuell wäre der intuitive Arbeitsfluss des Zeichenvorgangs behindert, sofern die Layer automatisch entfernt würde. Das Zeichnen auf einem Graphic Tablet funktioniert flüssig. 
\par
Die Verwendung von absoluter Rotationen von Elementen hat Vor- und Nachteile. Relative Rotationen könnten ein intuitives Arbeiten ermöglichen und würden sich Kopfrechenschritte erübrigen. Jedoch haben absolute Rotationen den Vorteil, dass die Rotation immer gleich ausgeführt wird und somit besser nachvollziehbar ist. Durch einen Winkel von 0 Grad kann die Rotation sogar zurückgesetzt werden. Ein Nachteil der absoluten Rotation liegt darin, dass bei der Rotation von mehreren Elementen im Layer Mode bereits bestehende Rotationen von Elementen mit der aktuellen Rotation überschrieben werden. In diesem Fall sollten die Elemente idealerweise im Box Mode einzeln rotiert werden. Hätte ich eine relative Rotation implementiert, würden vorherige Rotationen der Elemente bestehen bleiben. Die Generierung von neuen Layers nach einer Rotation von Drawings ist dann entbehrlich. 
\par
Die Verwendung von relativer Skalierung hat den Vorteil, dass im Layer Mode unabhängig große Objekte mit dem gleichen Faktor skaliert werden können. Zumindest bei Shaper und Imager können Elemente auch absolut in Breite und Höhe verkleinert oder vergrößert werden. Im Drawer existiert lediglich der relativen Skalierungsfaktor, der Breite und Höhe individuell skaliert. Drawings sind eher ungenau und brauchen keine exakte Größe. Zusätzlich sollte ich zukünftig noch eine untere und obere Begrenzung für die Größe von Elementen nach relativer Skalierung programmieren. 
\par
Was sicherlich für den Benutzer ungewohnt ist, dass sich ein Element im Editor nicht mitbewegt, wenn dessen controlBox verschoben wird. Dieses Verhalten habe ich aus Performance-Gründen implementiert. Auf diese Weise lässt sich besser erfassen, was sich unter dem Element an der Zielposition befindet. Der Nachteil ist, dass Elemente weniger präzise platziert werden können. Im Box Mode können nur controlBoxes bearbeitet werden, die nicht von anderen controlBoxes verdeckt sind, wie z.B. im Falle von kopierten Elementen. Sofern die Move-Operation im Layer Mode auf von anderen controlBoxes verdeckten Elementen anwendet werden soll (Layers sind ausgewählt), ist dies ebenso nicht möglich. Diese Probleme können behoben werden, indem die betreffenden Layers der Elemente in den Vordergrund verschoben werden. Der Benutzer der PDF Web App könnte zu der Ansicht gelangen, dass die controlBoxes etwas zu groß gestaltet sind. Ich habe sie deshalb so groß dimensioniert, weil ich bin der Meinung, dass die anklickbaren Boxen von grafischen Elementen in Programmen wie Illustrator zu klein sind. Daher ist es für die Augen anstrengender in Illustrator zu arbeiten. Die Mauscursorpositionen der PDF Web App zeigen keine Floats an und es sind noch keine Hilfsmittel wie Lineale oder Hilfslinien für exakte Layoutgestaltung implementiert. Die PDF Web App erhebt jedoch nicht den Anspruch, eine möglichst exakte Layoutanwendung zu sein, sondern soll die gängigsten schnellen Bearbeitungsmöglichkeiten von PDF-Dokumenten bieten. Ein weiterer ausbaubarer Aspekt ist die Tatsache, dass die controlBoxes eine fixe Farbe besitzen. Beispielsweise besitzt ein Shape mit einer orange controlBox eine Füllung im ähnlichen oder identischen Orange, würde die controlBox visuell nicht mehr sichtbar sein. In solch einem Fall sollte die controlBox die Farbe wechseln oder generell zusätzlich eine andere Füllfarbe als die Kontur aufweisen. Eine weitere Option bestünde darin, eine gestrichelte oder punktierte Kontur zu implementieren. Sofern ein Element zu weit an den Rand bewegt wird, verschwindet die controlBox im Editorhintergrund. Eine Verbesserung für die Zukunft wäre, dass controlBoxes immer sichtbar sind, selbst wenn das Element aus dem Seitenbereich manövriert wird.
\par
Der Layer Mode als Programmdesignentscheidung kam mir in den Sinn, um mehrere Elemente zeitsparend zu modifizieren. Der Vorteil meiner Implementierung der Layers liegt darin, dass Elemente nicht partiell ausgewählt werden müssen, sondern mit einer Layer verknüpft sind. Somit kann ein Element allumfassend, unabhängig von der Lage in Vorder- bzw. Hintergrund oder anderen Elementen, modifiziert werden. Durch die Änderbarkeit der Reihenfolge von Layers können sogar Elemente auf andere Seiten gezogen werden. Eine benutzerdefinierte Gruppierung von Layers würde die Layers-Implementierung noch bereichern. Verschiebt man ein Element im Editor auf eine andere Seite, die maßgeblich kleiner als die souce Seite ist, verschwindet das Element im Hintergrund, da die Seitenkoordinaten zu sehr voneinander abweichen. Allerdings wird das Element wieder sichtbar, wenn es auf eine Seite mit gleicher oder ähnlicher Größe gezogen wird. Eine Erweiterung der aktuellen Layers-Implementierung wäre, dass eine benutzerdefinierte Gruppe mehrerer Layers von einer Seite auf eine andere mit einer Operation verschoben werden könnte. Ebenso sinnvoll wäre ein Ein- und Ausblenden von mehreren selektierten Layers mit einem Klick. Auf ähnliche Weise könnte der selection und deselection filter nach ein- und ausgeblendeten Layers filtern. 
\par   
Da die PDF Web ab nicht installiert werden muss, bleiben keine störenden Deinstallationsdateien übrig. Außerdem ist die Speichergröße der PDF Web App für die Offline-Verwendung aktuell unter 3,5 MB groß. Viele PDF-Programme auf dem Markt verbrauchen wesentlich mehr Speicherplatz. Ich unterstütze keine Verschlüsselung von PDF-Dokumenten, da PDF-Dateien unsicher sind und nicht für vertraulichen Inhalt verwendet werden sollten. In Zukunft könnte ich versuchen, eine sichere Verschlüsselung zu implementieren, was eher mehr Zeit in Anspruch nehmen würde.
\par
Insgesamt ist die leichte Bedienbarkeit der PDF Web App zumindest im Editor nicht ganz gelungen. Für das Verständnis von Box Mode und Layer Mode, sowie deren Vorzüge, muss man die Tutorials gelesen haben. Auch die Select All und Deselct All Filter sind nicht ganz eingängig. Als einzige Buttons sind Spin CW (clockwise) und Spin CCW (counterclockwise) weniger aussagekräftig beschriftet. Eine Unannehmlichkeit ist die aktuelle Funktion, wenn man von einem anderen Modul als dem Editor auf ein beliebiges Editormodul klickt im Hauptmenü, gelangt man immer zuerst zum Texteditor. Besser wäre es, wenn man direkt zum jeweiligen Editormodul gelangen würde.

\section{Ideen für zukünftige Features}
Einige wichtige Funktionalitäten sind noch nicht implementiert und ich plane diese noch in die PDF Web App in den nächsten Releases einzubauen. In meiner PDF Web App fehlt die Spiegelung von Text, Zeichnungen, Geometrie und Bildern als wichtige affine Transformation. Die anderen Transformationen Verschiebung, Skalierung und Rotation sind bereits in der PDF Web App vorhanden. Außerdem fehlt die searchable PDF-Eigenschaft im Reader bzw. Editor, d.h. der Benutzer kann keinen Text oder Bilder kopieren bzw. markieren. Die searchable PDF-Eigenschaft kann nicht genutzt werden, da ich die PDF Seiten des source Dokuments in CANVAS HTML-Elemente rendere. Folglich werden alle Seiten gerastert und die searchable PDF-Eigenschaft des source Dokuments geht verloren. 
\par
Ein fortschrittliches zukünftige Feature wäre die Editierung von bereits bestehenden Objekten im source Dokument bearbeiten zu können. Im weiteren Verlauf des Projekts möchte ich unabdingbar Text, Geometrie und Zeichnungen mit Vektorpfaden ersetzen. In der aktuellen Version der PDF Web App sind alle Elemente pixelbasiert. Falls der Browser oder Computer abstürzt und das in der PDF Web App geöffnete Dokument nicht gedownloaded wurde, sind die Daten verloren. Als Gegenmaßnahme könnte man einen Export und Import Button konfigurieren, der die hinzugefügten PDF-Objekte als \gls{json}-Datei speichert. Bei Export würde diese Datei angelegt und im Ordner der PDF Web App gespeichert werden und mittels Import könnte man die PDF-Objekte dem geöffneten Dokument hinzufügen, sodass sie weiter bearbeitet werden können. Zusätzlich könnte man einen automatischen Export der PDF-Objekte als Backup implementieren, der z.B. alle 10 Minuten die hinzugefügten Elemente wie Text, Zeichnungen, Geometrie und Bilder in der \gls{json}-Datei sichert.
\par
Eine Funktionalität, die ich oft benötige, ist die Kompression von PDF-Dokumenten. Anfangs wollte ich gerne eine Funktion dafür bieten, jedoch hatte ich keine geeignete JavaScript-Library für Kompression gefunden. Ich habe darauf verzichtet die Kompression selbst zu implementieren. Ebenfalls häufig Anwendung würde eine Undo- bzw. Redo-Funktion finden, also dass man Schritte rückgängig machen und rückgängig gemachte Schritte wieder ausführen kann. In der Praxis verwende ich eher selten Formulare, aber Formulare auszufüllen und editierbare Formulare zu erstellen, wäre mit Sicherheit ein sehr nützliches Feature. Außerdem plane ich später die PDF Web App auf allen gängigen Browsern zu testen und anzupassen, selbst das GUI-Design. Bisher habe ich ausschließlich in Firefox entwickelt und getestet.