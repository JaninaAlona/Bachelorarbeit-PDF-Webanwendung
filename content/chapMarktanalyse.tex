\chapter{Stand der Technik}
Die Popularität von PDF-Dateien ist seit 2008 rasant angestiegen in der globalen Informationsübertragung. Täglich werden weltweit 2,5 Milliarden PDF Dokumente erzeugt. Seine Beliebtheit verdankt PDF vor allem an der plattformübergreifenden Kompatibilität (Desktop-Computer, Tablets und Smartphones), denn PDF Dokumente ist auf mehr als 1,5 Milliarden Geräten ohne zusätzliche Software lesbar. Über 80\% der geschäftlichen Dokumente werden als PDF Datei weitergegeben. \cite{formilo} 90 \% der Büroangestellten wollen auf das PDF Dateiformat nicht mehr verzichten. Drei Viertel aller archivierten Dokumente sind PDF Dokumente (2023).
Bis 2025 werden über 3 Millarden Dollar jährlich für PDF Editoren ausgegeben werden (2023) \cite{kofax}. Im Jahr 2015 gab es 1,6 Milliarde PDF-Dokumente im Web und im Jahr 2019 wurden PDF-Dateien bei ca. 99 \% Firmen und Regierungsinstitutionen weltweit verwendet \cite{ccc-break-pdf}. \\
In PDF ist keine automatische Anpassung des Seiteninhalt-Layouts, wie z.B. in Microsoft Word, möglich. Daher kann ein PDF-Dokument nicht sinnvoll in das Word-Format umgewandelt werden ohne möglicherweise das ursprüngliche PDF-Layout zu beeinflussen und zu ändern, sowie die maximalen Bearbeitungsmöglichkeiten von Word ausschöpfen zu können.


\section{Generative AI}
Generative artificial intelligence (AI) Anwendungen können das Arbeiten mit Text, Bildern, Code und Dokumenten wie PDF erleichtern. Solche Anwendungen sind seit einigen Jahren wertvolle Tools in vielen Businessumgebungen und -workflows. Sie kommen in Tools für Teamkollaboration, wie Zoom, \gls{ccaas} Plattformen und Produktivitätsapps vor. Einer der prominentesten Beispiele sind Microsoft Copilot und Google Duet AI. Sie bauen auf einem \gls{llm} auf und bedienen sich Algorithmen zur Gesprächsführung. Mittels \gls{nlp} und hochmodernen Algorithmen können generative AI-Anwendungen mit Menschen interagieren. Durch eine Benutzeraufforderung (prompt) kann der Benutzer Fragen oder Befehle an das AI Tool übermitteln und erhält menschenähnliche Antworten in wenigen Sekunden \cite{copilot-duet}.

\subsubsection{Microsoft Copilot}
Microsoft Copilot ist u.a. in die Bing Suchmaschine, Microsoft Office, Microsoft Teams und Windows 11 integriert. Copilot basiert auf OpenAIs GPT model. Die Funktionalität von Copilot variiert, je nachdem wo es verwendet wird. Bei Word kann Copilot behilflich sein, Dokumente zu skizzieren, Quellen für Informationen in einem Dokument zu suchen, Wortvorschläge zu machen und Schreibhilfe zu leisten. Benutzer können Informationen aus anderen Microsoft Dokumenten, wie z.B. PowerPoint, beziehen, um ihr aktuelles Dokument zu füllen oder es an die Formatierung von einem anderen Dokument anzupassen. In Bezug auf Excel ist Copilot ein Analysetool, um Daten zu visuellen Repräsentationen zu transformieren oder für automatisierte Prozesse. Der Bot kann sogar Trends von Schlüsseldaten ableiten, Fehler korrigieren, Zellen automatisch vervollständigen und Berechnungen erklären. Bei PowerPoint kann Copilot Präsentationen basierend auf Informationen und Dokumenten des Microsoft Ökosystems erstellen. Präsentationsfolien können auf Grundlage von spezifischen Instruktionen passend, z.B. zum eigenen Stil oder der Stimme, gestaltet werden. 

\subsubsection{Google Duet AI}
Google Duet AI ist ein Bestandteil der Google Workspace Apps und Entwicklertools. Der Service unterstützt bei der Code Assistenz mehr als 20 Programmiersprachen. Benutzer können AppSheets verwenden, um intelligente Businessanwendungen und -workflows zu erstellen. Google bietet außerdem die Vertex Plattform für Softwareentwicklung an. Bei Google Docs gibt es ein Duet Feature Help Me Write zur Erstellung von Dokumenten. Basis dessen, ist ein prompt, der beschreibt, über was der Benutzer gerne schreiben möchte. Folglich arbeitet das AI-System ein passendes Dokument heraus. Das Feature Smart Chips wird für variable Informationen verwendet. Hierbei kann der Benutzer bestimmte Inhaltspassagen mit Bedingungen, bezogen auf einen Ort oder eines bestimmten Businesses, versehen. Help Me Write ist für viele Sprachen verfügbar und kann eine Reihe an unterschiedlichen Typen von Dokumenten, z.B. Blogs zu Jobbeschreibungen, kreieren. Im Falle von Google Sheets bietet Duet eine Datenanalyse an. Durch \gls{nlp}-Technologie assistiert Duet Benutzern bei der Dokumentennavigation und Erstellung von benutzerdefinierten Templates, um Daten zu organisieren. Mächtige Tools für Datenklassifizierung unterstützen den Benutzer Datenkontexte zu verstehen. Zusätzlich kann Duet Fehler finden und berichtigen, Zellen automatisch füllen und Vorschläge für eine Analyse machen. Bei Google Slides spielt Duet ebenfalls eine Rolle beim Erstellen und Optimieren von Präsentationen. Bilder können mittels des Help Me Visualize-Werkzeugs generiert werden und Bilddesigns mit verschiedenen Styles angepasst werden. Duet assistiert außerdem beim Notizen Schreiben für Präsentationen und hilft eine Layoutkonsistenz von Präsentationen beizubehalten. Sowohl Copilot als auch Duet können beim E-Mail Schreiben und Video-Meetings behilflich sein. Vor allem Duet bietet Assistenz beim Programmieren \cite{copilot-duet}.

\subsubsection{Donut}
Normalerweise wird bei der \gls{ocr}-Technologie zunächst das Bild in Graustufen konvertiert und eine Texterkennungsphase eingeleitet, in der Stellen mit Informationsgehalt identifiziert werden. An diesen Stellen werden Textboxen generiert. Danach liest eine \gls{ocr}-Engine, wie Tesseract, den Inhalt jeder Box im Bild und konvertiert ihn zu Text. Zuletzt wird ein Algorithmus eingesetzt, der eine post-\gls{ocr} pipeline oder language model sein kann, um extrahierte Informationen zu klassifizieren. Beispiele für solche Anwendungen sind Amazon Textract oder Microsoft Azure AI Document Intelligence. Eine bessere Alternative für \gls{ocr}-Anwendungen stellt die \gls{ocr}-freie Lösung Donut dar. Donut ist ein Visual Document Understanding model, was ein Bild als input aktzeptiert und textbasierte Aufgaben lösen kann, wie Informationsgewinnung und Beantwortung von visuellen Fragen. Mit Fokus auf dem deep learning img2seq model verwendet Donut als Ersatz zur \gls{ocr}-Technologie einen visuellen Encoder (Swin-B transformer) und einen Textdecoder (BART). Swin-B ist eine transformer-Architektur, die speziell auf image processing zugeschnitten ist. Bilder werden vom transformer in Segmente unterteilt und hierarchisch verarbeitet, um lokale und globale, kontextuelle Informationen zu erfassen. Die Bildsegmente werden mittels eines shifted window-based multi-head self-attention Moduls analysiert, um die Beziehungen zwischen benachbarten Segmenten zu erfassen. Danach ermöglicht eine two-layer \gls{mlp}, welche ein supervised learning Algorithmus eines supervised neural networks ist, dem model, das Schema in jedem Segment zu lernen, damit es ein besseres Verständnis des Bildinhalts entwickeln kann. Zum Schluss durchlaufen die token des Segments die Schichten, die die Segmente wieder zusammenfügen, was dem model ermöglicht, Informationen zu kumulieren und eine verständlichere Repräsentation des Bildes zu liefern. Zum Schluss wird der Output dieses Prozesses dem Decoder, einem multilingual BART model, übergeben \cite{transformers-ocr}.

\subsubsection{Amazon Textract}
Amazon Textract ist ein machine learning (ML) Service, der automatisch Text, Handschrift, Layoutelemente und Daten von gescannten Dokumenten, Bildern oder PDF-Dateien, ohne manuelle Konfiguration, extrahieren kann. Bei Textract handelt es sich um einen \gls{aws} Cloud-Dienst. Im Gegensatz zu traditionellen \gls{ocr}-Anwendungen kann Textract außerdem strukturierte Informationen aus Tabellen oder Formularen erfassen. Nebst Zeichenerkennung werden auch Formatierungen sowie die Struktur des Texts herausgearbeitet. Der Service kann per Konsole, Command Line Interface (CLI) oder API verwendet werden. Mittels der Detect Document Text API wird eine \gls{ocr}-Schnittstelle bereitgestellt, um handschriftliche oder gedruckte Texte aus Dokumenten zu entnehmen. Zusätzlich hat die Analyze Document API das Aufgabenfeld strukturierte Daten einzulesen und Beziehungen bzw. Schlüsselwertpaare aus Tabellen- oder Formularfelder zu erstellen. Extrahierte Informationen sind mit einem Confidence-Score versehen, um dem Benutzer mitzuteilen, wie exakt und verlässlich die Daten sind. Handschriftlicher und gedruckter Text kann mit hoher Genauigkeit und Zuverlässigkeit erkannt werden. Die Extraktion der Daten geschieht sehr performant in kurzer Zeit. Das Pricing des Produkts ist nutzungsabhängig \cite{textract}.

\subsubsection{Perplexity AI}
Perplexity AI ist ein Werkzeug für \gls{nlp} mit Dokumenten. Insgesamt kann man bei Perplexity zwischen den generativen AI assistants Perplexity, \gls{gpt4} oder Claude 2 von Anthropic bei Dateiuploads wählen. Eine PDF-Datei als Dateiformat, plaintext oder Code kann hochgeladen werden und Perplexity verwendet deren Dateiinhalte, um Antworten auf Fragen zum PDF, inklusive Zitate mit Quellenangaben, zu formulieren. Bei kurzen Dateien wird das gesamte Dokument vom language model analysiert. Umfangreiche PDFs können manuell in Themenbereiche unterteilt und als Input für \gls{gpt4} für Kreatives Schreiben verwendet werden. Wissenschaftliche Artikel können verglichen, ihre Unterschiede herausgearbeitet, themenverwandte Dokumente durch eine query gefunden, Daten analysiert, Überblicke von verschiedenen Quellen generiert, Daten visualisiert, Grafiken von Quellen erstellt und Text in eine andere Sprache übersetzt werden. Bei der kostenlosen Version ist der Benutzer auf eine bestimmte Anzahl an Anfragen begrenzt \cite{hackernoon-claude}. Der Claude 2 AI assistant kann PDFs parsen und die Dokumentstruktur mittels ML-Techniken erfassen. Es hat eine eingebaute PDF-analyser-Komponente. Die maximale Dateigröße ist 25 MB. Der Benutzer kann Claude Fragen stellen, die im PDF-Dokument behandelt werden, und Claude liefert die Antworten. Direkte Zitate können aus der PDF-Datei gefunden werden und Claude zeigt die Seitenzahl an, wo das Zitat vorkommt. Außerdem kann Claude PDF-Inhalt zusammenfassen. Copilot ist ebenfalls verfügbar, um schnellere Antworten auf prompts in einer menschlichen Art und Weise zu liefern. Bei umfangreichen Dateien werden die relevantesten Dateisegmente analysiert, um wichtige Antworten zu geben. Code kann erklärt und Dateien übersetzt werden. Claude und \gls{gpt4} gelten als intelligentere models, um große Dateien zu parsen \cite{perplexity}.

\subsubsection{Adobe Firefly}
Adobe Firefly ist eine generative AI zur Erstellung von Bildern durch prompts. Im März 2023 wurde die Beta-Phase gestartet und seitdem haben Nutzer mehr als 3 Milliarden Bilder generiert. Firefly ist in Photoshop, Illustrator und Adobe Express integriert und existiert als eigenständige Webanwendung. Bei Photoshop ist es als Generative Füllung und Generatives Erweitern eingebaut. Diese Funktionen ermöglichen dem Nutzer Bildinhalte per Text prompt hinzuzufügen, zu entfernen, zu ersetzen oder zu erweitern, um Bilder umfassend zu ändern oder zu ergänzen. Im Bezug auf Illustrator wurden die Tools Generative Neufärbung, um Farbvarianten eines Designs zu erstellen, und Text to Vector Graphic, um aus einem prompt eine Vektorgrafik zu generieren, eingebaut. Adobe Express ergänzt Photoshop, Illustrator, Adobe Premiere Pro, sowie Acrobat und ermöglicht den Import, die Bearbeitung und Synchronisierung von Assets zwischen den Applikationen für Echtzeitzusammenarbeit. Es wurde für schnelle Aufgaben, wie das Entfernen von Hintergründen, die Erstellung von Inhalten für soziale Medien bzw. die Freigabe von Konzepten, auf den Markt gebracht. In Adobe Express wird Firefly für Text zu Bild, Texteffekte, Generative Füllung und Text to Template eingesetzt. Text to Template ermöglicht dem Benutzer durch einen prompt editierbare Templates zu generieren, um auf schnelle Art und Weise Social Media Posts, Poster, Flyer und digitale Karten zu erstellen. Die generierten Resultate von Text zu Bild und Texteffekte lassen sich in Adobe Express direkt als PDF-Datei speichern. Firefly unterstützt prompts in 100 Sprachen weltweit und die Webanwendung ist in 20 Sprachen verfügbar. In der Webanwendung ist kürzlich das Adobe Firefly Image 2 Model in der Beta-Version nebst Model 1 für mehr Gestaltungsmöglichkeiten und verbesserte Bildberechnung bzw. -qualität auf den Markt gebracht worden. Bei Text zu Bild wurde Generative Match implementiert, welcher einen Stil von mehreren, vorausgewählten Bildern auf ein neues generiertes Bild anwendet. Der Bildstil kann mit integrierten Stileffekten kombiniert werden \cite{adobe-firefly}. Einige der im Folgenden vorgestellten PDF-Programme bedienen sich ebenfalls der generativen AI-Technologie.
\section{Kostenpflichtige PDF Programme und Onlinedienste}
PDF Dateien lassen sich in vielen Programmen einfach über den Druckdialog erstellen. Apple hat das Lesen von PDF Dokumenten in seiner Apples Vorschau integriert. Viele Webbrowser stellen PDF Viewer bereit, so Google Chrome seit 2010 \cite{wiki-pdf-de}. Ich werde in den folgenden Abschnitten 7 kostenlose und kostenpflichtige Programme ausführlich behandeln und ihre interessantesten und wichtigsten Funktionen aufzeigen. Die Liste an PDF-Programmen lässt sich unendlich fortführen, deshalb habe ich mir die interessantesten herausgepickt. Andere kostenpflichtige PDF-Programme heißen pdf-it, Soda PDF, Nitro PDF Pro, Ashampoo PDF Pro 3, Infix 7, PDF Director 2 Pro und Perfect PDF.

\subsection{Adobe Acrobat Pro}
Produktseite: \url{https://www.adobe.com/de/acrobat/acrobat-pro.html} \\
Acrobat ist ein professionelles Werkzeug, um PDF-Dateien zu erstellen und bereits bestehende Inhalte zu bearbeiten. Text kann hinzugefügt, geändert, formatiert, gelöscht oder markiert werden. Ebenfalls kann der Text von Formularen bearbeitet werden. Bilder können gespiegelt, gedreht, zugeschnitten, ersetzt, ausgerichtet oder angeordnet werden. Beim Objekte Anordnen-Werkzeug können mehrere Objekte vor oder hinter anderen Objekten angeordnet werden. Das Arbeiten mit Ebenen ist möglich, jedoch eingeschränkter als bei Photoshop. PDF-Seiten können kopiert, ersetzt, gedreht, verschoben, gelöscht, extrahiert oder neu nummeriert werden. Eine PDF-Datei kann in mehrere Dokumente aufgeteilt und Kommentare können hinzugefügt werden. Webseiten können mit der Acrobat-Browsererweiterung in PDF konvertiert werden \cite{adobe-acrobat-um}. Eine PDF-Datei kann in eine PDF/A-Datei inklusive seiner Varianten, PDF/X, PDF/UA, PDF/VT oder PDF/E konvertiert werden. Außerdem kann die Kompatibilität mit diesen Formaten in Preflight-Profilen überprüft werden \cite{adobe-pdf-a}. Die Barrierefreiheit kann automatisch validiert oder ein neues Dokument kann direkt barrierefrei erstellt werden. Adobe Acrobat Pro kann andere Dokumentenformate wie HTML, DOC, DOCX, TXT und RTF in PDF konvertieren, PDF in andere Dateiformate wie Word exportieren oder Dokumente unterschreiben \cite{adobe-formate}. Der Benutzer kann PDF-Dokumente mit digitalen Signaturen unterzeichnen und das Programm kann die Signaturen validieren \cite{adobe-acrobat}. Für Vertraulichkeit können PDF-Dokumente mit Passwörtern und Zertifikaten versehen werden. Interaktive Objekte, Audio und Video können eingebettet werden. In der Pro-Variante gibt es Druckproduktionswerkzeuge u.a. für Druckermarken, Transparenz-Reduzierung, Farbkonvertierung oder Druckermanagement. Die Preflight-Option ist ebenfalls nur in Acrobat Pro verfügbar \cite{adobe-acrobat-um}. Mit dem Werkzeug Scan \& \gls{ocr} in Acrobat Pro kann man Pixelbilder als PDF und gescannte PDF-Dokumente in ein durchsuchbares PDF umwandeln \cite{adobe-search}. Lediglich Acrobat Pro kann PDFs vergleichen und schwärzen. Die Standard-Version ist nur unter Windows verfügbar. Hingegen ist Acrobat Pro für macOS und Windows erhältlich \cite{wondershare-acrobat}. Das Programm gibt es als kostenlosen Acrobat Reader mit sehr eingeschränkten Funktionen, Acrobat Standard für 15,46 Euro pro Monat und Acrobat Pro für 23,79 Euro pro Monat. Hierbei handelt es sich um ein Jahres-Abo mit monatlicher Zahlung. Wählt man das Monats-Abo, so kostet die Standard-Version 27,36 Euro im Monat und die Pro-Variante 35,69 Euro im Monat. Es gibt auch die Möglichkeit das Jahres-Abo mit Vorauszahlung zu erwerben \cite{adobe-acrobat}. Die Testversion von Acrobat ist 7 Tage nutzbar. Mit den Adobe Acrobat Onlinetools kann man über den Browser verschiedene Dateitypen in PDF umwandeln, unter anderem PDF in JPEG oder andere Bildformate, PDF Dateien bearbeiten und Kompression anwenden. Die Onlinetools können ebenfalls PDF in Word umwandeln. \cite{adobe-search} Der Adobe Acrobat PDF-Converter der Onlinetools kann DOCX, DOC, XLSX, XLS, PPTX, PPT, TXT, RTF, JPEG, PNG, TIFF, BMP, sowie Adobe eigene AI-, INDD- und PSD-Dateien in PDF konvertieren. \cite{adobe-formate} Die kostenlose Version des PDF-Converters kann nur begrenzt oft genutzt werden.

\subsection{Wondershare PDFelement}
Produktseite: \url{https://pdf.wondershare.com/} \\
Wondershare PDFelement ist eine KI gestützte PDF-Alternative zu Acrobat. Das Programm wird für Windows, macOS, iOS und Android angeboten und die aktuellste Version ist PDFelement 10. Bereits bestehende Bilder oder Text des geöffneten PDFs können bearbeitet werden. Vom Autor eingebettete Objekte können in 90 Grad-Schritten gedreht, horizontal und vertikal gespiegelt oder ausgerichtet werden. PDF-Dokumente können mit Highlights versehen und Geometrie bzw. Freihandzeichnungen können hinzugefügt werden. Dabei handelt es sich um Kommentartypen, die auch gedreht, skaliert oder verschoben werden können. Das Radiererwerkzeug funktioniert nur für Freihandzeichnungen. Der Hintergrund einer PDF-Datei kann seitenweise mit Farbe gefüllt werden oder durch ein Bild verschönert und leere PDFs können erstellt werden. Ausfüllbare Formulare können sogar automatisch erstellt werden. Es gibt Sicherheitsfunktionen für Passwortschutz, Signaturerstellung und Validierung von Signaturen. PDFelement besitzt ebenfalls eine \gls{ocr}-Funktion. Im Bezug auf AI-Tools gibt es die Optionen Erklären von Inhalten und Code, Korrekturlesen, Zusammenfassen, Chat mit PDF, Übersetzungen und eine Funktion zur Erkennung von AI erstellten Inhalten, sowie eine automatische Lesezeichenfunktion basierend auf der PDF-Struktur und Überschriften. Der AI reading assistant nennt sich Lumi und beruht auf ChatGPT. Mehrere PDF-Dateien können gleichzeitig geöffnet sein und Screenshots, Screen Recording, Lineale, Hilfslinien und Hilfsgatter sind verfügbar. Eine Stapelverarbeitung von mehreren PDF-Dateien ist für Konvertierung in andere Dateiformate, Komprimierung, \gls{ocr}, einige Bearbeitungsoptionen, Drucken und Verschlüsselung verfügbar. Konvertierungen von PDF zu Word, Powerpoint, Excel und JPEG, PNG, TIFF, GIF, Epub, HTML, \gls{xml} oder PDF/A sind u.a. möglich \cite{wondershare-um}. In der kostenlosen 14-tägigen-Testversion mit Anmeldung, bei der dann mehrere Funktionen zur Verfügung stehen, wird ein PDF mit Wasserzeichen gespeichert. Für einen Einzelbenutzer kostet eine Dauerlizenz 119 Euro, ein Jahresabo 89 Euro und ein 2-Jahresabo 109 Euro. Darin sind PDFelement Updates enthalten, sowie 20 GB Document Cloud Speicher. Es gibt einen Rabatt für Schüler, Studenten und Lehrkräfte \cite{wondershare-preis}. 

\subsection{UPDF Pro}
Produktseite: \url{https://updf.com/} \\
UPDF bietet eine kostenlose Testversion mit 1 GB Cloud-Speicher an, die fast alle Features der Pro-Variante freigeschaltet hat. Die kostenlose Testversion enthält mehr Funktionen, wenn man sich bei UPDF registriert. Es gibt bei UPDF verschiedene Anzeigemodi. Man kann auch 2 PDFs nebeneinander anzeigen lassen und sogar scrollen. Mehrere PDFs können in Tabs geöffnet werden. Ein Dokument kann als Slide Show mit dunklem Hintergrund für Präsentationszwecke angezeigt werden. Dabei kann man mit dem Eraser, Pen und Laser Pointer bestimmte Stellen im PDF während des Präsentationsmoduses hervorheben. Lesezeichen und Kommentare können hinzugefügt werden. Als Kommentarvarianten stehen Highlights, Text, Stift und Radierer, Geometrie, Sticker, Stempel und Dateianhänge zur Verfügung. Bei UPDF kann man Seiten neu ordnen, drehen, zerteilen, beschneiden, ersetzen, extrahieren und löschen. Eine \gls{ocr}-Funktionalität ist vorhanden. PDF-Formulare können erstellt und ausgefüllt werden. Eine PDF-Datei kann zu PDF/A, Word, Excel, PowerPoint, CSV, TXT, Bilder oder \gls{xml} konvertiert werden. Aus folgenden Dateiformaten kann ein PDF erstellt werden: CAJ, Word, PowerPoint, Excel, Visio und Bilddateien (.png, .jpeg, .bmp und .gif). Wenn man ein PDF-Dokument in der kostenlosen Testversion speichert wird es mit einem Wasserzeichen versehen. Jedes Mal, wenn man ein Dokument in der kostenlosen Testversion mit Login geöffnet hat, fragt UPDF nach, ob man sich die Pro-Version kaufen möchte, was sehr störend ist. Verschlüsselung ist durch eine Passworterstellung fürs Öffnen oder Rechte gewährleistet. Elektronische und digitale Signaturen mit Zertifikaten werden unterstützt. Man kann ein PDF in die UPDF Cloud hochladen und dem AI assistant Fragen über das hochgeladene PDF stellen. Basierend auf das PDF in der Cloud kann der AI assistant das PDF zusammenfassen, übersetzen und Inhalte erläutern. Es gibt auch einen Chatbot, bei dem man Text in den Prompt eingeben kann und der AI assistant kann vom Input Text übersetzen, zusammenfassen, erläutern, Rechtschreibung prüfen und Artikel schreiben. Bereits in der Testversion kann man source Dokument-Inhalte in Text- und Bildform bearbeiten. Bilder können um jeweils 90 Grad rotiert werden, extrahiert, zugeschnitten, ersetzt, verschoben, gelöscht und skaliert werden. Texte können skaliert, die Schriftart verändert, fett und kursiv gemacht, verschoben, gelöscht und die Farbe verändert werden. Text und Bilder können außerdem neu hinzugefügt werden. Beim Verschieben von Text, was durch das Ziehen der Textbox in verschiedene Richtungen möglich ist, bricht der Text automatisch um und passt sich der Textbox an. Hyperlinks und der Hintergrund können hinzugefügt, editiert und gelöscht werden. Stapelprozesse sind in der Pro-Version möglich. Als Prozesse kann man zwischen Konvertieren, Kombinieren, Einfügen, Drucken und Verschlüsseln wählen. PDFs können in maximum, high, medium oder low Stärken komprimiert werden. UPDF ist in 11 Sprachen inklusive Deutsch verfügbar \cite{updf-um}. UPDF kann auf Windows, macOS, iOS und Android installiert werden. Für eine jährliche Pauschale von 32,99 US Dollar ist UPDF Pro zu erwerben. Eine Dauerlizenz kostet 52,99 Dollar. Ausschließlich das AI Add-on kann für 59 Dollar pro Jahr gebucht werden \cite{updf-preis}. 

\subsection{Mathpix Snip}
Produktseite: \url{https://mathpix.com/snip} \\
Mathpix Snip ist ein PDF Editor mit Konvertierungsfeatures, dessen Zielgruppe Forscher, Lehrer und Studenten sind, d.h. Mathpix ist auf Forschungsdokumente und Research Papers ausgerichtet. Snip ist als Web App, im Apple Store, Google App Store und in der Huawei AppGallery verfügbar. Mittels \gls{ocr}-Technologie können sogar Formeln, Tabellen und zweispaltige PDFs in folgende Formate konvertiert werden: Word, LaTex, HTML und Markdown. Beim Online-Editor wird eine Mathpix Markdown Sprache verwendet und man kann auch Markdown, LaTeX und HTML verwenden um eine PDF-Datei zu erstellen. Für Desktop ist Snip ebenfalls für die Plattformen MacOS, Windows und Linux verfügbar. Es gibt außerdem eine Google Chrome Extension. Bilder können u.a. zu LaTeX, HTML, MathML, AsciiMath oder CSV konvertiert werden. Text und Matheformeln können direkt vom source PDF kopiert werden. Mit Search AI kann man Fragen über das Dokument stellen und erhält Antworten. Collaborative editing ermöglicht dem Ersteller, andere Benutzer zum Editiere hinzuzufügen und man kann live sehen wie bei Google Docs, wer gerade editiert. Handschriftliche Notizen können digitalisiert werden. Außerdem gibt es einen PDF Reader, der ebenfalls PDFs durch \gls{ocr} digitalisieren kann. Der Reader ist als Web App oder Mobil-Version verfügbar \cite{snip-um}. Die Snip App kostet 4,99 US Dollar pro Monat oder 49,90 Dollar pro Jahr. Die Gebühren für Organisationen, wie Schulen oder Firmen, kosten doppelt so viel. Es gibt eine kostenlose Version mit sehr begrenzter Anzahl an zu bearbeitenden PDF-Dateien. Sie reicht bei seltener Nutzung \cite{snip-price}.
\section{Freie PDF Programme und Onlinedienste}
PDF Dateien lassen sich in vielen Programmen einfach über den Druckdialog erstellen. Apple hat das Lesen von PDF Dokumenten in seiner Apples Vorschau integriert. Viele Webbrowser stellen PDF Viewer bereit, so Google Chrome seit 2010 \cite{wiki-pdf-de}.

\subsection{foxit}
Produktseite: \url{https://www.foxit.com/}
Die Firma foxit hat einen kostenlosen PDF Editor auf den Markt gebracht, der sehr umfangreiche Funktionen bietet.

\subsection{PDF24}
Produktseite: \url{https://www.pdf24.org/en/}

\subsection{PDFsam}
Produktseite: \url{https://pdfsam.org/}

\subsection{DrawboardPDF}
Produktseite: \url{https://www.drawboard.com/pdf/pdf}


\section{Relevanz von PDF in verschiedenen Marktbranchen}
Die PDF-Datei ersetzt als elektronisches Informationsaustauschformat in privaten Organisationen, Behörden und Bildungswesen papierbasierte Arbeitsprozesse. Vor allem in Behörden wird der E-Mail-Verkehr noch frequent verwendet. Starke Kompressionsalgorithmen ermöglichen es, speicherintensive PDF-Dateien auf leichtem Wege per E-Mail zu verschicken. Die PDF Spezifikation ist offen und sehr detailliert über sämtliche Aspekte des Dateiformats. Dadurch können Softwareunternehmen eigene Programme schreiben, die PDF Dokumente erzeugen, lesen und bearbeiten können. Searchable PDFs werden in Verträgen, Rechnungen und Geschäftsunterlagen verwendet, damit Mitarbeiter*innen Informationen gezielter suchen und Daten abteilungsübergreifend effizienter verwaltet werden können. In Forschungsarbeiten und wissenschaftlichen Artikeln werden searchable PDFs hauptsächlich bei der Überprüfung von Quellen oder dem Extrahieren von Zitaten verwendet. Behörden, Bibliotheken und Unternehmen digitalisieren Dokumente zur Archivierung und wandeln sie in ein searchable PDF um, was den langzeitigen Bestand der Dokumente sichert \cite{adobe-search}.
\par
Das PDF/A-Dateiformat wird in Bibliotheken und Archiven zur digitalen Archivierung von Büchern, Zeitschriften und historischen Dokumenten verwendet. Auch im Behördenzweig und Verwaltungssektor hat PDF/A für die Aufbewahrung von Verwaltungsakten und rechtlichen Dokumenten seine Existenzberechtigung. Im Gesundheitswegen wird es außerdem zur Speicherung von Patientenakten und medizinischen Unterlagen verwendet. Hingegen im Finanzwesen werden mit diesem Format Geschäftsunterlagen und Finanzdokumente verwahrt. Unternehmen und Organisationen können mit PDF/A gesetzliche und Compliance-Vorschriften einhalten \cite{adobe-pdf-a}. PDF/VT-Dateien werden im Direktmarketing verwendet. Personalisierte Werbematerialien erhöhen die Wahrscheinlichkeit einer positiven Reaktion bei den Kundi*nnen auf die Werbebotschaft und verbessert die Bindung von Unternehmen und Kund*innen. Der Transaktionsdruck findet bei Finanzdienstleistungen, Versicherungen und E-Commerce besonderen Anklang. Beliebte Transaktionsdokumente sind Rechnungen, Kontoauszüge, Versicherungspolicen oder Bestellbestätigungen \cite{adobe-pdf-vt}. PDF-Dokumente mit der PDF/UA-Kennzeichnung stärken den Ruf und die Reputation eines Unternehmens oder einer Organisation durch Engagement für Inklusion \cite{adobe-pdf-ua}. Digitale Signaturen werden bei digitalen Freigabe-, Abnahme-, Genehmigungs- und Vertragsprozesse verwendet. \gls{pades} wird in Rechtssystemen, Finanzwesen und Regierungssektor eingesetzt\cite{adobe-pdf-pades}. Im nächsten Kapitelabschnitt beleuchte ich die Verwendung von PDF-Dateien in der Druck- und Designindustrie genauer und beschreibe die Rolle von PDF im Agenturprozess.  
\section{Rolle von PDF in der Druckvorstufe und Designbranche}
Vor allem in der grafischen Industrie wird PDF gerne verwendet, weil es eine plattformübergreifende Visualisierung bietet auf allen Betriebssystemen. Schriften können bei Einbettung exakt wiedergegeben werden, unabhängig ob es sich um eine Windows oder MacOS Schrift handelt. Im Vergleich zu PostScript-Dateien erzielen die kompaktere Codierung von Seiteninhalten, dem einmaligen Speichern von identischen Bildern und die Verwendung von Kompressionsalgorithmen eine maßgeblich kleinere Dateigröße bei PDF. Korrekturänderungen in PDF-Dateien sind kurz vor dem Druck noch möglich und PDF entwickelte sich zunehmend zum Containerformat für alle grafischen Elemente. Die Produktion von Druckerzeugnissen wird somit wesentlich flexibler und sicherer. Downsampling und Kompression beschleunigt den Transport von der Agentur zum Dienstleistungsbüro enorm. In der Ausgabe ist die effektive Auflösung maßgeblich. Effektive Auflösung ist die Bildauflösung, die aus der Anzahl der Bildpunkte und der Fläche resultieren, auf der das Bild platziert wurde. Downsampling beeinflusst diese effektive Auflösung. Starke Artefakte fallen im Offsetdruck weniger auf als im Digitaldruck. \\ \cite{schneeberger}
Für die Betrachtung von Druckvorstufen-PDF-Dateien sollte immer Acrobat Pro bzw. Adobe Reader verwendet werden, da viele Drittanbieter-PDF-Viewer druckvorstufenrelevante Informationen nicht fehlerlos darstellen können. \cite{schneeberger}

In der Werbebranche werden PDF-Dateien vor allem für Korrekturabzüge verwendet. Ein Korrekturabzug ist eine Skizze bzw. Designvorlage des Werbeprodukts in vom Kunden gewünschten Position und Größe des Druckmotivs auf dem Werbeartikel. Als digitales Layout wird der Korrekturabzug mit dem Kunden abgestimmt und seine Änderungswünsche vor dem Druck entgegengenommen. \cite{korrektur}

Dank der Profile für unterschiedliche Geräte und Bedruckstoffe, kann eine Simulation des Ergebnisses am Monitor oder durch einen Prüfdruck bewerkstelligt werden. Dadurch steigt die Reproduktionssicherheit bei gleichbleibender Qualität und die Produktionszeiten verkürzen sich enorm. In den meisten Druckprojekten steht zu Beginn noch nicht fest, wann, wo und auf welchem Bedruckstoff gedruckt werden soll. Die zeitintensive Optimierung der Druckdaten kann durch PDF auf einen späteren Zeitpunkt verlegt werden.

Jeglicher Schutz sollte an einer druckvorstufentauglichen PDF-Datei vermieden werden, selbst wenn lediglich das Editieren gesperrt ist. Aktionen sind innerhalb druckfähiger Dateien untersagt, unabhängig von JavaScript oder Aktionen von Acrobat. \cite{schneeberger}

Da allgemeine Schriftinformationen immer eingebettet sind und die Zeilenlängen im Prinzip immer stimmen, können Druckvorstufenbetriebe zumindest immer erkennen, welche Schrift bzw. Schriftschnitt der Ersteller der PDF-Datei ursprünglich vorgesehen hatte, falls die Schrift nicht eingebettet wurde. \cite{schneeberger}

In der Praxis werden PDF-Dateien mit Hilfe von Prüfprofilen überprüft, Agenturen entwickeln nach eigenen Kriterien vorgefertigte Prüfroutinen und einen automatisch generierten Prüfbericht (Report). Der Prüfbericht sollte auf schnelle Fehlersuche im überprüften Dokument optimiert sein. Notwendige Korrekturen können im Originaldokument, bei PDF-Erstellung oder im vorliegenden PDF-Dokument ausgeübt werden je nach Schweregrad des Fehlers. Eine Erstellung von Korrekturprofilen kann dabei hilfreich sein. Man kann zwischen einer Vielzahl an voreingestellten Prüfprofilen und einer wesentlich kleineren Menge an Korrekturprofilen in Adobe Acrobat wählen. Außerdem kann der Prüfbericht als Kommunikationsmittel zwischen Auftraggeber und Agentur dienen, falls der Agentur das Originaldokument nicht ausgehändigt wurde. Das Preflight-Werkzeug in Acrobat Pro ist ein gängiges Werkzeug in der Druckvorstufe. Die Kontrolle von PDF-Dateien mit Preflight kann viel Geld sparen. Im tiefgreifenden Preflight Check, der ein System vor dessen Einsatz überprüft, kann man die besten Ergebnisse erzielen, wenn das zum Job passende Prüfprofil verwendet wird. Beim Preflighting-Prozess werden fehlerhafte und Objekte mit speziellen Anforderungen gefunden, ihr Zustand abgefragt und dann dementsprechende ein Fehler, eine Warnung oder eine Information ausgegeben. Im Warnungsfall muss die Agentur entscheiden, ob das Objekt, was die Warnung verursacht hat, sich negative auf das Endprodukt auswirkt. Folglich können durch Preflight frühzeitig Mängel im Produktionszyklus beseitigt werden. Ein Report sollte für die automatisierte Auswertung in weiterführenden Workflow-Systemen im \gls{xml}-Standard vorliegen. Erfolgreiche selbst erstellte Prüfprofile basieren auf qualitativ hochwertig gewählte Kriterien, wobei nicht zu viele verwendet werden sollten. Prüfprofile können als wichtigste Prüfkriterien das Dokument, Seiten, Bilder, Farbe, Zeichensätze, Rendering und Standard-Konformität enthalten. 
\cite{schneeberger}





