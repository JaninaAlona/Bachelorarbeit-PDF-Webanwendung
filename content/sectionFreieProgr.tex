\section{Freie PDF Programme und Onlinedienste}
In diesem Abschnitt stelle ich 3 kostenlose PDF-Programme vor, wobei 2 davon als Online-Programm verfügbar sind. Es gibt zudem PDFsam Basic zum Editieren von PDF-Seiten und DrawboardPDF, welches ausschließlich zum Zeichnen auf PDF-Dateien vorgesehen ist.

\subsection{Foxit PDF Reader}
Produktseite: \url{https://www.foxit.com/pdf-reader/} \\
Die Firma Foxit hat einen kostenlosen PDF-Reader auf den Markt gebracht, der sehr umfangreiche Funktionen bietet. Der Reader kann alle Seiten des PDFs in 90 Grad-Schritten rotieren, Lesezeichen hinzufügen, bestehende Texte kopieren und farbig markieren, sowie Text, Bilder, Audio und Video hinzufügen. Außerdem gibt es ein Messwerkzeug für 3D-Objekte. Der Reader hat eine Fill \& Sign-Funktion für elektronische Signaturen, bei der man seine Unterschrift zeichnen und sie für die folgende Unterschriften als Bild speichern kann. Digitale Signaturen mit Zertifikat können ebenfalls verwendet werden. Kommentare in Form von Zeichnungen, Geometrieformen und Text werden angeboten. Zeichnungen können wegradiert und Kommentarobjekte skaliert oder verschoben werden. Darüber hinaus können Kommentare zusammengefasst werden und eine erweiterte Suche für PDF-Inhalte ist möglich. Das Arbeiten mit Ebenen ist nur im Foxit PDF Editor realisiert und es können keine bestehenden PDF-Objekte editiert werden. Der Foxit PDF Reader ist als Webversion, für Windows, macOS, Linux, iOS und Android verfügbar. Zusätzlich gibt es auf Foxits Produktseite kostenlose PDF Converter Online Tools. Dort können mehrere PDFs zusammengefügt und komprimiert werden. PDF kann zu Word, JPEG, Powerpoint und Excel konvertiert und aus diesen Dateiformaten zurück zu PDF umgewandelt werden. Beim Foxit PDF Editor kann man PDF Dateien vergleichen und Wörter zählen. Im \gls{ocr}-Tool kann man die \gls{ocr}-Ergebnisse manuell korrigieren. Es gibt einen Preflight-Dialog. Ausschließlich in der Pro-Version kann man PDF-Dateien standardkonform zu PDF/A, PDF/E oder PDF/X Dateien umwandeln. Die Undo- und Redo-Optionen können Schritte rückgängig machen oder einen rückgängig gemachten Schritt wieder ausführen. Bezüglich der Pro-Variante ist ein Action Wizard für actions eingebaut. Text kann in der Standard-Variante editiert werden und Bilder in der Pro-Variante. Eine Prüfung der Rechtschreibung ist vorhanden. Das Arbeiten mit Ebenen, bezüglich Anzeigen bzw. Verbergen, Löschen und neu Ordnen ist freigeschaltet. Zusätzlich kann man in der Pro-Version Ebenen importieren, zusammenführen und reduzieren. Seiten können eingefügt, zerteilt, beschnitten, skaliert und basierend auf Lesezeichen neu angeordnet werden. ChatGPT ist im Editor integriert und bietet die Funktionen: Zusammenfassung, Umschreiben, Übersetzung, PDF-Chat, Inhaltserklärung, Korrektur der Rechtschreibung und ein AI Chatbot mit 50 prompts pro Tag. Eine 14-tägige kostenlose Testversion für den Foxit PDF Editor kann heruntergeladen werden. Die PDF Editor Suite für Einzelpersonen kostet 12,37 Euro im Monat, die Suite Pro 15,75 Euro pro Monat und die Cloud 6,63 Euro pro Monat. Man kann auch eine jährliche Zahlung leisten. Für den Bildungssektor muss lediglich eine sehr vergünstigte, jährliche Zahlung von Studenten, Lehrern und Institutionen geleistet werden \cite{foxit-um}. 

\subsection{PDF24 Tools}
Produktseite: \url{https://tools.pdf24.org/en/all-tools} \\
Bei den PDF24 Tools handelt es sich um eine Online-Lösung für PDF-Bearbeitung. Die Online-Lösung ist 100 \% kostenlos und die Verwendung ohne Limits. Es gibt auch eine Desktop-Version für Windows. PDF24 kann PDFs zusammenfügen, zerteilen, komprimieren, editieren und signieren. Beim Edit PDF-Tool kann man nur Elemente hinzufügen und keine Objekte aus dem source PDF bearbeiten. Es ist möglich Text, Zeichnungen, Bilder und Geometrie hinzuzufügen. Ergänzte Elemente können gelöscht, gedreht, skaliert oder verschoben werden. Wenn ein Element geklont wird, wird es leicht versetzt über dem Ursprungselement eingefügt. Bei Texten kann man auch eine Umrandungslinie mit Breite und Farbe, inklusive Deckkraft, anzeigen und es gibt eine begrenzte Zahl an Fonts: Sans-Serif, Serif und Monospace. Texte sind nach dem Speichern markierbar. Zeichnungen können auch mit Transparenz eingefärbt werden. Der Stift hat eine Liste an Pinseltypen zur Auswahl. Eine Radier-Option ist nicht vorhanden. Wenn man ein Element markiert, wird es in den Vordergrund geholt. Jedes Element ist einzeln selektierbar. Es gibt eine Undo- und Redo-Funktion. PDF-Dateien können zu anderen Dateiformaten und von anderen Formaten zu PDF konvertiert werden. Die unterstützten Dateiformate sind u.a. Word, Excel, Powerpoint, JPEG, PNG, TIFF, SVG, HTML und TXT. Alle Bilder aus einem PDF können extrahiert und als ZIP-Ordner heruntergeladen werden. PDFs können mit einem Passwort geschützt, Rechte vergeben und Passwörter entfernt werden. Seiten können gedreht, entfernt, extrahiert und neu geordnet werden. Eine Webseite kann als PDF gespeichert werden und eine \gls{ocr}-Funktionalität ist implementiert. Außerdem können PDFs verglichen, für Web optimiert und kommentiert werden. Metadaten können editiert oder gelöscht werden \cite{pdf24}. Im Großen und Ganzen bietet PDF24 eine breite Palette an kostenlosen Tools, die kaum Wünsche offen lassen.

\subsection{ASPOSE.PDF Apps}
Produktseite: \url{https://products.aspose.app/pdf/family} \\
Aspose offeriert eine Vielzahl an kostenlosen PDF Online Apps für verschiedene PDF-Szenarien. Es gibt u.a. folgende Apps: PDF Editor, Splitter zum Seiten Extrahieren, Converter, Merger zum PDFs Zusammenfügen, eine App für Metadaten, eine App zum Vergleichen von 2 PDF-Dateien, eine App für Signaturen, PDF Unlock zum Entfernen von Passwörtern, PDF Lock zum Sichern durch Passwörter, PDF \gls{xfa} to Acroform, PDF Search, eine App für das Extrahieren von Text und Bildern, eine App zum Suchen und Ersetzen von vertraulichen Informationen auch mit regular expressions, ein Übersetzer, PDF Compression, eine \gls{ocr} App, eine App zum Seiten Drehen, eine App zum Bilder Beschneiden, PDF Music Video Maker, PDF Meme Generator, eine App zum Formular Ausfüllen, eine App zum Entfernen von Kommentaren, Stempeln und Wasserzeichen, PDF Gif Maker und PDF Table Extraction. Der PDF-Converter konvertiert eine PDF-Datei zu einer Vielzahl an Formaten u.a. Word, Excel, PowerPoint, EPS, HTML, SVG, \gls{xml}, ZIP, LaTeX, PostScript, JPG, TIFF, MP4, MOV, WAP, MP3 und JSON. Beim PDF-Editor kann man lediglich Objekte hinzufügen: Text, Bilder, Zeichnungen, Ellipsen oder Rechtecke. Die Objekte können rotiert, skaliert und verschoben werden. Jedes Element wird als Ebene einer Seitengruppe hinzugefügt und hat einen unabänderlichen default Namen. Die Elemente können in der Reihenfolge verschoben werden, jedoch wirkt sich das nicht wirklich auf die z-Reihenfolge der Elemente aus, welche im Vorder- oder Hintergrund liegen. Wenn man ein Element auswählt, rückt es automatisch in den Vordergrund. Es gibt einen Remove all-Button, um alle hinzugefügten Elemente auf einen Schlag zu Entfernen. Eine Undo- und Redo-Option ist nicht vorhanden. Bei Text, Zeichnung und Geometrie kann man die Farben variieren. Selbst bei der Zeichnung wird die bereits gemalte Linie farblich verändert, sobald die Farbe angepasst wird. Einen Radier-Funktion ist nicht vorhanden. Text ist nach Speichern markierbar, jedoch kann man nur Times New Roman, Courier oder Helvetica jeweils in normal, fett oder kursiv anwenden. Bei allen Elementen kann die Deckkraft eingestellt werden. Aspose bietet eine ganze Reihe von PDF On Premise APIs an, die sich an Firmen richtet: Aspose.PDF for .NET, Java, C++, Python via.NET, Python via Java, Python via C++, SharePoint, Android via Java, Reporting Services, JasperReports, JavaScript via C++ und Node.js via C++. Die APIs sind spezialisiert auf verschiedene Aufgabenbereiche mit PDF-Dokumenten. Beispielsweise Aspose.PDF for Python via C++ kann PDFs lesen, schreiben, konvertieren und manipulieren. Eine kostenlose Testversion ist verfügbar. Preislich fängt die komplette API-Produktpalette bei 1679 US Dollar an. Mit dessen Libraries kann man neben PDF 100 andere Dateiformate, wie Word, Excel und PowerPoint manipulieren \cite{aspose-api}.

