\section{Rolle von PDF in der Druckvorstufe und Designbranche}
Seit PDF 1.3 werden \gls{icc}-Profile unterstützt, die die Farbeigenschaften, Helligkeit, Weißpunkt, Gammakurve und Farbumfang eines bestimmten Monitors eines spezifischen Geräts beschreiben, sprich ein \gls{icc}-Profil beschreibt, wie Farben von diesem Gerät dargestellt werden können. Außerdem wird die Transformation zwischen dem Gerät und dem Profilverbindungsraum \gls{pcs} definiert. Dabei gibt es die Variante Eingabeprofile für Kameras und Scanner und Ausgabeprofile für Monitore und Drucker. Zweck des \gls{icc}-Profils ist möglichst Farbübereinstimmungen zwischen verschiedenen Geräten zu erzielen. \cite{benq} \\
Beim \gls{pcs} handelt es sich um ein neutrales Farbmodell im \gls{icc}-Colormanagement, welches den Quellfarbraum mit dem Zielfarbraum verbindet und somit geräteunabhängig ist. Der \gls{pcs} kann entweder der LAB oder XYZ Farbraum sein. \cite{prepress} \\
Der DeviceN-Farbraum, der seit PDF 1.3 verwendet werden kann, wird auch in PostScript 3 unterstützt und erlaubt die willkürliche Kombinationen von Farbkanälen beim Composite-Druck. Dokumente mit Schmuckfarben müssen auf einem Gerät mit physikalisch getrennten Kanälen für jede verwendete Schmuckfarbe ausgegeben werden. Folglich kann kein CMYK- oder RGB-Gerät Dokumente mit Schmuckfarben farblich korrekt darstellen. Davon sind fast alle Farbdruckersysteme betroffen, sowie die von Adobe Acrobat erzeugte Bildschirmdarstellung von PDF Dokumenten mit Schmuckfarben. Ohne den DeviceN Farbraum können Bilder mit Kombinationen von z.B. CMYK und 2 Schmuckfarben oder Schwarz und eine Schmuckfarbe nicht im Composite-PostScript und Composite-PDF wiedergegeben werden, sondern höchstens mit CMYK als Näherung. \cite{helios} \gls{opi} ist ein Workflow Protokoll, welches in der elektronischen Druckvorstufe verwendet werden kann, um Desktop Publishing Systeme und high-end \gls{ceps} zu verknüpfen und optimiert die Übertragung von hochauflösenden Dateien in Netzwerken. \cite{printwiki} \\
Seit PDF 1.3 werden \gls{cid} Schrifttypen unterstützt. \gls{cid} ist ein Synonym für das PostScript Type o Format, das eine Adressierung von mehr als 256 Zeichen ermöglicht und für Fonts mit einer großen Zeichenanzahl verwendet wurde. \cite{typoinfo}

\subsection{Fontformate}
Composite-Fonts sind Basisschriften mit hierarchischem System. Die oberste hierarchische Ebene stellt den root font dar alle folgenden Fonts sind descendant fonts. Sie ermöglichen die Einführung von Type-1-Schriften im asiatischen Markt. \cite{schneeberger}