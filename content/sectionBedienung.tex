\section{Realisierte Funktionalität und Bedienung}
In der PDF Web App habe ich alle geplanten Module Reader, Creator, Splitter, Merger, Writer, Drawer, Shaper und Imager realisiert. Nach dem ersten Öffnen der PDF Web App, wird die in Screenshot \ref{fig:start} abgebildete Startseite vorgefunden.

\begin{figure}[!htbp]
	\centering
	\includegraphics[width=1\textwidth]{"images/startseite.png"}
	\caption{Startseite der PDF Web App}
	\label{fig:start}
\end{figure}

Bei fast allen Modulen gibt es die Möglichkeit Benutzereingaben zu machen. Die Benutzereingaben sind derart umgesetzt, dass sie bei ungültigem Input automatisch korrigiert werden oder die darauf bezogene Operation nicht ausgeführt wird (Input Control). Wird in ein input field, wo eine Zahl erwartet wird, ein String eingegeben, so wird die Funktion nicht angewendet. Liegt eine Benutzereingabe als Zahl unter oder über dem minimalen oder maximalen Schwellenwert des Eingabefeldes, so wird die Eingabe mit dem minimalen oder maximalen Wert substituiert. Manche Eingabefelder erwarten Integers, anstatt Floats, z.B. das input field für die aktuelle Seitenzahl. In diesem Fall wird die Nachkommastelle automatisch verworfen. Haben sich Leerzeichen in die Eingabe eingeschlichen, so wird der white space von der Anwendung automatisiert erkannt und entfernt. Falls eine andere Dateiart als PDF, unabhängig vom Modul der App, geöffnet wurde, erscheint die Fehlermeldung in Screenshot \ref{fig:errorfile}. Auch beim Versuch eine verschlüsselte PDF-Datei zu öffnen, wird eine in Screenshot \ref{fig:errorcrypt} dargestellte Fehlermeldung angezeigt.

\begin{figure}[!htbp]
	\centering
	\includegraphics[width=1\textwidth]{"images/errorfile.png"}
	\caption{Fehlermeldung bei einer nicht-PDF-Datei}
	\label{fig:errorfile}
\end{figure}

\begin{figure}[!htbp]
	\centering
	\includegraphics[width=1\textwidth]{"images/errorcrypt.png"}
	\caption{Fehlermeldung bei einem verschlüsselten PDF}
	\label{fig:errorcrypt}
\end{figure}

\begin{figure}[!htbp]
	\centering
	\includegraphics[width=1\textwidth]{"images/errorpages.png"}
	\caption{Fehlermeldung bei einem Input- oder Output-PDF mit mehr als 5000 Seiten}
	\label{fig:errorpages}
\end{figure}

Alle Module unterstützen Input- und Output-PDFs von maximal 5000 Seiten. Falls ein PDF mit mehr als 5000 Seiten geöffnet wird, erscheint eine entsprechende Fehlermeldung. Sie erscheint ebenfalls im Merger, ein PDF von über 5000 Seiten gemergt werden soll, was Screenshot \ref{fig:errorpages} visualisiert. Die Fehlermeldung verschwindet, sobald ein neues PDF-Dokument ausgewählt oder ein anderer Hauptmenüpunkt aufgerufen wurde. \\
Ist ein Modul der PDF Web App geöffnet, so wird der entsprechende Hauptmenübutton durch einen dunkelgrauen Hintergrund mit grüner Schrift symbolisiert. Initial ist der Reader ausgewählt. Der Button Create führt zum Creator für leere PDFs, Split zum Splitter für das Seiten Zerteilen, Merge zum Merger für das Zusammenfügen von PDF-Dateien, Text zum Writer für Textbearbeitung, Draw zum Drawer fürs Zeichnen, Shape zum Shaper zur Geometrieerstellung und Image zum Imager zur Bildbearbeitung. Falls aktuell der Reader, Merger, Splitter oder Creator geöffnet ist und anschließend auf ein Editormodul geklickt wird, ist standardmäßig der Writer ausgewählt. Um ein anderes Editormodul zu aktivieren, muss abermals auf das gewünschte Editormodul geklickt werden. Jedes Modul der PDF Web App hat einen grünen Save-Button, mit das aktuelle PDF als ZIP-Archiv gedownloaded werden kann. Es wird im Downloads-Ordner im lokalen Dateisystem abgelegt. Sobald die Maus über den Save-Button bewegt wird, erscheint eine schwarze Dialogbox, in der ein benutzerdefinierter Dateinamen mit maximal 50 Zeichen vergeben werden kann. Wurde ein benutzerdefinierter Dateinamen definiert, so erhält das Output-PDF und der ZIP-Ordner diesen Namen, ansonsten wird ein default Name verwendet, der aus dem Ursprungsdateinamen und einem Suffix besteht. Der Screenshot \ref{fig:save} bildet die Dialogbox zur Vergabe des Dateinamens ab. Bei Splitter und Merger wird der Dateiname, der im Dateibrowser selektierten Datei, auf 50 Zeichen gekürzt, falls der Dateiname größer oder gleich 54 Zeichen ist.

\begin{figure}[!htbp]
	\centering
	\includegraphics[width=0.6\textwidth]{"images/save.png"}
	\caption{Dateibenennungsdialog des Save-Buttons der PDF Web App}
	\label{fig:save}
\end{figure}

Bei Read, Text, Draw, Shape und Image erscheint zunächst der Choose file-Button, damit im Dateisystem ein PDF-Dokument zum Lesen oder Bearbeiten auswählt werden kann. Wird auf Choose file geklickt, öffnet sich der Dateibrowser und eine einzelne PDF-Datei kann zum Öffnen ausgewählt werden. Ein geöffnetes Dokument passt seine Zoomgröße basierend auf die Größe der ersten Seite automatisch an das Browserfenster an, sodass es formatfüllend mit Randabstand in den Viewport des Browserfensters passt. 

\subsection{Bedienung des Readers}
Der Reader mit geöffneter PDF-Datei präsentiert sich in Screenshot \ref{fig:reader}.

\begin{figure}[!htbp]
	\centering
	\includegraphics[width=1\textwidth]{"images/reader.png"}
	\caption{Geöffnetes PDF im Reader der PDF Web App}
	\label{fig:reader}
\end{figure}

Hat der Anwender eine PDF-Datei im Reader geöffnet, so entdeckt er 2 dunkelgraue Leisten mit Funktionsbuttons. Mittels Previous und Next kann der Benutzer zur vorherigen bzw. nächsten Seite blättern. Zwischen diesen Buttons informiert das pageCounter input field über die aktuelle Seite im Viewport und rechts daneben ist die Anzahl an Seiten im Dokument zu sehen. Im pageCounter input field kann man eine Seitenzahl eingeben, mit Enter bestätigen und der Reader springt direkt zu dieser Zielseite. Alternativ kann man mit dem Scrollbar am linken Browserfensterrand oder dem Scrollrad der Maus durch die Seiten scrollen. Mittels der Buttons Plus und Minus kann man in 20 \%-Schritten rein- bzw. rauszoomen. Der aktuelle Zoomwert in Prozent wird im input field dazwischen signalisiert. Der Anwender kann den Zoomwert auch auf einen gewünschten Wert mit oder ohne Prozentzeichen setzen und mit Enter bestätigen, damit der Zoomwert angewendet wird. Wird kein Prozentzeichen über die Tastatur eingegeben, sondern nur der Wert, so wird ein Prozentzeichen von der Anwendung hinzugefügt. Dabei werden keine Nachkommastellen berücksichtigt. Außerdem wird der minimale und maximale Zoomwert in Prozent angezeigt. Spin CW und Spin CCW dreht die aktuelle Seite, die der pageCounter signalisiert, in 90 Grad Schritten im Uhrzeigersinn (clockwise) und gegen den Uhrzeigersinn (counterclockwise). Durch den Button Drag kann man die aktuelle Seite im Viewport verschieben. Dabei klickt man zuerst auf Drag, hält die Maustaste auf der aktuellen Seite gedrückt und bewegt sie in die gewünschte Richtung. Dabei verschiebt sich nicht nur die aktuelle Seite, sondern alle Seiten und der Mauscursor wechselt das Aussehen zu einem weißen Kreuz mit Pfeilen an den Enden.

\subsection{Bedienung des Creators}
Der Creator ist mittels des Create-Buttons im Hauptmenü aufrufbar. Im Screenshot \ref{fig:creator} ist die GUI vom Creator dargestellt. 

\begin{figure}[!htbp]
	\centering
	\includegraphics[width=1\textwidth]{"images/creator.png"}
	\caption{Creator GUI der PDF Web App}
	\label{fig:creator}
\end{figure}

\begin{figure}[!htbp]
	\centering
	\includegraphics[width=0.7\textwidth]{"images/creator-sel.png"}
	\caption{Creator selection menu der PDF Web App}
	\label{fig:creator-sel}
\end{figure}

Man gibt eine Anzahl an gewünschten Seiten des leeren PDFs ein, sowie die Breite und Höhe in mm. Wahlweise kann man das grüne selection menu benutzen, um ein DIN A-Preset zu verwenden, was der Screenshot \ref{fig:creator-sel} darstellt. Mittels der Schnellauswahl kann man die Orientierung bestimmten: Portrait, Landscape oder Quadratic. Minimale und maximale Werte für die Anzahl an Seiten und die Breite und Höhe sind ebenfalls abgebildet. 

\subsection{Bedienung des Splitters}
Zum Splitter kann man mit dem Split Button gelangen, dessen GUI vom Screenshot \ref{fig:splitter} gezeigt wird.

\begin{figure}[!htbp]
	\centering
	\includegraphics[width=1\textwidth]{"images/splitter.png"}
	\caption{Splitter GUI der PDF Web App}
	\label{fig:splitter}
\end{figure}

\begin{figure}[!htbp]
	\centering
	\includegraphics[width=0.7\textwidth]{"images/splitter2.png"}
	\caption{Splitter selection menu der PDF Web App}
	\label{fig:splitter2}
\end{figure}

Hat man eine Datei ausgewählt, so wird der Dateiname und die Anzahl an Seiten des Dokuments angezeigt. Durch erneutes Klicken des Choose file-Buttons öffnet sich abermals der Dateidialog und man kann eine andere PDF-Datei auswählen. Dabei ersetzt die neue Datei die vorherige, denn man kann nicht mehrere PDF-Dateien gleichzeitig splitten. Im selection menu kann man zwischen Zerteilen nach every page, odd pages, even pages und list of pages wählen, was in Abbildung \ref{fig:splitter2} dargestellt wird. Selektiert man list of pages, so kann man die einzelnen mit Komma separierten Seitennummern oder auch nur eine einzelne Seitennummer eintippen. Die Seitenzahlen müssen nicht in aufsteigender Reihenfolge angegeben werden. Bei ungültigen Eingaben wird das input field für die Seitenliste automatisch geleert. 

\subsection{Bedienung des Mergers}
Der Merger ist mit dem Hauptmenüpunkt Merge zu öffnen. Abbildung \ref{fig:merger} zeigt die Startseite des Mergers. 

\begin{figure}[!htbp]
	\centering
	\includegraphics[width=1\textwidth]{"images/merger.png"}
	\caption{Merger-Startseite der PDF Web App}
	\label{fig:merger}
\end{figure}

Hier kann man nacheinander mehrere Dateien mittels Choose file öffnen und sie erscheinen, je nach Auswahlreihenfolge, untereinander in einer scrollbaren Liste, wobei die zuerst ausgewählte Datei am Anfang der Liste steht. Es können maximal 100 Dateien der Liste hinzugefügt werden. In der Liste kann man dann eine Quelldatei mit gedrückter Maustaste zu einer Zieldatei in der Liste ziehen. Lässt man die Maustaste los, so wird die mit der Maus bewegte Datei in diese Listenposition eingefügt. Man kann außerdem eine Datei in der Liste selektieren und mittels des Remove-Buttons wieder aus der Liste entfernen. Eine selektierte Datei wird durch einen schwarzen Hintergrund, was der Screenshot \ref{fig:mergelist} verdeutlicht, symbolisiert.

\begin{figure}[!htbp]
	\centering
	\includegraphics[width=0.7\textwidth]{"images/mergelist.png"}
	\caption{Merger Dateiliste der PDF Web App mit selektierter Datei}
	\label{fig:mergelist}
\end{figure}

Der Benutzer kann zu jeder Zeit erneut den Dateibrowser bedienen, unabhängig davon, ob die Dateiliste bereits modifiziert wurde. Der Save-Button mergt alle PDF-Dateien in der gegenwärtigen Liste mit der obersten Datei zuerst des Output-PDFs.

\subsection{Bedienung des Editors}
Der Editor ist über den Text, Draw, Shape oder Image Button erreichbar. Genau wie im Reader erscheint ein Button Choose file. Je nachdem ob man auf Text, Draw, Shape oder Image geklickt hat, wird als erstes der Text-, Zeichen-, Geometrie- oder Bildeditor geöffnet. Hat man eine Datei geöffnet, so befindet sich der Reader ohne die Operationen zum Seiten Drehen ebenfalls im Editor, sowie seine Drag und Save Funktionalitäten. Bei Save kann man wie bei anderen Modulen einen benutzerdefinierten Dateinamen vergeben und das aktuelle PDF wird in den Downloads-Ordner runtergeladen. Alle input fields im Editor sind mit dem gültigen Wertebereich für Benutzereingaben als Information Min: Max: versehen. Der Editor besteht aus einem grauen waagerechten Operations Bar, einem linken Ebenen Seitenmenü in Rosa und einem rechten grünen Tools Seitenmenü. Mit dem ganz linken grünen Button Layers im Operations Bar kann das Ebenen Seitenmenü aus- und eingeblendet werden. Daneben zeigt oder verbirgt der Button Tools das Tools Seitenmenü. 

\subsubsection{Textbearbeitung}
Hat man den Texteditor aufgerufen, so präsentiert sich einem der Editor in folgenden Abbildungen \ref{fig:texteditor} und \ref{fig:texteditor2}.

\begin{figure}[!htbp]
	\centering
	\includegraphics[width=1\textwidth]{"images/texteditor.png"}
	\caption{Startseite des Texteditors der PDF Web App}
	\label{fig:texteditor}
\end{figure}

\begin{figure}[!htbp]
	\centering
	\includegraphics[width=1\textwidth]{"images/texteditor2.png"}
	\caption{Mehr Tools der Startseite des Texteditors der PDF Web App}
	\label{fig:texteditor2}
\end{figure}

Mit dem Button Text in der grauen Leiste und nachfolgendem Klick aufs geöffnete Dokument kann man einen Text hinzufügen mit dem Platzhaltertext dummy. Unter dem Text erscheint eine dunkelrote control box, auf die man alle Operationen in der grauen Leiste und dem Tools Seitenmenü im Box Mode anwenden kann. Der Box Mode ist standardmäßig eingestellt. Alle Operationen im rechten Tools Seitenmenü beziehen sich jeweils auf das aktuelle Editor Element und sind nur auf diesem anwendbar. Ich werde zunächst alle Operationen im Box Mode beschreiben und später auf den Layer Mode eingehen. Man kann mehrere Texte ohne erneut Text drücken zu müssen dem PDF Dokument hinzufügen. Für jedes neu hinzugefügte Textelement wird eine Ebene mit einem  Element spezifischen Standardnamen erstellt, was im linken rosa Ebenenmenü zu sehen ist. Mit dem Delete Button und nachfolgendem Klick in eine oder mehrere control boxen im Box Mode können Texte wieder gelöscht werden. Move verschiebt einzelne Texte durch mit der Maus gedrückte control box. Wenn die Maus losgelassen, nachdem die control box verschoben wurde, springt der Text an die verschobene Stelle. Ganz oben im grünen Tools Seitenmenü werden dem Betrachter die x- und y-Koordinaten der Maus auf der PDF Seite angezeigt, wenn die Maus über eine Seite bewegt wird. Darunter kann man in der textarea den Text editieren. Es werden auch Zeilenumbrüche berücksichtigt. Nachdem man den dummy Text überschrieben hat, einem Klick auf den weißen Text Button und ein oder mehrere Klicks in control boxen, kann der Text angewendet werden. Alle Operationen in Tools werden genau gleich ausgeführt: Man tätigt seine Einstellung, drückt mit der linken Maustaste auf den weißen Button für die jeweilige Operation und klickt daraufhin auf ein oder mehrere Textelemente nacheinander. Darunter kann man den Zeilenabstand einstellen. Entweder verwendet man das selection menu mit voreingestellten Werten oder man gibt einen gewünschten Wert manuell in das input field ein. Alle selection Menüs und input fields in jedem Editor zeigen die default Werte, mit denen ein neu hinzugefügtes Element konfiguriert ist, an. Falls man zuletzt das selection Menü betätigt hat, überschreibt es den Wert im input field und umgekehrt. Maßgeblich ist, was man zuletzt betätigt hatte. Dieses Verhalten habe ich bei jeder selection menu und input field Kombination programmiert. Einen benutzerdefinierten Font kann man durch den dunkelgrauen Choose file Button auswählen und er erscheint in der Liste. Der zuletzt hochgeladene Font wird ausgewählt. Mittels clear kann man einen ausgewählten Font aus der Liste entfernen, was nicht heißt, dass er auch auf dem angewendeten Text entfernt wird. Abbildung \ref{fig:custom-font} zeigt 2 geöffnete .ttf Schriftdateien in der Liste.

\begin{figure}[!htbp]
	\centering
	\includegraphics[width=0.3\textwidth]{"images/custom-font.png"}
	\caption{Benutzerdefinierte Fontliste im Texteditor der PDF Web App}
	\label{fig:custom-font}
\end{figure}

Die Fontgröße kann man ebenfalls wie die Zeilenhöhe mit selection menu und input field justieren. Bei der Fontfarbe klickt man auf das initial schwarze Quadrat, was die aktuelle Farbe zeigt, und es öffnet sich ein color picker Menü. Hier kann man die Farbe und Transparenz einstellen. Die Werte kann man sich in RGBA, HSLA oder HEX formatieren lassen. Mit Klick auf die beiden kleinen senkrechten Pfeile im color picker wird jeweils das Format gewechselt. Das Fenster des color pickers für die Fontfarbe ist in Abbildung \ref{fig:fontcolor} abgebildet. 

\begin{figure}[!htbp]
	\centering
	\includegraphics[width=0.5\textwidth]{"images/fontcolor.png"}
	\caption{Color picker für die Fontfarbe des Texteditors der PDF Web App}
	\label{fig:fontcolor}
\end{figure}

Als vorletzte Option kann man den Text absolut drehen. Durch den weißen Button Rotation und der entsprechenden Benutzerinteraktion durch selection Menu oder input field wird das Textelement absolut gedreht. Das bedeutet, dass es eine feste Rotationsskala gibt mit denen das Element rotiert wird. Folglich passiert keine Veränderung, wenn man den gleichen Rotationswert 2 Mal hintereinander ausführt. Wählt man 0 Grad aus zum rotieren, wird das Element wieder zurück auf die Ausgangsrotation gedreht. Abschließend können alle Textelemente im Dokument mit dem Remove Button auf einen Schlag gelöscht werden. Beim Zeilenabstand und der Schriftgröße wird der Benutzer außerdem über den Wertebereich von Benutzereingaben informiert. Man kann generell nur Elemente hinzufügen und auf ihnen die Operationen anwenden. Man kann in der PDF Web App keine im PDF bereits bestehenden Elemente bearbeiten. Die Textoperationen werden in Abbildung \ref{fig:text} demonstriert.

\begin{figure}[!htbp]
	\centering
	\includegraphics[width=0.9\textwidth]{"images/text.png"}
	\caption{Bearbeiteter Text in der PDF Web App}
	\label{fig:text}
\end{figure}

\subsubsection{Zeichnungen erstellen}
Der Zeichneneditor präsentiert sich einem in Screenshot \ref{fig:drawer}. 

\begin{figure}[!htbp]
	\centering
	\includegraphics[width=1\textwidth]{"images/drawer.png"}
	\caption{Drawer der PDF Web App}
	\label{fig:drawer}
\end{figure}

Das Ebenenmenü und Tools Seitenmenü des Zeichneneditors erscheint selbst wenn man zuerst im Texteditor ein Dokument geöffnet hat. Nach einem geöffneten PDF kann man von jedem Editor in einen anderen wechseln ohne erneut ein PDF öffnen zu müssen. Mit dem Zeichnen kann man anfangen, wenn man auf Pencil klickt. Bei gedrückter Maustaste auf einer PDF Seite erscheint eine schwarze Linie dort wo die Maus sich bewegt hat. Zusätzlich wird dort wo man angefangen hat die Maus zu drücken eine magenta control box hinzugefügt. Das Zeichnen funktioniert auch mit einem Graphic Tablet. Es wird immer auf der zuletzt gezeichneten Ebene auf der Seite gemalt bzw. wenn man eine Ebene auswählt im linken Ebenenmenü wird auf der ausgewählten Ebene gezeichnet. Ein Klick auf New Layer und anschließender Zeichenmodus mit Pencil kreiert für die neue Zeichnung eine weitere Ebene. Die neue Zeichnung auf der Ebene erhält abermals eine magenta control box. Wurde auf einer Seite bisher noch nichts gezeichnet, so wird bei der ersten Zeichnung auf der Seite eine neue Ebene automatisch angelegt und man muss nicht New Layer drücken. Die Zeichenelemente sind die einzigen Objekte, bei denen der Nutzer selbst die Ebenen einer Seite zuweisen kann mit New Layer. Bei allen anderen Elementen, sei es Text, Geometrie oder Bilder, wird für jedes neue Element automatisch eine Ebene erstellt. Der Radierer ist  mit dem Eraser Button im grauen operations Menü anwendbar. Zuerst drückt man Eraser und geht dann mit gedrückter Maustaste über die Zeichnungen auf einer Seite, die man entfernen möchte. Dort wo bei gedrückter Maustaste die Maus die Linie berührt wird wegradiert. Zeichnen und Radieren bekommen jeweils ein neues Mauscursorsymbol: Beim Zeichnen hat man ein schwarzes dünnes Kreuz und beim Radieren ein weißes dickes Kreuz. Mit dem Delete Button kann man mehrere Zeichnungen nacheinander mit Klicks in die control boxen entfernen. Move und gedrückte Maustaste auf eine control box verschiebt diese. Delete und Move funktionieren analog zum Texteditor. In jedem Editor gibt es einen Delete und Move Button zum Löschen und Verschieben von Elementen. Genau wie alle Operationen im Tools Seitenmenü kann man mit Delete und Move nur die dem Editor zugehörigen Elemente editieren. Im Tools Seitenmenü kann man eine Zeichnung relativ skalieren, indem man einen Faktor eingibt. Der Faktor kann auch ein Float sein und multipliziert sich immer mit der aktuellen Größe, d.h. die aktuelle Größe ist 100 \%. Darunter kann man mit dem color picker menü die Farbe und Transparenz der Stiftfarbe definieren. Sie wird mit einem Klick auf Pencil Color auf die nächste Zeichenoperation angewendet. Außerdem definiert sie auch gleichzeitig die Radiererfarbe, was sich bei Transparenzen unter 1 bemerkbar macht. Sonst ist jede Radiererfarbe gleich, jedoch bei einer Transparenz von unter 1 radiert der Radierer weniger deckend, so als ob man einen Radiergrummi weniger stark auf das Papier drückt. Dann kann man die Größe des Stiftes bzw. Radierers einstellen. Sie greift auch ab der nächsten Zeichen- bzw. Radieroperationen. Ebenfalls kann man Zeichnung mit samt radierten Partien rotieren. Zum Schluss kann man mit Remove alle Zeichnungen im Dokument löschen. Teilweise transparente Zeichnungen werden im Bild \ref{fig:drawing} dargestellt. 

\begin{figure}[!htbp]
	\centering
	\includegraphics[width=1\textwidth]{"images/drawing.png"}
	\caption{Zeichnungen in der PDF Web App}
	\label{fig:drawing}
\end{figure}


\subsubsection{Geometrie hinzufügen}
Die Startseite des Geometrieeditors ist in den Screenshots \ref{fig:shaper} und \ref{fig:shaper2} abgebildet. 

\begin{figure}[!htbp]
	\centering
	\includegraphics[width=1\textwidth]{"images/shaper.png"}
	\caption{Geometrieeditor in der PDF Web App}
	\label{fig:shaper}
\end{figure}

\begin{figure}[!htbp]
	\centering
	\includegraphics[width=1\textwidth]{"images/shaper2.png"}
	\caption{Mehr Tools des Geometrieeditors in der PDF Web App}
	\label{fig:shaper2}
\end{figure}

Der Geometrietyp kann durch die Buttons Rectangle für Rechteck, Triangle für Dreieck oder Ellipse  und einem oder mehreren Klicks auf eine PDF-Seite bestimmt werden. Ebenfalls können alle Shapetypen durch Delete gelöscht und durch Move auf der Seite verschoben werden. Im Tools Seitenmenü des Geometrieeditors gibt es eine einzige Operation, die nur auf Dreiecke angewendet werden kann. Es handelt sich um die oberste Einstellung für die Position des dritten Punktes des Dreiecks. Hiermit kann der rechte Punkt der langen Spitze des default Dreiecks bearbeitet werden. Alle anderen Einstellmöglichkeiten können auf allen Geometrieelementen Rechteck, Dreieck und Ellipse arbeiten. Man hat 2 Möglichkeiten einen Shape zu skalieren. Zum einen kann man die Breite und Höhe unabhängig voneinander einstellen, was eine absolute Skalierung bedeutet, oder man verwendet den Skalierungsfaktor, der relativ vergrößert genauso wie beim Zeichneneditor. Für die Shape Umrandungslinien kann man auf der einen Seite die Farbe inklusive Deckkraft und auf der anderen Seite die Breite der Linie justieren. Die Strichfarbe muss mit der Checkbox Use Stroke in Grün eingeschaltet sein, was sie beim ersten Öffnen des Editors auch ist. Deaktiviert man die Use Stroke Checkbox schaltet sich automatisch die Use Fill Checkbox an. Man kann auch beide Checkboxen einschalten, aber nicht beide zusammen ausschalten. Ist Use Stroke rosa, d.h. deaktiviert, und man wendet die Strichbreite an, dann wird trotzdem ein Strich in der letzten Strichfarbe gesetzt in der gewünschten Breite. Use Fill muss Grün sein, um die Füllfarbe anzuwenden. Bei Strich- und Füllfarbe wird ein wie in dem Text- und Zeicheneditor der gleiche color picker verwendet. Alle Shapes können mit absoluter Rotation rotiert werden. Im ganzen Editor kann ausschließlich absolut rotiert werden. Zuunterst entfernt der Remove Button alle Geometrieelemente im geöffneten PDF. Der Screenshot \ref{fig:shaping} hebt mehrere bearbeitete Geometrieelemente hervor.

\begin{figure}[!htbp]
	\centering
	\includegraphics[width=1\textwidth]{"images/shaping.png"}
	\caption{Zeichnungen in der PDF Web App}
	\label{fig:shaping}
\end{figure}



\subsubsection{Bilder einfügen}

\subsubsection{Arbeiten im Layer Mode}

\subsubsection{Ebenensteuerung}



