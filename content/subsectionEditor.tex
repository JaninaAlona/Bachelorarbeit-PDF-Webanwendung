\subsection{Bedienung des Editors}
Der Editor ist über den Text-, Draw-, Shape- oder Image-Button erreichbar. Ebenso wie im Reader erscheint zuerst ein Button Choose file. Je nachdem ob nach erstmaligem Öffnen einer PDF-Datei auf Text, Draw, Shape oder Image geklickt wurde, wird als erstes der Writer-, Drawer-, Shaper- oder Imager-Bearbeitungsmodus geöffnet. Ist eine Datei geöffnet, so ist der Reader, ohne die Operationen zum Seiten Drehen, Teil jedes Editormoduls. Alle input fields im Editor sind mit dem gültigen Wertebereich für Benutzereingaben als Information Min: Max: versehen. Der Editor samt dargestellter PDF-Datei besteht aus einem grauen, waagerechten Operations Bar, einem linken Layers Seitenmenü in Rosa und einem rechten grünen Tools Seitenmenü. Mit dem ganz linken, grünen Button Layers im Operations Bar kann das Layers Seitenmenü aus- und eingeblendet werden. Daneben zeigt oder verbirgt der Button Tools das Tools Seitenmenü. Standardmäßig sind Layers und Tools ausgeklappt. Beim Öffnen einer PDF-Datei, wird eine Infobox, die über den Fortschritt der gerenderten PDF-Seiten informiert, angezeigt. Ebenso wird eine Infobox angezeigt, die Auskunft darüber gibt, dass gerade der Speicherprozess in Gange ist. Hat man Text, Drawings, Shapes oder Images hinzugefügt und klickt auf Save, so wird das PDF zuerst auf 500 \% gezoomt, damit die Elemente als Bilder in hochwertiger Qualität eingebettet werden können. Am Ende des Speichervorgangs wird wieder auf den aktuellen Zoomwert des Benutzers zurück gezoomt. Diese beiden Infoboxen, die in Screenshot \ref{fig:render-info} und \ref{fig:save-info} abgebildet sind, sind auch im Reader vorhanden. Die einzelnen Speicherschritte werden in der Webkonsole der Web Developer Tools des Browsers ausgegeben, was in Bild \ref{fig:save-progress-steps} zu sehen ist.

\begin{figure}[!htbp]
	\centering
	\includegraphics[width=1\textwidth]{"images/render-info.png"}
	\caption{Infobox über den Renderfortschritt der PDF Web App}
	\label{fig:render-info}
\end{figure}

\begin{figure}[!htbp]
	\centering
	\includegraphics[width=1\textwidth]{"images/save-info.png"}
	\caption{Infobox über den Speicherprozess der PDF Web App}
	\label{fig:save-info}
\end{figure}

\begin{figure}[!htbp]
	\centering
	\includegraphics[width=0.4\textwidth]{"images/save-progress-steps.png"}
	\caption{Speicherschritte der PDF Web App in der Webkonsole beim Speichern von Elementen}
	\label{fig:save-progress-steps}
\end{figure}

\subsubsection{Textbearbeitung}
Hat man den Writer aufgerufen, so präsentiert sich einem der Texteditor in den folgenden Abbildungen \ref{fig:texteditor} und \ref{fig:texteditor2}.

\begin{figure}[!htbp]
	\centering
	\includegraphics[width=1\textwidth]{"images/texteditor.png"}
	\caption{Startseite des Writers der PDF Web App}
	\label{fig:texteditor}
\end{figure}

\begin{figure}[!htbp]
	\centering
	\includegraphics[width=1\textwidth]{"images/texteditor2.png"}
	\caption{Mehr Tools der Startseite des Writers der PDF Web App}
	\label{fig:texteditor2}
\end{figure}

Mit dem Button Text im Operations Bar und nachfolgendem Klick auf das geöffnete Dokument, wird der Platzhaltertext dummy hinzugefügt. Unter dem Text erscheint eine dunkelrote controlBox, auf die man alle Operationen im Operations Bar und in Tools im Box Mode anwenden kann. Ich werde zunächst alle Operationen im Box Mode beschreiben und später auf den Layer Mode eingehen. Der Box Mode ist standardmäßig eingestellt. Operationen können durch die grünen Buttons zum Erstellen neuer Elemente, Delete und Move im Operations Bar und allen weißen Buttons in Tools getriggert werden. Hat man eine Operation getriggert, so befindet man sich im Modus dieser Operation im Box Mode. Darauffolgend können alle, im Dokument verfügbaren controBoxes angeklickt werden, um die Operation auf das betreffende Element auszuführen. Alle Operationen in Tools beziehen sich jeweils auf das Element des aktuellen Editormoduls und sind nur auf diesem anwendbar. Versucht man eine Operation eines Editormoduls auf ein Element eines anderes Editormoduls anzuwenden, wird die Operation nicht ausgeführt. In der Praxis des Box Modes kann man mehrere Texte, ohne erneut den grünen Text Button drücken zu müssen, dem PDF-Dokument hinzufügen. Für jedes neu hinzugefügte Editorelement wird eine Layer, mit einem elementspezifischen Standardnamen erstellt, die in Layers erscheint. Die obere linke Ecke der quadratischen controlBox von sämtlichen Editorelementen wird an der Stelle platziert, an der man mit der Maus auf die Seite geklickt hat. Alle controlBoxes werden nicht im  Output-PDF gespeichert, denn sie dienen lediglich der Steuerung von Editorelementen, um Operationen im Box Mode anwenden zu können. Mit dem Delete-Button und nachfolgendem Klick in eine oder mehrere controlBoxes im Box Mode, können Texte wieder gelöscht werden. Move verschiebt einzelne Texte durch eine mit der Maus gedrückten und zur Zielposition bewegten controlBox. Sobald die Maus losgelassen wird, nachdem die controlBox verschoben wurde, springt der Text an die Zielposition auf der Seite. Dabei ist wichtig zu beachten, dass man die controlBox langsam verschiebt, denn der Mauszeiger muss innerhalb der controlBox bleiben. Delete und Move kommen in allen Editormodulen vor und funktionieren immer gleich. Ganz oben in Tools werden dem Betrachter die x- und y-Koordinaten des Mauscursors auf der PDF-Seite angezeigt, während die Maus über eine Seite bewegt wird. Diese Mauscursorkoordinaten sind in jedem Editormodul präsent. Textelemente lassen sich in der textarea editieren. Zeilenumbrüche werden berücksichtig. Nachdem der dummy Text in der textarea überschrieben wurde, ein Klick auf den weißen Text-Button erfolgte und die Operation auf eine controlBox angewendet wurde, substituiert sich der Platzhaltertext mit dem aktuellen Text in der textarea. Die textarea kann in vertikaler Höhe expandiert werden. In allen Editormodulen von Tools werden alle Operationen exakt gleich ausgeführt: Man tätigt seine Einstellung, drückt mit der linken Maustaste auf den weißen Button für die jeweilige Operation und klickt daraufhin auf eine oder mehrere controlBoxes von Elementen. Unterhalb der Texteditierungsoperation kann der Zeilenabstand einstellt werden. Entweder verwendet man das selection menu mit voreingestellten Werten oder man gibt einen gewünschten Wert manuell in das input field ein. Neu hinzugefügte Elemente sind standardmäßig vorkonfiguriert. Alle Eingabefelder sämtlicher Editormodule zeigen dessen Standardwerte an. Bei jeder selection menu und input field Kombination ist maßgeblich, welche Schaltfläche zuletzt betätigt wurde. Einen benutzerdefinierten Font als .ttf oder .otf Datei kann man durch den dunkelgrauen Choose file-Button vom Dateisystem auswählen und wird in einer Liste abgebildet. Der zuletzt geöffnete Font wird ausgewählt. Die Selektion des aktuellen Fonts wird durch einen Klick auf den radio button des Fontdateinamens aktiviert. Mittels Clear werden alle Fonts aus der Liste entfernt. Fontdateinamen werden nach 15 Zeichen in die nächste Zeile umgebrochen. Abbildung \ref{fig:custom-font} zeigt 2 geöffnete .ttf Schriftdateien in der Liste.

\begin{figure}[!htbp]
	\centering
	\includegraphics[width=0.2\textwidth]{"images/custom-font.png"}
	\caption{Benutzerdefinierte Fontliste im Texteditor der PDF Web App}
	\label{fig:custom-font}
\end{figure}

Die Fontgröße kann mittels selection menu und input field justiert werden. Bezüglich der Fontfarbe lässt sich ein Color Picker Menü mit Klick auf das initial schwarze Quadrat ausklappen. Es können Farbe und Transparenz eingestellt werden. Die Farbwerte stehen in den Formten RGBA, HSLA oder HEX zur Verfügung. Mit Klick auf die beiden kleinen senkrechten Pfeile im Color Picker wird das Format gewechselt. Das Fenster des Color Pickers für die Fontfarbe ist in Abbildung \ref{fig:fontcolor} dargestellt. 

\begin{figure}[!htbp]
	\centering
	\includegraphics[width=0.5\textwidth]{"images/fontcolor.png"}
	\caption{Color picker für die Fontfarbe des Texteditors der PDF Web App}
	\label{fig:fontcolor}
\end{figure}

Als vorletzte Option kann der Text absolut gedreht werden. Durch den weißen Button Rotation und der entsprechenden Benutzerinteraktion durch selection Menu oder input field wird das Textelement rotiert. Absolute Rotation bedeutet, dass es eine feste Rotationsskala gibt, anhand der das Element rotiert wird. Bei einem Rotationswinkel von 0 Grad ein wird die Ausgangsrotation angewendet. Alle Editormodule arbeiten mit absoluter Rotation. Abschließend können alle Textelemente im Dokument mit dem Remove Button vollständig gelöscht werden. Ein neuer und modifizierter Text wird in Abbildung \ref{fig:text} demonstriert.

\begin{figure}[!htbp]
	\centering
	\includegraphics[width=1\textwidth]{"images/text.png"}
	\caption{Bearbeiteter Text im Writer der PDF Web App}
	\label{fig:text}
\end{figure}

\subsubsection{Drawings erstellen}
Der Drawer ist in Screenshot \ref{fig:drawer} abgebildet. 

\begin{figure}[!htbp]
	\centering
	\includegraphics[width=1\textwidth]{"images/drawer.png"}
	\caption{Drawer der PDF Web App}
	\label{fig:drawer}
\end{figure}

Die Zeichenoperation wird nach Klick auf Pencil initiiert. Mit gedrückter Maustaste wird ein Drawing erstellt. Zusätzlich wird an der Startstelle eine magenta farbige controlBox gesetzt. Zeichnen lässt sich auch mit einem Graphic Tablet. Auf einer ausgewählten Layer wird gezeichnet. Ist keine Layer ausgewählt, so wird auf der zuletzt gezeichneten Layer gearbeitet. Der New Layer-Button weist einem Drawing eine neue Layer zu. Wurde auf einer Seite bisher noch nichts gezeichnet, so wird beim ersten Drawing auf dieser Seite eine neue Layer automatisch angelegt und man muss dafür nicht New Layer drücken. Drawings sind die einzigen Elemente, bei denen der Nutzer selbst die Layers einer Seite zuweisen kann mit New Layer. Bei allen anderen Elementen, sei es Text, Shapes oder Images, wird jedes neue Element automatisch einer Layer zugeteilt. Der Radierer ist mit dem Eraser-Button im Operations Bar aufrufbar. Zuerst drückt man Eraser und geht dann mit gedrückter Maustaste über die Drawings auf einer Seite, die man entfernen möchte. Auch hier gilt, das auf der ausgewählten Layer radiert wird. Dort, wo bei gedrückter Maustaste die Maus die Linie berührt, wird die Linie wegradiert. Zeichnen und Radieren bekommen jeweils ein neues Mauscursorsymbol: Beim Zeichnen hat man ein schwarzes dünnes Kreuz und beim Radieren ein weißes dickes Kreuz. In Tools kann man ein Drawing samt radierten Partien relativ skalieren, indem man einen Faktor eingibt. Der Faktor kann auch ein Float sein und multipliziert sich immer mit der aktuellen Größe, d.h. die aktuelle Größe wird als 100 \% berechnet. Darunter kann man mit dem color picker menu die Farbe und Transparenz der Stiftfarbe definieren. Sie wird mit einem Klick auf Pencil Color auf die nächste Zeichenoperation angewendet. Außerdem definiert sie auch gleichzeitig die Radiererfarbe, was sich nur bei Transparenzen unter einem Wert von 1 bemerkbar macht. Ansonsten ist jede Radiererfarbe gleich, jedoch bei einer Transparenz von unter 1 radiert der Radierer weniger deckend. Des Weiteren kann man die Größe des Stiftes bzw. Radierers einstellen. Sie greift auch ab der nächsten Zeichen- bzw. Radieroperation. Ebenfalls kann man Drawings, samt radierten Partien, absolut rotieren. Wurde ein Drawing gedreht, so wird eine weitere Layer von der Anwendung neu angelegt, falls man versucht, auf ein rotiertes Drawing zu malen. Zum Schluss kann man mit Remove alle Drawings im gesamten Dokument löschen. Teilweise transparente Drawings werden im Bild \ref{fig:drawing} dargestellt. 

\begin{figure}[!htbp]
	\centering
	\includegraphics[width=1\textwidth]{"images/drawing.png"}
	\caption{Drawings im Drawer der PDF Web App}
	\label{fig:drawing}
\end{figure}


\subsubsection{Shapes hinzufügen}
Die Startseite des Shapers ist in den Screenshots \ref{fig:shaper} und \ref{fig:shaper2} abgebildet. 

\begin{figure}[!htbp]
	\centering
	\includegraphics[width=1\textwidth]{"images/shaper.png"}
	\caption{Shaper der PDF Web App}
	\label{fig:shaper}
\end{figure}

\begin{figure}[!htbp]
	\centering
	\includegraphics[width=1\textwidth]{"images/shaper2.png"}
	\caption{Mehr Tools des Shapers der PDF Web App}
	\label{fig:shaper2}
\end{figure}

Der Shapetyp kann durch die Buttons Rectangle für Rechteck, Triangle für Dreieck oder Ellipse und einem oder mehreren Klicks auf eine PDF-Seite bestimmt werden. Jedem Shape wird eine orangene controlBox mehr oder weniger mittig hinzugefügt. In Tools des Shapers gibt es eine einzige Operation, die nur auf Dreiecke angewendet werden kann. Es handelt sich um die oberste Einstellung für die Breite und Höhe des dritten Punktes des Dreiecks. Hiermit kann der rechte Punkt der langen Spitze des default Dreiecks bearbeitet werden. Alle anderen Einstellmöglichkeiten können auf allen Shapeelemente Rechteck, Dreieck und Ellipse arbeiten. Man hat 2 Möglichkeiten einen Shape zu skalieren. Zum einen kann man die Breite und Höhe unabhängig voneinander einstellen, was eine absolute Skalierung bedeutet, oder man verwendet den Skalierungsfaktor, der relativ und proportional vergrößert. Für die Umrandungslinien des Shapes kann man auf der einen Seite die Farbe inklusive Deckkraft und auf der anderen Seite die Breite der Linie justieren. Die Strichfarbe muss mit der checkbox Use Stroke in Grün eingeschaltet sein, was sie beim ersten Öffnen des Editors auch ist. Deaktiviert man die Use Stroke checkbox schaltet sich automatisch die Use Fill checkbox an und umgekehrt. Man kann auch beide checkboxes einschalten, aber nicht beide zusammen ausschalten. Ist Use Stroke Rosa, d.h. deaktiviert, und man wendet die Strichbreite an, dann hat Stroke Width keinen Effekt. Use Fill muss Grün sein, um die Füllfarbe anzuwenden. Bei Strich- und Füllfarbe wird ein wie in dem Writer und Drawer der gleiche color picker verwendet. Alle Shapes können mit absoluter Rotation rotiert werden. Die controlBoxes werden bei den Shapes mitgedreht. Zuunterst entfernt der Remove-Button alle Geometrieelemente im geöffneten PDF. Der Screenshot \ref{fig:shaping} hebt mehrere bearbeitete Shapes hervor.

\begin{figure}[!htbp]
	\centering
	\includegraphics[width=1\textwidth]{"images/shaping.png"}
	\caption{Shapes im Shaper der PDF Web App}
	\label{fig:shaping}
\end{figure}


\subsubsection{Images einfügen}
Der Imager ist im Bild \ref{fig:images} dargestellt. 

\begin{figure}[!htbp]
	\centering
	\includegraphics[width=1\textwidth]{"images/images.png"}
	\caption{Imager der PDF Web App}
	\label{fig:images}
\end{figure}

Um ein Image mit dem Image-Button im Operations Bar hinzuzufügen, muss man erst ein Image mittels des schwarzen Choose file-Buttons vom Dateisystem ausgewählt haben. Dann erscheint der Imagename, wie im Writer bei einem benutzerdefinierten Font, in der Liste unter Choose file. Man kann mehrere Images nacheinander im Dateidialog auswählen. Sie werden alle in der Liste untereinander angezeigt. Das zuletzt ausgewählte Image wird zuunterst in der Liste angefügt und ausgewählt. Zusätzlich werden die Originaldimensionen des Images angezeigt, sobald das Image auf der Seite platziert wurde. Das Image muss mit dem runden radio button in Blau ausgewählt sein, um es mit Image und einem Klick auf eine Dokumentenseite auf dem PDF zu platzieren. Weiße radio buttons stellen nicht ausgewählte Images dar. Mittels des dunkelgrauen Clear-Buttons wird die Imageliste gelöscht. Die Image-Selektionssektion funktioniert analog zum benutzerdefinierten Fontselektionsbereich im Writer. Bei der Imageplatzierung wird eine hellblaue controlBox dem Image hinzugefügt. Ein Image kann in Breite und Höhe unterschiedlich absolut skaliert werden und proportional mit einem Skalierungsfaktor relativ verkleinert bzw. vergrößert werden. Unter den Skalierungsoptionen kann man die Deckkraft eines Images bestimmen. Ebenfalls kann man ein Image absolut drehen. Zuletzt können alle hinzugefügten Images im PDF mit dem Remove-Button entfernt werden. Die Abbildung \ref{fig:imaging} zeigt den Imager in Aktion.

\begin{figure}[!htbp]
	\centering
	\includegraphics[width=1\textwidth]{"images/imaging.png"}
	\caption{Platzierte Images im Imager der PDF Web App}
	\label{fig:imaging}
\end{figure}



\subsubsection{Ebenensteuerung}
Das in Abbildung \ref{fig:ebenenmenu} gezeigte Layers Menu lässt sich mit einem Klick auf Menu in Layers hervorholen oder verbergen. Es ist möglich durch die Liste an hinzugefügten Layers zu scrollen.

\begin{figure}[!htbp]
	\centering
	\includegraphics[width=1\textwidth]{"images/ebenenmenu.png"}
	\caption{Ausgeklapptes Layers Menu im Editor der PDF Web App}
	\label{fig:ebenenmenu}
\end{figure}

Standardmäßig sind die Schaltflächen eingeklappt. Fügt man ein Element, unabhängig vom Typ, einer PDF-Seite hinzu, wird für dieses Element eine Layer angelegt, die dann automatisch ausgewählt ist. Layers werden nach aufsteigenden Seitenzahlen gruppiert. Eine ausgewählte Layer ist rosa und eine abgewählte schwarz. Es können mehrere Layers ausgewählt werden. Wenn eine Layer angelegt wird, bekommt sie einen Standardnamen gesetzt, der durch den Elementtyp und einem nummerischen Index, beginnend mit 1, gekennzeichnet ist. Der Standardname kann durch Tastatureingabe im grauen input field auf der Layer überschrieben werden. Layers werden nach Seitenzahlen in einer schwarzen Box, die oberhalb mit der Seitenzahl gekennzeichnet ist, gruppiert. Im Layers Menu kann man zwischen Box Mode und Layer Mode wechseln. Ist der Modus Button grün, so ist der betreffende Modus eingeschaltet. Hingegen ist ein ausgeschalteter Modus mit einem weißen Button versehen. Man kann sich entweder im Box Mode, oder im Layer Mode befinden, aber nicht in beiden Modi gleichzeitig. Mit dem dunkelgrauen Copy-Button kann der Benutzer ausgewählte Layers kopieren. Folglich werden die beinhaltenden Elemente dubliziert. Wird eine Layer kopiert, so wird an den Layernamen copy und eine Nummerierung angehängt. Mit Lock und Unlock kann man Layers sperren bzw. entsperren. Eine locked Layer kann nicht verändert werden, d.h. keine Operationen können auf ihr beinhaltendes Element angewendet werden. Eine locked, unausgewählte Layer ist weiß und eine locked, ausgewählte ist rosa mit weißer Umrandung. 

\begin{figure}[!htbp]
	\centering
	\includegraphics[width=0.3\textwidth]{"images/ebenen.png"}
	\caption{Teilweise locked Layers im Editor der PDF Web App}
	\label{fig:ebenen}
\end{figure}

Die dunkelgrauen Select All und Deselect All Buttons stellen Auswahlfilter dar. Bewegt man die Maus auf die Buttons klappt sich ein Selection Filter Menu auf, was im Bildausschnitt \ref{fig:filtermenu} gezeigt wird. 

\begin{figure}[!htbp]
	\centering
	\includegraphics[width=0.6\textwidth]{"images/filtermenu.png"}
	\caption{Selection Filter Menu im Editor der PDF Web App}
	\label{fig:filtermenu}
\end{figure}

Bei Select All kann man mehrere Layers nach Seiten auswählen, nach Elementtyp und, ob sie locked oder unlocked sind, d.h. die Layers werden rosa markiert. Ist ein Selection Filter aktiviert, werden die Buttons im Selection Filter Menu grün. Bei weißen Buttons oder einer leeren Liste an Seiten ist kein Selection Filter aktiviert. Bei der Seitenliste muss man die Seiten durch Komma trennen und sie müssen nicht in aufsteigender Reihenfolge angegeben werden. Die Filter werden mit einem Klick auf Select All angewendet. Ist kein Filter aktiviert, was der Standardzustand ist, so werden alle Layers mit Klick auf Select All ausgewählt. Deselect All hat die gleiche Funktionalität, nur dass die Deselection Filter zum Auswahl aufheben angewendet werden, d.h. Layers werden auf Schwarz gesetzt. Ein Klick auf Deselect All ohne Filter wählt alle Layers im Dokument ab. Im Bildausschnitt \ref{fig:filtering.} ist ein Beispiel von Deselect All Filtern abgebildet. Laut des Beispiels würde die Auswahl auf allen unlocked Textelementen auf Seite 1 und 2 aufgehoben werden. 

\begin{figure}[!htbp]
	\centering
	\includegraphics[width=0.6\textwidth]{"images/filtering.png"}
	\caption{Deselection Filter Menu mit aktivierten Filtern im Editor der PDF Web App}
	\label{fig:filtering}
\end{figure}

Sowohl bei Select All als auch bei Deselect All habe ich eine Benutzereingabenkontrolle beim der Seitenlistenfilter implementiert. Falls der Benutzer ungültige Eingaben gemacht hat, z.B. Seiten als Float oder Strings, so wird die Eingabe bei Auslösung des Filters gelöscht. Leerzeichen können zwischen gültigen Seitenzahlen verwendet werden. Folglich wird Select All bzw. Deselect All so angewendet, als ob kein Filter für die Seiten eingestellt wurde, d.h. aktivierte Filter für Elementtypen oder locked bzw. unlocked Layers greifen dennoch. Die Reihenfolge der Elemente in der z-Achse kann über die Layer gesteuert werden. Man kann ein einzelnes Element mit gedrückter Maustaste auf eine andere Layer verschieben, um die Position zu ändern, wie Elemente übereinander liegen. Dabei ändert sich das Maussymbol. Man muss zwecks Verschiebung in der z-Achse auf den Layernamen oder auf die rosa Fläche initial die Maus drücken und dann ziehen. Layers sind pro Seite gruppiert. Dabei kann man sogar ein Element in eine andere Seitengruppe ziehen, sodass das Element auf der entsprechenden Seite erscheint. Eine Seitengruppe entsteht erst, wenn man das erste Element auf einer Seite platziert. Links neben jedem Layernamen ist eine grüne checkbox abgebildet. Wenn man sie abwählt, färbt sie sich rosa und das Element wird samt controlBox unsichtbar. Dann können ebenfalls keine Operationen angewendet werden. Ein erneuter Klick auf die Layer checkbox schaltet sie wieder in Grün ein und das Element wird auf der Seite erneut sichtbar. 

\subsubsection{Arbeiten im Layer Mode}
Ich habe bisher alle Operationen im Box Mode beschrieben. Es gibt außerdem den Layer Mode, den man im Layers Menu aktivieren kann. Im Layer Mode kann der Benutzer ein oder mehrere Layers auf allen PDF-Seiten selektieren und dann Move, Delete und alle Operationen in Tools auf sie anwenden. Um dies umzusetzen, muss man bei Tools seine gewünschte Einstellung machen und auf den jeweiligen weißen Button klicken. Sofort werden die Einstellungen auf die selektierten Layers mit einem einzigen Klick angewendet. Es ist nicht mehr nötig in irgendeine controlBox zu klicken. Dabei sind die Einstellungen in Tools elementspezifisch. Wird beispielsweise eine Tools-Einstellung des Writers auf eine Shape Layers angewendet, wird sie ignoriert. Ist dabei auch noch eine Text Layer ausgewählt, so wird die Operation nur auf die Text Layer angewendet. Delete und Move können in jedem Editormodul im Layer Mode auf alle Elementtypen angewendet werden. Vor allem bei Move muss man in einer controlBox, dessen Layer ausgewählt ist, die Maus drücken und auf eine gewünschte Stelle auf der PDF-Seite ziehen. Alle anderen Layers, die zusätzlich ausgewählt wurden, werden mit gleichem proportionalem Abstand zueinander um die gewünschten Verschiebungskoordinaten verschoben. Wenn die controlBox losgelassen wird, springen die Elemente an die jeweiligen Stellen. Ob man lieber im Box Mode oder Layer Mode arbeitet, ist Geschmackssache. Jeder Modus bringt seine Vor- und Nachteile mit sich, die ich im Kapitel Diskussion und Kritik aufzeigen werde.