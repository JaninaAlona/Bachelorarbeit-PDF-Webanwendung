\chapter*{Bachelorarbeit}
\label{chap:abstract}
%
\textbf{Titel:} Entwicklung einer webbasierter Applikation zur Bearbeitung von PDF\\ Dateien

\textbf{Gutachter:}
\par
- Prof. Dr. Chunrong Yuan
\par
- Prof. Dr. René Wörzberger

\textbf{Zusammenfassung:} Für die Bachelorarbeit habe ich eine Open Source offline Webseite zur Bearbeitung von PDF Dateien im Firefox Browser programmiert. Seit Adobe den PDF Standard entwickelt hat, tauchten zahlreiche meist kostenpflichtige PDF Anwendungen, um PDF Dateien zu bearbeiten auf dem Markt auf. Ich habe den Markt an PDF Programmen analysiert und diese mit meiner Webapplikation verglichen. Daraufhin beleuchte ich den aktuellen Stand der Technik des PDF Standards. Im späteren Verlauf erkläre ich die Implementierung meiner Webapp und meine Erfahrungen mit anderen Browsern, sowie auf MacOS, Linux, Android und iOS. Die Javascript Libraries PDF.js und PDF-LIB sind das tragende Fundament meiner PDF Webapp. Die PDF Webapp vereint alle Funktionalitäten, die man für gängige PDF Bearbeitung benötigt. Man kann PDFs lesen, splitten, mergen, erstellen, sowie mit Texten, Bildern, Geometrie und Zeichnungen versehen. Am Ende diskutiere ich, was man hätte besser machen können, welche Funktionalitäten fehlen und welche Features in Zukunft noch geplant sind.

\textbf{Stichtwörter:} PDF Bearbeitung, Adobe, Javascript, Vue JS 3, auf PDF zeichnen, Splitten, Mergen, PDF.js, PDF-LIB

\textbf{Datum:} 04. März 2024