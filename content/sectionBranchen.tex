\section{Relevanz von PDF in verschiedenen Marktbranchen}
Die PDF-Datei ersetzt als elektronisches Informationsaustauschformat in privaten Organisationen, Behörden und Bildungswesen papierbasierte Arbeitsprozesse. Vor allem in Behörden wird der E-Mail-Verkehr noch frequent verwendet. Starke Kompressionsalgorithmen ermöglichen es, speicherintensive PDF-Dateien auf leichtem Wege per E-Mail zu verschicken. Die PDF-Spezifikation ist offen und sehr detailliert über sämtliche Aspekte des Dateiformats. Dadurch können Softwareunternehmen eigene Programme schreiben, die PDF-Dokumente erzeugen, lesen und bearbeiten können. Searchable PDFs werden in Verträgen, Rechnungen und Geschäftsunterlagen verwendet, damit Mitarbeiter Informationen gezielter suchen und Daten abteilungsübergreifend effizienter verwaltet werden können. In Forschungsarbeiten und wissenschaftlichen Artikeln werden searchable PDFs hauptsächlich bei der Überprüfung von Quellen oder dem Extrahieren von Zitaten verwendet. Behörden, Bibliotheken und Unternehmen digitalisieren Dokumente zur Archivierung und wandeln sie in ein searchable PDF um, um den langzeitigen Bestand der Dokumente zu sichern \cite{adobe-search}.
\par
Das PDF/A-Dateiformat wird in Bibliotheken und Archiven zur digitalen Archivierung von Büchern, Zeitschriften und historischen Dokumenten verwendet. Auch in Behörden und im Verwaltungssektor hat PDF/A für die Aufbewahrung von Verwaltungsakten und rechtlichen Dokumenten seine Existenzberechtigung. Im Gesundheitswegen wird es außerdem zur Speicherung von Patientenakten und medizinischen Unterlagen verwendet. Hingegen im Finanzwesen werden mit diesem Format Geschäftsunterlagen und Finanzdokumente verwahrt. Unternehmen und Organisationen können mit PDF/A gesetzliche und Compliance-Vorschriften einhalten \cite{adobe-pdf-a}. PDF/VT-Dateien werden im Direktmarketing eingesetzt. Personalisierte Werbematerialien erhöhen die Wahrscheinlichkeit einer positiven Reaktion auf die Werbebotschaft bei den Kunden und verbessert die Bindung von Unternehmen und Kunden. Der Transaktionsdruck findet bei Finanzdienstleistungen, Versicherungen und E-Commerce besonderen Anklang. Beliebte Transaktionsdokumente sind Rechnungen, Kontoauszüge, Versicherungspolicen oder Bestellbestätigungen \cite{adobe-pdf-vt}. PDF-Dokumente mit der PDF/UA-Kennzeichnung stärken den Ruf und die Reputation eines Unternehmens oder einer Organisation durch Engagement für Inklusion \cite{adobe-pdf-ua}. Digitale Signaturen werden bei digitalen Freigabe-, Abnahme-, Genehmigungs- und Vertragsprozessen verwendet. \gls{pades} wird in Rechtssystemen, im Finanzwesen und im Regierungssektor eingesetzt \cite{adobe-pdf-pades}. Im nächsten Kapitelabschnitt beleuchte ich die Verwendung von PDF-Dateien in der Druck- und Designindustrie genauer und beschreibe die Rolle von PDF im Agenturprozess.  