\section{Rolle von PDF in der Druckvorstufe und Designbranche}
Vor allem in der grafischen Industrie wird PDF gerne verwendet, weil es eine plattformübergreifende Visualisierung bietet auf allen Betriebssystemen. Schriften können bei Einbettung exakt wiedergegeben werden, unabhängig ob es sich um eine Windows oder MacOS Schrift handelt. Im Vergleich zu PostScript-Dateien erzielen die kompaktere Codierung von Seiteninhalten, dem einmaligen Speichern von identischen Bildern und die Verwendung von Kompressionsalgorithmen eine maßgeblich kleinere Dateigröße bei PDF. Korrekturänderungen in PDF-Dateien sind kurz vor dem Druck noch möglich und PDF entwickelte sich zunehmend zum Containerformat für alle grafischen Elemente. Die Produktion von Druckerzeugnissen wird somit wesentlich flexibler und sicherer. Downsampling und Kompression beschleunigt den Transport von der Agentur zum Dienstleistungsbüro enorm. In der Ausgabe ist die effektive Auflösung maßgeblich. Effektive Auflösung ist die Bildauflösung, die aus der Anzahl der Bildpunkte und der Fläche resultieren, auf der das Bild platziert wurde. Downsampling beeinflusst diese effektive Auflösung. Starke Artefakte fallen im Offsetdruck weniger auf als im Digitaldruck. \\ \cite{schneeberger}
Für die Betrachtung von Druckvorstufen-PDF-Dateien sollte immer Acrobat Pro bzw. Adobe Reader verwendet werden, da viele Drittanbieter-PDF-Viewer druckvorstufenrelevante Informationen nicht fehlerlos darstellen können. \cite{schneeberger}

In der Werbebranche werden PDF-Dateien vor allem für Korrekturabzüge verwendet. Ein Korrekturabzug ist eine Skizze bzw. Designvorlage des Werbeprodukts in vom Kunden gewünschten Position und Größe des Druckmotivs auf dem Werbeartikel. Als digitales Layout wird der Korrekturabzug mit dem Kunden abgestimmt und seine Änderungswünsche vor dem Druck entgegengenommen. \cite{korrektur}

Dank der Profile für unterschiedliche Geräte und Bedruckstoffe, kann eine Simulation des Ergebnisses am Monitor oder durch einen Prüfdruck bewerkstelligt werden. Dadurch steigt die Reproduktionssicherheit bei gleichbleibender Qualität und die Produktionszeiten verkürzen sich enorm. In den meisten Druckprojekten steht zu Beginn noch nicht fest, wann, wo und auf welchem Bedruckstoff gedruckt werden soll. Die zeitintensive Optimierung der Druckdaten kann durch PDF auf einen späteren Zeitpunkt verlegt werden.






\subsection{Preflight}

\subsection{Fontformate}
Da allgemeine Schriftinformationen immer eingebettet sind und die Zeilenlängen im Prinzip immer stimmen, können Druckvorstufenbetriebe zumindest immer erkennen, welche Schrift bzw. Schriftschnitt der Ersteller der PDF-Datei ursprünglich vorgesehen hatte, falls die Schrift nicht eingebettet wurde. \cite{schneeberger}


Composite-Fonts sind Basisschriften mit hierarchischem System. Die oberste hierarchische Ebene stellt den root font dar alle folgenden Fonts sind descendant fonts. Sie ermöglichen die Einführung von Type-1-Schriften im asiatischen Markt. \cite{schneeberger}

