\section{\gls{pdf} Implementierung}
\gls{pdf} ist eine vektorbasierte \gls{pdl} (Seitenbeschreibungssprache) und basiert auf dem PostScript-Format. Eine \gls{pdl} beschreibt den Seitenaufbau, wie die Seite in einem Ausgabeprogramm bzw. Ausgabegerät, z.B. einem Drucker, aussehen soll. \gls{pdl}s können Seiten mit Vektoren beschreiben. Das Ausgabeformat ist normalerweise nicht zur weiteren Bearbeitung vorgesehen. An den Drucker wird durch die \gls{pdl} ein Datenstrom der zu druckenden Aufgabe erzeugt und an den Drucker gesendet. Dabei müssen Drucker die \gls{pdl} nicht selbst verarbeiten. Im Common Unix Printing System, der Standard-Druckersteuerung von Linux hat der PostScript und der \gls{pdf}-Interpreter ghostscript die Aufgabe eines \gls{rip}, d.h. er ist für die Umwandlung in die gerasterte Druckausgabe auf dem Drucker zuständig. Viele APIs der Hardwareabstraktionsschicht im Computer wie \gls{gdi} oder OpenGL können in \gls{pdl} ausgeben.
