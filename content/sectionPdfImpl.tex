\section{PDF Implementierung}
PDF ist eine vektorbasierte \gls{pdl} (Seitenbeschreibungssprache) und basiert auf dem PostScript-Format. Der \gls{mime}-Type von PDF heißt application/pdf. Eine \gls{pdl} beschreibt den Seitenaufbau, wie die Seite in einem Ausgabeprogramm bzw. Ausgabegerät, z.B. einem Drucker, aussehen soll. \gls{pdl}s können Seiten mit Vektoren beschreiben. Vektorielle Seitenbeschreibung bedeutet, dass das Format beliebig skalierbar ist ohne Qualitätseinbußen, jedoch eingebettete Pixelgrafiken erhalten durchaus mittels genügend Skalierung Qualitätsverluste. Deren Ausgabeformat ist normalerweise nicht zur weiteren Bearbeitung vorgesehen. An den Drucker wird durch die \gls{pdl} ein Datenstrom der zu druckenden Aufgabe erzeugt und an den Drucker gesendet. Der \gls{rip} eines Druckers wandelt die Bildschirmausgabe in die gerasterte Druckerausgabe um. Viele APIs der Hardwareabstraktionsschicht im Computer wie \gls{gdi} oder OpenGL können in \gls{pdl} ausgeben. Speichert ein Satzprogramm den Seitenbeschreibungscode eines Dokuments in einer Datei, müssen Drucker die \gls{pdl} nicht selbst verarbeiten. Eine \gls{ppd} Druckerbeschreibungsdatei definiert Fonts, Papiergröße, Auflösung und andere Standardeigenschaften für einen bestimmten PostScript-Drucker. \cite{ppd-file} Im Common Unix Printing System, der Standard-Druckersteuerung von Linux hat der PostScript und der PDF-Interpreter ghostscript die Aufgabe eines \gls{rip}, d.h. er ist für die Umwandlung in die gerasterte Druckausgabe auf dem Drucker zuständig. Zudem stellen \gls{pdl}s eine Schnittstelle zum Quellcode eines Dokuments bzw. zu Programmen, die Quellcode verwalten oder das Dokument formatieren können, dar. Die \gls{pdl} PDF erweitert die Funktionalität von PostScript um anklickbare Links (Hypertextfunktionalität), die die Navigation im Dokument erleichtern oder um URLs, die sich automatisch im Browser öffnen. \cite{wiki-pdl} 

\subsection{PostScript}
Sowohl PostScript als auch PDF haben zum Ziel die Seiten eines Dokuments vollständig für die Ausgabe in der Druckvorstufe zu beschreiben. Die abwärtskompatible, stackorientierte, Turning-vollständige Hochsprache PostScript \gls{pdl} wurde in den 1980er Jahren von Adobe erfunden. \cite{adobe-postscript, wiki-postscript} Hinzu wurden weiter PostScript Technologien entwickelt, die aus der Programmiersprache PostScript, Grafik-, Textformatierungsanwendungen, Treibern und Abbildungssystemen bestehen. PostScript hat sich als Industriestandard etabliert. Die letzte Version ist PostScript 3 von 1997. Seine primäre Anwendung gemäß des Adobe Imaging Models findet sich in der Beschreibung von Textdarstellung, graphische Formen und Bildern auf gedruckten oder auf dem Bildschirm angezeigten Seiten. Dabei ist die Beschreibung des Dokuments geräteunabhängig und eine PostScript-Datei ist sequentiell organisiert. PostScript unterstützt unter anderem beliebige geometrische Formen, Zeichenoperationen in Graustufen, RGB, CMYK und CIE (Yxy-Farbraum) und  vorinstallierte oder benutzerdefinierte Fonts und Digitalbilder jeglicher Auflösung je nach Farbmodell und ein allgemeines Koordinatensystem.
In PostScript wird eine Seite, die ein Koordinatensystem umspannt, als Grafik betrachtet, die verschiedene Grafikelemente enthalten kann. Dabei werden die Textzeichen eines Fonts, gemäß des Adobe Imaging Models, als graphische Formen betrachtet auf denen Grafikoperationen möglich sind. Das Koordinatensystem unterstützt alle linearen Transformationen, die auf alle Seitenelemente angewandt werden können. Die Seitenbeschreibung in PostScript kann auf jedem Gerät, was einen PostScript Interpreter implementiert, gerendert werden. In diesem Prozess wird die high-level PostScript-Beschreibung in low-level Rasterdatenformate für das jeweilige Gerät übersetzt. PostScript Programme können erstellt, übertragen und als ASCII Quellcode interpretiert werden. \cite{adobe-postscript} Jede PostScript-Datei muss durch einen \gls{rip} interpretiert werden. Der PostScript-Interpreter rechnet die Benutzerkoordinaten in Gerätepixel um, wobei auch die technischen Eckdaten des jeweiligen Geräts mitberücksichtigt werden. Theoretisch kann derselbe PostScript-Code auf verschiedenen Endgeräten mit unterschiedlichen Auflösungen mehr oder weniger identische Ausgaben erreichen. Den Interpreter gab es früher als Hardware-\gls{rip}, der allerdings nicht mehr zum Einsatz kommt. Heute gibt es lediglich Software-\gls{rip}s, die von einem Betriebssystem kontrolliert werden und hardwareunabhängig arbeiten. Fast alle \gls{rip}-Hersteller orientieren sich am Standard und somit sind PostScript-Fehler in der Druckvorstufe minimiert worden. 
\cite{schneeberger}

\subsection{Adobe Imaging Model}
PDF und die PostScript Programmiersprache haben das Adobe Imaging Model als Gemeinsamkeit. Es kann nahtlos zwischen PDF und PostScript konvertiert werden und beide erzielen das gleiche Ausgabeergebnis beim Druck. Dennoch fehlt PDF das general-purpose Framework der PostScript Programmiersprache. Stattdessen stellt ein PDF Dokument eine statische Datenstruktur optimiert für den random-access dar und enthält zusätzlich Seitennavigationsinformationen für interaktives Lesen. Das high-level Imaging Model beschreibt die Elemente, die auf der Seite dargestellt werden, also Text, Geometrie oder Bilder, als abstrakte graphische Elemente aus Vektorobjekten und Bézierkurven, anstatt als Pixeldefinitionen. Dadurch wird das Imaging Model zu einem geräteunabhängigem Modell und kann hochwertige Ausgaben auf vielen verschiedenen Druckern und Bildschirmen liefern. Die \gls{pdl} beschreibt dieses Imaging Model. Eine Anwendung generiert zuerst die geräteunabhängige Beschreibung des gewünschten Ausgabegeräts in der \gls{pdl}. Daraufhin interpretiert eine Firmware oder Software eines spezifischen Ausgabegeräts für Rasterausgaben die Beschreibung und rendert sie im Ausgabegerät. Hierbei hat die \gls{pdl} die Rolle eines Austauschstandards für die Übertragung und Speicherung von druckbarem oder auf Displays darstellbaren Dokumenten. \cite{adobe-postscript} Später wurde das Imaging Model für die Unterstützung von Transparenzen erweitert. Diese Funktionalität wurde speziell für PDF implementiert und wird nicht von PostScript unterstützt. Bei PostScript überschreibt das zuletzt gezeichnete Objekt alle darunterliegenden Objekte im Hintergrund. 
Das Imaging Model beschreibt Pfad-Objekte aus verbundenen Punkten, Linien und Kurven. Text-Objekte bilden eine eigene Datenstruktur (Fonts), die als Glyphen aus Pfad-Objekten bestehen. Bild-Objekte sind aus einzelnen Pixelwerten in einer rechteckigen Fläche aufgebaut und enthalten eine eindeutige Position im Rechteck und einen Farbwert. Die Flexibilität des Adobe Imaging Models zeichnet sich durch seine flexible Ausgabefähigkeit auf jeglichen Rastergeräten und hochauflösenden Displays. Die Größe des Pixels wird durch die Ausgabeauflösung des Rastergeräts bestimmt, die bei Monitoren zwischen 75 und 110 \gls{ppi} und bei Tintentstrahl- bzw. Laserdruckern zwischen 300 und 1400 \gls{ppi} liegt.
\cite{schneeberger}

\subsection{Dateiformataufbau}
PDF ist ein reines objektbasiertes Dateiformat und PDF-Dateien enthalten Dokumentdaten in binärer Form. PDF-Dateien bestehen aus Sequenzen von 8-Bit-Binär bzw. 7-Bit-ASCII. \cite{schneeberger} 
Ein Dokument entspricht immer einer Datei. Das Einbetten von binären Dateien in beliebigen Formaten oder anderer PDF-Dateien ist möglich. Die Struktur besteht im Wesentlichen aus 4 Komponenten. Zunächst spezifiziert der Header die Version der PDF-Spezifikation (Signature) und den Charset Identifier. \cite{ccc-pdf-secrets} Der Body enthält die Daten der Objekte, aus denen das Dokument besteht und die Cross-Reference Table (Xref) deckt die Informationen über die Position der indirekten Objekte in der Datei ab. Die Xref-Sektion wird auch als Katalog bezeichnet und speichert genauer gesagt die Byte-Positionen der Objekte im Body. Ein Verweis auf ein Objekt im Katalog kann es für andere Seiten wiederverwenden. Zuletzt definiert der Trailer die Startposition der Cross-Reference Table als Pointer und von speziellen Objekten im Body. \cite{ccc-break-pdf} Außerdem enthält der Trailer einen Size Entry und die Markierung \%\%EOF für das Dateiende. \cite{ccc-pdf-secrets} Das erste Objekt in Xref hat eine ID von 0 und wird nicht verwendet. In der Xref besteht jede Objekteintragung aus der Byte Position, einer Generation Number und einem Zeichen, was angibt, ob das Objekt verwendet wird oder nicht. Hierbei steht f für not used und n für used. Danach kommt object und endobject, was die Objektdefinition umschließt. 
Die Objekte im Body sind in einer komplizierten hierarchischen Struktur, dem Dokument, verknüpft. Zur Dateigrößenoptimierung werden komplexe Verbindungen zwischen den Daten hergestellt und die Daten eines mehrfach vorkommenden Objektes müssen nur einmal gespeichert werden. 
\cite{softx}
Text in AcroForms wird als Stream gespeichert. In Streams kann alles gespeichert werden und sie werden nicht interpretiert. Objekte können Referenzen auf andere Objekte vor allem Seiten enthalten. Der Trailer enthält eine Referenz zum root Element im Body \cite{ccc-break-pdf} und Metadaten.
PDF-Dokumente können Dictionary Objekte enthalten, was ein Paar von Objekten darstellt, genannt Entries. Aktionen werden beispielsweise als Entries gespeichert. \cite{ccc-badpdf}
Die PDF-Dokumentstruktur ist auf einen schnellen, wahlfreien Zugriff (random-access) auf beliebigen Seiten optimiert. Im Unterschied dazu sind PostScript-Dateien seriell organisiert. \cite{schneeberger} PDF Viewer prozessieren PDF-Dateien im Prinzip vom Ende bis zum Anfang, d.h. vom Trailer zum Body. \cite{ccc-break-pdf}
Beim Parsen wird zunächst die Signature überprüft, dann wird die Position von \%\%EOF und startxref gesucht, was die Position der Xref angibt. Die Xref stellt die Offsets jedes Objektes zur Verfügung und der Trailer zeigt auf den /Root Entry des Root-Objekts. Nachfolgend werden alle Objekte geparst und überprüft, ob /Root den /Pages Entry, /Pages ein Seiten-Array, jede /Page eine Größe der /MediaBox hat, /Contents als Stream-Objekt vorliegt und /Resources das /Font dictionary definiert. Zuletzt wird die Seite gerendert durch BeginText, Auswahl des Fonts, Bewegung des Cursors, Anzeigen des Strings und EndText. \cite{ccc-pdf-secrets} Referenzen werden nicht in der parse time ausgewertet, sondern nur nach Verwendung. Strings können als <Length> <string> oder <string> <terminating symbol> definiert werden. \cite{ccc-wtf-pdf}
Metadaten wurden bis PDF 1.7 durch den \gls{xmp} Standard codiert und als \gls{xml} formatierte Daten in PDF-Dateien abgelegt.
\cite{softx} 
\gls{xml} ist eine Sprache zur Markierung von Inhalten mit Hilfe von Tags, um die Struktur zu beschreiben und Elemente zu identifizieren.
\cite{schneeberger}
Unicode wird in den Metadaten unterstützt. 
Seitenobjekte und die meisten PDF-Strukturen sind gerichtete azyklische Graphen. \cite{ccc-wtf-pdf}


\subsection{Implementierung von Fonts}
Die Beschreibung von Glypen ist bei eingebettete Schriften als Datenstrom im Eintrag FontFile registriert. Falls die Schrift nicht eingebettet wurde fehlt dieser Eintrag. Ein optionales Unicode-Mapping ToUnicode ist von Nöten, damit die Glyphen auch über Unicode verarbeitet werden kann. Ist dieses Mapping nicht vorhanden, so kann keine Textsuche und das Kopieren von Text stattfinden. Fehler im Mapping oder Modifikation von Schriften können zu falsche Ausgabebuchstaben, mangelnde Wiederverwendung und fehlerhafte Textkonvertierung führen. Jede Glyphe im Dokument wird über einen Character-Code prozessiert. Daraufhin erfolgt eine Zuordnung des Character Codes zum hinterlegten Encoding (Mapping). Zuletzt wird die Glyphe im aktuellen Font über die Glyphen-ID zum Zeichen der Glyphe aufgerufen. Folglich erzielt das Mapping des Codes und der Aufruf der Glyphe die benötigte Konturbeschreibung. Schriftsubstitution findet immer dann statt, wenn der Character-Code nicht mit der Encoding-Tabelle übereinstimmt. Häufig fehlen bestimmte Glypen im Font. Falls eine Outline-Beschreibung des Fonts zum Erstellungszeitpunkt nicht verfügbar ist, wird die Einbettung des Fonts verhindert. Dies kommt vor allem dann vor, wenn ein Font ein Schutzflag besitzt. Weitere Probleme bei der Schrifteingbettung sind u.a. Laufweitenfehler in Schriften, Fehler in der Buchstabenbeschreibung oder beim Cachen von Fonts. Zwecks der Schriftsubstitution müssen folgende allgemeine Informationen zu einem Font in der PDF-Datei gespeichert sein: Name der Schrift, Typ, Subtyp, Schriftstärke, Zeichenbreite, Laufweite, maximale Ausprägung der FontBox, Dickteninformationen, Positionsangaben über Versal- und x-Höhe und Winkel für Italic (kursiv). Diese Informationen sind selbst bei nicht eingebetteten Schriften vorhanden. Für jeden verwendeten Font wird ein Font Descriptor in der Datei hinterlegt. \cite{schneeberger} 

\subsection{Implementierung von Transparenzen}
Wird eine PostScript oder PDF-Datei erstellt, werden die Transparenzen vom Flattener reduziert (verflacht). Um den gewohnten visuellen Effekt der Transparenzen beizubehalten gibt es unterschiedliche Verfahren bei der Reduzierung auf Vektor- und Pixelebene. 