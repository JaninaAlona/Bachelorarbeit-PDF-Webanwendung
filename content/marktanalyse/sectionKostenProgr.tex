\section{Kostenpflichtige PDF Programme und Onlinedienste}

\subsection{Adobe Acrobat}
Die nicht-Pro Version von Acrobat kann prüfen, ob es sich bei dem geöffneten PDF-Dokument um ein PDF/A-Dokument handelt und auf dessen Konformität prüfen. Zusätzlich kann man sich die Kompatibilität mit anderen PDF-Dateiformaten PDF/X, PDF/E, PDF/VT und PDF/UA anzeigen lassen. \cite{adobe-pdf-a} Acrobat kann über JavaScript ferngesteuert werden. Dazu muss man die Berechtigung zur Ausführung von JavaScript erteilen. \cite{schneeberger}

\subsection{Adobe Acrobat Pro}
PDF-Inhalte können bearbeitet werden und Dokumente unterzeichnet. \cite{adobe-search}
Eine PDF-Datei kann in eine PDF/A-Datei inklusive seiner Varianten, PDF/X, PDF/UU oder PDF/E konvertiert werden. Außerdem kann die Kompatibilität mit diesen Formaten überprüft werden in Preflight-Profilen. \cite{adobe-pdf-a} Die Barrierefreiheit kann automatisch validiert werden oder ein neues Dokument kann direkt barrierefrei erstellt werden.
Adobe Acrobat Pro kann andere Dokumentenformate wie HTML, DOC, DOCX, TXT und RTF in PDF konvertieren, PDF in andere Dateiformate wie Microsoft Word exportieren oder Dokumente unterschreiben. \cite{adobe-formate} 
Mit dem Werkzeug Scan \& \gls{ocr} kann man Pixelbilder als PDF und gescannte PDF-Dokumente in ein durchsuchbares PDF umwandeln. \cite{adobe-search}

\subsection{Onlinetools von Acrobat}
Produktseite: \\
\url{https://www.adobe.com/de/acrobat/online.html} \\
\url{https://www.adobe.com/de/acrobat/online/convert-pdf.html}
Mit den Adobe Acrobat Onlinetools kann man über den Browser verschiedene Dateitypen in PDF umwandeln, unter anderem PDF in JPEG oder andere Bildformate, PDF Dateien bearbeiten und Kompression anwenden. Die Onlinetools können außerdem PDF in Word umwandeln. \cite{adobe-search}
Der Adobe Acrobat PDF-Converter der Onlinetools kann DOCX, DOC, XLSX, XLS, PPTX, PPT, TXT, RTF, JPEG, PNG, TIFF, BMP, sowie Adobe eigene AI-, INDD- und PSD-Dateien in PDF konvertieren. \cite{adobe-formate} Die kostenlose Version des PDF-Converters kann nur begrenzt oft genutzt werden.

\subsection{Microsoft Word}
Aus einem Word-Dokument lässt sich in Microsofts Word eine PDF-Datei, inklusive im PDF/A-Dateiformat, erstellen. \cite{adobe-pdf-a}
