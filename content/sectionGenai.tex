\section{Generative AI}
Generative artificial intelligence (AI) Anwendungen können das Arbeiten mit Text, Bildern, Code und Dokumenten wie PDF erleichtern. Solche Anwendungen sind seit einigen Jahren wertvolle Tools in vielen Businessumgebungen und -workflows. Sie kommen in Tools für Teamkollaboration, wie Zoom, \gls{ccaas} Plattformen und Produktivitätsapps vor. Einer der prominentesten Beispiele sind Microsoft Copilot und Google Duet AI. Sie bauen auf einem \gls{llm} auf und bedienen sich Algorithmen zur Gesprächsführung. Mittels \gls{nlp} und hochmodernen Algorithmen können generative AI-Anwendungen mit Menschen interagieren. Durch eine Benutzeraufforderung (prompt) kann der Benutzer Fragen oder Befehle an das AI Tool übermitteln und erhält menschenähnliche Antworten in wenigen Sekunden \cite{copilot-duet}.

\subsubsection{Microsoft Copilot}
Microsoft Copilot ist u.a. in die Bing Suchmaschine, Microsoft Office, Microsoft Teams und Windows 11 integriert. Copilot basiert auf OpenAIs GPT model. Die Funktionalität von Copilot variiert, je nachdem wo es verwendet wird. Bei Word kann Copilot behilflich sein, Dokumente zu skizzieren, Quellen für Informationen in einem Dokument zu suchen, Wortvorschläge zu machen und Schreibhilfe zu leisten. Benutzer können Informationen aus anderen Microsoft Dokumenten, wie z.B. PowerPoint, beziehen, um ihr aktuelles Dokument zu füllen oder es an die Formatierung von einem anderen Dokument anzupassen. In Bezug auf Excel ist Copilot ein Analysetool, um Daten zu visuellen Repräsentationen zu transformieren oder für automatisierte Prozesse. Der Bot kann sogar Trends von Schlüsseldaten ableiten, Fehler korrigieren, Zellen automatisch vervollständigen und Berechnungen erklären. Bei PowerPoint kann Copilot Präsentationen basierend auf Informationen und Dokumenten des Microsoft Ökosystems erstellen. Präsentationsfolien können auf Grundlage von spezifischen Instruktionen passend, z.B. zum eigenen Stil oder der Stimme, gestaltet werden. 

\subsubsection{Google Duet AI}
Google Duet AI ist ein Bestandteil der Google Workspace Apps und Entwicklertools. Der Service unterstützt bei der Code Assistenz mehr als 20 Programmiersprachen. Benutzer können AppSheets verwenden, um intelligente Businessanwendungen und -workflows zu erstellen. Google bietet außerdem die Vertex Plattform für Softwareentwicklung an. Bei Google Docs gibt es ein Duet Feature Help Me Write zur Erstellung von Dokumenten. Basis dessen, ist ein prompt, der beschreibt, über was der Benutzer gerne schreiben möchte. Folglich arbeitet das AI-System ein passendes Dokument heraus. Das Feature Smart Chips wird für variable Informationen verwendet. Hierbei kann der Benutzer bestimmte Inhaltspassagen mit Bedingungen, bezogen auf einen Ort oder eines bestimmten Businesses, versehen. Help Me Write ist für viele Sprachen verfügbar und kann eine Reihe an unterschiedlichen Typen von Dokumenten, z.B. Blogs zu Jobbeschreibungen, kreieren. Im Falle von Google Sheets bietet Duet eine Datenanalyse an. Durch \gls{nlp}-Technologie assistiert Duet Benutzern bei der Dokumentennavigation und Erstellung von benutzerdefinierten Templates, um Daten zu organisieren. Mächtige Tools für Datenklassifizierung unterstützen den Benutzer Datenkontexte zu verstehen. Zusätzlich kann Duet Fehler finden und berichtigen, Zellen automatisch füllen und Vorschläge für eine Analyse machen. Bei Google Slides spielt Duet ebenfalls eine Rolle beim Erstellen und Optimieren von Präsentationen. Bilder können mittels des Help Me Visualize-Werkzeugs generiert werden und Bilddesigns mit verschiedenen Styles angepasst werden. Duet assistiert außerdem beim Notizen Schreiben für Präsentationen und hilft eine Layoutkonsistenz von Präsentationen beizubehalten. Sowohl Copilot als auch Duet können beim E-Mail Schreiben und Video-Meetings behilflich sein. Vor allem Duet bietet Assistenz beim Programmieren \cite{copilot-duet}.

\subsubsection{Donut}
Normalerweise wird bei der \gls{ocr}-Technologie zunächst das Bild in Graustufen konvertiert und eine Texterkennungsphase eingeleitet, in der Stellen mit Informationsgehalt identifiziert werden. An diesen Stellen werden Textboxen generiert. Danach liest eine \gls{ocr}-Engine, wie Tesseract, den Inhalt jeder Box im Bild und konvertiert ihn zu Text. Zuletzt wird ein Algorithmus eingesetzt, der eine post-\gls{ocr} pipeline oder language model sein kann, um extrahierte Informationen zu klassifizieren. Beispiele für solche Anwendungen sind Amazon Textract oder Microsoft Azure AI Document Intelligence. Eine bessere Alternative für \gls{ocr}-Anwendungen stellt die \gls{ocr}-freie Lösung Donut dar. Donut ist ein Visual Document Understanding model, was ein Bild als input aktzeptiert und textbasierte Aufgaben lösen kann, wie Informationsgewinnung und Beantwortung von visuellen Fragen. Mit Fokus auf dem deep learning img2seq model verwendet Donut als Ersatz zur \gls{ocr}-Technologie einen visuellen Encoder (Swin-B transformer) und einen Textdecoder (BART). Swin-B ist eine transformer-Architektur, die speziell auf image processing zugeschnitten ist. Bilder werden vom transformer in Segmente unterteilt und hierarchisch verarbeitet, um lokale und globale, kontextuelle Informationen zu erfassen. Die Bildsegmente werden mittels eines shifted window-based multi-head self-attention Moduls analysiert, um die Beziehungen zwischen benachbarten Segmenten zu erfassen. Danach ermöglicht eine two-layer \gls{mlp}, welche ein supervised learning Algorithmus eines supervised neural networks ist, dem model, das Schema in jedem Segment zu lernen, damit es ein besseres Verständnis des Bildinhalts entwickeln kann. Zum Schluss durchlaufen die token des Segments die Schichten, die die Segmente wieder zusammenfügen, was dem model ermöglicht, Informationen zu kumulieren und eine verständlichere Repräsentation des Bildes zu liefern. Zum Schluss wird der Output dieses Prozesses dem Decoder, einem multilingual BART model, übergeben \cite{transformers-ocr}.

\subsubsection{Amazon Textract}
Amazon Textract ist ein machine learning (ML) Service, der automatisch Text, Handschrift, Layoutelemente und Daten von gescannten Dokumenten, Bildern oder PDF-Dateien, ohne manuelle Konfiguration, extrahieren kann. Bei Textract handelt es sich um einen \gls{aws} Cloud-Dienst. Im Gegensatz zu traditionellen \gls{ocr}-Anwendungen kann Textract außerdem strukturierte Informationen aus Tabellen oder Formularen erfassen. Nebst Zeichenerkennung werden auch Formatierungen sowie die Struktur des Texts herausgearbeitet. Der Service kann per Konsole, Command Line Interface (CLI) oder API verwendet werden. Mittels der Detect Document Text API wird eine \gls{ocr}-Schnittstelle bereitgestellt, um handschriftliche oder gedruckte Texte aus Dokumenten zu entnehmen. Zusätzlich hat die Analyze Document API das Aufgabenfeld strukturierte Daten einzulesen und Beziehungen bzw. Schlüsselwertpaare aus Tabellen- oder Formularfelder zu erstellen. Extrahierte Informationen sind mit einem Confidence-Score versehen, um dem Benutzer mitzuteilen, wie exakt und verlässlich die Daten sind. Handschriftlicher und gedruckter Text kann mit hoher Genauigkeit und Zuverlässigkeit erkannt werden. Die Extraktion der Daten geschieht sehr performant in kurzer Zeit. Das Pricing des Produkts ist nutzungsabhängig \cite{textract}.

\subsubsection{Perplexity AI}
Perplexity AI ist ein Werkzeug für \gls{nlp} mit Dokumenten. Insgesamt kann man bei Perplexity zwischen den generativen AI assistants Perplexity, \gls{gpt4} oder Claude 2 von Anthropic bei Dateiuploads wählen. Eine PDF-Datei als Dateiformat, plaintext oder Code kann hochgeladen werden und Perplexity verwendet deren Dateiinhalte, um Antworten auf Fragen zum PDF, inklusive Zitate mit Quellenangaben, zu formulieren. Bei kurzen Dateien wird das gesamte Dokument vom language model analysiert. Umfangreiche PDFs können manuell in Themenbereiche unterteilt und als Input für \gls{gpt4} für Kreatives Schreiben verwendet werden. Wissenschaftliche Artikel können verglichen, ihre Unterschiede herausgearbeitet, themenverwandte Dokumente durch eine query gefunden, Daten analysiert, Überblicke von verschiedenen Quellen generiert, Daten visualisiert, Grafiken von Quellen erstellt und Text in eine andere Sprache übersetzt werden. Bei der kostenlosen Version ist der Benutzer auf eine bestimmte Anzahl an Anfragen begrenzt \cite{hackernoon-claude}. Der Claude 2 AI assistant kann PDFs parsen und die Dokumentstruktur mittels ML-Techniken erfassen. Es hat eine eingebaute PDF-analyser-Komponente. Die maximale Dateigröße ist 25 MB. Der Benutzer kann Claude Fragen stellen, die im PDF-Dokument behandelt werden, und Claude liefert die Antworten. Direkte Zitate können aus der PDF-Datei gefunden werden und Claude zeigt die Seitenzahl an, wo das Zitat vorkommt. Außerdem kann Claude PDF-Inhalt zusammenfassen. Copilot ist ebenfalls verfügbar, um schnellere Antworten auf prompts in einer menschlichen Art und Weise zu liefern. Bei umfangreichen Dateien werden die relevantesten Dateisegmente analysiert, um wichtige Antworten zu geben. Code kann erklärt und Dateien übersetzt werden. Claude und \gls{gpt4} gelten als intelligentere models, um große Dateien zu parsen \cite{perplexity}.

\subsubsection{Adobe Firefly}
Adobe Firefly ist eine generative AI zur Erstellung von Bildern durch prompts. Im März 2023 wurde die Beta-Phase gestartet und seitdem haben Nutzer mehr als 3 Milliarden Bilder generiert. Firefly ist in Photoshop, Illustrator und Adobe Express integriert und existiert als eigenständige Webanwendung. Bei Photoshop ist es als Generative Füllung und Generatives Erweitern eingebaut. Diese Funktionen ermöglichen dem Nutzer Bildinhalte per Text prompt hinzuzufügen, zu entfernen, zu ersetzen oder zu erweitern, um Bilder umfassend zu ändern oder zu ergänzen. Im Bezug auf Illustrator wurden die Tools Generative Neufärbung, um Farbvarianten eines Designs zu erstellen, und Text to Vector Graphic, um aus einem prompt eine Vektorgrafik zu generieren, eingebaut. Adobe Express ergänzt Photoshop, Illustrator, Adobe Premiere Pro, sowie Acrobat und ermöglicht den Import, die Bearbeitung und Synchronisierung von Assets zwischen den Applikationen für Echtzeitzusammenarbeit. Es wurde für schnelle Aufgaben, wie das Entfernen von Hintergründen, die Erstellung von Inhalten für soziale Medien bzw. die Freigabe von Konzepten, auf den Markt gebracht. In Adobe Express wird Firefly für Text zu Bild, Texteffekte, Generative Füllung und Text to Template eingesetzt. Text to Template ermöglicht dem Benutzer durch einen prompt editierbare Templates zu generieren, um auf schnelle Art und Weise Social Media Posts, Poster, Flyer und digitale Karten zu erstellen. Die generierten Resultate von Text zu Bild und Texteffekte lassen sich in Adobe Express direkt als PDF-Datei speichern. Firefly unterstützt prompts in 100 Sprachen weltweit und die Webanwendung ist in 20 Sprachen verfügbar. In der Webanwendung ist kürzlich das Adobe Firefly Image 2 Model in der Beta-Version nebst Model 1 für mehr Gestaltungsmöglichkeiten und verbesserte Bildberechnung bzw. -qualität auf den Markt gebracht worden. Bei Text zu Bild wurde Generative Match implementiert, welcher einen Stil von mehreren, vorausgewählten Bildern auf ein neues generiertes Bild anwendet. Der Bildstil kann mit integrierten Stileffekten kombiniert werden \cite{adobe-firefly}. Einige der im Folgenden vorgestellten PDF-Programme bedienen sich ebenfalls der generativen AI-Technologie.