\section{PDF Implementierung}
PDF ist eine vektorbasierte \acrfull{pdl} (Seitenbeschreibungssprache) und basiert auf dem PostScript-Format. Eine \acrshort{pdl} beschreibt den Seitenaufbau, wie die Seite in einem Ausgabeprogramm bzw. Ausgabegerät, z.B. einem Drucker, aussehen soll.  Seitenbeschreibungssprachen können Seiten mit Vektoren beschreiben. Das Ausgabeformat ist normalerweise nicht zur weiteren Bearbeitung vorgesehen. An den Drucker wird durch die \ac{pdl} ein Datenstrom der zu druckenden Aufgabe erzeugt und an den Drucker gesendet. Dabei müssen Drucker die \acrshort{pdl} nicht selbst verarbeiten. Im Common Unix Printing System, der Standard-Druckersteuerung von Linux hat der PostScript und PDF-Interpreter ghostscript die Aufgabe eines \ac{rip}, d.h. er ist für die Umwandlung in die gerasterte Druckausgabe auf dem Drucker zuständig. Viele APIs der Hardwareabstraktionsschicht im Computer wie \acrfull{gdi} oder OpenGL können in \acrshort{pdl} ausgeben.
