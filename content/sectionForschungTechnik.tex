\section{Aktueller Stand von Forschung und Technik}
Generative Artificial Intelligence (AI) Anwendungen können das Arbeiten mit Text, Bildern, Code und Dokumenten wie PDF erleichtern. Solche Anwendungen sind seit einigen Jahren wertvolle Tools in vielen Businessumgebungen und -workflows. Sie kommen in Tools für Teamkollaboration, wie Zoom, \gls{ccaas} Plattformen und Produktivitätsapps vor. Einer der prominentesten Beispiele sind Microsoft Copilot und Google Duet AI. Sie bauen auf ein \gls{llm} auf und bedienen sich Algorithmen zur Gesprächsführung. Mittels \gls{nlp} und hochmodernen Algorithmen können generative AI Anwendungen mit Menschen interagieren. Durch eine Benutzeraufforderung (prompt) kann der Benutzer Fragen oder Befehle an das Tool übermitteln und erhält menschenähnliche Antworten in wenigen Sekunden. Copilot ist u.a. in die Bing Suchmaschine, Microsoft Office, Microsoft Teams und Windows 11 integriert. Seine Funktionalität variiert, je nachdem wo es verwendet wird. Bei Word kann Copilot behilflich sein, um Dokumente zu skizzieren, Quellen für Informationen in einem Dokument zu suchen, Wortvorschläge zu machen und Schreibhilfe zu leisten. Benutzer können Informationen aus anderen Microsoft Dokumenten, wie z.B. PowerPoint, ziehen, um ihr aktuelles Dokument zu füllen oder sein aktuelles Dokument an die Formatierung von einem anderen Dokument anpassen. In Bezug auf Excel ist Copilot ein Analysetool, um Daten zu visuelle Repräsentationen zu transformieren oder bei automatisierten Prozessen. Der bot kann sogar Trends von Schlüsseldaten, und Fehler korrigieren, Zellen automatisch vervollständigen und Berechnungen erklären. Bei PowerPoint kann Copilot Präsentationen basierend auf Informationen und Dokumente des Microsoft Ökosystems erstellen. Präsentationsfolien können auf Grundlage von spezifischen Instruktionen, wie passend zum eigenen Stil oder der Stimme, gestaltet werden. Google Duet AI ist ein Bestandteil des Google Workspace Apps und Entwicklertools. Der Service unterstützt mehr als 20 Programmiersprachen bei der Code Assistenz. Benutzer können AppSheets verwenden, um intelligente Businessanwendungen und Workflows zu erstellen. Google bietet außerdem die Vertex Platform für Entwicklung an. Bei Google Docs gibt es ein Duet Feature Help Me Write zur Erstellung von Dokumenten. Basis ist ein prompt, der beschreibt, über was der Benutzer gerne schreiben möchte und das AI System arbeitet ein passendes Dokument heraus. Das Feature Smart Chips wird für variable Informationen verwendet. Hierbei kann der Benutzer bestimmte Inhaltspassagen mit Bedingungen bezogen auf einen Ort oder eines bestimmten Businesses versehen. Help Me Write ist für viele Sprachen verfügbar und kann eine Reihe an unterschiedlichen Typen von Dokumenten von Blogs zu Jobbeschreibungen kreieren. Im Falle von Google Sheets bietet Duet eine Datenanalyse an. Durch \gls{nlp} Technologie assistiert Duet Benutzern bei der Dokumentennavigation und Erstellung von benutzerdefinierten Templates, um Daten zu organisieren. Mächtige Tools für Datenklassifizierung unterstützen den Benutzer Datenkontexte zu verstehen. Zusätzlich kann Duet Fehler finden und berichtigen, Zellen automatisch füllen und Vorschläge für die Analyse machen. 

Sowohl Copilot als auch Duet können beim E-Mail Schreiben behilflich sein. 

\gls{ie} ist eine Unterkategorie von \gls{nlp} und ist relevant für die Identifizierung von relevanten Informationen in Text und die Extrahierung dieser Informationen zu spezifischen Output-Formaten. Bei der \gls{re} werden relevante Einheiten im Text in Beziehung zueinander gesetzt. Normalerweise wird zunächst das Bild in Graustufen konvertiert und eine Texterkennungsphase wird eingeleitet, in der Stellen identifiziert werden, wo sich Informationen befinden. Auf diese Stellen werden Textboxen generiert. Danach liest die \gls{ocr}-Engine wie Tesseract den Inhalt jeder Box im Bild und konvertiert ihn zu Text. Zuletzt wird ein Algorithmus eingesetzt, der eine post-\gls{ocr} pipeline oder language model sein kann, um extrahierte Informationen zu klassifizieren. Beispiele für solche Anwendungen sind Amazon Textract oder Microsoft Azure AI Document Intelligence. Eine bessere Alternative für \gls{ocr}-Anwendungen stellt die \gls{ocr}-freie Lösung Donut dar. Donut ist ein Visual Document Understanding model, was ein Bild als input aktzeptiert und textbasierte Aufgaben lösen kann, wie Informationsgewinnung und Beantwortung von visuellen Fragen. Mit Fokus auf deep learning img2seq models verwendet Donut als Ersatz zur \gls{ocr}-Technologie einen visuellen Encoder (Swin-B Transformer) und einen Textdecoder (BART). Swin-B ist eine Transformer-Architektur, die speziell auf image processing zugeschnitten ist. Bilder werden vom Transformer in Segmente unterteilt und hierarchisch verarbeitet, um lokale und globale kontextuelle Informationen zu erfassen. Die Bildsegmente werden mittels eines shifted window-based multi-head self-attention Moduls analysiert, um die Beziehungen zwischen benachbarten Segmenten zu erfassen. Danach ermöglicht eine two-layer \gls{mlp} dem model das Schema in jedem Segment zu lernen, damit es ein besseres Verständnis der Bildinhalts entwickeln kann. Zum Schluss durchlaufen die tokens des Segments die Schichten, die die Segmente wieder zusammenfügen, was dem model ermöglicht, Informationen zu kumulieren und ein verständlichere Repräsentation des Bildes zu liefern. Der output dieses Prozesses wird Decoder, einem multilingual BART model, übergeben \cite{transformers-ocr}. \\

Amazon Textract ist ein ML-Service, der automatisch Text, Handschrift, Layoutelemente und Daten von gescannten Dokumenten, Bildern oder PDF-Dateien ohne manuelle Konfiguration extrahieren kann. Bei Textract handelt es sich um einen \gls{aws} Cloud-Dienst. Im Gegensatz zu traditionellen \gls{ocr}-Anwendungen kann Textract außerdem strukturierte Informationen aus Tabellen oder Formularen erfassen. Nebst Zeichenerkennung werden auch Formatierungen, sowie die Struktur des Text herausgearbeitet. Der Service kann per Konsole, Command Line Interface (CLI) oder Application Programming Interface (API) verwendet werden. Mittels der Detect Document Text API wird eine \gls{ocr}-Schnittstelle bereitgestellt, um handschriftliche oder gedruckte Texte aus Dokumenten zu entnehmen. Zusätzlich hat die Analyze Document API das Aufgabenfeld, strukturierte Daten einzulesen und Beziehungen bzw. Schlüsselwertpaare aus Tabellen- oder Formularfelder zu erstellen. Extrahierte Informationen sind mit einem Confidence-Score versehen, um dem Benutzer mitzuteilen, wie exakt und verlässlich die Daten sind. Handschriftlicher und gedruckter Text kann mit hoher Genauigkeit und Zuverlässigkeit erkannt werden. Die Extraktion der Daten geschieht sehr performant in kurzer Zeit. Das Pricing des Produkts ist nutzungsabhängig \cite{textract}.


Perplexity AI ist ein Werkzeug für \gls{nlp} mit Dokumenten. Ingesamt kann man bei Perplexity zwischen den generativen AI assistants Perplexity, \gls{gpt4} oder Claude 2 bei Dateiuploads wählen. Eine PDF-Datei als Dateiformat, plaintext oder Code kann hochgeladen werden und Perplexity verwendet dessen Dateiinhalte, um Antworten auf Fragen zum PDF inklusive Zitate mit Quellenangaben zu formulieren. Bei kurzen Dateien wird das gesamte Dokument beim language model analysiert. Umfangreiche PDFs können manuell in Themenbereiche unterteilt werden und als input für \gls{gpt4} für Kreatives Schreiben verwendet werden. Wissenschaftliche Artikel können verglichen, ihre Unterschiede herausgearbeitet, themenverwandte Dokumente durch eine query gefunden, Daten analysiert, Überblicke von verschiedenen Quellen generiert, Daten visualisiert, Grafiken von Quellen erstellt und Text in eine andere Sprache übersetzt werden. Bei der kostenlosen Version ist der Benutzer auf eine bestimmte Anzahl an Anfragen begrenzt \cite{hackernoon-claude}. Claude 2 model von Anthropic ist verfügbar in Perplexity beim Dateiupload. Der AI assistant kann PDFs parsen und die Dokumentstruktur mittels Machine Learning (ML) Techniken erfassen. Es hat eine eingebaute PDF analyser Komponente. Die maximale Dateigröße ist 25 MB. Der Benutzer kann Claude Fragen stellen, die im PDF-Dokument behandelt werden und Claude liefert die Antworten. Direkte Zitate können aus der PDF-Datei gefunden werden und Claude zeigt die Seitenzahl an, wo das Zitat vorkommt. Außerdem kann Claude PDF-Inhalt zusammenfassen. Copilot ist ebenfalls verfügbar, um schnellere Antworten auf prompts in einer menschlichen Art und Weise zu liefern. Bei umfangreichen Dateien werden die relevantesten Dateisegmente analysiert, um wichtige Antworten zu geben. Code kann erklärt und Dateien übersetzt werden. Claude und \gls{gpt4} gelten als intelligentere models, um große Dateien zu parsen \cite{perplexity}.

ChatGPT Plus Mitglieder können seit Oktober 2023 ein PDF-Analysefeature als Beta-Version genießen. Benutzer können PDF-Dateien und andere Dokumente hochgeladen werden und der chatbot kann Zusammenfassungen erstellen und Graphen oder Tabellen basierend auf die Dokument-Daten erstellen \cite{hackernoon-claude}.

