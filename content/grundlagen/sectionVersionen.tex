\section{PDF Dateiversionen}

\subsection{PDF 1.0}
PDF 1.0 wurde 1992/1993 entwickelt und ist keine Norm. 1992 wurde die Spezifikation als Buch verkauft und 1993 wurde das der Spezifikation entsprechende digitale Format entwickelt, welches ausschließlich den RGB Farbraum darstellen konnte. Medien, die einen anderen Farbraum besaßen wurden in RGB umkonvertiert. Der RGB Farbraum ist nur für die Bildschirmdarstellung geeignet und beschreibt die für den Menschen 16,7 Mio. sichtbaren Farben mit Hilfe von additiver Farbmischung. In der Druckindustrie ist jedoch der CMYK Farbraum von Bedeutung und daher war PDF 1.0 nicht für den Printbereich ausgelegt. Damals war Adoba Acrobat 1.0 das einzige Programm, um mit dieser Dateiversion zu arbeiten. \cite{proj-consult}

\subsection{PDF 1.1}
Genauso ist das 1994 kreierte PDF 1.1 keine Norm und implementiert weiterhin nur den RGB Farbraum, jedoch geräteunabhängig. Zusätzlich benötigte man ein Update von Adobe Acrobat auf Version 2.0. Erstmals sind in diesem Format das Einbetten von externen Links, mehrseitige Artikel und Threads, Passwortverschlüsselung und Notizen und Anmerkungen erschienen. \cite{proj-consult}

\subsection{PDF 1.2}
Das ebenfalls 1996 erschienene PDF 1.2 wurde keine Norm, jedoch ermöglichte es erstmals den druckbaren CMYK Farbraum und Sonderfarben zu verwenden. Des weiteren wurden interaktive Formularfunktionen, Unicode Unterstützung, Multimedia Kompatibilität, Unterstützung der \gls{opi} 1.3 Spezifikationen und eine Druckrasterfunktion implementiert. \cite{proj-consult} In PDF 1.2 wurden erstmal AcroForms (Acrobat forms) vorgestellt.

\subsection{PDF 1.3}
1999 wurde PDF 1.3 auf den Markt gebracht und trug seinen Teil 2001 und 2002 bei zur Standardisierung des \gls{iso} PDF/X Standards bei. Es ist kompatibel mit PostScript 3 und bietet die Neuerungen der 2-Byte \gls{cid} Schrifttypen, OPI 2.0 Unterstützung, Farbraumerweiterung für Sonderfarben durch \gls{icc}-Profile, DeviceN Farbraum, weiche Schatten und Farbübergänge (Smooth Shading), digitale Signaturen, RC4-Verschlüsselung (40 Bit in Acrobat 4 und 56 Bit in Acrobat 4.05) und JavaScript. \cite{proj-consult}


\subsection{PDF 1.5}
2003 kam PDF 1.5 auf den Markt und hat sich nicht zur Norm entwickelt. In dieser Version wurden erstmals Ebenen implementiert, die erlauben dass man mehrere Elemente wie eine Gruppe auf einer Ebene speichern kann und diese Elemente auf einmal nach Bedarf ein- und ausblenden kann, sperren oder Operationen anwenden kann. Diese Funktionalität enthalten auch die Adobe Programme Photoshop, InDesign und Illustrator. Des weiteren wurden gesteigerte Kompressionstechniken einschließlich Objekt-Streams und JPEG 2000-Kompression, sowie eine verbesserte XRef-Tabelle und XRef-Streams implementiert.
Die XRef-Tabelle enthält die Positionen der indirekten Objekte innerhalb der Datei. Streams binden Dateien ein. 12 weitere Seitenübergänge für Präsentationen, verbesserte Unterstützung für Tagged PDF und die Adobe proprietäre Technologie \gls{xfa} wurden außerdem hinzugefügt. \cite{proj-consult} \gls{xfa}s Haupterweiterung zu \gls{xml} sind rechnergestützte, aktive Tags uns sein Datenformat ist kompatibel mit anderen Systemen, Anwendungen und Technologiestandards. \cite{wiki-xfa}

\subsection{PDF 1.4}
Der erste PDF \gls{iso}-Standard \gls{iso} 16612-1:2005 wurde endlich verabschiedet.

\subsection{PDF 1.6}

\subsection{PDF 1.7}
Veröffentlichung am 1. Juli 2008 ist PDF in Version 1.7 als \gls{iso} 32000-1:2008 ein Offener Standard

\subsection{PDF 2.0}
\gls{xfa} ist in PDF 2.0 vom \gls{iso} Gremium als veraltet markiert.