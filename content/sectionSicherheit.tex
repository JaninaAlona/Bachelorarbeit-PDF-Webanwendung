\section{PDF Sicherheitsaspekte}
Etwa 40 \% der Unternehmen setzen PDFs für geschützte Inhalte ein. In den letzten 2 Jahren ist die Nutzung der elektronischen Signaturfunktion in PDFs um mehr als 150 \% gestiegen. \cite{formilo} Diese Statistik zeigt keine gute Entwicklung, da PDF-Dokumente nicht für vertrauliche Inhalte verwendet werden sollten. Obwohl PDF Verschlüsselung verwendet erlaubt das Dateiformat eine Vielzahl an Angriffen, um die Sicherheit von PDF Dokumenten zu untergraben. Ich werde im Folgenden die implementierten Sicherheitsmechanismen von PDF und die möglichen Angriffe beleuchten.
\par
In den Sicherheitseinstellungen eines PDF-Dokuments können Dokumentensicherheit und Zugriffsregeln justiert werden. PDF unterstützt Verschlüsselung und die Vergabe von Passwörtern. Eventuell kann beim Öffnen einer Datei ein Passwort gefordert werden oder das Kopieren von Teilinhalten, jeglichem Inhalt, Ausfüllen von Formularfeldern, Dokumentveränderungen (z.B. Struktur, Inhalt, Kommentare), das Einbetten und Editieren von Schriften oder das Ausdrucken kann vom Ersteller des Dokuments gesperrt worden sein. Das Schwärzen von Dokumenten, d.h. ein schwarzer Balken liegt auf dem Text, führt nicht dazu, dass der geschwärzte Text aus der Datei verschwindet. Er kann weiterhin ausgelesen werden, ebenso Metadaten. 
\cite{adobe-pdf-pades}

\subsection{Digitale Signatur}
Digitale Unterschriften sollen die Identität des Unterzeichners des Dokuments authentifizieren und dass der Inhalt nach der digitalen Unterschrift nicht geändert wurde. Der Verfasser benötigt für die digitale Signatur ein Signatur-Zertifikat. Das Zertifikat bescheinigt die Echtheit der Unterschrift und der Herkunft und wird von einem Zertifizierungsanbieter ausgestellt. Zusätzlich können Zertifikate ablaufen oder entzogen werden und müssen gültig sein. Dabei sollte ein vertrauenswürdiger Zertifizierungsanbieter gewählt werden. Jede Unterschrift kann mit einem Zeitstempel versehen werden. Ein vertrauenswürdiger \gls{zsa} belegt den Zeitpunkt, wann diese Unterschrift geleistet wurde. \cite{softx} \\
Eine digitale Signatur ist eine spezielle Art von elektronischer Signatur, die kryptographische Techniken implementiert. Im Gegensatz dazu kann eine elektronische Signatur verschiedene Formen annehmen, beispielsweise eine gescannte handschriftliche Unterschrift, ein getippter Name, eine biometrische Signatur oder eine digitale Signatur. Elektronische Signaturen sind der Oberbegriff und digitale Signaturen sind eine Teilmenge davon. Elektronische Signaturen bieten variierende Sicherheitsgrade. \cite{adobe-pdf-pades} \\
Digitale Signaturen verwenden asymmetrische Kryptographie. Anfangs wird ein Algorithmus zur Schlüsselgenerierung wird gestartet. Als Ausgabe werden ein Private Key und Public Key erzeugt. Ein Signierungsalgorithmus erstellt den einzigartigen Hash-Wert des PDF-Dokuments und verschlüsselt ihn mit dem geheimen Private Key, was die digitale Signatur produziert. Zuletzt überprüft ein Signaturvalidierungsalgorithmus die Authentizität des PDF-Dokuments, indem er erneut einen Hash des Dokuments berechnet. Die Signatur wird mit dem Public Key entschlüsselt und gibt einen Hash aus. Falls der berechnete Hash im Signaturvalidierungsalgorithmus mit dem entschlüsselten Hash übereinstimmt, ist die Authentizität des PDF-Dokuments gesichert. \cite{signature} \\
Eine PDF-Datei ermöglicht mehrere digitale Unterschriften, jedoch muss jede neue Unterschrift in einem inkrementellen Update geleistet werden. Jede Unterschrift muss mit einem Unterschriftsfeld im Dokument verbunden sein. Optional kann das Unterschriftsfeld mit einem Widget gekoppelt sein. Dann wird die Unterschrift graphisch dargestellte. Unterschriften ohne Widgets sind versteckte Unterschriften. \cite{softx} Es gibt Desktop-Anwendungen und online Validierungsservices zur Überprüfung der digitalen Signatur.


\subsection{PDF Signature Spoofing}
Die PDF-Spezifikation ist sehr ungenau formuliert im Bezug auf Signaturen und auf welche Art und Weise sie validiert werden müssen. PDF-Viewer haben eine hohe Toleranz beim Öffnen, Validieren und Anzeigen von beschädigten PDF-Dateien. \\
Im Jahr 2019 wurden 3 Attacken zum Vortäuschen von validen PDF Signaturen gefunden: \gls{isa}, Signature Wrapping Attack und Universal Signature Forgery. Mittels dieser Attacken ist es den Forschern gelungen zu zeigen, dass 21 von 22 PDF-Viewer anfällig waren. Das Attack Scenario sieht wie folgt aus: Der Angreifer besitzt das signierte PDF mit einer gültigen Signatur und manipuliert es. Das manipulierte  signierte PDF wird an das Opfer geschickt und die Signatur bleibt gültig, obwohl der Inhalt geändert wurde.
\par
Die \gls{isa} nutzt das Incremental Update-Feature von PDF aus, um bösartigen Inhalt ins PDF zu schleusen. Im Prozess des Signierens wird ein Incremental Update verwendet, um die Signatur zu speichern. Am Ende des Original-Trailers wird ein neuer Katalog und ein neues Signatur-Objekt als Body Updates angehängt, was den Signaturwert und Informationen über den Ersteller der Signatur enthält. Danach kommt eine updated Xref-Section und ein Updated Trailer. Beim Inremental Update können Objekte mit anderem Inhalt neu definiert werden. Der Updated Body wird hinter dem Originaltrailer angehängt, darunter zeigt die Updated Xref-Section auf das neue Objekt und der Updated Trailer wird ganz am Ende angefügt. Zuerst hat man geprüft, ob die PDF-Reader gegen eine Attacke anfällig waren, bei der hinter dem Incremental Update, was die Signatur enthält, eine weitere Sektion mit Body Updates, Xref Table und Trailer angefügt wurde. Ausschließlich LibreOffice wurde durch diese Vorgehensweise getäuscht. Als Nächstes haben die Forscher einfach hinter dem Updated Trailer des Incremental Updates der Signatur Body Updates hinzugefügt, was eigentlich eine beschädigte PDF-Datei darstellt. Einige getestete PDF-Viewer haben lediglich überprüft, ob eine neue Xref-Sektion und Trailer vorhanden sind. Da sie nicht vorhanden waren, blieb die Signatur gültig und die zusätzlichen Body Updates wurden ausgeführt und die Modifikation blieb ohne Warnung unbemerkt. Andere PDF-Viewer benötigten neben zusätzlichen Body-Updates noch einen Trailer ohne Xref dazwischen, damit keine Warnung geworfen wurde. Die komplexeste Vorgehensweise, bei der neue Body Updates mit einer Kopie des Signatur-Objekts am Dateiende angebracht wurde, hat einige PDF-Viewer wie Foxit gezwungen die Signatur 2 Mal zu validieren und die Body Updates wurden eingefügt.
\par
Bei der Signature Wrapping Attack werden die Werte der signierten ByteRange modifiziert und Platz wird für das Einschleusen von bösartigen Inhalt geschaffen. Das Signatur-Objekt besteht aus einem /Contents-Eintrag mit dem Signaturwert und einem /ByteRange-Eintrag mit 4 Werten. Die ersten 2 Einträge der ByteRange beziehen sich auf den Beginn des Dokuments bis zum Anfang des Signaturwerts. Hingegen definieren die letzten beiden ByteRange-Einträge den Bereich nach dem Signaturwert bis zu \%\%EOF. Diese beiden Bereiche wurden nicht angerührt. Die erfolgreiche Idee war eine zweite ByteRange hinter dem Signaturwert mit einem angepassten dritten Wert, der Platz für bösartige Objekte, etwas Padding und eine bösartige Xref, die auf die bösartigen Objekte zeigt, zu setzen. Lediglich die bösartige Xref-Position ist im Trailer vorgegeben, der nicht verändert werden kann. 
\par
Mittels Universal Signature Forgery wird die Signaturvalidierung außer Kraft gesetzt. Dennoch wird die Meldung „PDF is validly signed" dem Benutzer angezeigt.

