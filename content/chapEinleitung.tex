\chapter{Einleitung}

\section{Motivation}
Grundsätzlich benutze ich 4 PDF-Programme, um alle für mich zufriedenstellenden, häufigen Anforderungen der PDF-Bearbeitung zu erledigen: Adobe Reader zum Anzeigen von PDF, Drawboard PDF zum Zeichnen, Foxit Reader für Texte und PDF Sam Basic zum Splitten und Mergen. Einige dieser Programme sind sehr unintuitiv und unübersichtlich. Ein Nachteil von PDF Sam Basic ist, dass zusätzliche Werbeversionen installiert werden. Kein PDF-Programm vereinte alle gewünschten Funktionalitäten. Aus diesem Grund entwickelte ich meine eigene, kompakte PDF-Anwendung. Ein maßgeblicher Anspruch war es, dass die PDF-Anwendung benutzerfreundlich ist. Die PDF Web App ist kein abgeschlossenes Projekt. Dieses Hobby-Projekt möchte ich nach der Bachelorarbeit auf meiner Github-Seite fortführen. Möglicherweise gibt es weitere Entwickler, die mein Repository forken oder bug fixes hinzusteuern.

\section{Aufbau der Arbeit}
Die Arbeit besteht im Kern aus 5 Kapiteln. Beginnend mit den Grundlagen des PDF-Dateiformats, werden die wichtigsten Features, die Implementierung von PDF, die bisher veröffentlichten Formatvarianten und -versionen, sowie die kritischen Sicherheitsaspekte beleuchtet. Im anschließenden 2. Kapitel, Stand der Technik, werden die neusten Generativen AI-Entwicklungen aufgezeigt. Der Funktionsumpfang von 7 repräsentativen PDF-Anwendungen wird analysiert. Die Rolle von PDF in gängigen Marktbranchen und speziell, in der Druckvorstufe bzw. Designbranche, wird erläutert. Im 3. Kapitel Anforderungen und Konzept werden die Rahmenbedingungen, inklusive Qualitätsanforderungen, für die von mir entwickelte PDF-Anwendung vorgestellt. Das anknüpfende 4. Kapitel Implementierung erläutert und illustriert die Bedienung der PDF Web App. Der Abschnitt Umsetzung in Code geht detailliert auf die Programmierung ein, Programmtests schließen das Kapitel ab. Das letzte 5. Kapitel Diskussion und Kritik setzt sich mit Verbesserungsvorschlägen und Ideen für zukünftige Releases auseinander. 