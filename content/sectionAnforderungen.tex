\section{Problemstellung und Anforderungen}
Das Thema der Bachelorarbeit \textit{\glqq Entwicklung einer webbasierten Applikation zur Bearbeitung von PDF Dateien\grqq} habe ich selbst gewählt. Ziel der Arbeit ist eine PDF-Webapplikation zu entwickeln, die gängige Funktionalitäten, die man bei üblicher PDF-Bearbeitung benötigt, bieten soll. Gängige Funktionalitäten sind für mich ein PDF-Reader, das Editieren von Seiten und die grundsätzlichsten Grafikoperationen. Mir war außerdem wichtig, dass die PDF Web App auf allen Desktop-Plattformen Windows, macOS und Linux einsetzbar ist, da PDF genauso ein plattformunabhängiges und hardwareunabhängiges Format ist. Ich habe mir selbst die Vorgabe gestellt, dass die PDF Web App lediglich als Desktop-App funktionieren muss und nicht optimiert für Tablet oder Smartphone sein soll. PDF-Bearbeitung macht für mich am meisten Sinn in einer Desktopumgebung, da der Bildschirm auf Tablets und vor allem bei Smartphones nicht besonders groß bis sehr klein ist. PDF-Bearbeitung nimmt Zeit in Anspruch und am besten setzt man sich an den Schreibtisch vor einen großen Bildschirm und setzt seine PDF-Modifikationen mit der Maus und Tastatur um. \\
Ursprünglich war die PDF Web App als offline Webseite angedacht. Die Webseite sollte man auch ohne Internetverbindung wie ein Programm verwenden können, nur dass keine Installation notwendig ist. Das einzige Programm, was man benötigt für die Ausführung der PDF Web App soll ein Browser sein. Durch Öffnen der index.html im Browser kann man die PDF Webapp auf einfache Art und Weise starten und benutzen. Eine Anforderung an mich selbst ist, dass die PDF Web App als Open Source-Projekt für jeden kostenlos nutzbar sein soll, da ich selbst nur kostenlose PDF-Bearbeitungsprogramme verwende. So soll die PDF Web App auch u.a. für Studenten mit wenig Geld zur Verfügung stehen. Das Open Source-Projekt soll als öffentliches Repository auf Github vorhanden sein und jeder kann es downloaden. Da das PDF-Dateiformat ein offenes Dateiformat ist, finde ich, dass die Bearbeitungsprogramme für PDF-Dateien ebenfalls kostenlos sein sollten. Auf dem öffentlichen Github Repository soll man lediglich die PDF Web App runterladen können ohne Werbeprogramme oder ähnliches. \\
Des Weiteren wollte ich mich nur auf HTML, CSS und JavaScript beschränken und kein Backend verwenden, was Daten speichert. Dadurch habe ich die Programmierung und den Verwaltungsaufwand der PDF Web App simpler gemacht. Die Bedienung sollte möglichst intuitiv, eingängig und einfach gehalten sein, damit ein Nutzer nicht seitenlange Tutorials lesen muss und die Funktionen der Einstellmöglichkeiten möglichst durch Ausprobieren erfassen kann. Um diese Anforderung zu erreichen, sollte eine GUI mit simplem, übersichtlichem und strukturiertem Design verwendet werden. \\
Eine weitere Vorgabe der Gutachterin der Bachelorarbeit ist eine Marktübersicht inklusive der Beschreibung anderen PDF Wettbewerberprogramme zu erarbeiten. \\

Daraufhin habe ich mir einige freie und kostenpflichtige PDF-Programme und Onlinedienste ausgesucht: 

\begin{itemize}
	\item Adobe Acrobat
	\item foxit
	\item pdf-it
	\item PDF24
	\item PDFelement
	\item Soda PDF
	\item Nitro PDF Pro 
	\item Ashampoo PDF Pro 3
	\item Infix 7
	\item PDFsam Basic und Advanced
	\item DrawboardPDF
	\item PDF Director 2 PRO
	\item Perfect PDF
\end{itemize}

