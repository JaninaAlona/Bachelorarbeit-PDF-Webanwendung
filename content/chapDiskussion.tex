\chapter{Diskussion und Kritik}
Es gibt einige Aspekte meiner PDF Web App, die zur weiteren Diskussion anregen. Die Erweiterung der PDF Web App um eine Druckfunktion kann zu einem späteren Zeitpunkt noch erfolgen. Eine Druckfunktionalität wird durch die same-orgin policy des Browsers beeinflusst. Diese same-origin policy blockiert den Lesezugriff von Ressourcen, die von einer anderen Herkunft geladen werden. Diese Sicherheitsfunktionalität regelt, wie Dokumente und Scripte mit Ressourcen anderen Ursprungs interagieren \cite{same-origin}. 
\par
Aktuell kann der Creator keine exakte Größe des Seitenformats erstellen. Im Nachkommabereich ist die eingestellte Größe der PDF-Datei um einen Bruchteil kleiner. In visueller Hinsicht macht diese Eigenschaft keinen Unterschied. Ursächlich ist, dass die PDF-Spezifikation keine expliziten Breiten und Höhen von Dokumenten implementiert. Stattdessen wird die Seitengröße durch die MediaBox, CropBox, BeleedBox, TrimBox und ArtBox definiert. Daher lassen sich die Dimensionen der Seite nicht direkt erstellen. Es gibt einige Optimierungsmöglichkeiten von bestehenden Funktionalitäten. Im Falle der input fields für die Seitenliste könnte die gültige Benutzereingabe durch die Möglichkeit eines Bindestrichs ergänzt werden, z.B. Seiten x-y. 
\par
Kopierte Texts und Images sind minimal, kaum wahrnehmbar verschoben. Ihre controlBoxes sind sehr wohl kongruent. Dieser Effekt verschwindet, sobald eine Elementoperation ausführt wird. Komplett wegradierte Layers im Drawer könnten automatisch entfernt werden. Im Kontrast dazu müssen im aktuellen Release vollständig ausradierte Layers manuell gelöscht werden. Eventuell wäre der intuitive Arbeitsfluss des Zeichenvorgangs behindert, sofern die Layer automatisiert von der Anwendung entfernt würde. Das Zeichnen auf einem Graphic Tablet funktioniert flüssig. 
\par
Die Verwendung der absoluten Rotation von Elementen hat Vor- und Nachteile. Relative Rotation könnte ein intuitives Arbeiten ermöglichen und Kopfrechenschritte würden sich erübrigen. Hingegen hat absolute Rotation den Vorteil, dass die Rotation immer gleich ausgeführt wird und somit besser nachvollziehbar ist. Durch einen Winkel von 0 Grad kann die Rotation sogar zurückgesetzt werden. Ein Nachteil der absoluten Rotation liegt darin, dass bei der Rotation von mehreren Elementen im Layer Mode bereits bestehende Rotationen durch die aktuelle Rotation überschrieben werden. In diesem Fall sollten die Elemente idealerweise im Box Mode einzeln rotiert werden. Hätte ich eine relative Rotation implementiert, würden vorherige Rotationen der Elemente bestehen bleiben. Die Generierung von neuen Layers nach einer Rotation von Drawings ist dann entbehrlich. 
\par
Die Verwendung von relativer Skalierung hat den Vorteil, dass im Layer Mode unabhängig große Objekte mit dem gleichen Faktor skaliert werden können. Zumindest bei Shaper und Imager können Elemente auch absolut in Breite und Höhe verkleinert oder vergrößert werden. Im Drawer existiert lediglich der relative Skalierungsfaktor, der Breite und Höhe individuell skaliert. Drawings sind eher ungenau und brauchen keine exakte Größe. Zusätzlich sollte ich zukünftig noch eine untere und obere Begrenzung für die Größe von Elementen nach relativer Skalierung programmieren. 
\par
Was sicherlich für den Benutzer ungewohnt ist, dass sich ein Element im Editor nicht mitbewegt, wenn dessen controlBox verschoben wird. Dieses Verhalten habe ich aus Performance-Gründen implementiert. Auf diese Weise lässt sich besser erfassen, was sich unter dem Element an der Zielposition befindet. Der Nachteil ist, dass Elemente weniger präzise platziert werden können. Im Box Mode können nur controlBoxes bearbeitet werden, die nicht von anderen controlBoxes verdeckt sind, wie z.B. im Falle von kopierten Elementen. Sofern die Move-Operation im Layer-Modus auf Elemente angewendet werden soll, die von anderen ControlBoxen verdeckt werden (Layer sind ausgewählt), ist dies ebenfalls nicht möglich. Diese Unannehmlichkeiten können behoben werden, indem die betreffenden Layers der Elemente in den Vordergrund verschoben werden. Der Benutzer der PDF Web App könnte zu der Ansicht gelangen, dass die controlBoxes etwas zu groß gestaltet sind. Ich habe sie deshalb in der Größe dimensioniert, weil ich der Meinung bin, dass die anklickbaren Boxen von grafischen Elementen in Programmen wie Illustrator derart klein sind, dass es für die Augen sehr anstrengend ist in Illustrator zu arbeiten. Die Mauscursorpositionen der PDF Web App zeigen keine Floats an. Außerdem sind noch keine Hilfsmittel wie Lineale oder Hilfslinien für exakte Layoutgestaltung implementiert. Die PDF Web App erhebt jedoch nicht den Anspruch, eine möglichst exakte Layoutanwendung zu sein, sondern soll die gängigsten, schnellsten Bearbeitungsmöglichkeiten von PDF-Dokumenten bieten. Ein weiterer ausbaubarer Aspekt ist die Tatsache, dass die controlBoxes mit einer fixe Konturfarbe gestaltet sind. Gesetzt dem Fall, dass der Shape, dem eine orange controlBox anhaftet, mit einer Füllfarbe im ähnlichen oder identischen Orange konfiguriert wurde, würde die controlBox visuell nicht mehr sichtbar sein. In solch einem Fall sollte die controlBox die Farbe wechseln oder generell zusätzlich eine andere Füllfarbe als die Kontur der controlBox aufweisen. Eine weitere Option bestünde darin, eine gestrichelte oder punktierte Kontur zu implementieren. Sofern ein Element zu weit an den Rand bewegt wird, verschwindet die controlBox im Editorhintergrund. Eine Verbesserung für die Zukunft wäre, dass controlBoxes immer sichtbar sind, selbst wenn das Element aus dem Seitenbereich heraus manövriert wird.
\par
Der Layer Mode als Programmdesignentscheidung kam mir in den Sinn, um mehrere Elemente zeitsparend zu modifizieren. Der Vorteil meiner Implementierung der Layers liegt darin, dass Elemente nicht partiell wie in Photoshop ausgewählt werden müssen, sondern mit einer Layer verknüpft sind. Somit kann ein Element allumfassend, unabhängig von der Lage in Vorder- bzw. Hintergrund oder anderen Elementen, editiert werden. Durch die Änderbarkeit der Reihenfolge von Layers können sogar Elemente auf andere Seiten gezogen werden. Eine benutzerdefinierte Gruppierung von Layers würde die Layers-Implementierung noch bereichern. Verschiebt man ein Element im Editor auf eine andere Seite, die maßgeblich kleiner als die souce Seite ist, verschwindet das Element im Hintergrund, da die Seitenkoordinaten zu sehr voneinander abweichen. Allerdings wird das Element wieder sichtbar, indem es auf eine Seite mit gleicher oder ähnlicher Größe gezogen wird. Eine Erweiterung der aktuellen Layers-Implementierung wäre, dass eine benutzerdefinierte Gruppe mehrerer Layers von einer Seite auf eine andere mit einer Operation verschoben werden könnte. Ebenso sinnvoll wäre ein Ein- und Ausblenden von mehreren selektierten Elementen mit einem Klick. Auf ähnliche Weise könnte der selection und deselection filter nach ein- und ausgeblendeten Elementen filtern. 
\par   
Da die PDF Web App nicht installiert werden muss, bleiben keine störenden Deinstallationsdateien übrig. Die Speichergröße der PDF Web App für die Offline-Verwendung liegt derzeit unter 3,5 MB. Viele PDF-Programme auf dem Markt belegen wesentlich mehr Speicherplatz im GB-Bereich. Trotz der leichten Bedienbarkeit der PDF Web App lässt sich das Verständnis von Box Mode, Layer Mode, selection und deselection filter durch die Tutorials auf Github vertiefen.

\section{Ideen für zukünftige Features}
Einige wichtige Funktionalitäten der PDF Web App plane ich für die nächsten Releases. Auf jeden Fall möchte ich die Spiegelung von Text, Drawings, Shapes und Images als wichtige affine Transformation implementieren. Die searchable PDF-Eigenschaft im Reader bzw. Editor, d.h. der Benutzer kann Text oder Bilder kopieren bzw. markieren, sollte nicht unangetastet bleiben. Die searchable-Eigenschaft kann aktuell nicht ausgeschöpft werden, da ich die Seiten des source Dokuments in CANVAS-HTML-Elemente rendere. Folglich werden alle Seiten gerastert. Dennoch sind nicht in der PDF Web App hinzugefügte Texte und Bilder in anderen PDF-Viewern weiterhin markierbar, selbst nachdem das PDF durch die PDF Web App modifiziert wurde. Da ich die Opacity von Text und Image durch die Änderung der Pixel der Canvas (editimg), die ein Element rendert, selbst implementiert habe, konnte ich Text- und Image-Elemente nicht direkt in das Output-PDF einbetten. Stattdessen müssen PNG-Bilder der Elemente als DataURLs in das Output-PDF integriert werden. Dadurch kann die searchable-Eigenschaft von Text und Image nicht ihre Wirkung entfalten.
\par
Ein fortschrittliches zukünftige Feature wäre die Editierung von bereits bestehenden Objekten im source Dokument. Im weiteren Verlauf des Projekts möchte ich unabdingbar Text, Geometrie und Zeichnungen durch Vektorpfade ersetzen. In der aktuellen Version sind alle Elemente pixelbasiert. Das Eraser-Werkzeug möchte ich auf Text, Shape und Image Elemente erweitern, damit sie freigestellt werden können. Falls der Browser oder Computer abstürzt und das in der PDF Web App geöffnete Dokument nicht gedownloaded wurde, sind die Änderungen verloren. Als Gegenmaßnahme könnte ich eine Export- und Importfunktionalität programmieren, die hinzugefügte Elemente als \gls{json}-Datei speichert. Beim Export würde diese Datei angelegt und auf der Festplatte gesichert. Ein Import-Button würde aus der gespeicherten \gls{json}-Datei die hinzugefügten Elemente wiederherstellen, damit diese weiter editiert werden können. Zusätzlich könnte ein automatischer Export der Elemente als Backup erstellt werden, der z.B. alle 10 Minuten die \gls{json}-Datei mit den neusten Änderungen aktualisiert.
\par
Eine häufig von mir benötigte Funktionalität ist die Kompression von PDF-Dokumenten. Ebenfalls gängig angewendet ist eine Undo- bzw. Redo-Funktion, d.h. Bearbeitungsschritte können rückgängig gemacht und rückgängig gemachte Schritte wieder ausgeführt werden. In der Praxis verwende ich eher selten Formulare, jedoch das Ausfüllen von Formularen und die Erstellung editierbarer Formulare ist mit Sicherheit ein beliebtes Feature. Die von mir verwendete PDF-LIB-Library bietet Unterstützung für Formulare. Im Verlauf des Projekts beabsichtige ich, die PDF Web App auf allen gängigen Browsern zu testen und so anzupassen, dass Cross-Browser-Kompatibilität vorhanden ist.