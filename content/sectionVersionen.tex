\section{PDF Dateiversionen}
Gestartet mit Version 1.0 war PDF lediglich ein proprietäres Dateiformat von Adobe. Die Freigabe von PDF als offenes und kostenlose Dateiformat führte letztendlich erst zu seiner weltweiten Verbreitung und Anerkennung. Erst im Jahr 2005 entwickelte sich Version 1.4 zu einem internationalen \gls{iso} Standard. Die letzte Dateiversion 2.0 von 2017 ist schon eine Weile her und es hat sich zeitlich nur das PDF/R-Dateiformat später entwickelt als PDF 2.0.

\subsection{PDF 1.0}
PDF 1.0 wurde 1992/1993 entwickelt und ist wurde nicht normiert. 1992 wurde die Spezifikation als Buch verkauft und 1993 das der Spezifikation entsprechende digitale Format entwickelt, welches ausschließlich den RGB Farbraum darstellen konnte. Medien, die einen anderen Farbraum besitzen wurden in RGB konvertiert. Der RGB Farbraum eignet sich lediglich für die Bildschirmdarstellung und beschreibt die für den Menschen 16,7 Mio. sichtbaren Farben mit Hilfe von additiver Farbmischung. In der Druckindustrie ist jedoch der CMYK Farbraum von Bedeutung. Folglich ist PDF 1.0 nicht für den Printbereich geeignet. Damals war Adoba Acrobat 1.0 das einzige Programm mit dem man diese Dateiversion bearbeiten konnte. \cite{proj-consult}

\subsection{PDF 1.1}
Genauso ist das 1994 kreierte PDF 1.1 keine Norm und implementiert weiterhin nur den RGB Farbraum, jedoch geräteunabhängig. Zusätzlich benötigte man ein Update von Adobe Acrobat auf Version 2.0. Erstmals sind in diesem Format das Einbetten von externen Links, mehrseitige Artikel und Threads, Passwortverschlüsselung und Notizen bzw. Anmerkungen erschienen. \cite{proj-consult}

\subsection{PDF 1.2}
Das 1996 erschienene PDF 1.2 wurde ebenfalls nicht standardisiert, jedoch ermöglichte es erstmals den druckbaren CMYK-Farbraum und Sonderfarben zu verwenden. Des weiteren wurden interaktive Formularfunktionen, Unicode, Multimedia Kompatibilität, Unterstützung der \gls{opi} 1.3 Spezifikationen und eine Druckrasterfunktion implementiert. \cite{proj-consult} In PDF 1.2 wurden erstmalig AcroForms vorgestellt.

\subsection{PDF 1.3}
1999 wurde PDF 1.3 mit fehlender Normierung auf den Markt gebracht und trug seinen Teil 2001 und 2002 bei zur Standardisierung des \gls{iso} PDF/X Standards. Es ist kompatibel mit PostScript 3 und bietet die Neuerungen der 2-Byte \gls{cid} Schrifttypen, \gls{opi} 2.0 Unterstützung, Farbraumerweiterung für Sonderfarben durch \gls{icc}-Profile und den DeviceN Farbraum, weiche Schatten und Farbübergänge (Smooth Shading), digitale Signaturen, RC4-Verschlüsselung (40 Bit in Acrobat 4 und 56 Bit in Acrobat 4.05) und JavaScript. \cite{proj-consult}


\subsection{PDF 1.5}
Im Jahr 2003 kam PDF 1.5 auf den Markt und hat sich nicht zur Norm entwickelt. In dieser Version wurden erstmals Ebenen implementiert. Des weiteren wurden gesteigerte Kompressionstechniken einschließlich Objekt-Streams und JPEG 2000-Kompression, sowie eine verbesserte XRef-Tabelle und XRef-Streams implementiert. 12 weitere Seitenübergänge für Präsentationen, verbesserte Unterstützung für Tagged PDF und die Adobe proprietäre Technologie \gls{xfa} wurden außerdem hinzugefügt. \cite{proj-consult} \gls{xfa}s Haupterweiterung zu \gls{xml} sind rechnergestützte, aktive Tags und sein Datenformat ist kompatibel mit anderen Systemen, Anwendungen und Technologiestandards. \cite{wiki-xfa}

\subsection{PDF 1.4}
Der erste PDF \gls{iso}-Standard \gls{iso} 16612-1:2005 wurde zeitlich nach Version 1.5 verabschiedet. In diesem Format sind Transparenzen, JavaScript 1.5, bessere Integration von Datenbanken, Titel, Textblockdefinition, JBIG2-Komprimierung und 128-Bit-RC4-Verschlüsselung erstmalig eingeführt worden. \cite{proj-consult}

\subsection{PDF 1.6}
In diesem \gls{iso} 15930-8:2008 Standard sind erstmals folgende Technologien in diesem Format eingeführt worden: NChannel, welches eine Erweiterung von NDevice mit Sonderfarben ist, JPEG 2000-Kompression, \gls{aes} Verschlüsselung, direkte Einbettung von OpenType-Schriften, Containerisierung, 3D-Daten (U3D) und \gls{xml} Formulare. \cite{proj-consult}

\subsection{PDF 1.7}
Veröffentlichung am 1. Juli 2008 ist PDF in Version 1.7 als \gls{iso} 32000-1:2008 als Offener Standard definiert worden. Es wurden komplexer 3D-Objekte, Kontrolle über 3D-Animationen und Einbettung von Standard-Druckeinstellungen wie Papierauswahl, Anzahl der Kopien und Skalierung hinzugefügt.
\cite{proj-consult}

\subsection{PDF 2.0}
\gls{xfa} ist in PDF 2.0 als \gls{iso} 32000-2:2017 vom \gls{iso}-Gremium als veraltet markiert worden. Mehr Einstellungsmöglichkeiten und Funktionen sind diesem Format beigefügt worden: Erweiterte Definitionen der Halbtöne der Rasterung, z.B. für Flex- oder Tiefdruck, konsistente Transparenz, erweiterte Tagged-PDF Funktion für Barrierefreiheit, Definition von Sonderfarben über Spektralfarben, alternierende Reihenfolge der zu  druckenden Farben, Steuerung der Schwarzpunktkompensation, \gls{aes}-256-Bit-Verschlüsselung und Einbettung von 3D-Messungen oder Querschnittsdaten.
\cite{proj-consult}