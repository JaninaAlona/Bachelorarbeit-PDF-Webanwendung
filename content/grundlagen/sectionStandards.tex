\section{PDF Dateiformate}

\subsection{PDF-X}
Das PDF-X Dateiformat (\gls{iso} 15930) dient des simpleren Datenaustausches in der Druckvorstufe. Es beschreibt Eigenschaften von Druckvorlagen und vereinfacht die Datenübermittlung von der Druckvorstufe bis zum finalen Druck.

\subsection{PDF-VT}
Das PDF-VT Dateiformat stellt ein spezielles Austauschformat im variablen Datendruck und Transaktionsdruck dar.

\subsection{PDF-A}
Das PDF-A Dateiformat wurde zur Langzeitarchivierung von digitalen Dokumenten entwickelt.
Langzeitarchivierung von PDF-Dateien (als PDF/A-1 in \gls{iso} 19005-1:2005) 

\subsection{PDF-E}
Das PDF-E Dateiformat wurde speziell für das Ingenieurwesen entworfen und kann interaktive 3D-Elemente darstellen.

\subsection{PDF-UA)}
Das PDF-UA Dateiformat dient der Erstellung barrierefreier Dokumente.

\subsection{Durchsuchbares PDF}
Das Durchsuchbare PDF kann mit Suchfunktionalitäten eines PDF Readers durchsucht werden.

\subsection{\gls{pades}}
\gls{pades} ergänzt den Funktionsumfang um Werkzeuge, um elektronische Signaturen anzupassen.

\subsection{PDF-H}
Das PDF-H Dateiformat soll im Gesundheitswesen Patientendaten erfassen, austauschen und archivieren.
