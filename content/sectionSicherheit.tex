\section{PDF Sicherheitsaspekte}
Etwa 40 \% der Unternehmen setzen PDFs für geschützte Inhalte ein. In den letzten 2 Jahren ist die Nutzung der elektronischen Signaturfunktion in PDFs um mehr als 150 \% gestiegen. \cite{formilo} Diese Statistik zeigt keine gute Entwicklung, da PDF-Dokumente nicht für vertrauliche Inhalte verwendet werden sollten. Obwohl PDF Verschlüsselung verwendet erlaubt das Dateiformat eine Vielzahl an Angriffen, um die Sicherheit von PDF Dokumenten zu untergraben. Ich werde im Folgenden die implementierten Sicherheitsmechanismen von PDF und die möglichen Angriffe beleuchten.
\par
In den Sicherheitseinstellungen eines PDF-Dokuments können Dokumentensicherheit und Zugriffsregeln justiert werden. PDF unterstützt Verschlüsselung und die Vergabe von Passwörtern. Eventuell kann beim Öffnen einer Datei ein Passwort gefordert werden oder das Kopieren von Teilinhalten, jeglichem Inhalt, Ausfüllen von Formularfeldern, Dokumentveränderungen (z.B. Struktur, Inhalt, Kommentare), das Einbetten und Editieren von Schriften oder das Ausdrucken kann vom Ersteller des Dokuments gesperrt worden sein. Das Schwärzen von Dokumenten, d.h. ein schwarzer Balken liegt auf dem Text, führt nicht dazu, dass der geschwärzte Text aus der Datei verschwindet. Er kann weiterhin ausgelesen werden, ebenso Metadaten. 
\cite{adobe-pdf-pades}

\subsection{Digitale Signatur}
Digitale Unterschriften sollen die Identität des Unterzeichners des Dokuments authentifizieren und dass der Inhalt nach der digitalen Unterschrift nicht geändert wurde. Der Verfasser benötigt für die digitale Signatur ein Signatur-Zertifikat. Das Zertifikat bescheinigt die Echtheit der Unterschrift und der Herkunft und wird von einem Zertifizierungsanbieter ausgestellt. Zusätzlich können Zertifikate ablaufen oder entzogen werden und müssen gültig sein. Dabei sollte ein vertrauenswürdiger Zertifizierungsanbieter gewählt werden. Jede Unterschrift kann mit einem Zeitstempel versehen werden. Ein vertrauenswürdiger \gls{zsa} belegt den Zeitpunkt, wann diese Unterschrift geleistet wurde. \cite{softx} \\
Eine digitale Signatur ist eine spezielle Art von elektronischer Signatur, die kryptographische Techniken implementiert. Im Gegensatz dazu kann eine elektronische Signatur verschiedene Formen annehmen, beispielsweise eine gescannte handschriftliche Unterschrift, ein getippter Name, eine biometrische Signatur oder eine digitale Signatur. Elektronische Signaturen sind der Oberbegriff und digitale Signaturen sind eine Teilmenge davon. Elektronische Signaturen bieten variierende Sicherheitsgrade. \cite{adobe-pdf-pades} \\
Digitale Signaturen verwenden asymmetrische Kryptographie. Anfangs wird ein Algorithmus zur Schlüsselgenerierung wird gestartet. Als Ausgabe werden ein Private Key und Public Key erzeugt. Ein Signierungsalgorithmus erstellt den einzigartigen Hash-Wert des PDF-Dokuments und verschlüsselt ihn mit dem geheimen Private Key, was die digitale Signatur produziert. Zuletzt überprüft ein Signaturvalidierungsalgorithmus die Authentizität des PDF-Dokuments, indem er erneut einen Hash des Dokuments berechnet. Die Signatur wird mit dem Public Key entschlüsselt und gibt einen Hash aus. Falls der berechnete Hash im Signaturvalidierungsalgorithmus mit dem entschlüsselten Hash übereinstimmt, ist die Authentizität des PDF-Dokuments gesichert. \cite{signature} \\
Eine PDF-Datei ermöglicht mehrere digitale Unterschriften, jedoch muss jede neue Unterschrift in einem inkrementellen Update geleistet werden. Jede Unterschrift muss mit einem Unterschriftsfeld im Dokument verbunden sein. Optional kann das Unterschriftsfeld mit einem Widget gekoppelt sein. Dann wird die Unterschrift graphisch dargestellte. Unterschriften ohne Widgets sind versteckte Unterschriften. \cite{softx} 

