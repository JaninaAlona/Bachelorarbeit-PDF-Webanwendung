\section{\gls{pdf} Dateiformate}

\subsection{\gls{pdf}-X}
Das \gls{pdf}-X Dateiformat (\gls{iso} 15930) dient des simpleren Datenaustausches in der Druckvorstufe. Es beschreibt Eigenschaften von Druckvorlagen und vereinfacht die Datenübermittlung von der Druckvorstufe bis zum finalen Druck.

\subsection{\gls{pdf}-VT}
Das \gls{pdf}-VT Dateiformat stellt ein spezielles Austauschformat im variablen Datendruck und Transaktionsdruck dar.

\subsection{\gls{pdf}-A}
Das \gls{pdf}-A Dateiformat wurde zur Langzeitarchivierung von digitalen Dokumenten entwickelt.
Langzeitarchivierung von \gls{pdf}-Dateien (als \gls{pdf}/A-1 in \gls{iso} 19005-1:2005) 

\subsection{\gls{pdf}-E}
Das \gls{pdf}-E Dateiformat wurde speziell für das Ingenieurwesen entworfen und kann interaktive 3D-Elemente darstellen.

\subsection{\gls{pdf}-UA)}
Das \gls{pdf}-UA Dateiformat dient der Erstellung barrierefreier Dokumente.

\subsection{Durchsuchbares \gls{pdf}}
Das Durchsuchbare \gls{pdf} kann mit Suchfunktionalitäten eines PDF Readers durchsucht werden.

\subsection{\gls{pades}}
\gls{pades} ergänzt den Funktionsumfang um Werkzeuge, um elektronische Signaturen anzupassen.

\subsection{\gls{pdf}-H}
Das \gls{pdf}-H Dateiformat soll im Gesundheitswesen Patientendaten erfassen, austauschen und archivieren.
