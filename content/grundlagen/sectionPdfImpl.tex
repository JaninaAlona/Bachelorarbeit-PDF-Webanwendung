\section{\gls{pdf} Implementierung}
\gls{pdf} ist eine vektorbasierte \gls{pdl} (Seitenbeschreibungssprache) und basiert auf dem PostScript-Format. Eine \gls{pdl} beschreibt den Seitenaufbau, wie die Seite in einem Ausgabeprogramm bzw. Ausgabegerät, z.B. einem Drucker, aussehen soll. \gls{pdl}s können Seiten mit Vektoren beschreiben. Das Ausgabeformat ist normalerweise nicht zur weiteren Bearbeitung vorgesehen. An den Drucker wird durch die \gls{pdl} ein Datenstrom der zu druckenden Aufgabe erzeugt und an den Drucker gesendet. Der \gls{rip} eines Druckers wandelt die Bildschirmausgabe in die gerasterte Druckerausgabe um. Viele APIs der Hardwareabstraktionsschicht im Computer wie \gls{gdi} oder OpenGL können in \gls{pdl} ausgeben. Speichert ein Satzprogramm den Seitenbeschreibungscode eines Dokuments in einer Datei, müssen Drucker die \gls{pdl} nicht selbst verarbeiten. Im Common Unix Printing System, der Standard-Druckersteuerung von Linux hat der PostScript und der \gls{pdf}-Interpreter ghostscript die Aufgabe eines \gls{rip}, d.h. er ist für die Umwandlung in die gerasterte Druckausgabe auf dem Drucker zuständig. Zudem stellen \gls{pdl}s eine Schnittstelle zum Quellcode eines Dokuments bzw. zu Programmen, die Quellcode verwalten oder das Dokument formatieren können, dar. Die \gls{pdl} \gls{pdf} erweitert die Funktionalität der Vorschau am Bildschirm um anklickbare Links (Hypertextfunktionalität), die die Navigation im Dokument erleichtern oder um URLs, die sich automatisch im Browser öffnen. [5]
\par
\subsection{PostScript}
Die PostScript \gls{pdl} wurde in den 1980er Jahren von Adobe erfunden. [6] Hinzu wurden weiter PostScript Technologien entwickelt, die aus der stackorientierten, Turning-vollständigen, interpretierten Programmiersprache PostScript [7], Grafik-, Textformatierungsanwendungen, Treibern und Abbildungssystemen bestehen. PostScript hat sich als Industriestandard etabliert. Die letzte Version ist PostScript 3 von 1997. Seine primäre Anwendung gemäß des Adobe imaging models findet sich in der Beschreibung von Textdarstellung, graphische Formen und Bildern auf gedruckten oder auf dem Bildschirm angezeigten Seiten. Dabei ist die Beschreibung des Dokuments geräteunabhängig. PostScript unterstützt unter anderem beliebige geometrische Formen, Zeichenoperationen in Graustufen, RGB, CMYK und CIE (Yxy-Farbraum) und  vorinstallierte oder benutzerdefinierte Fonts und Digitalbilder jeglicher Auflösung je nach Farbmodell und ein allgemeines Koordinatensystem.
Dabei werden die Textzeichen eines Fonts, gemäß des Adobe imaging models, als graphische Formen betrachtet auf denen Grafikoperationen möglich sind. Das Koordinatensystem unterstützt alle linearen Transformationen, die auf alle Seitenelemente angewandt werden können. Die Seitenbeschreibung in PostScript kann auf jedem Gerät, was einen PostScript Interpreter implementiert, gerendert werden. In diesem Prozess wird die high-level PostScript-Beschreibung in low-level Rasterdatenformate für das jeweilige Gerät übersetzt. PostScript Programme können erstellt, übertragen und als ASCII Quellcode interpretiert werden. [6]
\par
\subsection{Adobe imaging model}
\gls{pdf} und die PostScript Programmiersprache haben das Adobe imaging model als Gemeinsamkeit. Es kann nahtlos zwischen \gls{pdf} und PostScript konvertiert werden und beide erzielen das gleiche Ausgabeergebnis beim Druck. Dennoch fehlt \gls{pdf} das general-purpose Framework der PostScript Programmiersprache. Stattdessen stellt ein \gls{pdf} Dokument eine statische Datenstruktur optimiert für den random access dar und enthält zusätzlich Seitennavigationsinformationen für interaktives Lesen. Das high-level imaging model beschreibt die Elemente, die auf der Seite dargestellt werden, also Text, Geometrie oder Bilder, als abstrakte graphische Elemente, anstatt als Pixeldefinitionen. Dadurch wird das imaging model zu einem geräteunabhängigem Modell und kann hochwertige Ausgaben auf vielen verschiedenen Druckern und Bildschirmen liefern. Die \gls{pdl} beschreibt dieses imaging model. Eine Anwendung generiert zuerst die geräteunabhängige Beschreibung des gewünschten Ausgabegeräts in der \gls{pdl}. Daraufhin interpretiert eine Firmware oder Software eines spezifischen Ausgabegeräts für Rasterausgaben die Beschreibung und rendert sie im Ausgabegerät. Hierbei hat die \gls{pdl} die Rolle eines Austauschstandards für die Übertragung und Speicherung von druckbarem oder auf Displays darstellbaren Dokumenten. [6]
