\subsection{Performance Tests}
Zu Beginn habe ich die Renderdauer mit variierend großen PDF-Dateien im Bezug auf Speichergröße und Seitenanzahl gemessen. Anschließend habe ich die Ausführungszeit jedes Moduls überprüft. Im Editor ermittelte ich die Downloaddauer von hinzugefügten Elementen.

\subsubsection{Renderdauer}
Die renderPage-Funktion wurde mit verschiedenen PDF-Testdateien ausgeführt. Die Tabelle \ref{table:render-dur} bezieht sich ausschließlich auf Tests im Reader. 

\begin{table}[!htbp]
	\centering
	\begin{tabular}{|p{4cm}|p{3cm}|p{3cm}|p{3cm}|}
		\hline
		\textbf{Datei}													& \textbf{Seitenanzahl} 	& \textbf{Dateigröße} 	& \textbf{Execution in ms}	\\ 
		\hline
		\parbox[t]{4cm}{vivaoptik\_Gutschein\_\\50euro}					& 1 						& 33,22 KB  			& 28						\\ 
		02-Sensoren														& 9 						& 1,17 MB  				& 190						\\ 		
		the-metamorphosis-franz-kafka 									& 88 						& 298,86 KB  			& 802						\\   
		09. Beyond Good and Evil author Friedrich Nietzsche				& 301 						& 795,91 KB  			& 1924 						\\ 
		02. The Critique of Pure Reason author Immanuel Kant			& 1277 						& 1,78 MB  				& 9419						\\ 
		DevOps with Kubernetes											& 520 						& 13,7 MB  				& 9985						\\  
		01. War and Peace author Leo Tolstoy 							& 2882 						& 7,21 MB  				& 30353						\\ 
		blank\_pdf-5000-dina6											& 5000						& 69,55 KB  			& 37042 					\\ 
		Animal Crossing Amiibo Card Art									& 50 						& 167,05 MB  			& 50767						\\   
		UNIX and Linux System Administration Handbook - Fifth Edition	& 1809						& 71,94 MB  			& 51878 					\\ 
		l11manual\_en 													& 850 						& 91,8 MB  				& 100658					\\
		\hline
	\end{tabular}
	\caption{Execution Times der renderPage Funktion für verschiedene PDF-Dateien}
	\label{table:render-dur}
\end{table}

\subsubsection{Modulleistung}
In diesem Abschnitt messe ich die Leistung der übrigen Module. Konkret beobachte ich die Ausführungszeit des jeweiligen Moduls, das PDF zu erstellen und zu Downloaden. Die Testergebnisse eines Moduls fließen in eine Tabelle ein. Diese Tabelle \ref{table:creator-dur} zeigt Extremfälle der PDF-Erstellung. Im Creator habe ich ein PDF erstellt, welches die Maximalgröße von 10000 mm aufweist. Diese Größe kann nicht mehr vom Adobe Reader angezeigt werden. Die Seiten dieses PDFs werden in diesem Programm auf ca. 5079 mm verkleinert. 

\begin{table}[!htbp]
	\centering
	\begin{tabular}{|p{3cm}|p{3cm}|p{2cm}|p{2cm}|p{2cm}|p{2cm}|}
		\hline
		\textbf{Outputdatei}						& \textbf{Seitengröße}	& \textbf{Seiten-anzahl}	& \textbf{Download-größe}	& \textbf{Execution in ms} 	\\ 
		\hline
		blank\_pdf-5000-a6							& DIN A6 				& 5000 						& 36,33 KB 					& 2131  					\\
		blank\_pdf-5000								& DIN A4 				& 5000 						& 36,96 KB 					& 2156   					\\
		blank\_pdf-5000-a1							& DIN A1 				& 5000 						& 31,28 KB 					& 2136   					\\
		blank\_pdf-1p-10000s						& 10000 x 10000			& 1 						& 764 Bytes					& 11						\\	
		blank\_pdf-500p-10000s						& 10000 x 10000			& 500 						& 5,05 KB					& 131						\\	
		\hline
	\end{tabular}
	\caption{Execution Times des Creators}
	\label{table:creator-dur}
\end{table}

Als Nächstes untersuche ich den Splitter. Dessen Testergebnisse sind in Tabelle \ref{table:splitter-dur} gezeigt. Eine Liste an Seitenzahlen symbolisiert die Split-Methode list of pages.

\begin{table}[!htbp]
	\centering
	\begin{tabular}{|p{2.5cm}|p{3cm}|p{3cm}|p{2cm}|p{2cm}|p{2cm}|}
		\hline
		\textbf{Inputdatei}					& \textbf{Split-Methode}					& \textbf{Outputdatei}								& \textbf{Download-größe}	& \textbf{Execution in ms} 	\\ 
		\hline
		Animal Crossing Amiibo Card Art		& 2,5,10,30,49 								& Animal Crossing Amiibo Card Art\_split-pagelist 	& 150,39 MB					& 3976  						\\
		the-metamorphosis-franz-kafka		& 4, 5,5, 5, 6 ,80, 30,30, 29, 71, 21 		& the-metamorphosis-franz-kafka\_split 				& 309,23 KB				
											& 254  			\\
		DevOps with Kubernetes				& every page								& DevOps with Kubernetes\_split 					& 67,2 MB					& 4146  						\\
		02-Sensoren							& odd pages 								& 02-Sensoren\_split-odd 							& 1,45 MB					& 215  							\\
		02-Sensoren							& even pages								& 02-Sensoren\_split-even 							& 1,46 MB					& 236 							\\
		\hline
	\end{tabular}
	\caption{Execution Times des Splitters}
	\label{table:splitter-dur}
\end{table}

In den Testresultaten des Mergers in Tabelle \ref{table:merger-dur} liste ich die Inputdateien in der Merge-Reihenfolge auf.

\begin{table}[!htbp]
	\centering
	\begin{tabular}{|p{4cm}|p{1.7cm}|p{2.3cm}|p{2cm}|p{2cm}|p{2cm}|}
		\hline
		\textbf{Inputdateien}	& \textbf{Seiten-anzahl Output}		& \textbf{Outputdatei}		& \textbf{Download-größe}	& \textbf{Execution in ms} 	\\ 
		\hline
		\parbox[t]{4cm}{vivaoptik\_Gutschein\_\\50euro,} 02-Sensoren, the-metamorphosis-franz-kafka	& 98 & merged\_pdf1 & 1,72 MB		& 834  					\\
		
		DevOps with Kubernetes, \parbox[t]{4cm}{vivaoptik\_Gutschein\_\\50euro,} DevOps with Kubernetes, 02-Sensoren, \parbox[t]{4cm}{vivaoptik\_Gutschein\_\\50euro,} \parbox[t]{4cm}{vivaoptik\_Gutschein\_\\50euro}		& 1052 		& merged\_pdf2		& 134,66 MB		& 18441   						\\
		
		DevOps with Kubernetes, \parbox[t]{4cm}{vivaoptik\_Gutschein\_\\50euro,} DevOps with Kubernetes, 02-Sensoren, \parbox[t]{4cm}{vivaoptik\_Gutschein\_\\50euro,} \parbox[t]{4cm}{vivaoptik\_Gutschein\_\\50euro,} 02-Sensoren		& 1061 		& merged\_pdf3				& 136,1 MB					& 18983   						\\
		\hline
	\end{tabular}
	\caption{Execution Times des Mergers}
	\label{table:merger-dur}
\end{table}

Als abschließenden Schritt messe ich die Leistung der Downloadfunktion des Editors mit der zweiseitigen PDF-Datei Animal-Crossing-Amiibo-s1-2. Ihre Performance ist in der Tabelle \ref{table:editor-dur} aufgelistet. Ich habe verschiedene Elemente dem PDF hinzugefügt. Zu Darstellungszwecken habe ich die Elemente mit t für Text, d für Drawing, s für Shape und i für Image abgekürzt. Alle Elemente sind Standardelemente. Jeweils 10 Elemente aller verwendeten Typen pro Testdurchlauf wurden auf der ersten Seite und die übrigen 10 Elemente auf der anderen Seite wahllos platziert. Nach jedem Downloadvorgang habe ich das PDF erneut geöffnet. Die Downloadfunktion verwendet ZIP-Kompression und ermöglicht sogar Downloads von großen PDF-Dateien.

\begin{table}[!htbp]
	\centering
	\begin{tabular}{|p{2cm}|p{4cm}|p{2cm}|p{2cm}|p{2cm}|}
		\hline
		\textbf{Anzahl Elemente}			& \textbf{Outputdatei}							& \textbf{Download-größe}	& \textbf{Execution in ms} 	\\ 
		\hline
		keine Elemente 						& Animal-Crossing-Amiibo-s1-2-no-elems			& 8,57 MB					& 431      					\\
		20 t 								& Animal-Crossing-Amiibo-s1-2-20t				& 8,69 MB					& 19278       				\\
		20 d 								& Animal-Crossing-Amiibo-s1-2-20d				& 8,65 MB					& 19486    					\\
		20 s 								& Animal-Crossing-Amiibo-s1-2-20s				& 8,65 MB					& 19455       				\\
		20 i 								& Animal-Crossing-Amiibo-s1-2-20i				& 10,26 MB					& 19843      				\\
		20 t, 20 s 							& Animal-Crossing-Amiibo-s1-2-20t-20s			& 8,77 MB					& 35296      				\\
		20 t, 20 s, 20 i 					& Animal-Crossing-Amiibo-s1-2-20t-20s-20i		& 10,53 MB					& 50735      				\\
		20 t, 20 d, 20 s, 20 i 				& Animal-Crossing-Amiibo-s1-2-20t-20d-20s-20i	& 10,57 MB					& 68632    					\\
		\hline
	\end{tabular}
	\caption{Execution Times des Editors}
	\label{table:editor-dur}
\end{table}