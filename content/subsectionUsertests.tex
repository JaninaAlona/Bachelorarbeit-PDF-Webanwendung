\subsection{GUI Tests}
Ich fange mit den GUI Tests an, die sich hauptsächlich auf den Editor beziehen. Die anderen Module teste ich in den Performance Tests. Im Vorgang werde ich Screenshots von Operationen auf Elementen zur Verfügung stellen, die das Vorher (Eingabe) und Nachher (Ausgabe) zeigen.

\subsubsection{Input Control Tests}
In diesen Tests möchte ich die Funktionalität der implementierten Input Control bei input HTML-Elementen vom Typ Text auf die Probe stellen. Dies hat zum Ziel, dass meine Eingaben die minimalen und maximalen Werte für die input fields austesten und von der Anwendung Usereingaben automatisch korrigiert werden. Die Tests habe ich mit der Datei merged-sensoren-wissschreib.pdf mit 11 Seiten durchgeführt. 

\begin{table}[!htbp]
	\centering
	\begin{tabular}{|p{4cm}|p{3cm}|p{3cm}|p{3cm}|}
		\hline
		\textbf{input field}													& \textbf{Input} 	& \textbf{Output-Korrektur}		\\ 
		\hline
		Aktuelle Seitenzahl												& 0 						& 1		\\ 
		Aktuelle Seitenzahl												& 10 						& 9 					\\ 
		Aktuelle Seitenzahl												& 1.2 						& 1 					\\ 
		Zoom 													& 0						& 1\%  							\\
		Zoom 									& 801 \% 						& 800\% 					\\ 
		Zoom 									& 20 .1 						& 20 \% 					\\ 
		\hline
	\end{tabular}
	\caption{input field Tests im Reader}
	\label{table:reader-input}
\end{table}	

\begin{table}[!htbp]
	\centering
	\begin{tabular}{|p{4cm}|p{3cm}|p{3cm}|p{3cm}|}
		\hline
		\textbf{input field}													& \textbf{Input} 	& \textbf{Output-Korrektur}		\\ 
		\hline
		Number of Pages 							& 0 						& 1 					\\ 
		Number of Pages									& 5001 						& 5000  					\\  
		Width und Height											& 9 						& 10						\\  
		Width und Height			& 10001 						& 10000 						\\  
		\hline
	\end{tabular}
	\caption{input field Tests im Creator}
	\label{table:creator-input}
\end{table}

\begin{table}[!htbp]
	\centering
	\begin{tabular}{|p{4cm}|p{3cm}|p{3cm}|p{3cm}|}
		\hline
		\textbf{input field}													& \textbf{Input} 	& \textbf{Output-Korrektur}		\\ 
		\hline
		list of pages	& 1,2,11						& Eingabe gelöscht  						\\ 
		list of pages	& 1,2,2,5						& 1,2,5 						\\ 
		list of pages					& 1 0,2						& 2,10		\\ 
		\hline
	\end{tabular}
	\caption{input field Tests im Splitter}
	\label{table:splitter-input}
\end{table}

\begin{table}[!htbp]
	\centering
	\begin{tabular}{|p{4cm}|p{3cm}|p{3cm}|p{3cm}|}
		\hline
		\textbf{input field}													& \textbf{Input} 	& \textbf{Output-Korrektur}		\\ 
		\hline
		Line Height 													& 0 						& 1  							\\
		Line Height												& 301 						& 300 					\\ 
		Font Size 													& 2						& 3  							\\
		Font Size									& 501						& 500 					\\ 
		\hline
	\end{tabular}
	\caption{input field Tests im Texteditor}
	\label{table:text-input}
\end{table}

\begin{table}[!htbp]
	\centering
	\begin{tabular}{|p{4cm}|p{3cm}|p{3cm}|p{3cm}|}
		\hline
		\textbf{input field}													& \textbf{Input} 	& \textbf{Output-Korrektur}		\\ 
		\hline
		Rotate													& 850 						& 91,8 MB  							\\
		Rotate												& 10 						& 9 					\\ 
		Font Size 													& 0						& 1\%  							\\
		Font Size									& 801 \% 						& 800\% 					\\ 
		\hline
	\end{tabular}
	\caption{input field Tests im Editor}
	\label{table:text-input}
\end{table}

