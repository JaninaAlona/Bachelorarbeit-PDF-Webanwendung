\section{PDF zu Word Programme und Onlinedienste}
In PDF ist keine automatische Anpassung des Seiteninhalt-Layouts, wie z.B. in Microsoft Word, möglich. Daher kann ein PDF-Dokument nicht sinnvoll in das Word-Format umgewandelt werden ohne möglicherweise das ursprüngliche PDF-Layout zu beeinflussen und zu ändern, sowie die maximalen Bearbeitungsmöglichkeiten von Word ausschöpfen zu können.

\subsection{Adobe Acrobat Pro DC}
Produktseite: \url{https://www.adobe.com/de/acrobat/acrobat-pro.html}
Acrobat ist ein professionelles Werkzeug, um PDF-Dateien zu erstellen und bereits bestehende Inhalte zu bearbeiten. Text kann hinzugefügt, geändert, formatiert, gelöscht oder markiert werden. Ebenfalls kann der Text von Formularen bearbeitet werden. Bilder können gespiegelt, gedreht, zugeschnitten, ersetzt, ausgerichtet oder angeordnet werden. Beim Objekte Anordnen-Werkzeug können mehrere Objekte vor oder hinter anderen Objekten angeordnet werden. Das Arbeiten mit Ebenen ist möglich, jedoch eingeschränkter als bei Photoshop. PDF-Seiten können kopiert, ersetzt, gedreht, verschoben, gelöscht, extrahiert oder neu nummeriert werden. Eine PDF-Datei kann in mehrere Dokumente aufgeteilt werden und Kommentare können hinzugefügt werden. Webseiten können mit der Acrobat-Browsererweiterung in PDF konvertiert werden \cite{adobe-acrobat-um}. Eine PDF-Datei kann in eine PDF/A-Datei inklusive seiner Varianten, PDF/X, PDF/UA, PDF/VT oder PDF/E konvertiert werden. Außerdem kann die Kompatibilität mit diesen Formaten überprüft werden in Preflight-Profilen \cite{adobe-pdf-a}. Die Barrierefreiheit kann automatisch validiert werden oder ein neues Dokument kann direkt barrierefrei erstellt werden. Adobe Acrobat Pro kann andere Dokumentenformate wie HTML, DOC, DOCX, TXT und RTF in PDF konvertieren, PDF in andere Dateiformate wie Microsoft Word exportieren oder Dokumente unterschreiben \cite{adobe-formate}. PDF-Dokumente mit digitalen Signaturen unterzeichnet werden und die Signaturen können validiert werden \cite{adobe-acrobat}. Für Vertraulichkeit können PDF-Dokumente mit Passwörtern und Zertifikaten versehen werden. Interaktive Objekte, Audio und Video kann eingebettet werden. In der Pro-Variante gibt es Druckproduktionswerkzeuge u.a. für Druckermarken, Transparenz-Reduzierung, Farbkonvertierung oder Druckermanagement. Die Preflight-Option ist ebenfalls nur in Acrobat Pro verfügbar \cite{adobe-acrobat-um}. Mit dem Werkzeug Scan \& \gls{ocr} in Acrobat Pro kann man Pixelbilder als PDF und gescannte PDF-Dokumente in ein durchsuchbares PDF umwandeln \cite{adobe-search}. Das Programm gibt es als kostenlosen Acrobat Reader, Acrobat Standard für 15,46 Euro pro Monat und Acrobat Pro für 23,79 Euro pro Monat. Hierbei handelt es sich um ein Jahres-Abo mit monatlicher Zahlung. Wählt man das Monats-Abo, so kostet die Standard-Version 27,36 Euro im Monat und die Pro-Variante 35,69 Euro im Monat. Es gibt auch die Möglichkeit das Jahres-Abo mit Vorauszahlung zu erwerben \cite{adobe-acrobat}. Mit den Adobe Acrobat Onlinetools kann man über den Browser verschiedene Dateitypen in PDF umwandeln, unter anderem PDF in JPEG oder andere Bildformate, PDF Dateien bearbeiten und Kompression anwenden. Die Onlinetools können ebenfalls PDF in Word umwandeln. \cite{adobe-search} Der Adobe Acrobat PDF-Converter der Onlinetools kann DOCX, DOC, XLSX, XLS, PPTX, PPT, TXT, RTF, JPEG, PNG, TIFF, BMP, sowie Adobe eigene AI-, INDD- und PSD-Dateien in PDF konvertieren. \cite{adobe-formate} Die kostenlose Version des PDF-Converters kann nur begrenzt oft genutzt werden.

\subsection{UPDF}


\subsection{WPS}


\subsection{LightPDF}