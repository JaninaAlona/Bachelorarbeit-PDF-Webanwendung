\section{Testdurchführung}
Für alle Tests habe ich Input- und Output-Dateien auf dem abgegebenen USB-Stick verwendet.

\subsection{Funktionale User Tests}

\subsection{Performance Tests}
Hat man eine PDF-Datei im Reader oder Editor geöffnet und löscht diese, so gibt es keine Fehlermeldung, wie in vielen lokalen Programmen.

\subsubsection{Renderdauer}
Die renderPage Funktion wurde mit verschiedenen PDF-Testdateien ausgeführt. Die Tabelle bezieht sich ausschließlich auf Tests im Reader. 

\begin{table}[!htbp]
	\centering
	\begin{tabular}{|p{4cm}|p{3cm}|p{3cm}|p{3cm}|}
		\hline
		\textbf{Datei}													& \textbf{Seitenanzahl} 	& \textbf{Dateigröße} 	& \textbf{Execution in ms}	\\ 
		\hline
		\parbox[t]{4cm}{vivaoptik\_Gutschein\_\\50euro}					& 1 						& 33,22 KB  			& 27						\\ 
		02-Sensoren														& 9 						& 1,17 MB  				& 182						\\ 
		l11manual\_en 													& 850 						& 91,8 MB  				& 99914						\\
		the-metamorphosis-franz-kafka 									& 88 						& 298,86 KB  			& 714						\\ 
		01. War and Peace author Leo Tolstoy 							& 2882 						& 7,21 MB  				& 29115						\\ 
		Animal Crossing Amiibo Card Art									& 50 						& 167,05 MB  			& 53545						\\  
		DevOps with Kubernetes											& 520 						& 13,7 MB  				& 9883						\\  
		02. The Critique of Pure Reason author Immanuel Kant			& 1277 						& 1,78 MB  				& 9428						\\  
		UNIX and Linux System Administration Handbook - Fifth Edition	& 1809						& 71,94 MB  			& 47366						\\ 
		\hline
	\end{tabular}
	\caption{Execution Times der renderPage Funktion für verschiedene PDF-Dateien}
	\label{table:render-dur}
\end{table}

\subsubsection{Modulleistung}
In diesem Abschnitt messe ich die Leistung der übrigen Module der PDF Web App. Konkret messe ich die Ausführungszeit des jeweiligen Moduls das PDF zu erstellen und zu Downloaden. Ich fange mit dem Creator an.

\begin{table}[!htbp]
	\centering
	\begin{tabular}{|p{3cm}|p{2cm}|p{2cm}|p{2cm}|p{2cm}|p{2cm}|}
		\hline
		\textbf{Output-datei}					& \textbf{Seitengröße}	& \textbf{Seiten-anzahl}	& \textbf{Download-größe}	& \textbf{Execution in ms} 	\\ 
		\hline
		blank\_pdf5000							& DIN A4 				& 5000 						& 36,96 KB 					& 2180  					\\
		\parbox[t]{4cm}{blank\_pdf500\\p10000s}	& 10000 x 10000			& 500 						& 4,92 KB					& 170 						\\
		\hline
	\end{tabular}
	\caption{Execution Times des Creators}
	\label{table:creator-dur}
\end{table}

\subsection{Testbewertung}