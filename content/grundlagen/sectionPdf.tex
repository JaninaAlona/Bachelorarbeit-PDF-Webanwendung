\section{\gls{pdf} Einführung}
Die Popularität von \gls{pdf} (Portable Document Format) Dateien ist seit 2008 rasant angestiegen in der globalen Informationsübertragung. Täglich werden weltweit 2,5 Milliarden \gls{pdf}-Dokumente erzeugt. Seine Beliebtheit verdankt \gls{pdf} vor allem an der plattformübergreifenden Kompatibilität (Desktop-Computer, Tablets und Smartphones), denn \gls{pdf} Dokumente ist auf mehr als 1,5 Milliarden Geräten ohne zusätzliche Software lesbar. Über 80\% der geschäftlichen Dokumente werden als \gls{pdf} Datei weitergegeben. [1] 90 \% der Büroangestellten wollen auf das \gls{pdf} Dateiformat nicht mehr verzichten. Drei Viertel aller archivierten Dokumente sind \gls{pdf} Dokumente. [3] Das \gls{pdf} Dateiformat steht für Plattformunabhängigkeit, Konsistenz in Formatierung und Layout und soll ein möglichst originalgetreues Druckergebnis liefern.
\par
\gls{pdf} wurde 1993 von Adobe veröffentlicht und ging aus dem 1991 von Adobe-Mitbegründer John Warnock gestarteten „Project Camelot" hervor. Ziel dieses Projektes war, ein Dateiformat für elektronische Dokumente zu kreieren, sodass diese Anwendungsprogramm, Betriebssystem und Hardware unabhängig originalgetreu wiedergegeben werden können. Anfangs war der Adobe Reader kostenpflichtig und \gls{pdf} war für einen langen Zeitraum ein proprietäres Dateiformat, welches offengelegt im \gls{pdf} Reference Manual von Adobe dokumentiert ist. Die \gls{iso} übernahm \gls{pdf} 2007 in den Standardisierungsprozess und seit der Veröffentlichung von \gls{pdf} Version 1.7 am 1. Juli 2008 gilt \gls{pdf} als Offener Standard. [2] Der Begriff Offener Standard bezeichnet einen Standard, der für alle Teilhaber am Markt besonders leicht zugänglich, weiterentwickelbar und einsetzbar ist. Das bedeutet, dass der Standard von einer gemeinnützigen Organisation eingeführt, veröffentlicht, weiterentwickelt und gleichmäßige Einflussnahme aller interessierten Parteien ermöglicht. [4]
