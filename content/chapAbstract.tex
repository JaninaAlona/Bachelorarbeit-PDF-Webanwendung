\chapter*{Bachelorarbeit}
\label{chap:zusammenfassung}
%
\textbf{Titel:} Entwicklung einer webbasierter Applikation zur Bearbeitung von PDF-Dateien

\textbf{Gutachter:}
\par
- Prof. Dr. Chunrong Yuan
\par
- Prof. Dr. René Wörzberger

\textbf{Zusammenfassung:} Für meine Bachelorarbeit habe ich eine Open-Source-Webseite zur Bearbeitung von PDF-Dateien im Firefox-Browser programmiert. Seit Adobe den PDF-Standard entwickelt hat, sind zahlreiche kostenpflichtige PDF-Anwendungen auf dem Markt erschienen. In meiner Arbeit erläutere ich zunächst das Hintergrundwissen zum PDF-Dateiformat und zeige anschließend die damit verbundenen Sicherheitsprobleme auf. Ich beleuchte den aktuellen Stand der Technik und analysiere sieben PDF-Programme auf dem Markt. Später erkläre ich die Implementierung meiner PDF Web App und führe GUI- und Performance-Tests durch. Die JavaScript-Bibliotheken pdf.js und PDF-LIB bilden das tragende Fundament meiner Anwendung. Sie vereinen alle Funktionen, die für gängige PDF-Bearbeitungen benötigt werden. PDFs können gelesen, gesplittet, gemergt und erstellt werden. Außerdem können sie mit Texten, Bildern, Geometrie und Zeichnungen versehen werden. Abschließend werden Anregungen für Erweiterungen und zukünftige Features diskutiert.

\textbf{Stichtwörter:} PDF-Editor, Adobe, JavaScript, Vue JS 3, Splitten, Mergen, pdf.js, PDF-LIB

\textbf{Datum:} 04. März 2024


\newpage
\chapter*{Bachelor Thesis}
\label{chap:abstract}
%
\textbf{Title:} Development of a web-based application for editing PDF files

\textbf{Reviewers:}
\par
- Prof. Dr. Chunrong Yuan
\par
- Prof. Dr. René Wörzberger

\textbf{Abstract:} As part of my bachelor thesis, I created an open-source website for editing PDF files in the Firefox browser.  While there are many PDF applications available for editing PDF files, most of them require a fee. In my thesis, I provided background knowledge about the PDF file format and discussed security issues related to this format. Additionally, I presented the latest technology and analysed seven PDF programs. In the later part of my work, I explain the implementation of my PDF Web App and conduct GUI and performance tests. The JavaScript libraries pdf.js and PDF-LIB form the foundation of my PDF Web App, providing all the necessary functionalities for common PDF editing, such as reading, splitting, merging, creating PDF files, and adding text, images, geometric shapes, and drawings. Finally, I discuss the functions that are still evolving and propose features planned for the future.

\textbf{Keywords:} PDF editing, Adobe, JavaScript, Vue JS 3, splitting, merging, pdf.js, PDF-LIB

\textbf{Date:} 04 March 2024