\section{PDF Sicherheitsaspekte}
Etwa 40 \% der Unternehmen setzen PDFs für geschützte Inhalte ein. In den letzten 2 Jahren ist die Nutzung der elektronischen Signaturfunktion in PDFs um mehr als 150 \% gestiegen. \cite{formilo}
\par
In den Sicherheitseinstellungen eines PDF-Dokuments können Dokumentensicherheit und Zugriffsregeln justiert werden. PDF unterstützt Verschlüsselung und die Vergabe von 2 Passworttypen. Eventuell kann beim Öffnen einer Datei ein Passwort gefordert werden oder das Kopieren von Teilinhalten, jeglichem Inhalt, Ausfüllen von Formularfeldern, Dokumentveränderungen (z.B. Struktur, Inhalt, Kommentare) oder das Ausdrucken kann vom Ersteller des Dokuments gesperrt worden sein. 

\subsection{Digitale Unterschrift}
Digitale Unterschriften sollen die Identität des Unterzeichners des Dokuments authentifizieren und dass der Inhalt nach der digitalen Unterschrift nicht geändert wurde. Der Verfasser kann sein PDF-Dokument mit einem digitalen Zertifikat signieren. Das Zertifikat bescheinigt die Echtheit der Unterschrift und der Herkunft und wird von einem Zertifizierungsanbieter ausgestellt. Zusätzlich können Zertifikate ablaufen oder entzogen werden und müssen gültig sein. Dabei sollte ein vertrauenswürdiger Zertifizierungsanbieter gewählt werden. Digitale Signaturen werden durch einen Hash basierend auf das erstellte PDF-Dokument berechnet und geben der PDF-Datei einen eindeutigen Fingerabdruck. Dieser Hash wird im Dokument gespeichtert und wird überprüft, wenn die Unterschrift validiert werden soll, indem er neu berechnet wird. Unterscheiden sich beide Hashs voneinander wurde die PDF Datei verändert. Jede Unterschrift kann mit einem Zeitstempel versehen werden. Ein vertrauenswürdiger \gls{zsa} belegt den Zeitpunkt, wann diese Unterschrift geleistet wurde. \\
Eine PDF-Datei ermöglicht mehrere digitale Unterschriften, jedoch muss jede neue Unterschrift in einem inkrementellen Update geleistet werden. Jede Unterschrift muss mit einem Unterschriftsfeld im Dokument verbunden sein. Optional kann das Unterschriftsfeld mit einem Widget gekoppelt sein. Dann wird die Unterschrift graphisch dargestellte. Unterschriften ohne Widgets sind versteckte Unterschriften. \cite{softx}