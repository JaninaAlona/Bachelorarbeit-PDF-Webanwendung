\section{Freie PDF Programme und Onlinedienste}


\subsection{Foxit PDF Reader}
Produktseite: \url{https://www.foxit.com/pdf-reader/} \\
Die Firma Foxit hat einen kostenlosen PDF Reader auf den Markt gebracht, der sehr umfangreiche Funktionen bietet. Der Reader kann alle Seiten des PDFs rotieren in 90 Grad-Schritten, Lesezeichen hinzufügen, bestehenden Texte kopieren und farbig markieren, sowie Text, Bilder, Audio und Video hinzufügen. Außerdem gibt es ein Messwerkzeug für 3D-Objekte. Der Reader hat eine Fill \% Sign-Funktion für elektronische Signaturen, bei der man seine Unterschrift zeichnen und sie für die nächste Unterschrift als Bild speichern kann. Digitale Signaturen mit Zertifikat können ebenfalls verwendet werden. Kommentare gibt es in Form von Zeichnungen, Geometrieformen und Text. Zeichnungen können wegradiert und Kommentar-Objekte können skaliert oder verschoben werden. Darüber hinaus können Kommentare zusammengefasst werden und eine erweiterte Suche für PDF-Inhalte ist möglich. Das Arbeiten mit Ebenen ist nur im Foxit PDF Editor möglich und es können keine bestehen PDF-Objekte editiert werden. Der Foxit PDF Reader ist als Webversion, für Windows, macOS, Linux, iOS und Android verfügbar \cite{foxit-reader}. Zusätzlich gibt es auf Foxits Produktseite kostenlose PDF Converter Online Tools. Dort können mehrere PDFs zusammengefügt werden und komprimiert werden. PDF kann zu Word, JPEG, Powerpoint und Excel konvertiert und aus diesen Dateiformaten zurück zu PDF umgewandelt werden. Beim Foxit PDF Editor kann man PDF Dateien vergleichen und Wörter zählen. Im \gls{ocr}-Tool kann man die \gls{ocr}-Ergebnisse manuell korrigieren. Es gibt einen Preflight-Dialog. Ausschließlich in der Pro-Version kann man PDF-Dateien standardkonform zu PDF/A, PDF/E oder PDF/X Dateien umwandeln. Die Undo und Redo-Optionen können Schritte rückgängig machen oder einen rückgängig gemachten Schritt wieder ausführen. Bezüglich der Pro-Variante ist ein Action Wizard für actions eingebaut. Text kann in der Standard-Variante editiert werden und Bilder in der Pro-Variante. Eine Prüfung der Rechtschreibung ist vorhanden. Das Arbeiten mit Ebenen bezüglich anzeigen bzw. verbergen, löschen und neu ordnen ist freigeschaltet. Zusätzlich kann man in der Pro-Version Ebenen importieren, zusammenführen und reduzieren. Seiten können eingefügt, zerteilt, beschnitten, skaliert und basierend auf den Lesezeichen neu angeordnet werden. ChatGPT ist im Editor integriert und bietet die Funktionen: Zusammenfassung, Umschreiben, Übersetzung, PDF Chat mit Fragen Stellen und Antworten Erhalten vom AI assistant, Inhaltserklärung, Korrektur der Rechtschreibung und ein AI Chatbot mit 50 prompts pro Tag \cite{foxit-um}. Eine 14-tägige kostenlose Testversion für den Foxit PDF Editor kann heruntergeladen werden. Die PDF Editor Suite für Einzelpersonen kostet 12,37 Euro im Monat, die Suite Pro 15,75 Euro pro Monat und die Cloud 6,63 Euro pro Monat. Man kann auch eine jährliche Zahlung leisten. Für den Bildungssektor kann lediglich eine sehr vergünstigte jährliche Zahlung von Studenten, Lehrern und Institutionen geleistet werden \cite{foxit-editor}. 

\subsection{PDF24 Tools}
Produktseite: \url{https://tools.pdf24.org/en/all-tools} \\
Bei den PDF24 Tools handelt es sich um eine Online-Lösung für PDF-Bearbeitung. Die Online-Lösung ist 100 \% kostenlos und die Verwendung ohne Limits. Es gibt auch eine Desktop-Version für Windows. PDF24 kann PDFs zusammenfügen, zerteilen, komprimieren, editieren und signieren. Beim Edit PDF-Tool kann man nur Elemente hinzufügen und keine Objekte aus dem source PDF bearbeiten. Es ist möglich Text, Zeichnungen, Bilder und Geometrie hinzuzufügen. Hinzugefügte Elemente können gelöscht, gedreht, skaliert, verschoben werden. Wenn ein Element geclont wird, wird es leicht versetzt über dem Ursprungselement eingefügt. Texte sind markierbar nach dem Speichern. Es gibt eine Undo und Redo-Funktion. Bei Texten kann man auch eine Umrandungslinie mit Breite und Farbe inklusive Deckkraft anzeigen und es gibt nur 3 verschiedene Fontarten: Sans-Serif, Serif und Monospace. Zeichnungen können auch mit Transparenz eingefärbt werden. Ich habe keinen Radierer im Tool gesehen. Der Stift hat ein paar Pinseltypen zur Auswahl. Wenn man ein Element markiert, wird es in den Vordergrund geholt. Jedes Element ist einzeln selektierbar. 
\cite{pdf24}

\subsection{Aspose.PDF}
Produktseite: \url{https://products.aspose.app/pdf/family} \\

%%%%%%%%%%%%%%%%%%%%%%%%%%%%%%%

\subsection{PDFsam}
Produktseite: \url{https://pdfsam.org/}

\subsection{DrawboardPDF}
Produktseite: \url{https://www.drawboard.com/pdf/pdf}

