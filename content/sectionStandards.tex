\section{PDF Dateiformate}
PDF hat zahlreiche Dateiformate, von denen die meisten standardisiert wurden, hervorgebracht. Jedes Dateiformat ist einem individuellen Anwendungsbereich zugeordnet und adressiert spezifische Industriebranchen: PAdES für elektronische Signaturen, PDF/X für den professionellen Druck, PDF/A für die Archivierung, PDF/E für den Ingenieurbereich, PDF/H für das Gesundheitswesen, PDF/VT für den Druck mit variablen Daten, PDF/UA für Barrierefreiheit, PDF/R für gescannte Dokumente und Durchsuchbare PDFs für Stichwortsuche. Im Folgenden stelle ich jedes Format vor und beschreibe seine speziellen Merkmale. 

\subsection{PAdES}
\gls{pades} ergänzt den Funktionsumfang um Werkzeuge mit denen man elektronische Signaturen erzeugen, anpassen und prüfen kann. Es wurde vom \gls{etsi} veröffentlicht, 1999 in PDF 1.3 eingeführt und basiert auf der \gls{iso} 32000-1 Spezifikation. Nachfolgend wurde dessen Konzept weiterentwickelt. \gls{pades} erweitert PDF um kryptographische Techniken und ermöglicht sichtbare und unsichtbare digitale Signaturen. Resultierend soll dieses Dateiformat die Integrität, Authentizität, Verbindlichkeit und Rechtssicherheit von digital signierten PDF-Dokumenten herstellen. \gls{pades} implementiert verschiedene Signaturformate wie \gls{cades} und \gls{xades}, unterstützt Zeitstempel und die Validierung des Zertifikatwiderrufsstatuses. Zertifikatbasierte Signaturen sollen die Identität des Unterzeichners und die Unabänderlichkeit des Dokuments sichern. Eine zertifizierte PDF-Datei ermöglicht die Umsetzung bestimmter Nutzungsrechte, wie eingeschränkte Bearbeitung, Ausfüllen von Formularen oder gesperrtes Drucken. Eine elektronische Unterschrift kann mit dem Programm Adobe Acrobat Sign erstellt werden \cite{adobe-pdf-pades}. 


\subsection{PDF/X}
Speziell für den simpleren Datenaustausch in der Druckvorstufe und der professionellen Druckindustrie wurde PDF/X (Exchange) als \gls{iso} 15930:2001 entwickelt. Dieser erst 2001 publizierte Dateiformatstandard beschreibt die speziellen Eigenschaften von Druckvorlagen und vereinfacht die Datenübermittlung von der Design-Agentur und Druckvorstufe bis zum finalen Druck. Besonderen Wert wurde darauf gelegt, dass in offenen Dateiformaten aus Layoutprogrammen keine Informationen über Farbe und Schrift verloren gehen und einer Verfälschung im Druckergebnis vorgebeugt werden kann \cite{adobe-pdf-e}. Die Entwicklung von PDF/X zielt auf eine Verminderung von Druckfehlern und Mehraufwand in der Druckerei. In der Umsetzung bedeutet das, dass Elemente, die sich nicht sinnvoll drucken lassen, z.B. Video und Audio, nicht berücksichtigt werden. PDF/X-kompatibel bezeichnet die Eigenschaft von Dokumenten, dass sie ohne vorherige Prüfung von der Druckerei direkt verwendet werden können \cite{adobe-pdf-x}. \\
Beschnitt, Farbangaben und verwendete Schriften sind u.a. für den Druck notwendig und sollten verwendet werden. Qualitätsanforderungen, die sich auf bestimmte Druckverfahren beziehen, sind nicht implementiert, sondern werden abstrakter definiert. Besondere Qualitätsanforderungen liegen vor allem im Zeitungsdruck, Akzidenzdruck oder Bilderdruck vor. Vielmehr geht es bei PDF/X darum, Grundvoraussetzungen für den Druck sicherzustellen, z.B. ob der richtige Farbraum gesetzt wurde oder korrekte Einstellungen für Überdrucken und Überfüllung vorliegen. Neben Aussparen und Unterfüllung werden Überdrucken und Überfüllung zum Oberbegriff Trapping zusammengefasst. Bei der Überfüllung werden bei verschieden farbigen Objekten das hellere Objekt auf dunklem Hintergrund minimal vergrößert, sodass es das dunklere Objekt leicht überlappt. Solche Überlappungen nennt man Traps. Dies beugt weißen Blitzern (Papierweiß scheint durch) beim Druckergebnis vor. Die umgekehrte Vorgehensweise wird bei der Unterfüllung angewendet. Liegt ein dunkles Objekt auf hellem Hintergrund, so wird das helle Objekt an den Rändern zur dunkleren Farbe hin verengt. Vordergrundobjekte stehen in Layout- und Grafikprogrammen standardmäßig auf Aussparen. Dessen Fläche wird im Hintergrund ausgeschnitten, um unerwünschte Farbmischungen zu vermeiden. Im Falle von schwarzen oder sehr dunklem Text auf farbigem Hintergrund sollte man Layoutprogramm diese Vordergrundelemente überdrucken lassen \cite{kompendium}. Generell werden schwarze Schrift oder Linien im Drucker durch 3 oder 4 Farben zusammengesetzt und fehlende Schriften werden häufig durch den Font Courier kompensiert. Im Druck sollten keine verlustlosen Kompressionsalgorithmen für Bilder verwendet werden wie JPEG, da Artefakte auftreten können. Ebenso gibt es keine automatischen Einstellungen für passende Auflösungen von Vollton-, Halbton- oder Strichbildern \cite{adobe-pdf-x}. Eine Farbe mit 100 \% Deckkraft wird als Volltonfarbe bezeichnet. Halbtöne sind Farben mit geringerer Deckkraft \cite{halb-voll}. Als Volltonfarben werden auch speziell vorgemischte Druckfarben bezeichnet die anstelle von oder zusätzlich zu den üblichen Prozessdruckfarben in CMYK verwendet werden. Für Volltonfarben ist eine eigene Druckplatte in der Druckmaschine vonnöten \cite{adobe-voll}. Der CMYK-Farbraum wird im Druck verwendet durch subtraktive Farbmischung mit den Farben Cyan, Magenta, Yellow und Key (Schwarz). Einige Farben können nicht von CMYK reproduziert werden, dann spricht man von Sonderfarben. Strichbilder sind Bilder mit ausschließlich weißen und schwarzen Partien \cite{strich}. \\
Bei allen PDF/X-Dateivarianten außer den Varianten, die ein p enthalten, muss ein Output-Intent im Katalog eingetragen sein. Jede Seite einer PDF/X-Datei muss mindestens die TrimBox oder ArtBox, aber nicht beide gemeinsam, beinhalten. JavaScript bzw. Aktionen, \gls{opi} Kommentare und Verschlüsselung sind verboten \cite{schneeberger}. 
\par
Es gibt verschiedene Varianten von PDF/X, die jeweils einen verbesserten Farbspielraum ermöglichen. 

\subsubsection{PDF/X-1a}
PDF/X-1a ist eine Subvariante von PDF/X-1 (Norm \gls{iso} 15930-1). In der a-Version sind lediglich CMYK, Graustufen, Schwarz-Weiß (Bitmap) und Sonderfarben möglich und es muss eine Composite-Datei vorliegen. Vorseparierte Dateien sind nicht zulässig. Farben können nicht auf Grundlage von \gls{icc} Profilen bei PDF/X-1a definiert werden \cite{adobe-pdf-x, schneeberger}. Erst seit PDF 1.3 werden \gls{icc}-Profile unterstützt, die die Farbeigenschaften, Helligkeit, Weißpunkt, Gammakurve und Farbumfang eines bestimmten Monitors eines spezifischen Geräts beschreiben, sprich ein \gls{icc}-Profil definiert wie Farben vom Zielgerät dargestellt werden können. Außerdem wird die Transformation zwischen dem Gerät und dem Profilverbindungsraum \gls{pcs} definiert. Dabei gibt es die Variante Eingabeprofile für Kameras oder Scanner und Ausgabeprofile für Monitore oder Drucker. Zweck des \gls{icc}-Profils ist möglichst Farbübereinstimmungen zwischen verschiedenen Geräten zu erzielen \cite{benq}. Beim \gls{pcs} handelt es sich um ein neutrales Farbmodell im \gls{icc}-Colormanagement, welches den Quellfarbraum mit dem Zielfarbraum verbindet und somit geräteunabhängig ist. Der \gls{pcs} kann entweder der LAB oder XYZ Farbraum sein \cite{prepress}. \\
Es können  TIFF-, TIFF/IT-, EPS- und \gls{dcs}-Dateien integriert werden. Transparenzen, Ebenen, LZW- und JBIG2-Kompression, Formularfunktionen und interaktive Elemente dürfen nicht verwendet werden bzw. sind nicht implementiert. Dieser Standard wurde von \gls{iso} 15930-1:2001 auf \gls{iso} 15930-4:2003 überarbeitet. Lediglich in der überarbeiteten Version, die die Version von 2001 ersetzt, werden auch Sonderfarben unterstützt \cite{proj-consult, schneeberger}. 

\subsubsection{PDF/X-2}
Die 2. Variante ist als \gls{iso} 15930-6:2003 erschienen und bietet die Voraussetzungen zum farbigen Qualitätsdruck wie Farbmanagement, CMYK- und Sonderfarbdaten in beliebiger Kombination. PDF/X-2 hat in der Praxis wegen der Unvollständigkeit der PDF-Datei für den Datenaustausch in der Druckvorstufe keinerlei Relevanz. Dieser Standard wird durch PDF/X-5 vollkommen ersetzt \cite{proj-consult, schneeberger}. 

\subsubsection{PDF/X-3}
Zusätzlich erweitert Version 3 als \gls{iso} 15930-3:2002 \cite{proj-consult} um die Farbräume RGB und LAB, sowie alle \gls{icc}-Profile. Möglicherweise wird in der Druckvorstufe der im Dokument eingestellte Farbraum in CMYK umgewandelt. Es findet eine automatische Tranzparenz- und Ebenenreduzierung statt \cite{adobe-pdf-x}. Bei der Transparenzreduzierung werden einzelne Bildsegmente in vektorbasierte und gerasterte Bereiche unterteilt, überlappende Bereiche der Transparenzen zerschnitten und auf einer Ebene reduziert \cite{adobe-transp, primus}.

\subsubsection{PDF/X-4}
Transparenzen, Ebenen, JPEG2000 und OpenType-Schriften können in dieser PDF/X-Variante als \gls{iso} 15930-7:2008 \cite{proj-consult, schneeberger} verwendet werden, wodurch sie für das Bedrucken von Textilien besonders gut geeignet ist  \cite{adobe-pdf-x}. Es gibt außerdem die PDF/X-4p (Profil) Variante \cite{schneeberger}. 

\subsubsection{PDF/X-5}
PDF/X-5 wurde als \gls{iso} 15930-8:2010 als vorletzter Standard des PDF/X-Formats verabschiedet und inkludiert externe Elemente und Multichannel-\gls{icc}-Profile \cite{proj-consult}. Multichannel-Profile unterstützen mehr als 4 Farbkanäle und können somit für Drucker mit mehr als 4 Druckpatronen eingesetzt werden \cite{adobe-profil}. Außerdem gibt es die folgenden Varianten: PDF/X-g (external graphic content), PDF/X-pg (external \gls{icc} color profiles and external graphical content) und PDF/X-gn (n-colorant \gls{icc}profile) für den mehrfarbigen Verpackungsdruck \cite{schneeberger}.

\subsubsection{PDF/X-6}
Der letzte PDF/X-Standard als Version 6 wurde in \gls{iso} 15930-9:2020 offenbart und basiert auf dem PDF 2.0 Standard. In dieser Version sind neben maßgeblichen Neuerungen für die heutigen Print-Anforderungen Lockerungen im Vergleich zu vorherigen PDF/X-Standards eingeführt worden. Die wichtigsten Neuerungen sind Parameter für Tiefenkompensierung, separate Ausgabebedingungen, DPart Metadaten, Informationen zu Sonderfarben im CxF/X-4 Standard und Mixing Hints \cite{proj-consult}.
Tiefenkompensierung ist eine rechnerische Korrektur und wird bei der Konvertierung eines Farbraums mit großem Tonwertumfang in einen mit kleinem Tonwertumfang angewendet. Dabei soll die ursprüngliche Differenzierung der Tonwerte dunkler Bildteile beibehalten werden können \cite{tiefen}. 
DPart-Metadaten wurden ursprünglich für den PDF/VT-Standard spezifiziert und ermöglichen mehrteilige PDF-Dateien in Datensätze zu unterteilen. Diese PDF-Dateien können dann automatisch verarbeitet werden \cite{pdfa-dpart}.  Mixing Hints enthalten Informationen, z.B. wichtig bei Prüfdruck, über erwartete Ergebnisse, wenn mehrere Volltonfarben im Druck miteinander interagieren. Diese Informationen sind beispielsweise für einen Prüfdruck wissenswert \cite{mixing-hints}. \\
Zu den Lockerungen zählen die Möglichkeit von Notizen und grafischen Anmerkungen, strukturelle Aktionen, Formularfelder und digitale Signaturen. Außerdem besteht dieser Standard aus 2 Konformitätsstufen: PDF/X-6p zur Referenzierung von \gls{icc}-Profilen und PDF/X-6n für Multicolor-Profile \cite{proj-consult}. 


\subsection{PDF/A}
Das PDF/A Dateiformat (Archivable) wurde zur gesetzeskonformen Langzeitarchivierung von digitalen Dokumenten entwickelt und solche Dokumente sind zunächst schreibgeschützt. Der \gls{iso}-Standard definiert die Konformität der Form von Elementen wie Schriften oder Layout für eine Langzeitarchivierung. Dadurch ist die Lesbarkeit der Dokumente über lange Zeiträume gesichert und die Bedingungen einer revisionssicheren Archivierung gewährleistet \cite{adobe-pdf-a}. Revisionssichere Archivierung bedeutet, dass gespeicherte Daten vor nachträglichen Modifikationen, Fälschung oder Manipulation geschützt sind \cite{adobe-revisions}.  Der Fokus in diesem Dateiformat liegt auf langfristige und einfache Speicherung der PDF-Dateien. Folglich ist die Einbettung von Audio und Video nicht implementiert, aktive Komponenten wie Links, sowie externe Ressourcen wie Grafiken und Schriftarten werden nicht unterstützt, sondern müssen direkt eingebettet werden. Ebenso können Dokumente nicht verschlüsselt werden. Die Einbettung von Metadaten als \gls{xmp} wird unterstützt, was die Identifizierung und Suche von Dokumenten erleichtert. Es gibt einige Nachteile von PDF/A. Nicht alle Dokumente können problemlos in dieses Dateiformat umgewandelt werden, wie beispielsweise Dokumente mit Audio, Video oder JavaScript. Nach solch einer Konvertierung zu PDF/A kann es zu Fehlern in der visuellen Darstellung kommen und die Dateigröße kann enorm werden, da alle Elemente direkt eingebettet werden müssen \cite{adobe-pdf-a}.

\subsubsection{PDF/A-1}
Seit der ersten Version von PDF/A ist das Dateiformat in die \gls{iso}-Norm als PDF/A-1 in \gls{iso} 19005-1:2005 übernommen worden \cite{proj-consult}. Die Originalversion stellt sicher, dass alle externen Quellen wie Schriften oder Bilder eingebettet sind, unterstützt digitale Signaturen und Hyperlinks. PDF/A-1 ist abwärtskompatibel. Es gibt 2 Qualitätsebenen von PDF/A-1: PDF/A-1b (Basic) und PDF/A-1a (Accessible). Die Basic Variante legt Wert darauf, dass Dokumente eindeutig visuell reproduzierbar sind und Accessible ist zusätzlich für Barrierefreiheit optimiert. Bei Accessible können Text und inhaltliche Struktur von einem Screenreader durch Tagged PDF vorgelesen werden \cite{adobe-pdf-a}. Des Weiteren werden, Sprach-Angabe und Unicode-Mappings unterstützt \cite{proj-consult}.

\subsubsection{PDF/A-2}
Im Jahr 2011 wurde die PDF/A-2 Version als \gls{iso} 19005-2:2011 auf den Markt gebracht. Sie ermöglicht die Kompression von Grafikformaten mit JPEG-2000, Transparenzen, PDF-Ebenen, Portfolios, Object Level \gls{xmp} Metadaten, Kommentartypen, Annotationen und digitale Signaturen \cite{proj-consult}. PDF/A-1-Dateien können in PDF/A-2-Dateien eingebunden werden. Es gibt 3 Varianten von PDF/A-2: PDF/A-2b (Basic), PDF/A-2u (Unicode-Textsemantik) und PDF/A-2a (Accessible). Basic gewährleistet das unveränderte Erscheinungsbild eines Dokuments und definiert die Mindestanforderungen. Die Unicode-Version ergänzt um Unicode-Unterstützung und Indexierung. Accessible setzt alle Anforderungen der \gls{iso}-Norm 19005-2 um \cite{adobe-pdf-a}. 

\subsubsection{PDF/A-3}
Ein Jahr später wurde PDF/A-3 im Standard \gls{iso}-19005-3:2012 veröffentlicht. Er basiert auf PDF 1.7 und ermöglicht die Einbettung dynamischer, zur Laufzeit interpretierbaren Komponenten und Dateiformate. Gleichfalls definiert PDF/A-3 die Konformitätsebenen 3b, 3u und 3a. Die u-Variante bietet eine Vereinfachung in der Durchsuchbarkeit von Texten und das Kopieren von Unicode-Text \cite{proj-consult}.

\subsubsection{PDF/A-4}
Viel später im Jahr 2020 wurde PDF/A-4 als \gls{iso} 19005-4:2020 herausgebracht. Dieser Standard basiert auf der PDF 2.0 Dateiversion. Sie spezifiziert 2 neue Konformitätsebenen: PDF/A-4f für nicht-PDF/A konforme Dateianhänge und PDF/A-4e für Einbindung von 3D-Inhalten in den Formaten U3D oder PRC für den Engineering-Bereich \cite{proj-consult}.


\subsection{PDF/E}
PDF/E (Engineering) gilt als international standardisiertes Austauschformat als Norm \gls{iso} 24517 für technische Dokumente und wird im Maschinenbau, in der Fertigung und im Baugewerbe für Fertigungspläne, Konstruktionszeichnungen oder technische Dokumentationen verwendet. Das PDF/E Dateiformat von 2008 ist speziell für das Ingenieurwesen entworfen und kann interaktive 3D-Elemente und Animationen darstellen. Im einzelnen können \gls{cad}-Dateien im 3D- und 2D-Format eingebettet werden. Die 3D-Elemente können im Dokument ausgeklappt oder gedreht werden. Metadaten und interaktive Funktionen wie Lesezeichen, Formulare oder Hyperlinks werden unterstützt \cite{adobe-pdf-e}.


\subsection{PDF/H}
Das PDF/H (Healthcare) Dateiformat soll im Gesundheitswesen Patientendaten erfassen, austauschen und archivieren. Hierbei wird besonders Wert auf die Anforderungen des Datenschutzes in gesundheitsspezifischen Ämtern, Institutionen und Arztpraxen gelegt. PDF/H wurde zwar im Jahr 2008 kreiert, jedoch wurde es nicht in den Normierungsprozess der \gls{iso} eingebunden \cite{proj-consult}. Es handelt sich eher um eine Best-Practice für die Umstellung von Papiergesundheitsakten auf PDF als e-Akte. Die Digitalisierung soll gesundheitsbezogene Daten strukturieren, verwalten und so präsentieren, dass Forscher*innen und Beschäftigte im Gesundheitswesen effizienter auf sie zugreifen können.


\subsection{PDF/VT}
Basierend auf PDF/X wurde PDF/VT als spezielles Austauschformat im variablen Datendruck (Variable Data) und Transaktionsdruck (Transactional Printing) im Jahr 2010 auf den Markt gebracht \cite{adobe-pdf-vt}.  Wiederkehrende Elemente wie Texte, Grafiken oder Bilder sollen effizienter verarbeitet und an den Drucker übertragen werden können \cite{adobe-pdf-e}. Variabler Datendruck bezeichnet ein digitales Druckverfahren bei dem einzelne Parameter von Printprodukten individuell variiert werden können, wobei das Grundlayout beständig bleibt. Folglich können große Mengen von Printprodukten mit Personalisierung hergestellt werden, z.B. Werbebriefe mit konstanten grafischen Elementen wie Firmenlogos und individuellen Namen der Kund*innen. Dadurch können Firmen ihr Corporate Identity-Layout behalten und ihre Kund*innen persönlicher ansprechen. Der Begriff Transaktionsdruck definiert das Drucken von Transaktionsdokumentationen wie Rechnungen, Mahnungen, Lieferscheine oder Quittungen im Waren- und Dienstleistungssektor. Herausstechend ist, dass PDF/VT große Mengen an variablen Daten in einer einzigen PDF-Datei speichern kann, wobei es immer einen Satz von statischen und variablen Daten gibt. Diese Vorgehensweise spart Zeit, Kosten und reduziert Fehler. Vorteilhafterweise werden \gls{icc}-Profile unterstützt \cite{adobe-pdf-vt}. Die erste Version als \gls{iso} 16612-1:2005 konnte sich auf dem Markt nicht durchsetzen \cite{proj-consult}.

\subsubsection{PDF/VT-2}
Die zweite Version als \gls{iso} 16612-2:2010 implementiert die Verwendung externer grafischer Inhalte und das Streamen von mehrteiligen \gls{mime} Paketen in der Version PDF/VT-2s. \gls{mime}-Typen (media types) zeigen Art und Format von Dokumenten, Dateien oder einem Sortiment von Bytes an und sind standardisiert. Ein \gls{mime}-Typ besteht üblicherweise immer aus einem durch ein / getrennten Typ und Subtyp. Der Typ repräsentiert die generelle Kategorie des Datentyps und der Subtyp die exakte Datenart des Typs. Jeder Typ hat seine eigene Menge an Subtypen \cite{mime}. Zum Lesen von PDF/VT-2 Dokumenten wird ein PDF/X-4- bzw. PDF/X-5- oder PDF/VT-konformer PDF-Reader benötigt \cite{proj-consult}.

\subsubsection{PDF/VT-3}
Spezialisiert auf die Integration von variablen Daten (DPart-Metadaten) und den Transaktionsdruck ist die dritte PDF/VT Variante als \gls{iso} 16612-3:2020 \cite{proj-consult}.


\subsection{PDF/UA}
Das PDF/UA (Universal Accessibility) Dateiformat dient der Erstellung barrierefreier Dokumente. Die PDF/UA-Kennzeichnung stellt eine Klassifizierung für barrierefreie Dokumente dar und orientiert sich an den Anforderungen der \gls{wcag} 2.0 des World Wide Web Consortiums. Als Rechtliche Grundlage dient die \gls{bitv} 2.0. Um die Anforderungen der PDF/UA-Kennzeichnung zu erfüllen, müssen Dokumente bestimmte technische und inhaltliche Vorgaben erfüllen. Auf Basis des Matterhorn-Protokolls, welches aus 31 Prüfpunkten und 136 Konformitätskriterien besteht, müssen Aufbau von Texten, Bildern, Listen, Tabellen und Formularefeldern festgelegte Vorgaben erfüllen. Diese Konformitätskriterien können auf der einen Seite nur von einer Software geprüft werden und auf der anderen Seite nur von Menschen. \\ Menschen mit Einschränkungen sollen das Dokument optimal nutzen können. Zur Erleichterung des Verständnisses sollten Überschriften, alternative Texte für Bilder, Beschreibungen für Tabellen, Tags und eine klare Lesereihenfolge verwendet werden \cite{adobe-pdf-ua}. Es gibt die PDF/UA-1 Version aus \gls{iso} 14289-1:2012 und die überarbeitete Version PDF/UA-1 aus \gls{iso} 14289-1:2014, die erst 2020 als gültig erklärt wurde \cite{proj-consult}.


\subsection{PDF/R}
Speziell für die Speicherung, den Transport und Austausch von gescannten Dokumenten gibt es das \gls{iso} 23504-1:2020 standardisierte Format PDF/R-1. Es bietet die grundsätzlichen Funktionalitäten von TIFF, bitonalen, Graustufen- und Echtfarbbilder \cite{proj-consult}. Bitonale Bilder bestehen lediglich aus einer Vordergrund- und Hintergrundfarbe. 


\subsection{Searchable PDF}
Searchable PDFs können mit Suchfunktionalitäten eines PDF-Readers durchsucht werden. Es kann gezielt nach Zahlen oder Stichwörtern durchsucht und Inhalte können zur Bearbeitung in anderen Programmen kopiert werden. Man erkennt durchsuchbare PDFs daran, dass man den Text markieren kann. Diese PDF-Art wird üblicherweise durch die \gls{ocr} Technologie erstellt. Bei \gls{ocr} handelt es sich um optische Zeichenerkennung, die Textzeichen und Dokumentstruktur analysiert. Auf diese Weise können gescannte Dokumente oder Pixelbilder als PDF abgespeichert und in ein Durchsuchbares PDF in Acrobat umgewandelt werden. Während der Umwandlung wird dem Dokument eine zusätzliche unsichtbare Textebene, die unter der Bildebene liegt und durchsuchbar ist, auf der Seite hinzugefügt. In Acrobat ist es außerdem möglich mit der einfachen Suche innerhalb einer Datei nach Suchbegriffen zu suchen, mit der erweiterten Suche oder der Suchen-Werkzeugleiste mehrere PDF-Dokumente zu durchsuchen und speziell in der erweiterten Suche u.a. Objektdaten und Bildern zu lokalisieren. Textbasierte PDF-Dateien können grundsätzlich durchsucht werden und auch in andere Dateiformate wie Microsoft Word, Excel oder PowerPoint umgewandelt werden. Durchsuchbare PDFs ermöglichen Barrierefreiheit. Sie können von Bildschirmleseprogrammen für Sehbehinderte vorgelesen oder vergrößert werden\cite{adobe-search}. 
