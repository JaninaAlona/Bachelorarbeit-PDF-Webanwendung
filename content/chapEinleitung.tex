\chapter{Einleitung}


\section{Motivation}
Zu Hause benutze ich 4 PDF-Programme, um alle für mich zufriedenstellenden häufigen Anforderungen der PDF-Bearbeitung zu erledigen: Adobe Reader zum Anzeigen von PDF, Drawboard PDF zum Zeichnen, Foxit Reader zum Schreiben und PDF Sam Basic zum Splitten und Mergen. Ich habe dann später herausgefunden, dass man mit Foxit Reader auch Splitten, Mergen und Zeichnen kann, jedoch finde ich dieses Programm sehr unintuitiv und ich vergesse immer wo die Einstellmöglichkeiten für diese Funktionalitäten waren. Oft habe ich dann keine Lust die Einstellmöglichkeiten im Internet zu googlen und nehme dann lieber PDF Sam Basic zum Splitten von Seiten. PDF Sam Basic ist da einfacher, jedoch ist es sehr störend, dass man bei seiner Installation auch eine andere Werbeversion von PDF Sam zusätzlich installiert. Ich habe mir gedacht, dass es nicht sein kann, dass mir kein kostenloses PDF Programm so wirklich gefällt und deshalb beschlossen meine eigene PDF Web App zu entwickeln. Weiter habe ich überlegt: Es wäre sinnvoll eine Webseite zu programmieren, die man auch offline nutzen kann. Alles was man benötigen soll ist ein Browser und keine Installationen mit zusätzlichen Werbeprogrammen, die man eigentlich nicht installieren wollte. Wie wäre es mit einer Open Source Web App, die ich meinen Freunden zeigen kann und mit denen ich mich auf dem Arbeitsmarkt bewerben kann? Gesagt, getan - ich habe diese PDF Web App für diese Bachelorarbeit programmiert und sie ist neben ein paar Ecken und Kanten besser geworden, als ich es je erwartet hätte. Ich habe während der Programmierung viel gelernt über JavaScript, asynchrone Programmierung, Event Handler, usw. und bin generell sicherer geworden im Programmieren. Vor allem habe ich Wert darauf gelegt, dass die Web App eher einfach und intuitiv zu bedienen ist. Einige Features waren nicht geplant, aber mir hat es Spaß gemacht sie zu implementieren, da ich die Motivation hatte ein für andere und mich gutes Tool zu entwickeln. Natürlich gibt es noch eine lange Liste an Features, die ich in Zukunft noch implementieren möchte. Die PDF Web App ist kein abgeschlossenes Projekt. Dieses Hobby-Projekt möchte ich gerne nach der Bachelorarbeit weiter auf meiner Github-Seite maintainen. Vielleicht gibt es andere Entwickler, die mein Repository forken oder bug fixes hinzusteuern wollen. Vielleicht kann ich andere Entwickler dafür begeistern an der PDF Web App mitzuarbeiten und sie noch besser zu machen. 

\section{Aufbau der Arbeit}