\chapter{Stand der Technik}
Die Popularität von PDF-Dateien ist seit 2008 rasant angestiegen in der globalen Informationsübertragung. Täglich werden weltweit 2,5 Milliarden PDF Dokumente erzeugt. Seine Beliebtheit verdankt PDF vor allem an der plattformübergreifenden Kompatibilität (Desktop-Computer, Tablets und Smartphones), denn PDF Dokumente ist auf mehr als 1,5 Milliarden Geräten ohne zusätzliche Software lesbar. Über 80\% der geschäftlichen Dokumente werden als PDF Datei weitergegeben. \cite{formilo} 90 \% der Büroangestellten wollen auf das PDF Dateiformat nicht mehr verzichten. Drei Viertel aller archivierten Dokumente sind PDF Dokumente (2023).
Bis 2025 werden über 3 Millarden Dollar jährlich für PDF Editoren ausgegeben werden (2023) \cite{kofax}. Im Jahr 2015 gab es 1,6 Milliarde PDF-Dokumente im Web und im Jahr 2019 wurden PDF-Dateien bei ca. 99 \% Firmen und Regierungsinstitutionen weltweit verwendet \cite{ccc-break-pdf}. \\

Generative Artificial Intelligence (AI) Anwendungen können das Arbeiten mit Text, Bildern, Code und Dokumenten wie PDF erleichtern. Solche Anwendungen sind seit einigen Jahren wertvolle Tools in vielen Businessumgebungen und -workflows. Sie kommen in Tools für Teamkollaboration, wie Zoom, \gls{ccaas} Plattformen und Produktivitätsapps vor. Einer der prominentesten Beispiele sind Microsoft Copilot und Google Duet AI. Sie bauen auf ein \gls{llm} auf und bedienen sich Algorithmen zur Gesprächsführung. Mittels \gls{nlp} und hochmodernen Algorithmen können generative AI Anwendungen mit Menschen interagieren. Durch eine Benutzeraufforderung (prompt) kann der Benutzer Fragen oder Befehle an das Tool übermitteln und erhält menschenähnliche Antworten in wenigen Sekunden. Copilot ist u.a. in die Bing Suchmaschine, Microsoft Office, Microsoft Teams und Windows 11 integriert. Seine Funktionalität variiert, je nachdem wo es verwendet wird. Bei Word kann Copilot behilflich sein, um Dokumente zu skizzieren, Quellen für Informationen in einem Dokument zu suchen, Wortvorschläge zu machen und Schreibhilfe zu leisten. Benutzer können Informationen aus anderen Microsoft Dokumenten, wie z.B. PowerPoint, ziehen, um ihr aktuelles Dokument zu füllen oder sein aktuelles Dokument an die Formatierung von einem anderen Dokument anpassen. In Bezug auf Excel ist Copilot ein Analysetool, um Daten zu visuelle Repräsentationen zu transformieren oder bei automatisierten Prozessen. Der bot kann sogar Trends von Schlüsseldaten, und Fehler korrigieren, Zellen automatisch vervollständigen und Berechnungen erklären. Bei PowerPoint kann Copilot Präsentationen basierend auf Informationen und Dokumente des Microsoft Ökosystems erstellen. Präsentationsfolien können auf Grundlage von spezifischen Instruktionen, wie passend zum eigenen Stil oder der Stimme, gestaltet werden. Google Duet AI ist ein Bestandteil des Google Workspace Apps und Entwicklertools. Der Service unterstützt mehr als 20 Programmiersprachen bei der Code Assistenz. Benutzer können AppSheets verwenden, um intelligente Businessanwendungen und Workflows zu erstellen. Google bietet außerdem die Vertex Platform für Entwicklung an. Bei Google Docs gibt es ein Duet Feature Help Me Write zur Erstellung von Dokumenten. Basis ist ein prompt, der beschreibt, über was der Benutzer gerne schreiben möchte und das AI System arbeitet ein passendes Dokument heraus. Das Feature Smart Chips wird für variable Informationen verwendet. Hierbei kann der Benutzer bestimmte Inhaltspassagen mit Bedingungen bezogen auf einen Ort oder eines bestimmten Businesses versehen. Help Me Write ist für viele Sprachen verfügbar und kann eine Reihe an unterschiedlichen Typen von Dokumenten von Blogs zu Jobbeschreibungen kreieren. Im Falle von Google Sheets bietet Duet eine Datenanalyse an. Durch \gls{nlp} Technologie assistiert Duet Benutzern bei der Dokumentennavigation und Erstellung von benutzerdefinierten Templates, um Daten zu organisieren. Mächtige Tools für Datenklassifizierung unterstützen den Benutzer Datenkontexte zu verstehen. Zusätzlich kann Duet Fehler finden und berichtigen, Zellen automatisch füllen und Vorschläge für eine Analyse machen. Bei Google Slides spielt Duet ebenfalls eine Rolle beim Erstellen und Optimieren von Präsentationen. Bilder können mittels des Help Me Visualize Werkzeugs generiert werden und Bilddesigns können mit verschiedenen Styles angepasst werden. Duet assistiert ebenfalls beim Notizen Schreiben für Präsentationen und dass das Layout der Präsentation konsistent aussieht. Sowohl Copilot als auch Duet können beim E-Mail Schreiben und Video-Meetings behilflich sein. Vor allem Duet bietet Assistenz beim Programmieren. Sowohl Copilot als auch Duet sind noch im preview Prozess, d.h. sie werden in Zukunft kostenpflichtig sein \cite{copilot-duet}. 

\gls{ie} ist eine Unterkategorie von \gls{nlp} und ist relevant für die Identifizierung von relevanten Informationen in Text und die Extrahierung dieser Informationen zu spezifischen Output-Formaten. Bei der \gls{re} werden relevante Einheiten im Text in Beziehung zueinander gesetzt. Normalerweise wird zunächst das Bild in Graustufen konvertiert und eine Texterkennungsphase wird eingeleitet, in der Stellen identifiziert werden, wo sich Informationen befinden. Auf diese Stellen werden Textboxen generiert. Danach liest die \gls{ocr}-Engine wie Tesseract den Inhalt jeder Box im Bild und konvertiert ihn zu Text. Zuletzt wird ein Algorithmus eingesetzt, der eine post-\gls{ocr} pipeline oder language model sein kann, um extrahierte Informationen zu klassifizieren. Beispiele für solche Anwendungen sind Amazon Textract oder Microsoft Azure AI Document Intelligence. Eine bessere Alternative für \gls{ocr}-Anwendungen stellt die \gls{ocr}-freie Lösung Donut dar. Donut ist ein Visual Document Understanding model, was ein Bild als input aktzeptiert und textbasierte Aufgaben lösen kann, wie Informationsgewinnung und Beantwortung von visuellen Fragen. Mit Fokus auf deep learning img2seq models verwendet Donut als Ersatz zur \gls{ocr}-Technologie einen visuellen Encoder (Swin-B Transformer) und einen Textdecoder (BART). Swin-B ist eine Transformer-Architektur, die speziell auf image processing zugeschnitten ist. Bilder werden vom Transformer in Segmente unterteilt und hierarchisch verarbeitet, um lokale und globale kontextuelle Informationen zu erfassen. Die Bildsegmente werden mittels eines shifted window-based multi-head self-attention Moduls analysiert, um die Beziehungen zwischen benachbarten Segmenten zu erfassen. Danach ermöglicht eine two-layer \gls{mlp} dem model das Schema in jedem Segment zu lernen, damit es ein besseres Verständnis der Bildinhalts entwickeln kann. Zum Schluss durchlaufen die tokens des Segments die Schichten, die die Segmente wieder zusammenfügen, was dem model ermöglicht, Informationen zu kumulieren und ein verständlichere Repräsentation des Bildes zu liefern. Der output dieses Prozesses wird Decoder, einem multilingual BART model, übergeben \cite{transformers-ocr}. \\

Amazon Textract ist ein ML-Service, der automatisch Text, Handschrift, Layoutelemente und Daten von gescannten Dokumenten, Bildern oder PDF-Dateien ohne manuelle Konfiguration extrahieren kann. Bei Textract handelt es sich um einen \gls{aws} Cloud-Dienst. Im Gegensatz zu traditionellen \gls{ocr}-Anwendungen kann Textract außerdem strukturierte Informationen aus Tabellen oder Formularen erfassen. Nebst Zeichenerkennung werden auch Formatierungen, sowie die Struktur des Text herausgearbeitet. Der Service kann per Konsole, Command Line Interface (CLI) oder Application Programming Interface (API) verwendet werden. Mittels der Detect Document Text API wird eine \gls{ocr}-Schnittstelle bereitgestellt, um handschriftliche oder gedruckte Texte aus Dokumenten zu entnehmen. Zusätzlich hat die Analyze Document API das Aufgabenfeld, strukturierte Daten einzulesen und Beziehungen bzw. Schlüsselwertpaare aus Tabellen- oder Formularfelder zu erstellen. Extrahierte Informationen sind mit einem Confidence-Score versehen, um dem Benutzer mitzuteilen, wie exakt und verlässlich die Daten sind. Handschriftlicher und gedruckter Text kann mit hoher Genauigkeit und Zuverlässigkeit erkannt werden. Die Extraktion der Daten geschieht sehr performant in kurzer Zeit. Das Pricing des Produkts ist nutzungsabhängig \cite{textract}.


Perplexity AI ist ein Werkzeug für \gls{nlp} mit Dokumenten. Ingesamt kann man bei Perplexity zwischen den generativen AI assistants Perplexity, \gls{gpt4} oder Claude 2 bei Dateiuploads wählen. Eine PDF-Datei als Dateiformat, plaintext oder Code kann hochgeladen werden und Perplexity verwendet dessen Dateiinhalte, um Antworten auf Fragen zum PDF inklusive Zitate mit Quellenangaben zu formulieren. Bei kurzen Dateien wird das gesamte Dokument beim language model analysiert. Umfangreiche PDFs können manuell in Themenbereiche unterteilt werden und als input für \gls{gpt4} für Kreatives Schreiben verwendet werden. Wissenschaftliche Artikel können verglichen, ihre Unterschiede herausgearbeitet, themenverwandte Dokumente durch eine query gefunden, Daten analysiert, Überblicke von verschiedenen Quellen generiert, Daten visualisiert, Grafiken von Quellen erstellt und Text in eine andere Sprache übersetzt werden. Bei der kostenlosen Version ist der Benutzer auf eine bestimmte Anzahl an Anfragen begrenzt \cite{hackernoon-claude}. Claude 2 model von Anthropic ist verfügbar in Perplexity beim Dateiupload. Der AI assistant kann PDFs parsen und die Dokumentstruktur mittels Machine Learning (ML) Techniken erfassen. Es hat eine eingebaute PDF analyser Komponente. Die maximale Dateigröße ist 25 MB. Der Benutzer kann Claude Fragen stellen, die im PDF-Dokument behandelt werden und Claude liefert die Antworten. Direkte Zitate können aus der PDF-Datei gefunden werden und Claude zeigt die Seitenzahl an, wo das Zitat vorkommt. Außerdem kann Claude PDF-Inhalt zusammenfassen. Copilot ist ebenfalls verfügbar, um schnellere Antworten auf prompts in einer menschlichen Art und Weise zu liefern. Bei umfangreichen Dateien werden die relevantesten Dateisegmente analysiert, um wichtige Antworten zu geben. Code kann erklärt und Dateien übersetzt werden. Claude und \gls{gpt4} gelten als intelligentere models, um große Dateien zu parsen \cite{perplexity}.

ChatGPT Plus Mitglieder können seit Oktober 2023 ein PDF-Analysefeature als Beta-Version genießen. Benutzer können PDF-Dateien und andere Dokumente hochgeladen werden und der chatbot kann Zusammenfassungen erstellen und Graphen oder Tabellen basierend auf die Dokument-Daten erstellen \cite{hackernoon-claude}.



\section{Freie PDF Programme und Onlinedienste}


\subsection{Foxit PDF Reader}
Produktseite: \url{https://www.foxit.com/pdf-reader/} \\
Die Firma Foxit hat einen kostenlosen PDF Reader auf den Markt gebracht, der sehr umfangreiche Funktionen bietet. Der Reader kann alle Seiten des PDFs rotieren in 90 Grad-Schritten, Lesezeichen hinzufügen, bestehenden Texte kopieren und farbig markieren, sowie Text, Bilder, Audio und Video hinzufügen. Außerdem gibt es ein Messwerkzeug für 3D-Objekte. Der Reader hat eine Fill \% Sign-Funktion für elektronische Signaturen, bei der man seine Unterschrift zeichnen und sie für die nächste Unterschrift als Bild speichern kann. Digitale Signaturen mit Zertifikat können ebenfalls verwendet werden. Kommentare gibt es in Form von Zeichnungen, Geometrieformen und Text. Zeichnungen können wegradiert und Kommentar-Objekte können skaliert oder verschoben werden. Darüber hinaus können Kommentare zusammengefasst werden und eine erweiterte Suche für PDF-Inhalte ist möglich. Das Arbeiten mit Ebenen ist nur im Foxit PDF Editor möglich und es können keine bestehen PDF-Objekte editiert werden. Der Foxit PDF Reader ist als Webversion, für Windows, macOS, Linux, iOS und Android verfügbar \cite{foxit-reader}. Zusätzlich gibt es auf Foxits Produktseite kostenlose PDF Converter Online Tools. Dort können mehrere PDFs zusammengefügt werden und komprimiert werden. PDF kann zu Word, JPEG, Powerpoint und Excel konvertiert und aus diesen Dateiformaten zurück zu PDF umgewandelt werden. Beim Foxit PDF Editor kann man PDF Dateien vergleichen und Wörter zählen. Im \gls{ocr}-Tool kann man die \gls{ocr}-Ergebnisse manuell korrigieren. Es gibt einen Preflight-Dialog. Ausschließlich in der Pro-Version kann man PDF-Dateien standardkonform zu PDF/A, PDF/E oder PDF/X Dateien umwandeln. Die Undo und Redo-Optionen können Schritte rückgängig machen oder einen rückgängig gemachten Schritt wieder ausführen. Bezüglich der Pro-Variante ist ein Action Wizard für actions eingebaut. Text kann in der Standard-Variante editiert werden und Bilder in der Pro-Variante. Eine Prüfung der Rechtschreibung ist vorhanden. Das Arbeiten mit Ebenen bezüglich anzeigen bzw. verbergen, löschen und neu ordnen ist freigeschaltet. Zusätzlich kann man in der Pro-Version Ebenen importieren, zusammenführen und reduzieren. Seiten können eingefügt, zerteilt, beschnitten, skaliert und basierend auf den Lesezeichen neu angeordnet werden. ChatGPT ist im Editor integriert und bietet die Funktionen: Zusammenfassung, Umschreiben, Übersetzung, PDF Chat mit Fragen Stellen und Antworten Erhalten vom AI assistant, Inhaltserklärung, Korrektur der Rechtschreibung und ein AI Chatbot mit 50 prompts pro Tag \cite{foxit-um}. Eine 14-tägige kostenlose Testversion für den Foxit PDF Editor kann heruntergeladen werden. Die PDF Editor Suite für Einzelpersonen kostet 12,37 Euro im Monat, die Suite Pro 15,75 Euro pro Monat und die Cloud 6,63 Euro pro Monat. Man kann auch eine jährliche Zahlung leisten. Für den Bildungssektor kann lediglich eine sehr vergünstigte jährliche Zahlung von Studenten, Lehrern und Institutionen geleistet werden \cite{foxit-editor}. 

\subsection{PDF24 Tools}
Produktseite: \url{https://www.pdf24.org/en/} \\
Bei den PDF24 Tools handelt es sich um eine Online-Lösung für PDF-Bearbeitung.

\subsection{Aspose.PDF}
Produktseite: \url{https://products.aspose.app/pdf/family} \\

%%%%%%%%%%%%%%%%%%%%%%%%%%%%%%%

\subsection{PDFsam}
Produktseite: \url{https://pdfsam.org/}

\subsection{DrawboardPDF}
Produktseite: \url{https://www.drawboard.com/pdf/pdf}


\section{Kostenpflichtige PDF Programme und Onlinedienste}

\subsection{Adobe Acrobat}
Die nicht-Pro Version von Acrobat kann prüfen, ob es sich bei dem geöffneten PDF-Dokument um ein PDF/A-Dokument handelt und auf dessen Konformität prüfen. Zusätzlich kann man sich die Kompatibilität mit anderen PDF-Dateiformaten PDF/X, PDF/E, PDF/VT und PDF/UA anzeigen lassen. \cite{adobe-pdf-a} Acrobat kann über JavaScript ferngesteuert werden. Dazu muss man die Berechtigung zur Ausführung von JavaScript erteilen. \cite{schneeberger}

\subsection{Adobe Acrobat Pro}
PDF-Inhalte können bearbeitet und Dokumente mit digitalen Signaturen unterzeichnet werden. \cite{adobe-search}
Eine PDF-Datei kann in eine PDF/A-Datei inklusive seiner Varianten, PDF/X, PDF/UA, PDF/VT oder PDF/E konvertiert werden. Außerdem kann die Kompatibilität mit diesen Formaten überprüft werden in Preflight-Profilen. \cite{adobe-pdf-a} Die Barrierefreiheit kann automatisch validiert werden oder ein neues Dokument kann direkt barrierefrei erstellt werden.
Adobe Acrobat Pro kann andere Dokumentenformate wie HTML, DOC, DOCX, TXT und RTF in PDF konvertieren, PDF in andere Dateiformate wie Microsoft Word exportieren oder Dokumente unterschreiben. \cite{adobe-formate} 
Mit dem Werkzeug Scan \& \gls{ocr} kann man Pixelbilder als PDF und gescannte PDF-Dokumente in ein durchsuchbares PDF umwandeln. \cite{adobe-search}
Installiert man Acrobat Pro in Windows, steht dem Anwender ein Adobe PDF-Drucker mit entsprechender \gls{ppd}-Datei zur Verfügung. \cite{schneeberger}

\subsection{Onlinetools von Acrobat}
Produktseite: \\
\url{https://www.adobe.com/de/acrobat/online.html} \\
\url{https://www.adobe.com/de/acrobat/online/convert-pdf.html}
Mit den Adobe Acrobat Onlinetools kann man über den Browser verschiedene Dateitypen in PDF umwandeln, unter anderem PDF in JPEG oder andere Bildformate, PDF Dateien bearbeiten und Kompression anwenden. Die Onlinetools können außerdem PDF in Word umwandeln. \cite{adobe-search}
Der Adobe Acrobat PDF-Converter der Onlinetools kann DOCX, DOC, XLSX, XLS, PPTX, PPT, TXT, RTF, JPEG, PNG, TIFF, BMP, sowie Adobe eigene AI-, INDD- und PSD-Dateien in PDF konvertieren. \cite{adobe-formate} Die kostenlose Version des PDF-Converters kann nur begrenzt oft genutzt werden.

\subsection{pdf-it}
Produktseite: \url{https://www.pdf-it.com/}

\subsection{PDFelement}
Produktseite: \url{https://pdf.wondershare.com/}

\subsection{Soda PDF}
Produktseite: \url{https://www.sodapdf.com/}

\subsection{Nitro PDF Pro}
Produktseite: \url{https://www.gonitro.com/}

\subsection{Ashampoo PDF Pro 3}
Produktseite: \url{https://www.ashampoo.com/de-ch/pdf-pro}

\subsection{Infix 7}
Produktseite: \url{https://www.iceni.com/infix.htm}

\subsection{PDF Director 2 Pro}
Produktseite: \url{https://pdfdirector.de/funktionen}

\subsection{Perfect PDF}
Produktseite: \url{https://soft-xpansion.de/products/perfect-pdf-12/}
\section{PDF zu Word Programme und Onlinedienste}
In PDF ist keine automatische Anpassung des Seiteninhalt-Layouts, wie z.B. in Microsoft Word, möglich. Daher kann ein PDF-Dokument nicht sinnvoll in das Word-Format umgewandelt werden ohne möglicherweise das ursprüngliche PDF-Layout zu beeinflussen und zu ändern, sowie die maximalen Bearbeitungsmöglichkeiten von Word ausschöpfen zu können.
\section{PDF zu Latex Programme und Onlinedienste}
\section{Relevanz von PDF in verschiedenen Marktbranchen}
Durchsuchbare PDFs werden in Verträgen, Rechnungen und Geschäftsunterlagen verwendet, damit Mitarbeiter*innen Informationen gezielter suchen und Daten abteilungsübergreifend effizienter verwaltet werden können. In Forschungsarbeiten und wissenschaftlichen Artikeln werden durchsuchbare PDFs bei der Überprüfung von Quellen oder dem Extrahieren von Zitaten hauptsächlich verwendet. Behörden, Bibliotheken und Unternehmen digitalisieren Dokumente zur Archivierung und wandeln sie in ein durchsuchbares PDF um, was den langzeitigen Bestand der Dokumente sichert. \cite{adobe-search}

Das PDF/A-Dateiformat wird in Bibliotheken und Archiven zur digitalen Archivierung von Büchern, Zeitschriften und historischen Dokumenten verwendet. Auch im Behördenzweig und Verwaltungssektor hat PDF/A für die Aufbewahrung von Verwaltungsakten und rechtlichen Dokumenten seine Existenzberechtigung. Im Gesundheitswegen wird es außerdem zur Speicherung von Patientenakten und medizinischen Unterlagen verwendet. Hingegen im Finanzwesen werden mit ihm Geschäftsunterlagen und Finanzdokumente verwahrt. Unternehmen und Organisationen können mit PDF/A gesetzliche und Compliance-Vorschriften einhalten. \cite{adobe-pdf-a}

PDF/VT-Dateien werden im Direktmarketing verwendet. Personalisierte Werbematerialien erhöhen die Wahrscheinlichkeit einer positiven Reaktion bei den Kundi*nnen auf die Werbebotschaft und verbessert die Bindung von Unternehmen und Kund*innen. Der Transaktionsdruck findet bei Finanzdienstleistungen, Versicherungen und E-Commerce besonderen Anklang. Beliebte Transaktionsdokumente sind Rechnungen, Kontoauszüge, Versicherungspolicen oder Bestellbestätigungen. \cite{adobe-pdf-vt}

PDF-Dokumente mit der PDF/UA-Kennzeichnung stärken den Ruf und die Reputation eines Unternehmens oder einer Organisation durch Engagement für Inklusion. \cite{adobe-pdf-ua}

Digitale Signaturen werden bei digitalen Freigabe-, Abnahme-, Genehmigungs- und Vertragsprozesse verwendet. \gls{pades} wird in Rechtssystemen, Finanzwesen und Regierungssektor eingesetzt. \cite{adobe-pdf-pades} 
\section{Rolle von PDF in der Druckvorstufe und Designbranche}
PDF hat sich zum Standard für den elektronischen Austausch von Dokumenten entwickelt und unterstützt beliebige Seitenformate. Vor allem in der grafischen Industrie wird PDF gerne verwendet, weil es eine plattformübergreifende Visualisierung auf allen Betriebssystemen bietet. Außerdem kann PDF für die schnelle Webansicht optimiert werden. Statische PDF-Dateien sind nicht responsive, d.h. eine optimale Anpassung an verschiedene Endgeräte ist nicht garantiert und sie haben eine lange Ladezeit. Vor allem Bilder lassen sich ohne sichtbaren Qualitätsverlust und geringem Datenvolumen speichern, da sehr leistungsfähige Kompressionsalgorithmen in PDF implementiert wurden. Im Vergleich zu PostScript-Dateien erzielen die kompaktere Codierung von Seiteninhalten, sowie das einmalige Speichern von identischen Bildern und die Verwendung von Kompressionsalgorithmen eine maßgeblich kleinere Dateigröße bei PDF. PDF entwickelte sich zunehmend zum Containerformat für alle grafischen Elemente. Korrekturänderungen in PDF-Dateien sind kurz vor dem Druck noch möglich. Die Produktion von Druckerzeugnissen wird somit wesentlich flexibler und sicherer. Downsampling und Kompression beschleunigen den Transport von der Agentur zum Dienstleistungsbüro enorm. Heute ist es das meist eingesetzte Format bei der Produktion von Druckvorlagen in der digitalen Druckvorstufe. Das Dateiformat funktioniert mit allen Druckern und überwindet Kompatibilitätsprobleme. \\
Textmodifikationen sind am meisten gefordert in der Druckvorstufe. Farbänderungen bzw. -konvertierungen sind ebenso erforderliche Eingriffe. Änderungen am Layout oder Neugestaltungen von Seiten müssen im Originalprogramm erledigt werden. Das modifizierte PDF muss in das Druck-PDF integriert werden. Ständige visuelle Kontrollen der PDF-Datei des Kunden mit der korrigierten Version werden durch eine Fachkraft im Betrieb durchgeführt. Am Ende jeder Korrektur steht die Prüfung auf PDF/X-Konformität, um als technisches Gütesiegel weitere Fehler für den Druck zu reduzieren. Für die Betrachtung von Druckvorstufen-PDF-Dateien sollte immer Acrobat Pro bzw. Adobe Reader verwendet werden, da viele Drittanbieter-PDF-Viewer druckvorstufenrelevante Informationen nicht fehlerlos darstellen können. Dank der Profile für unterschiedliche Geräte und Bedruckstoffe, kann eine Simulation des Ergebnisses am Monitor oder durch einen Prüfdruck bewerkstelligt werden. Dadurch steigt die Reproduktionssicherheit bei gleichbleibender Qualität und die Produktionszeiten verkürzen sich enorm. In den meisten Druckprojekten steht zu Beginn noch nicht fest, wann, wo und auf welchem Bedruckstoff gedruckt werden soll. Die zeitintensive Optimierung der Druckdaten kann durch PDF auf einen späteren Zeitpunkt verlegt werden \cite{schneeberger}. In der Werbebranche werden PDF-Dateien vor allem für Korrekturabzüge verwendet. Ein Korrekturabzug ist eine Skizze bzw. Designvorlage des Werbeprodukts, welches die vom Kunden gewünschte Position und Größe des Druckmotivs auf dem Werbeartikel enthält. Als digitales Layout wird der Korrekturabzug mit dem Kunden abgestimmt und seine Änderungswünsche vor dem Druck entgegengenommen \cite{korrektur}. Zum Erstellen von PDF-Dateien mit geringer Dateigröße für den Online-Bereich muss der Output-Intent mit dem angehängten \gls{icc}-Profil entfernt werden, da \gls{icc}-Profile eine eher höhere Dateigröße aufweisen. Für eine korrekte farbliche Simulation am Monitor in der Druckvorstufe ist es in einigen Fällen wünschenswert, dass ein Output-Intent gesetzt wird. Bei der Erzeugung einer \gls{iso}-konformen PDF-Datei, z.B. bei PDF/X, PDF/A oder PDF/E wird immer ein Output-Intent angelegt. Je nach Standard wird auch das Zielprofil eingebettet. Die direkte Ausgabe von PDF-Dateien über den Druckdialog in Acrobat Pro, ist in der Praxis nicht mehr gängig. Alle gängigen Workflow-Systeme basieren auf einem PostScript-3-\gls{rip}. Intern konvertiert der PostScript-\gls{rip} die PDF-Datei automatisch in eine PostScript-Datei. Die Ausgabe von PostScript aus Acrobat Pro heraus spielt noch für Laserdrucker, Farbkopierer und Broschüren auf Farbdruckern eine Rolle. \\
In der Ausgabe ist die effektive Auflösung maßgeblich. Effektive Auflösung ist die Bildauflösung, die aus der Anzahl der Bildpunkte und der Fläche resultiert, auf der das Bild platziert wurde. Downsampling beeinflusst diese effektive Auflösung. Starke Artefakte fallen im Offsetdruck weniger als im Digitaldruck auf. Bilder sollten nach der PDF-Bearbeitung anschließend geschärft werden. Falls im Druck-PDF ein falscher Zielfarbraum eingestellt wurde, wirkt sich das auf den Gesamtfarbauftrag aus. Eine Farbraumkonvertierung in den, für das Druckergebnis gewünschten Farbraum, muss vorgenommen und der maximale Gesamtfarbauftrag angepasst werden. Bilder, die nicht im Zielfarbraum vorliegen, müssen erst in den Ausgabefarbraum überführt, separiert ausgegeben und Transparenzen reduziert werden. Jegliche Transparenzen in Objekten, vor allem Texte, Grafiken oder Bilder, werden beim Speichern in ein Dateiformat, welches keine Live-Transparenzen unterstützt, reduziert. Zu diesen Dateiformaten gehören u.a. \gls{eps}, JPEG, GIF, BMP, PDF 1.3 und niedriger. Nach der Transparenzreduzierung sollten für die Ausgabe Quellfarbprofile von den reduzierten Objekten entfernt werden, da sie oftmals zu unerwünschten Farbtransformationen im Ausgabegerät führen können. Das Anbringen von Transparenzen auf Objekte mit Volltonfarben ist, mit Ausnahme der Deckkraftänderung, verboten. Bilder können unscharf werden, wenn in den meisten Fällen eine zu niedrige Reduzierungsauflösung global eingestellt wurde. Der Prozess der Transparenzreduzierung sollte auf den letztmöglichen Zeitpunkt in der Produktion von digitalen Daten verschoben werden.
\par
Von einer medienneutralen Druckproduktion spricht man, wenn die Farbraumkonvertierung der neutral abgespeicherten Daten ausschließlich in RGB oder in RGB und CMYK zu einem möglichst späten Zeitpunkt stattfindet. Bei Logos werden medienneutrale PDF-Dokumente nicht bevorzugt. Durch das medienneutrale Konzept gewinnt man an Flexibilität und reduziert Mehrarbeit im Produktionsprozess, z.B. wenn ein später Wechsel des Druckverfahrens und der Papierklasse erfolgt. Der Gegensatz zur medienneutralen Arbeitsweise ist die verfahrensangepasste Arbeitsweise. In der Praxis ist primär der PDF-Empfänger für die Rastereinstellungen des \gls{rip}s verantwortlich. Die Anzahl der Sonderfarben ist in keinem PDF/X-Standard beschränkt. In der Verpackungsindustrie werden Mehrkanalproduktionen verwendet. Der Begriff Binding bezeichnet im Prozess der Druckdatenerstellung die Farbkonvertierung in die gewünschten Druckbedingungen. Beim Eearly Binding liegen alle verfahrensangepasste Daten schon im gewünschten Zielfarbraum vor. Hingegen bei medienneutralen Daten kann bei der PDF-Erzeugung eine Konvertierung in den Zielfarbraum erfolgen. Dies wird als Intermediate Binding bezeichnet. Im Prozess des Late Bindings wird die Konvertierung in den Zielfarbraum bei medienneutralen Daten erst in der Ausgabe vorgenommen. \\
Vor allem Office-Dokumente müssen für den Druck aufbereitet werden. Marketingabteilungen von Industrieunternehmen sind gezwungen, bei der Druck-PDF-Erzeugung auf die zur Verfügung stehenden Werkzeuge, den Microsoft-Office-Programmen bzw. OpenOffice-Programmen zurückzugreifen. Meistens liegen dann alle Objekte in RGB vor, es fehlt die TrimBox und der Anschnitt, sowie die Einzelseiten-PDFs für das Ausschießen sind nicht vorhanden \cite{schneeberger}. Ausschießen bezeichnet den Prozess, dass die einzelnen Seiten des Druckerzeugnisses in der Seitenmontage seitenverkehrt so angeordnet werden, dass sie nach dem Druck und der Weiterverarbeitung (Falzen) die korrekte Reihenfolge vorweisen \cite{kompendium}. Alle RGB-Farbräume müssen nach CMYK umgewandelt werden. Im Besonderen müssen beim Offsetdruck schwarze Texte auf den Farbaufzug K in CMYK gebracht werden. Schmuckfarben werden in CMYK oder in eine andere Schmuckfarbe konvertiert bzw. gemappt. Die hinterlegten Alternativfarbwerte für Schmuckfarben weichen in der Vielfalt der Programme voneinander ab. Ursächlich ist, dass unterschiedliche Pantone Libraries implementiert wurden. Häufig sind in PDF-Dateien verschiedene Farbwerte für dieselbe Farbe anzutreffen. \\
Schriften können bei Einbettung in PDF exakt wiedergegeben werden, unabhängig davon, ob es sich um eine Windows- oder macOS-Schrift handelt. Da allgemeine Schriftinformationen immer eingebettet sind und die Zeilenlängen im Prinzip immer stimmen, können Druckvorstufenbetriebe zumindest immer erkennen, welche Schrift bzw. welcher Schriftschnitt der Ersteller der PDF-Datei ursprünglich vorgesehen hatte, falls die Schrift nicht eingebettet wurde. Fehlende Schriften in PDF-Dateien sind in der Druckvorstufe katastrophal. Beim nachträglichen Einbetten von Schriften in letzter Minute, muss die richtige Schrift gesucht werden, denn der richtige Schriftname beschreibt nicht notwendigerweise dieselbe Schrift mit derselben Codierung bzw. Laufweite. Fehlende Zeichen, überlappende Buchstaben, unregelmäßige Abstände zwischen den Buchstaben, usw. sind gängige Probleme. Schriften können auch durch die PDF-Bearbeitung extrahiert (ausgebettet) werden. In der Druckvorstufe müssen Schriftbezeichnungen für die Ausgabe eindeutig benannt werden. \\
In der Praxis werden PDF-Dateien mit Hilfe von Prüfprofilen überprüft. Agenturen entwickeln nach eigenen Kriterien vorgefertigte Prüfroutinen und einen automatisch generierten Prüfbericht (Report). Der Prüfbericht sollte auf schnelle Fehlersuche im überprüften Dokument optimiert sein. Notwendige Korrekturen können im Originaldokument, bei PDF-Erstellung oder im vorliegenden PDF-Dokument je nach Schweregrad des Fehlers ausgeübt werden. Korrekturen im Originaldokument sind vorzuziehen, da sie Folgefehler minimieren können, jedoch liegt der Agentur oftmals nicht das Originaldokument oder ausschließlich eine unvollständige PDF-Datei des Originaldokuments vor. Eine Erstellung von Korrekturprofilen für Änderungen an der gesamten PDF-Datei kann dabei hilfreich sein. Man kann zwischen einer Vielzahl an voreingestellten Prüfprofilen und einer wesentlich kleineren Menge an Korrekturprofilen im Preflight in Acrobat Pro wählen. Außerdem kann der Prüfbericht als Kommunikationsmittel zwischen Auftraggeber und Agentur dienen, falls der Agentur das Originaldokument nicht ausgehändigt wurde. Das Preflight-Werkzeug in Acrobat Pro ist ein gängiges Werkzeug in der Druckvorstufe. Die Kontrolle von PDF-Dateien mit Preflight kann viel Geld sparen. Im tiefgreifenden Preflight-Check, der ein System vor dessen Einsatz überprüft, kann man die besten Ergebnisse erzielen, wenn das zum Job passende Prüfprofil verwendet wird. Beim Preflighting-Prozess werden fehlerhafte Objekte und Objekte mit speziellen Anforderungen gefunden, ihr Zustand abgefragt und dann dementsprechend ein Fehler, eine Warnung oder eine Information ausgegeben. Im Warnungsfall muss die Agentur entscheiden, ob das Objekt, was die Warnung verursacht hat, sich negative auf das Endprodukt auswirkt. Folglich können durch Preflight frühzeitig Mängel im Produktionszyklus beseitigt werden. Ein Report sollte für die automatisierte Auswertung in weiterführenden Workflow-Systemen im XML-Standard vorliegen. Erfolgreiche selbst erstellte Prüfprofile basieren auf qualitativ hochwertig gewählten Kriterien, wobei nicht zu viele verwendet werden sollten. Prüfprofile können als wichtigste Prüfkriterien das Dokument, Seiten, Bilder, Farbe, Zeichensätze, Rendering und Standard-Konformität enthalten. Nach der Einbringung von Korrekturen im PDF sollte eine technische und visuelle Überprüfung der Druckdatei stattfinden. In einem korrigierten Kunden-PDF sind die, für die Druckerei wichtigen Druck-, Schneide- und Passermarken, sowie der Bereich des Anschnitts meistens unerwünscht. Um diesen Bereich und alle Druckmarken unsichtbar zu machen, wird die CropBox auf die Größe der TrimBox skaliert. PDF-Dateien, die nicht aus professionellen Anwendungen der Druckvorstufe erstellt worden sind, fehlen in den meisten Fällen die TrimBox bzw. die BleedBox. Hat der Datenersteller Schnittmarken sehr genau manuell platziert, so können fast alle PDF-Bearbeitungswerkzeuge die TrimBox bzw. BleedBox davon ableiten. Umschläge und Rücken werden häufig als Druckbogen und einseitiges PDF ausgegeben. Falls der Umschlag versehentlich als PDF-Datei aus Einzelseiten erstellt wurde, müssen diese Seiten zu einem Druckbogen zusammengeführt werden. Für die Weiterverarbeitung sollten leere Seiten und nicht druckbare Objekte, die als solche gekennzeichnet sind oder sich außerhalb des druckbaren Bereichs befinden, entfernt werden. Große Dateien sind bei der Verarbeitung sehr zeitaufwendig und weniger ressourcenschonend. PDF-Dateien können sich durch Output-Intents, eingebettete Dateien, unzureichende Komprimierung von Bildern und Objekten, zu hohe Auflösung und die Präsenz nicht druckbarer Objekte in ihrer Dateigröße aufblähen. Häufig wird in der Agentur ein Abzug-PDF zur Druckfreigabe an den Auftraggeber versandt. Die Seiteninhalte eines Abzug-PDFs sollten gerastert werden, da fehlerhafte Grundeinstellungen im Adobe Reader zu abweichenden Darstellungen des PDFs, im Vergleich zum Aussehen nach dem Druck, führen können. Abzug-PDFs sollten eine Qualität besitzen, die ausreichend ist für eine Ansicht, jedoch für die Produktion bei einem anderen Druckdienstleister nicht genügt und der Output-Intent sollte zwecks Dateigrößenoptimierung entfernt werden. Falls viele Vektoren in einem Druckdokument enthalten sind, würde, trotz PDF-Optimierung, weiterhin eine hohe Auflösung in Vektorgrafiken bestehen bleiben und folglich nur eine geringe Dateigrößenreduktion erzielt werden. Das Rastern des Inhalts bietet einen praxisnahen Lösungsweg. Hierbei werden zu überdruckende Objekte in das Bild eingerechnet und Vektoren gerendert, wodurch eine maßgebliche Dateigrößenoptimierung erzielt werden kann. Speziell bei PDF/X-Dateien müssen bestimmte Inhalte als Metadaten vorliegen, wie der Dokumententitel, das Änderungsdatum der Datei, das PDF-Erstellerprogramm und das Setzes des Trapping Keys. Jeglicher Schutz oder Sperren sollte an einer druckvorstufentauglichen PDF-Datei vermieden werden. Aktionen sind innerhalb druckfähiger Dateien untersagt, unabhängig von JavaScript oder Aktionen von Acrobat. \\
Das Entfernen von Druckmarken ist für den Datenersteller und die Druckvorstufe, die mehrere Seiten auf einem Druckbogen ausschießen muss, ein gängiger Arbeitsschritt. Die letzte Arbeitsphase im PDF-Workflow ist das Überfüllen bzw. Unterfüllen, um weiße Blitzer zu vermeiden. Spezialisten in der Druckvorstufe sollten das Hinzufügen von Traps übernehmen bzw. über dessen Notwendigkeit entscheiden. Der Druckvorstufenbetrieb gibt dem Auftraggeber den \gls{iso}-Standard der PDF-Datei für das Druckerzeugnis vor. Nur wenn sich die Ausgangsdatei mit der zum Druck verwendeten Enddatei in allen Einzelheiten deckt, kann man sich Reklamationen, Geld, Zeit und Diskussionen sparen. Das Ausschießen wird üblicherweise von der Druckvorstufe digital ausgeführt. In Produktionsumgebungen, wo auf Digitaldrucksystemen direkt ausgegeben wird, muss das Ausschießen auch vom Datenersteller selbst vor Ort vorgenommen werden. Ziel des Ausschusses ist eine schnellere und kostengünstigere Produktion durch die bestmöglichste Ausnutzung des Papiers bzw. Materials. Nachfolgenden Schritte (Zusammentragen, Binden und Schneiden) der Endverarbeitung sollten auf einfache Art und Weise durchführbar sein. In der Praxis wird am häufigsten der Ausschuss nach dem Designprozess durch eine Softwarelösung praktiziert. Vor allem in der Magazin- und Zeitungsproduktion, wo Personen zu gleicher Zeit auf unterschiedlichen Seiten arbeiten und diese zu schwankenden Zeitpunkten fertigstellen, wird nach dem Designprozess ausgeschossen. Es gibt jedoch auch die Möglichkeiten im Erstellungsprogramm, beim Drucken auf dem Drucker oder im \gls{rip} auszuschießen. Da der wirtschaftliche Druck auf die produzierenden Unternehmen immer mehr zunimmt, müssen einzelne Produktionsschritte immer mehr zeitlich gestaucht werden und im Idealfall automatisch ablaufen. Für einen automatisierten PDF-Worklflow muss dem Trapping-Key ein eindeutiger Wert zugewiesen werden. Elemente, die nicht für den Druck vorgesehen werden, müssen aus dem druckbaren Bereich entfernt werden \cite{schneeberger}. Das nächste Kapitel stellt die Anforderungen an meine PDF Web App vor.



