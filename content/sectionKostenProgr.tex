\section{Kostenpflichtige PDF Programme und Onlinedienste}
In PDF ist keine automatische Anpassung des Seiteninhalt-Layouts, wie z.B. in Microsoft Word, möglich. Daher kann ein PDF-Dokument nicht sinnvoll in das Word-Format umgewandelt werden, ohne möglicherweise das ursprüngliche PDF-Layout zu beeinflussen und zu ändern, sowie die maximalen Bearbeitungsmöglichkeiten von Word ausschöpfen zu können. PDF-Dateien lassen sich in vielen Programmen einfach über den Druckdialog erstellen. Apple hat das Lesen von PDF-Dokumenten in Apples Vorschau integriert. Viele Webbrowser stellen PDF-Viewer bereit, so Google Chrome seit 2010 \cite{wiki-pdf-de}. Weitere kostenpflichtige PDF-Programme heißen pdf-it, Soda PDF, Nitro PDF Pro, Ashampoo PDF Pro 3, Infix 7, PDF Director 2 Pro und Perfect PDF.

\subsection{Adobe Acrobat Pro}
Produktseite: \url{https://www.adobe.com/de/acrobat/acrobat-pro.html} \\
Acrobat ist ein professionelles Werkzeug, um PDF-Dateien zu erstellen und bereits bestehende Inhalte zu bearbeiten. Text kann hinzugefügt, geändert, formatiert, gelöscht oder markiert werden. Ebenfalls kann der Text von Formularen bearbeitet werden. Bilder können gespiegelt, gedreht, zugeschnitten, ersetzt, ausgerichtet oder angeordnet werden. Das Arbeiten mit Ebenen ist möglich, jedoch eingeschränkter als bei Photoshop. PDF-Seiten können kopiert, ersetzt, gedreht, verschoben, gelöscht, extrahiert oder neu nummeriert werden. Eine PDF-Datei kann in mehrere Dokumente aufgeteilt und Kommentare können hinzugefügt werden. Webseiten können mit der Acrobat-Browsererweiterung in PDF konvertiert werden \cite{adobe-acrobat-um}. Eine PDF-Datei kann in eine PDF/A-Datei inklusive seiner Varianten, PDF/X, PDF/UA, PDF/VT oder PDF/E konvertiert werden. Außerdem kann die Kompatibilität mit diesen Formaten in Preflight-Profilen überprüft werden \cite{adobe-pdf-a}. Die Barrierefreiheit kann automatisch validiert oder ein neues Dokument direkt barrierefrei erstellt werden. Adobe Acrobat Pro kann andere Dokumentenformate wie HTML, DOC, DOCX, TXT und RTF zu PDF konvertieren, PDF in andere Dateiformate, wie Word, exportieren oder Dokumente unterschreiben \cite{adobe-formate}. Der Benutzer kann PDF-Dokumente mit digitalen Signaturen unterzeichnen und das Programm kann die Signaturen validieren. Für Vertraulichkeit können PDF-Dokumente mit Passwörtern und Zertifikaten versehen werden. Interaktive Objekte, Audio und Video können eingebettet werden. In der Pro-Variante gibt es Druckproduktionswerkzeuge, u.a. für Druckermarken, Transparenz-Reduzierung, Farbkonvertierung oder Druckermanagement. Die Preflight-Option ist ebenfalls nur in Acrobat Pro verfügbar \cite{adobe-acrobat-um}. Mit dem Werkzeug Scan \& \gls{ocr} in Acrobat Pro kann man Pixelbilder als PDF und gescannte PDF-Dokumente in ein searchable PDF umwandeln \cite{adobe-search}. Lediglich Acrobat Pro kann PDFs vergleichen und schwärzen. Die Standard-Version ist nur unter Windows verfügbar. Hingegen ist Acrobat Pro für macOS und Windows erhältlich \cite{wondershare-acrobat}. Das Programm gibt es als kostenlosen Acrobat Reader mit sehr eingeschränkten Funktionen, Acrobat Standard für 15,46 Euro pro Monat und Acrobat Pro für 23,79 Euro pro Monat. Hierbei handelt es sich um ein Jahres-Abo mit monatlicher Zahlung. Wählt man das Monats-Abo, so kostet die Standard-Version 27,36 Euro im Monat und die Pro-Variante 35,69 Euro im Monat. Es gibt auch die Möglichkeit das Jahres-Abo mit Vorauszahlung zu erwerben. Die Testversion von Acrobat ist 7 Tage nutzbar. Mit den Adobe Acrobat Onlinetools kann man über den Browser verschiedene Dateitypen in PDF umwandeln, unter anderem PDF in JPEG oder andere Bildformate, PDF Dateien bearbeiten und Kompression anwenden. Die Onlinetools können ebenfalls PDF zu Word-Dokumenten umwandeln. \cite{adobe-search} Der Adobe Acrobat PDF-Converter der Onlinetools kann DOCX, DOC, XLSX, XLS, PPTX, PPT, TXT, RTF, JPEG, PNG, TIFF, BMP, sowie Adobe eigene AI-, INDD- und PSD-Dateien, zu PDF konvertieren. \cite{adobe-formate} Die kostenlose Version des PDF-Converters kann nur begrenzt oft genutzt werden.

\subsection{Wondershare PDFelement}
Produktseite: \url{https://pdf.wondershare.com/} \\
Wondershare PDFelement ist eine KI gestützte PDF-Alternative zu Acrobat. Das Programm wird für Windows, macOS, iOS und Android angeboten und die aktuellste Version heißt PDFelement 10. Bereits bestehende Bilder oder Text des geöffneten PDFs können bearbeitet werden. Vom Autor eingebettete Objekte können in 90 Grad-Schritten gedreht, horizontal und vertikal gespiegelt oder ausgerichtet werden. PDF-Dokumente können mit Highlights versehen und Geometrie bzw. Freihandzeichnungen hinzugefügt werden. Dabei handelt es sich um Kommentartypen, die auch gedreht, skaliert oder verschoben werden können. Das Radiererwerkzeug funktioniert nur für Freihandzeichnungen. Der Hintergrund einer PDF-Datei kann seitenweise mit Farbe gefüllt werden oder durch ein Bild verschönert und leere PDFs können erstellt werden. Ausfüllbare Formulare können sogar automatisch erstellt werden. Es gibt Sicherheitsfunktionen für Passwortschutz, Signaturerstellung und Validierung von Signaturen. PDFelement besitzt ebenfalls eine \gls{ocr}-Funktion. Im Bezug auf AI-Tools gibt es die Optionen Erklären von Inhalten und Code, Korrekturlesen, Zusammenfassen, Chat mit PDF, Übersetzungen und eine Funktion zur Erkennung von AI-erstellten Inhalten, sowie eine automatische Lesezeichenfunktion, basierend auf der PDF-Struktur und Überschriften. Der AI reading assistant nennt sich Lumi und beruht auf ChatGPT. Mehrere PDF-Dateien können gleichzeitig geöffnet sein und Screenshots, Screen Recording, Lineale, Hilfslinien und Hilfsgatter sind verfügbar. Eine Stapelverarbeitung von mehreren PDF-Dateien ist für Konvertierung in andere Dateiformate, Komprimierung, \gls{ocr}, einige Bearbeitungsoptionen, Drucken und Verschlüsselung verfügbar. Konvertierungen von PDF zu Word, Powerpoint, Excel und JPEG, PNG, TIFF, GIF, Epub, HTML, \gls{xml} oder PDF/A sind u.a. möglich. In der kostenlosen 14-tägigen-Testversion mit Anmeldung, bei der dann mehrere Funktionen zur Verfügung stehen, wird ein PDF mit Wasserzeichen gespeichert. Für einen Einzelbenutzer kostet eine Dauerlizenz 119 Euro, ein Jahresabo 89 Euro und ein 2-Jahresabo 109 Euro. Darin sind PDFelement Updates enthalten, sowie 20 GB Document Cloud-Speicher. Es gibt einen Rabatt für Schüler, Studenten und Lehrkräfte \cite{wondershare-um}.  

\subsection{UPDF Pro}
Produktseite: \url{https://updf.com/} \\
UPDF bietet eine kostenlose Testversion mit 1 GB Cloud-Speicher an, die fast alle Features der Pro-Variante freigeschaltet hat. Die kostenlose Testversion enthält mehr Funktionen, wenn man sich bei UPDF registriert. Es gibt bei UPDF verschiedene Anzeigemodi. Man kann auch 2 PDFs nebeneinander anzeigen lassen und sogar scrollen. Mehrere PDFs können in Tabs geöffnet werden. Ein Dokument kann als Slide Show mit dunklem Hintergrund für Präsentationszwecke angezeigt werden. Dabei kann man mit dem Eraser, Pen und Laser Pointer bestimmte Stellen im PDF während des Präsentationsmoduses hervorheben. Lesezeichen und Kommentare können hinzugefügt werden. Als Kommentarvarianten stehen Highlights, Text, Stift und Radierer, Geometrie, Sticker, Stempel und Dateianhänge zur Verfügung. Bei UPDF kann man Seiten neu ordnen, drehen, zerteilen, beschneiden, ersetzen, extrahieren und löschen. Eine \gls{ocr}-Funktionalität ist vorhanden. PDF-Formulare können erstellt und ausgefüllt werden. Eine PDF-Datei kann zu PDF/A, Word, Excel, PowerPoint, CSV, TXT, Bilder oder \gls{xml} konvertiert werden. Aus folgenden Dateiformaten kann ein PDF erstellt werden: CAJ, Word, PowerPoint, Excel, Visio und Bilddateien (PNG, JPEG, BMP und GIF). Wenn man ein PDF-Dokument in der kostenlosen Testversion speichert, wird es mit einem Wasserzeichen versehen. Jedes mal, wenn man ein Dokument in der kostenlosen Testversion mit Login geöffnet hat, fragt UPDF nach, ob man sich die Pro-Version kaufen möchte, was sehr störend ist. Verschlüsselung ist durch eine Passworterstellung fürs Öffnen oder Rechtevergabe gewährleistet. Elektronische und digitale Signaturen mit Zertifikaten werden unterstützt. Man kann ein PDF in die UPDF Cloud hochladen und dem AI assistant Fragen über das hochgeladene PDF stellen. Basierend auf dem PDF in der Cloud, kann der AI assistant das PDF zusammenfassen, übersetzen und Inhalte erläutern. Es gibt auch einen Chatbot, bei dem man Text in den prompt eingeben kann und der AI assistant kann vom Input Text übersetzen, zusammenfassen, erläutern, Rechtschreibung prüfen und Artikel schreiben. Bereits in der Testversion kann man source Dokument-Inhalte in Text- und Bildform bearbeiten. Bilder können um 90 Grad rotiert, extrahiert, zugeschnitten, ersetzt, verschoben, gelöscht und skaliert werden. Texte können skaliert, die Schriftart verändert, fett und kursiv gemacht, verschoben, gelöscht und die Farben variiert werden. Text und Bilder können außerdem neu hinzugefügt werden. Beim Verschieben von Text, was durch das Ziehen der Textbox in verschiedene Richtungen möglich ist, bricht der Text automatisch um und passt sich der Textbox an. Hyperlinks und ein Hintergrund können hinzugefügt, editiert und gelöscht werden. Stapelprozesse sind in der Pro-Version möglich. Als Prozesse kann man zwischen Konvertieren, Kombinieren, Einfügen, Drucken und Verschlüsseln wählen. PDFs können in maximum, high, medium oder low Stärken komprimiert werden. UPDF ist in 11 Sprachen inklusive Deutsch verfügbar. UPDF kann auf Windows, macOS, iOS und Android installiert werden. Für eine jährliche Pauschale von 32,99 US Dollar ist UPDF Pro zu erwerben. Eine Dauerlizenz kostet 52,99 Dollar. Das AI Add-on kann für 59 Dollar pro Jahr gebucht werden \cite{updf-um}. 

\subsection{Mathpix Snip}
Produktseite: \url{https://mathpix.com/snip} \\
Mathpix Snip ist ein PDF-Editor mit Konvertierungsfeatures, dessen Zielgruppe Forscher, Lehrer und Studenten sind, d.h. Mathpix ist auf Forschungsdokumente und Research Papers ausgerichtet. Snip ist als Web App, im Apple Store, Google App Store und in der Huawei AppGallery verfügbar. Mittels \gls{ocr}-Technologie können sogar Formeln, Tabellen und zweispaltige PDFs in folgende Formate konvertiert werden: Word, LaTex, HTML und Markdown. Beim Online-Editor wird eine Mathpix Markdown Sprache angewendet
und man kann auch Markdown, LaTeX und HTML verwenden um eine PDF-Datei zu erstellen. Für Desktop ist Snip ebenfalls für die Plattformen macOS, Windows und Linux verfügbar. Es gibt außerdem eine Google Chrome Extension. Bilder können u.a. zu LaTeX, HTML, MathML, AsciiMath oder CSV konvertiert werden. Text und mathematische Formeln können direkt vom source PDF kopiert werden. Mit Search AI kann man Fragen über das Dokument stellen und erhält Antworten. Collaborative editing ermöglicht dem Ersteller, andere Benutzer zum Editiere hinzuzufügen und man kann live sehen, welcher Nutzer editiert. Handschriftliche Notizen können digitalisiert werden. Außerdem gibt es einen PDF-Reader, der ebenfalls PDFs durch \gls{ocr} digitalisieren kann. Der Reader ist als Web App oder Mobil-Version verfügbar. Die Snip App kostet 4,99 US Dollar pro Monat oder 49,90 Dollar pro Jahr. Die Gebühren für Organisationen, wie Schulen oder Firmen, kosten doppelt so viel. Es gibt eine kostenlose Version mit einer sehr begrenzten Anzahl an zu bearbeitenden PDF-Dateien \cite{snip-um}.