\chapter*{Bachelorarbeit}
\label{chap:zusammenfassung}
%
\textbf{Titel:} Entwicklung einer webbasierter Applikation zur Bearbeitung von PDF-Dateien

\textbf{Gutachter:}
\par
- Prof. Dr. Chunrong Yuan
\par
- Prof. Dr. René Wörzberger

\textbf{Zusammenfassung:} Für die Bachelorarbeit habe ich eine Open Source-Webseite zur Bearbeitung von PDF-Dateien im Firefox-Browser programmiert. Seit Adobe den PDF-Standard entwickelt hat, tauchten zahlreiche, meist kostenpflichtige PDF-Anwendungen auf dem Markt auf, um PDF Dateien zu bearbeiten. Zuerst stelle ich das Hintergrundwissen über das PDF-Dateiformat bereit und zeige die Sicherheitsprobleme des Dateiformats auf. Aufbauend beleuchte ich den aktuellen Stand der Technik und analysiere 7 PDF-Programme. Im späteren Verlauf erkläre ich die Implementierung meiner PDF Web App und praktiziere GUI- und Performance-Tests. Die JavaScript Libraries pdf.js und PDF-LIB sind das tragende Fundament meiner Webanwendung. Sie vereinen alle Funktionalitäten, die man für gängige PDF-Bearbeitungen benötigt. Man kann PDFs lesen, splitten, mergen, erstellen, sowie PDFs mit Texten, Bildern, Geometrie und Zeichnungen versehen. Am Ende diskutiere ich, was noch ausbaufähig ist, welche Funktionalitäten fehlen und welche Features in Zukunft noch geplant sind.

\textbf{Stichtwörter:} PDF-Bearbeitung, Adobe, JavaScript, Vue JS 3, Splitten, Mergen, pdf.js, PDF-LIB

\textbf{Datum:} 04. März 2024


\newpage
\chapter*{Bachelor Thesis}
\label{chap:abstract}
%
\textbf{Title:} Development of a web-based application for editing PDF files

\textbf{Reviewers:}
\par
- Prof. Dr. Chunrong Yuan
\par
- Prof. Dr. René Wörzberger

\textbf{Abstract:} As a result of this bachelor thesis, I developed an open source website for editing PDF files in Firefox browser. Since Adobe released the PDF standard, various mostly fee-paying PDF applications for editing PDF files arose on the market. At first, I provide the backgound knowledge about the PDF file format and depicted the security issues regarding this file format. In addition, I present the state-of-the-art technology and analysed 7 PDF programmes. Later in my work, I explain the implementation of my PDF Web App and execute GUI and performance tests. The JavaScript libraries pdf.js and PDF-LIB form the foundation of my PDF Web App. It unites all functionalities for common PDF editing. One can read, split, merge, create PDF files and add text, images, geometric shapes and drawings. In the end, I discuss which functions are still evolving, lacking and propose features planned in the future.

\textbf{Keywords:} PDF editing, Adobe, JavaScript, Vue JS 3, Splitting, Merging, pdf.js, PDF-LIB

\textbf{Date:} 04 March 2024