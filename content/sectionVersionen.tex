\section{PDF Dateiversionen}
Gestartet mit Version 1.0, war PDF lediglich ein proprietäres Dateiformat von Adobe. Die Freigabe von PDF als offenes und kostenlose Dateiformat führte letztendlich erst zu seiner weltweiten Verbreitung und Anerkennung. Erst im Jahr 2005 entwickelte sich Version 1.4 zu einem internationalen \gls{iso} Standard. Die letzte Dateiversion 2.0 von 2017 ist schon eine Weile her und es hat sich zeitlich nur das PDF/R-Dateiformat später entwickelt als PDF 2.0.

\subsection{PDF 1.0}
PDF 1.0 wurde 1992/1993 entwickelt und ist wurde nicht normiert. 1992 wurde die Spezifikation als Buch verkauft und 1993 das der Spezifikation entsprechende digitale Format entwickelt, welches ausschließlich den RGB Farbraum darstellen konnte. Medien, die einen anderen Farbraum besitzen, wurden in RGB konvertiert. In der Druckindustrie ist jedoch der CMYK-Farbraum von Bedeutung. Folglich ist PDF 1.0 nicht für den Printbereich geeignet. Damals war Adoba Acrobat 1.0 das einzige Programm, mit dem man diese Dateiversion bearbeiten konnte \cite{proj-consult}. 

\subsection{PDF 1.1}
Genauso ist das 1994 kreierte PDF 1.1 keine Norm und implementiert weiterhin nur den RGB-Farbraum, jedoch geräteunabhängig. Zusätzlich benötigte man ein Update von Adobe Acrobat auf Version 2.0. Erstmals ist in diesem Format das Einbetten von Hyperlinks, optional gebunden an Aktionen, mehrseitige Artikel und Threads, Passwortverschlüsselung und Notizen bzw. Anmerkungen, erschienen \cite{proj-consult}. Hier kann man bereits Benutzungseinschränkungen geltend machen, wie Schutz vor unerlaubtem Öffnen des PDF-Dokuments, das Sperren von Teilfunktionen, z.B. Entnahme von Texten und Bildern, sowie deaktiviertes Drucken. Verschlüsselt wird mit einer 40-Bit-Schlüssellänge durch den RC4-Algorithmus. TrueType-Fonts können nativ eingebettet werden und einige geräteunabhängige, d.h. \gls{cie} basierte Farbräume (Internationale Beleuchtungskommission), können eingestellt werden \cite{schneeberger}. Die \gls{cie} ist eine unabhängige Non-Profit-Organisation für Licht, Beleuchtung, Farbe und Farbräume. Sie entwickelt und publiziert hierzu Standards und Messverfahren \cite{wiki-cie-de, wiki-cie-engl}. Binäres Abspeichern der PDF-Datei ist nun möglich, womit eine Reduktion der Dateigröße um 25 \% erreicht werden kann \cite{schneeberger}.

\subsection{PDF 1.2}
Fast alle, für die Druckvorstufe nötigen Parameter aus PostScript, wurden in Version 1.2 umgesetzt. Das 1996 erschienene PDF 1.2 wurde ebenfalls nicht standardisiert, jedoch ermöglichte es erstmals den druckbaren CMYK-Farbraum und Sonderfarben zu verwenden. Des Weiteren wurden interaktive Formularfunktionen, Unicode, Unterstützung der \gls{opi} 1.3 Spezifikationen und eine Druckrasterfunktion implementiert\cite{proj-consult}. \gls{opi} ist ein Workflow-Protokoll, welches in der elektronischen Druckvorstufe verwendet werden kann, um Desktop Publishing Systeme und high-end \gls{ceps} zu verknüpfen und optimiert die Übertragung von hochauflösenden Dateien in Netzwerken \cite{printwiki}. In PDF 1.2 wurden erstmalig AcroForms vorgestellt. Zusätzlich ist das Abspielen von Video und Audio ausschließlich als Link zu einer externen Datei möglich. Composite-Fonts, \gls{cid} fonts und alle \gls{cie} font-basierte Farbräume werden unterstützt \cite{schneeberger}.

\subsection{PDF 1.3}
1999 wurde PDF 1.3 mit fehlender Normierung auf den Markt gebracht und trug seinen Teil 2001 und 2002 bei zur Standardisierung des \gls{iso} PDF/X Standards. Es ist kompatibel mit PostScript 3 und bietet die Neuerungen der 2-Byte-\gls{cid} font-Typen, \gls{opi} 2.0-Unterstützung, Farbraumerweiterung um Sonderfarben durch \gls{icc}-Profile und den DeviceN-Farbraum, weiche Schatten und Farbübergänge bzw. Verläufe in einem auflösungsunabhängigen Modus (Smooth Shading), digitale Signaturen, RC4-Verschlüsselung (40 Bit in Acrobat 4 und 56 Bit in Acrobat 4.05) und JavaScript \cite{proj-consult, schneeberger}. \\ 
Der DeviceN-Farbraum wird auch in PostScript 3 unterstützt und erlaubt die willkürliche Kombinationen von Farbkanälen beim Composite-Druck. Dokumente mit Schmuckfarben müssen auf einem Gerät mit physikalisch getrennten Kanälen für jede verwendete Schmuckfarbe ausgegeben werden. Schmuckfarben sind festgelegte Farbtöne, die nicht aus Prozessfarben gemischt werden. Sie können Kosten sparen und vermeiden Farbschwankungen. Besonders verbreitet sind HKS und Patone \cite{kompendium}. Folglich kann kein CMYK- oder RGB-Gerät Dokumente mit Schmuckfarben farblich korrekt darstellen. Davon sind fast alle Farbdruckersysteme betroffen, sowie die von Adobe Acrobat erzeugte Bildschirmdarstellung von PDF-Dokumenten mit Schmuckfarben. Ohne den DeviceN-Farbraum können Bilder mit Kombinationen von z.B. CMYK und 2 Schmuckfarben oder Schwarz und eine Schmuckfarbe, nicht im Composite-PostScript und Composite-PDF wiedergegeben werden, sondern höchstens mit CMYK als Näherung \cite{helios}. 
Dateianlagen jeglichen Typs können in Form von Streams im Body direkt eingebettet werden. Folglich kann eine PDF-Datei containerisiert werden. 3 weitere Boxen werden dem PDF-Format hinzugefügt: TrimBox, BleedBox und ArtBox. Für Überfüllungen in der Druckvorstufe können Traps als form XObject gespeichert werden. Bilder können auf Basis von Image XObjects für komplexere Pfadresultate maskiert werden. Eine Maske kann Sektionen eines Bildes ausblenden ohne diese zu löschen \cite{schneeberger}.

\subsection{PDF 1.5}
Im Jahr 2003 kam PDF 1.5 auf den Markt und hat sich nicht zur Norm entwickelt. In dieser Version wurden erstmals Ebenen und 16-Bit-Farbtiefe in eingebetteten Bildern implementiert. Des Weiteren wurden gesteigerte Kompressionstechniken einschließlich Objekt-Streams und JPEG 2000-Kompression, sowie eine verbesserte Cross-Reference Tabelle und XRef-Streams implementiert. 12 weitere Seitenübergänge für Präsentationen, verbesserte Unterstützung für Tagged PDF und die Adobe proprietäre Technologie \gls{xfa} wurden außerdem hinzugefügt \cite{proj-consult, schneeberger}. 

\subsection{PDF 1.4}
Der erste PDF \gls{iso}-Standard \gls{iso} 16612-1:2005 wurde zeitlich nach Version 1.5 verabschiedet. In diesem Format sind Transparenzen, JavaScript 1.5, bessere Integration von Datenbanken, Titel, Textblockdefinition, JBIG2-Komprimierung und 128-Bit-RC4-Verschlüsselung erstmalig integriert worden\cite{proj-consult}. Außerdem wurde die Kennzeichnung der Ausgabeabsicht über den Output-Intent mit einer Kompatibilität zu Version 1.3 implementiert \cite{schneeberger}.

\subsection{PDF 1.6}
In diesem \gls{iso} 15930-8:2008 Standard sind erstmals folgende Technologien in diesem Format eingeführt worden: NChannel-Farbraum, welches eine Erweiterung von NDevice mit Sonderfarben bzw. Schmuckfarben ist, JPEG 2000-Kompression, \gls{aes} Verschlüsselung, direkte Einbettung von OpenType-Schriften, 3D-Daten (U3D) und \gls{xml} Formulare \cite{proj-consult}.

\subsection{PDF 1.7}
Veröffentlichung am 1. Juli 2008 ist PDF in Version 1.7 als \gls{iso} 32000-1:2008 als Offener Standard definiert worden. Es wurden komplexer 3D-Objekte, Kontrolle über 3D-Animationen und Einbettung von Standard-Druckeinstellungen wie Papierauswahl, Anzahl der Kopien und Skalierung hinzugefügt\cite{proj-consult}.

\subsection{PDF 2.0}
Der PDF 2.0 Standard wurde erstmals im Juli 2017 veröffentlicht. Die überarbeitete Ausgabe der \gls{iso} 32000-2:2020 bestimmte, dass \gls{xfa} in PDF 2.0 als veraltet markiert werden sollte. Mehr Einstellungsmöglichkeiten und Funktionen sind diesem Format beigefügt worden: Erweiterte Definitionen der Halbtöne der Rasterung, z.B. für Flex- oder Tiefdruck, konsistente Transparenz, erweiterte Tagged PDF-Funktion für Barrierefreiheit, Definition von Sonderfarben über Spektralfarben, alternierende Reihenfolge der zu druckenden Farben, Steuerung der Schwarzpunktkompensation, \gls{aes}-256-Bit-Verschlüsselung, DPart-Metadaten und Einbettung von 3D-Messungen oder Querschnittsdaten \cite{proj-consult}. Spektralfarben bezeichnet die Regenbogenfarben, die entstehen, wenn weißes Licht durch ein Prisma gebrochen wird. Schwarzpunktkompensation stammt begrifflich aus der Bildverarbeitung. Bei diesem Prozess wird der Schwarzwert eines Bildes so angepasst, dass die Darstellung von dunkleren Bereichen verbessert wird. Ist der Schwarzwert falsch eingestellt, so können dunkle Bildbereiche grau oder „ausgewaschen" erscheinen, was zu einer Reduktion an Detail und Kontrast führt \cite{schwarz}.