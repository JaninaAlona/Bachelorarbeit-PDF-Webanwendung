\section{Problemstellung und Anforderungen}
Das Thema der Bachelorarbeit \textit{\glqq Entwicklung einer webbasierten Applikation zur Bearbeitung von PDF Dateien\grqq} habe ich selbst gewählt. Ziel der Arbeit ist eine PDF-Webapplikation zu entwickeln, die gängige Funktionalitäten, die man bei üblicher PDF-Bearbeitung benötigt, bieten soll. Gängige Funktionalitäten sind für mich ein PDF-Reader, das Editieren von Seiten und die grundsätzlichsten Grafikoperationen. Mir war außerdem wichtig, dass die PDF Web App auf allen Desktop-Plattformen Windows, macOS und Linux einsetzbar ist, da PDF genauso ein plattformunabhängiges und hardwareunabhängiges Format ist. Ich habe mir selbst die Vorgabe gestellt, dass die PDF Web App lediglich als Desktop-App funktionieren muss und nicht optimiert für Tablet oder Smartphone sein soll. PDF-Bearbeitung macht für mich am meisten Sinn in einer Desktopumgebung, da der Bildschirm auf Tablets und vor allem bei Smartphones nicht besonders groß bis sehr klein ist. PDF-Bearbeitung nimmt Zeit in Anspruch und am besten setzt man sich an den Schreibtisch vor einen großen Bildschirm und setzt seine PDF-Modifikationen mit der Maus und Tastatur um. \\
Ursprünglich war die PDF Web App als offline Webseite angedacht. Die Webseite sollte man auch ohne Internetverbindung wie ein Programm verwenden können, nur dass keine Installation notwendig ist. Das einzige Programm, was man benötigt für die Ausführung der PDF Web App soll ein Browser sein. Durch Öffnen der index.html im Browser kann man die PDF Webapp auf einfache Art und Weise starten und benutzen. Eine Anforderung an mich selbst ist, dass die PDF Web App als Open Source-Projekt für jeden kostenlos nutzbar sein soll, da ich selbst nur kostenlose PDF-Bearbeitungsprogramme verwende. So soll die PDF Web App auch u.a. für Studenten mit wenig Geld zur Verfügung stehen. Das Open Source-Projekt soll als öffentliches Repository auf Github vorhanden sein und jeder kann es downloaden. Da das PDF-Dateiformat ein offenes Dateiformat ist, finde ich, dass die Bearbeitungsprogramme für PDF-Dateien ebenfalls kostenlos sein sollten. Auf dem öffentlichen Github Repository soll man lediglich die PDF Web App runterladen können ohne Werbeprogramme oder ähnliches. \\
Des Weiteren wollte ich mich nur auf HTML, CSS und JavaScript beschränken. Folglich ist die PDF Web App eine reine Frontend Web Anwendung und ich habe kein Backend oder eine Datenbank implementiert. Dadurch habe ich die Programmierung und den Verwaltungsaufwand der PDF Web App simpler gemacht. Die Bedienung sollte möglichst intuitiv, eingängig und einfach gehalten sein, damit ein Nutzer nicht seitenlange Tutorials lesen muss und die Funktionen der Einstellmöglichkeiten möglichst durch Ausprobieren erfassen kann. Um diese Anforderung zu erreichen, sollte eine GUI mit simplem, übersichtlichem und strukturiertem Design verwendet werden. \\
Eine weitere Vorgabe der Gutachterin der Bachelorarbeit ist eine Marktübersicht inklusive der Beschreibung anderen PDF Wettbewerberprogramme zu erarbeiten. \\

Daraufhin habe ich mir einige freie und kostenpflichtige PDF-Programme und Onlinedienste ausgesucht: 

\begin{itemize}
	\item Adobe Acrobat
	\item foxit
	\item pdf-it
	\item PDF24
	\item PDFelement
	\item Soda PDF
	\item Nitro PDF Pro 
	\item Ashampoo PDF Pro 3
	\item Infix 7
	\item PDFsam Basic und Advanced
	\item DrawboardPDF
	\item PDF Director 2 PRO
	\item Perfect PDF
\end{itemize}


\subsection{Vorgaben für den Funktionsumfang der PDF Web App}
Bei der Planung der PDF Web App habe ich mir einen Funktionsumfang ausgesucht.\\
Im Prinzip sollte die PDF Web App aus 8 Modulen bestehen:
\begin{itemize}
	\item Reader
	\item Creator
	\item Merger
	\item Splitter
	\item Writer
	\item Drawer
	\item Geometry Editor
	\item Images Editor
\end{itemize}

Der Reader sollte eine Navigationseinheit enthalten, die die aktuelle Seitenzahl anzeigt, zu einer beliebigen Seite springen kann, eine Zoomfunktionalität implementieren und man soll zur vorherigen oder nächsten Seite blättern können. Zusätzlich soll eine Druckfunktionalität enthalten sein. Der Creator ist ein Modul zur Erstellung von leeren PDF-Dokumenten als Download mit beliebiger Größe, Anzahl an Seiten, Möglichkeiten die Orientierung (Portrait, Landscape, quadratisch) zu justieren und Presets zur DIN A Größe. Es gibt 2 Module zur Verwaltung von PDF-Dokumentkombinationen. Zum einen soll der Merger mehrere PDF-Dateien zusammenfügen und als eine Datei speichern können, dabei soll die Reihenfolge der einzelnen PDF-Dateien in einer Liste anpassbar sein und eine Druckfunktion enthalten. Hingegen soll der Splitter ein PDF zerteilen und als einzelne PDF-Dateien speichern können. Dabei gibt es verschiedene Split-Operationen: Nach jeder, nach jeder ungeraden, nach jeder geraden, nach einer bestimmten und nach einer Liste an Seiten kann der Splitter das PDF zerteilen. In meiner PDF Web App ist Speichern ein Synonym für den Download eines Output-PDFs. Beim Download soll man immer einen benutzerdefinierten Dateinamen setzen können. Diese 4 Module sind im Diagramm \ref{fig:modules4} graphisch dargestellt. 

\begin{figure}[!htb]
	\centering
	\includegraphics[width=0.9\textwidth]{"images/app-funktionen-anforderungen.png"}
	\caption{Anforderungen an den Reader, Creator, Merger und Splitter der PDF Web App}
	\label{fig:modules4}
\end{figure}

Der Editor besteht aus den restlichen 4 Modulen. Alle Editoroperationen beziehen sich lediglich auf bereits hinzugefügte Elemente. Es sollen keine schon im PDF bestehenden Objekte bearbeitet werden können. In jedem Editorteil soll es möglich sein zu drucken und das PDF zu downloaden, sowie alle neuen Elemente auf einen Schlag zu löschen. Außerdem sollen Elemente in beliebiger Reihenfolge übereinander stapelbar sein, d.h. man kann Elemente in den Vordergrund holen oder weiter nach hinten verschieben. Im Diagramm \ref{fig:editor}, was zeigt wie die Editormodule aufgebaut sind, ist die zuletzt genannte Funktionalität mit z-axis order gekennzeichnet. Mittels des Writers soll Text hinzugefügt werden können. Dieser neue Text kann gelöscht, verschoben, gedreht, die Schriftgröße und font family angepasst und eingefärbt werden. Als Font kann man zwischen Times Roman, Helvetic oder Courrir wählen bzw. einen benutzerdefinierten Font verwenden. Im Drawer hat man die Möglichkeit zu zeichnen und zu radieren. Dabei kann man die Größe und Farbe des Stifts bzw. Radierers anpassen. Das Geometrymodul soll geometrische Formen, wie Rechteck, Dreieck oder Ellipse hinzufügen, ihre Größe und Farbe verändern, Formen löschen, verschieben und drehen können. Zuletzt soll der Images-Editor Bilder vom Dateisystem hinzufügen, bereits platzierten Bilder löschen, verschieben und drehen können.


\begin{figure}[!htb]
	\centering
	\includegraphics[width=0.9\textwidth]{"images/editor-funktionen-anforderungen.png"}
	\caption{Anforderungen an den Editor der PDF Web App}
	\label{fig:editor}
\end{figure}


\subsection{Qualitätsanforderungen}
Ich habe den Ansporn, dass die PDF Web App sehr stabil und mit minimalen zeitlichen Verzögerungen arbeiten soll. Konkret bedeutet das, dass sich die Seite schnell aufbauen und die Seiten von geöffneten PDFs in einer akzeptablen Zeitspanne laden sollen. Genauso wichtig ist mir eine gute Funktionalität, d.h. so wenig Laufzeitfehler wie möglich und dass die PDF Web App immer genau gleich funktioniert und nicht abstürzt. Maximale Effizienz sollte gegeben sein, das die PDF Web App als reine Frontend App ohne Backend auskommen soll. Es müssen keine Daten gespeichert werden. Genauso wichtig ist eine gute Benutzbarkeit. Im Einzelnen ist damit gemeint, dass der Anwender sich die Funktionalität so gut wie möglich selbst durch Ausprobieren erschließen kann. Die Buttonbeschriftung sollte selbsterklärend sein und die Layoutfarben so eingesetzt, dass sie eine Designkonsistenz zur Vermittlung von Funktionalitätsklassen aufweist. Die Änderbarkeit und Erweiterbarkeit sollte auf einfache Weise möglich sein, sodass auch andere Entwickler an der PDF Web App arbeiten können. Des Weiteren sollte die Portierbarkeit hoch sein, denn die PDF Web App sollte auf Windows, macOS und Linux immer gleich funktionieren. Eine Tablet oder Mobilversion ist nicht geplant. Mir war wichtig, dass man die PDF Web App als Offlineseite verwenden kann. Folglich habe ich keine \gls{cdn} Links verwendet, sondern die verwendeten Libraries als Dateien eingebunden im Ordner lib, damit man die PDF Web App im Browser durch index.html öffnen kann und keinen Internetzugang benötigt.
