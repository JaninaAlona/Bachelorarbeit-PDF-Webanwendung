\section{PDF Implementierung}
PDF ist eine vektorbasierte \gls{pdl} (Seitenbeschreibungssprache) und basiert auf dem PostScript-Format. Der \gls{mime}-Type von PDF heißt application/pdf. Eine \gls{pdl} beschreibt den Seitenaufbau, wie die Seite in einem Ausgabeprogramm bzw. Ausgabegerät, z.B. einem Drucker, aussehen soll. \gls{pdl}s können Seiten mit Vektoren beschreiben. Vektorielle Seitenbeschreibung bedeutet, dass das Format beliebig skalierbar ist ohne Qualitätseinbußen, jedoch eingebettete Pixelgrafiken erhalten durchaus mittels genügend Skalierung Qualitätsverluste. Das Ausgabeformat ist normalerweise nicht zur weiteren Bearbeitung vorgesehen. An den Drucker wird durch die \gls{pdl} ein Datenstrom der zu druckenden Aufgabe erzeugt und an den Drucker gesendet. Der \gls{rip} eines Druckers wandelt die Bildschirmausgabe in die gerasterte Druckerausgabe um. Viele APIs der Hardwareabstraktionsschicht im Computer wie \gls{gdi} oder OpenGL können in \gls{pdl} ausgeben. Speichert ein Satzprogramm den Seitenbeschreibungscode eines Dokuments in einer Datei, müssen Drucker die \gls{pdl} nicht selbst verarbeiten. Im Common Unix Printing System, der Standard-Druckersteuerung von Linux hat der PostScript und der PDF-Interpreter ghostscript die Aufgabe eines \gls{rip}, d.h. er ist für die Umwandlung in die gerasterte Druckausgabe auf dem Drucker zuständig. Zudem stellen \gls{pdl}s eine Schnittstelle zum Quellcode eines Dokuments bzw. zu Programmen, die Quellcode verwalten oder das Dokument formatieren können, dar. Die \gls{pdl} PDF erweitert die Funktionalität der Vorschau am Bildschirm um anklickbare Links (Hypertextfunktionalität), die die Navigation im Dokument erleichtern oder um URLs, die sich automatisch im Browser öffnen. \cite{wiki-pdl} PDF-Dateien sind komprimiert und haben üblicherweise einen Bruchteil der Größe des Ursprungsformats oder von Bilddateien. 

\subsection{PostScript}
Die PostScript \gls{pdl} wurde in den 1980er Jahren von Adobe erfunden. \cite{adobe-postscript} Hinzu wurden weiter PostScript Technologien entwickelt, die aus der stackorientierten, Turning-vollständigen, interpretierten Programmiersprache PostScript \cite{wiki-postscript}, Grafik-, Textformatierungsanwendungen, Treibern und Abbildungssystemen bestehen. PostScript hat sich als Industriestandard etabliert. Die letzte Version ist PostScript 3 von 1997. Seine primäre Anwendung gemäß des Adobe imaging models findet sich in der Beschreibung von Textdarstellung, graphische Formen und Bildern auf gedruckten oder auf dem Bildschirm angezeigten Seiten. Dabei ist die Beschreibung des Dokuments geräteunabhängig. PostScript unterstützt unter anderem beliebige geometrische Formen, Zeichenoperationen in Graustufen, RGB, CMYK und CIE (Yxy-Farbraum) und  vorinstallierte oder benutzerdefinierte Fonts und Digitalbilder jeglicher Auflösung je nach Farbmodell und ein allgemeines Koordinatensystem.
Dabei werden die Textzeichen eines Fonts, gemäß des Adobe imaging models, als graphische Formen betrachtet auf denen Grafikoperationen möglich sind. Das Koordinatensystem unterstützt alle linearen Transformationen, die auf alle Seitenelemente angewandt werden können. Die Seitenbeschreibung in PostScript kann auf jedem Gerät, was einen PostScript Interpreter implementiert, gerendert werden. In diesem Prozess wird die high-level PostScript-Beschreibung in low-level Rasterdatenformate für das jeweilige Gerät übersetzt. PostScript Programme können erstellt, übertragen und als ASCII Quellcode interpretiert werden. \cite{adobe-postscript}

\subsection{Adobe imaging model}
PDF und die PostScript Programmiersprache haben das Adobe imaging model als Gemeinsamkeit. Es kann nahtlos zwischen PDF und PostScript konvertiert werden und beide erzielen das gleiche Ausgabeergebnis beim Druck. Dennoch fehlt PDF das general-purpose Framework der PostScript Programmiersprache. Stattdessen stellt ein PDF Dokument eine statische Datenstruktur optimiert für den random access dar und enthält zusätzlich Seitennavigationsinformationen für interaktives Lesen. Das high-level imaging model beschreibt die Elemente, die auf der Seite dargestellt werden, also Text, Geometrie oder Bilder, als abstrakte graphische Elemente, anstatt als Pixeldefinitionen. Dadurch wird das imaging model zu einem geräteunabhängigem Modell und kann hochwertige Ausgaben auf vielen verschiedenen Druckern und Bildschirmen liefern. Die \gls{pdl} beschreibt dieses imaging model. Eine Anwendung generiert zuerst die geräteunabhängige Beschreibung des gewünschten Ausgabegeräts in der \gls{pdl}. Daraufhin interpretiert eine Firmware oder Software eines spezifischen Ausgabegeräts für Rasterausgaben die Beschreibung und rendert sie im Ausgabegerät. Hierbei hat die \gls{pdl} die Rolle eines Austauschstandards für die Übertragung und Speicherung von druckbarem oder auf Displays darstellbaren Dokumenten. \cite{adobe-postscript} Später wurde das imaging model für die Unterstützung von Transparenzen erweitert. Diese Funktionalität wurde speziell für PDF implementiert und wird nicht von PostScript unterstützt. Bei PostScript überschreibt das zuletzt gezeichnete Objekt alle darunterliegenden Objekte im Hintergrund. \cite{schneeberger}

\subsection{Dateiformataufbau}
PDF-Dateien enthalten Dokumentdaten in binärer Form. Ein Dokument entspricht immer einer Datei. Das Einbetten von binären Dateien in beliebigen Formaten oder anderer PDF-Dateien ist möglich. Die Struktur besteht im Wesentlichen aus 4 Komponenten. Zunächst spezifiziert der Header die Version der PDF-Spezifikation. Der Body enthält die Daten der Objekte, aus denen das Dokument besteht und die Cross-Reference Table deckt die Informationen über die Position der Objekte in der Datei ab. Zuletzt definiert der Trailer die Position der Cross-Reference Table und von speziellen Objekten im Body. Die Objekte im Body sind in einer komplizierten hierarchischen Struktur, dem Dokument, verknüpft. Zur Dateigrößenoptimierung werden komplexe Verbindungen zwischen den Daten hergestellt und die Daten eines mehrfach vorkommenden Objektes müssen nur einmal gespeichert werden. \cite{softx} Die PDF-Dokumentstruktur ist auf einen schnellen, wahlfreien Zugriff (random-access) auf beliebigen Seiten optimiert. Im Unterschied dazu sind PostScript-Dateien seriell organisiert. \cite{schneeberger}
\par
Metadaten werden durch den XMP Standart kodiert und als \gls{xml} formatierte Daten in PDF-Dateien abgelegt. Unicode wird in den Metadaten unterstützt. \cite{softx} \\
Die XRef-Tabelle enthält die Positionen der indirekten Objekte innerhalb der Datei. Streams binden Dateien ein. 

\subsection{Implementierung von Fonts}
Die Beschreibung von Glypen ist bei eingebettete Schriften als Datenstrom im Eintrag FontFile registriert. Falls die Schrift nicht eingebettet wurde fehlt dieser Eintrag. Ein optionales Unicode-Mapping ToUnicode ist von Nöten, damit die Glyphen auch über Unicode verarbeitet werden kann. Ist dieses Mapping nicht vorhanden, so kann keine Textsuche und das Kopieren von Text stattfinden. Fehler im Mapping oder Modifikation von Schriften können zu falsche Ausgabebuchstaben, mangelnde Wiederverwendung und fehlerhafte Textkonvertierung führen. Jede Glyphe im Dokument wird über einen Character-Code prozessiert. Daraufhin erfolgt eine Zuordnung des Character Codes zum hinterlegten Encoding (Mapping). Zuletzt wird die Glyphe im aktuellen Font über die Glyphen-ID zum Zeichen der Glyphe aufgerufen. Folglich erzielt das Mapping des Codes und der Aufruf der Glyphe die benötigte Konturbeschreibung. Schriftsubstitution findet immer dann statt, wenn der Character-Code nicht mit der Encoding-Tabelle übereinstimmt. Häufig fehlen bestimmte Glypen im Font. Falls eine Outline-Beschreibung des Fonts zum Erstellungszeitpunkt nicht verfügbar ist, wird die Einbettung des Fonts verhindert. Dies kommt vor allem dann vor, wenn ein Font ein Schutzflag besitzt. Weitere Probleme bei der Schrifteingbettung sind u.a. Laufweitenfehler in Schriften, Fehler in der Buchstabenbeschreibung oder beim Cachen von Fonts. Zwecks der Schriftsubstitution müssen folgende allgemeine Informationen zu einem Font in der PDF-Datei gespeichert sein: Name der Schrift, Typ, Subtyp, Schriftstärke, Zeichenbreite, Laufweite, maximale Ausprägung der FontBox, Dickteninformationen, Positionsangaben über Versal- und x-Höhe und Winkel für Italic (kursiv). Diese Informationen sind selbst bei nicht eingebetteten Schriften vorhanden. \cite{schneeberger}

\subsection{Implementierung von Transparenzen}
Wird eine PostScript oder PDF-Datei erstellt, werden die Transparenzen vom Flattener reduziert (verflacht). Um den gewohnten visuellen Effekt der Transparenzen beizubehalten gibt es unterschiedliche Verfahren bei der Reduzierung auf Vektor- und Pixelebene. 