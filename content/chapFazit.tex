\chapter{Fazit und Ausblick}
Der Markt an PDF-Programmen ist mannigfaltig, jedoch werden wenige Open Source-PDF-Tools angeboten. Die meisten Anbieter verlangen hohe Gebühren für ihre PDF-Dienste. Ist das nachvollziehbar, wenn man bedenkt, dass PDF ein freies Format ist? Sollte man ein proprietäres Programm erwerben, welches Sicherheitsfunktionen für ein unsicheres Dateiformat unterstützt? Mit meiner PDF Web App beabsichtige ich eine kostenlose Alternative zu der PDF-Programmlandschaft anzubieten. Meine PDF Web App sollte sich auf die wichtigsten, gebräuchlichsten Operationen für PDF-Dateien beschränken. Sie sollte verständlich und intuitiv benutzbar sein. Im Bearbeitungsprozess sollte die Bedienung durch Ausprobieren der Schaltflächen leicht und eingängig sein. Herausstechend bei der Analyse anderer PDF-Programme ist, dass Zeichenoperationen eingeschränkt funktionieren. Die meisten Programme implementieren Zeichnungen lediglich als Kommentarobjekte. Eine Radierfunktion fehlt bei vielen gänzlich. Bei der Konzipierung der PDF Web App sollten vor allem die Zeichenfunktionen einen breiten Möglichkeitsspielraum offerieren. Meine Ebenenimplementierung besitzt eine vollkommen neue Funktionsweise, die nicht analog zu Photoshop-Ebenen arbeitet. Die selection und deselection filter sind in keinem anderen Programm anzutreffen. Diese Programmdesignentscheidung erleichtert das Editieren von mehreren Elemente im Editor maßgeblich. Obwohl mehrere PDF-Programmvariationen im Internet zu finden sind, sticht meine PDF Web App mit einzigartigen Features heraus.
\par
In puncto Dokumentensicherheit ist die PDF-Spezifikation von PDF 2.0 noch maßgeblich ausbaufähig. Allein die Dateistruktur, die verschlüsselte und unverschlüsselte Inhalte erlaubt, ermöglicht einem Angreifer schadhafte Inhalte in das PDF-Dokument des Opfers einzuschleusen. Folgende Mängel sollten in Zukunft behoben werden: Die Spezifikation des Dateiformats ist zu ungenau beschrieben, sie lässt zu viel Toleranz in der Implementierung von PDF-Programmen zu und sogar beschädigte PDF-Dateien können in Readern oder Editoren geöffnet werden. Das incremental update-Feature sollte in einer anderen Art und Weise realisiert werden, da in der aktuellen Form zu viel Schaden angerichtet werden kann. Die 256-Bit-\gls{aes}-Verschlüsselung ist zwar ein guter Ansatz im Bezug auf Dokumentensicherheit, jedoch sollte sie mit einem \gls{mac}-Integritätsschutz verbunden sein. Eine überarbeitete Spezifikation als PDF-Version 3.0 ist längst überfällig und ein robusteres Dateiformat ist vonnöten. PDF-Reader und -Editoren sollten die Datei aus sicherheitsrelevanten Gründen auf strukturelle Abweichungen prüfen, bevor sie das Dokument anzeigen. Die Struktur des Dateiformats sollte überarbeitet und konkreter definiert werden, damit Angriffe auf PDF reduziert werden können. Im Jahr 2019 wurde auf der Electronic Document Conference die Blockchain-Technologie zur Verschlüsselung von Dokumenten vorgeschlagen. Die Blockchain-Technologie könnte verwendet werden, um digitale Signaturen in der Blockchain zu speichern, anstatt in der PDF-Datei selbst. Dieser Ansatz würde schnellere Signaturprozesse anstoßen, Dokumentensicherheit unterstützen, sowie Echtzeitverfolgung von Signaturen ermöglichen und Versionskontrolle implementieren \cite{pdf-association-blockchain}.
\par
Generative AI ist die Zukunft und wird möglicherweise in den nächsten Jahren in fast allen PDF-Programmen Verwendung finden. Die Analyse von Dokumenten durch \gls{ocr}-Technologie wird möglicherweise durch \gls{ocr}-freie Technologien mittels transformer abgelöst, wie im genannten Donut-Beispiel. Sie liefern bessere Ergebnisse bei der Erkennung von Text auf Bildern. Bisher existiert noch keine Generative AI-Implementierung, die komplette PDF-Dokumente auf Basis eines prompts generieren kann. Ich denke ein Text to Document-AI-Tool wird bald aufkommen, da PDF ein wichtiges und verbreitetes Format weltweit ist. Die Funktion Text to Template von Adobe Firefly in Adobe Express geht schon in die richtige Richtung. 