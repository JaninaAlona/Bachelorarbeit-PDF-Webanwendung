\section{PDF Vorstellung}
Die Popularität von \gls{pdf} Dateien ist seit 2008 rasant angestiegen in der globalen Informationsübertragung. Täglich werden weltweit 2,5 Milliarden PDF Dokumente erzeugt. Seine Beliebtheit verdankt PDF vor allem an der plattformübergreifenden Kompatibilität (Desktop-Computer, Tablets und Smartphones), denn PDF Dokumente ist auf mehr als 1,5 Milliarden Geräten ohne zusätzliche Software lesbar. Über 80\% der geschäftlichen Dokumente werden als PDF Datei weitergegeben. \cite{formilo} 90 \% der Büroangestellten wollen auf das PDF Dateiformat nicht mehr verzichten. Drei Viertel aller archivierten Dokumente sind PDF Dokumente. \cite{kofax} Das PDF Dateiformat steht für Plattformunabhängigkeit und Hardwareunabhängigkeit, Konsistenz in Formatierung und Layout und soll ein möglichst originalgetreues Druckergebnis liefern. Der Leser soll ein PDF Dokument immer in der Form betrachten und ausdrucken können wie vom Ersteller des Dokuments festgelegt.
\par
PDF wurde 1993 von Adobe veröffentlicht und ging aus dem 1991 von Adobe-Mitbegründer John Warnock gestarteten „Project Camelot" hervor. Ziel dieses Projektes war, ein Dateiformat für elektronische Dokumente zu kreieren, sodass diese Anwendungsprogramm, Betriebssystem und Hardware unabhängig originalgetreu wiedergegeben werden können. Anfangs war der Adobe Reader kostenpflichtig und PDF war für einen langen Zeitraum ein proprietäres Dateiformat, welches offengelegt im PDF Reference Manual von Adobe dokumentiert ist. Die \gls{iso} übernahm PDF 2007 in den Standardisierungsprozess und seit der Veröffentlichung von PDF Version 1.7 am 1. Juli 2008 gilt PDF als Offener Standard. \cite{wiki-pdf-de} Der Begriff Offener Standard bezeichnet einen Standard, der für alle Teilhaber am Markt besonders leicht zugänglich, weiterentwickelbar und einsetzbar ist. Das bedeutet, dass der Standard von einer gemeinnützigen Organisation eingeführt, veröffentlicht, weiterentwickelt und gleichmäßige Einflussnahme aller interessierten Parteien ermöglicht. \cite{wiki-standard} Heute gehört PDF zur \gls{iso} 32000-Standards Normenserie und wird aktuell von der PDF Association weiterentwickelt.
