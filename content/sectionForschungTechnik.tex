\section{Aktueller Stand von Forschung und Technik}
Artificial Intelligence (AI) Tools können können das Arbeiten mit PDF erleichtern. \gls{ie} ist eine Unterkategorie von \gls{nlp} und ist relevant für die Identifizierung von relevanten Informationen in Text und die Extrahierung dieser Informationen zu spezifischen Output-Formaten. Bei der \gls{re} werden relevante Einheiten im Text in Beziehung zueinander gesetzt. 


Claude 2 von Anthropic ist ein AI assistant, der PDFs parsen und die Dokumentstruktur mittels Machine Learning (ML) Techniken erfassen kann. Es hat eine eingebaute PDF analyser Komponente. Die maximale Dateigröße ist 10 MB. Der Benutzer kann Claude Fragen stellen, die im PDF-Dokument behandelt werden und Claude liefert die Antworten. Direkte Zitate können aus der PDF-Datei gefunden werden und Claude zeigt die Seitenzahl an, wo das Zitat vorkommt. Unglücklicherweise ist Claude zurzeit nicht für Deutschland verfügba \cite{hackernoon-claude}r.

Perplexity AI ist ein Werkzeug für \gls{nlp} mit Dokumenten. Eine PDF-Datei als Dateiformat, plaintext oder Code kann hochgeladen werden und Perplexity verwendet dessen Dateiinhalte, um Antworten auf Fragen zum PDF inklusive Zitate mit Quellenangaben zu formulieren. Bei kurzen Dateien wird das gesamte Dokument beim language model analysiert. Umfangreiche PDFs können manuell in Themenbereiche unterteilt werden und als input für \gls{gpt4} für Kreatives Schreiben verwendet werden. Wissenschaftliche Artikel können verglichen, ihre Unterschiede herausgearbeitet, themenverwandte Dokumente durch eine query gefunden, Daten analysiert, Überblicke von verschiedenen Quellen generiert, Daten visualisiert, Grafiken von Quellen erstellt und Text in eine andere Sprache übersetzt werden. Bei der kostenlosen Version ist der Benutzer auf eine bestimmte Anzahl an Anfragen begrenzt \cite{hackernoon-claude}.

ChatGPT Plus Mitglieder können seit Oktober 2023 ein PDF-Analysefeature als Beta-Version genießen. Benutzer können PDF-Dateien und andere Dokumente hochgeladen werden und der chatbot kann Zusammenfassungen erstellen und Graphen oder Tabellen basierend auf die Dokument-Daten erstellen \cite{hackernoon-claude}.

Amazon Textract ist ein ML-Service, der automatisch Text, Handschrift, Layoutelemente und Daten von gescannten Dokumenten, Bildern oder PDF-Dateien ohne manuelle Konfiguration extrahieren kann. Bei Textract handelt es sich um einen Cloud-Dienst von Amazon Web Services. Im Gegensatz zu traditionellen \gls{ocr}-Anwendungen kann Textract außerdem strukturierte Informationen aus Tabellen oder Formularen erfassen. Nebst Zeichenerkennung werden auch Formatierungen, sowie die Struktur des Text herausgearbeitet. Der Service kann per Konsole, Command Line Interface (CLI) oder Application Programming Interface (API) verwendet werden. Mittels der Detect Document Text API wird eine \gls{ocr}-Schnittstelle bereitgestellt, um handschriftliche oder gedruckte Texte aus Dokumenten zu entnehmen. Zusätzlich hat die Analyze Document API das Aufgabenfeld, strukturierte Daten einzulesen und Beziehungen bzw. Schlüsselwertpaare aus Tabellen- oder Formularfelder zu erstellen. Extrahierte Informationen sind mit einem Confidence-Score versehen, um dem Benutzer mitzuteilen, wie exakt und verlässlich die Daten sind. Handschriftlicher und gedruckter Text kann mit hoher Genauigkeit und Zuverlässigkeit erkannt werden. Die Extraktion der Daten geschieht sehr performant in kurzer Zeit. Das Pricing des Produkts ist nutzungsabhängig \cite{textract}.