\section{PDF-Dateiversionen}
Gestartet mit Version 1.0 war PDF lediglich ein proprietäres Dateiformat von Adobe. Die Freigabe von PDF als offenes und kostenlose Dateiformat führte letztendlich erst zu seiner weltweiten Verbreitung und Anerkennung. Erst im Jahr 2005 entwickelte sich Version 1.4 zu einem internationalen \gls{iso} Standard. Die letzte Dateiversion 2.0 von 2017 wurde erst im Jahr 2020 eingeführt.

\subsection{PDF 1.0}
PDF 1.0 wurde 1992/1993 entwickelt und wurde nicht normiert. 1992 wurde die Spezifikation als Buch verkauft und 1993 das der Spezifikation entsprechende digitale Format entwickelt, welches ausschließlich den RGB-Farbraum darstellen konnte. Medien, die einen anderen Farbraum besitzen, wurden in RGB konvertiert. Folglich ist PDF 1.0 nicht für den Printbereich geeignet. Damals war Acrobat 1.0 das einzige Programm, mit dem man diese Dateiversion bearbeiten konnte \cite{proj-consult}. 

\subsection{PDF 1.1}
Genauso ist das 1994 kreierte PDF 1.1 keine Norm und implementiert weiterhin nur den RGB-Farbraum, jedoch geräteunabhängig. Erstmals erschien in diesem Format das Einbetten von Hyperlinks, optional gebunden an Aktionen, mehrseitigen Artikeln und Threads, Passwortverschlüsselung und Notizen bzw. Anmerkungen \cite{proj-consult}. Hier können bereits Benutzungseinschränkungen geltend gemacht werden, wie Schutz vor unerlaubtem Öffnen des PDF-Dokuments, das Sperren von Teilfunktionen, z.B. Entnahme von Texten und Bildern, sowie deaktiviertes Drucken. Verschlüsselt wird mit einer 40-Bit-Schlüssellänge durch den RC4-Algorithmus. TrueType-Fonts können nativ eingebettet werden und einige geräteunabhängige (\gls{cie}-basierte Farbräume) können eingestellt werden \cite{schneeberger}. Binäres Abspeichern der PDF-Datei ist nun möglich, womit eine Reduktion der Dateigröße um 25 \% erreicht werden kann \cite{schneeberger}.

\subsection{PDF 1.2}
Fast alle, für die Druckvorstufe nötigen Parameter aus PostScript, wurden in Version 1.2 umgesetzt. Das 1996 erschienene PDF 1.2 wurde ebenfalls nicht standardisiert, jedoch ermöglichte es erstmals den CMYK-Farbraum und Sonderfarben zu verwenden. Des Weiteren wurden interaktive Formularfunktionen, Unicode, Unterstützung der \gls{opi} 1.3 Spezifikationen und eine Druckrasterfunktion implementiert \cite{proj-consult}. \gls{opi} ist ein Workflow-Protokoll, welches in der elektronischen Druckvorstufe verwendet werden kann, um Desktop Publishing Systeme und high-end \gls{ceps} zu verknüpfen und optimiert die Übertragung von hochauflösenden Dateien in Netzwerken \cite{printwiki}. In PDF 1.2 wurden erstmalig AcroForms vorgestellt. Zusätzlich ist das Abspielen von Video und Audio ausschließlich als Link zu einer externen Datei möglich. Composite-Fonts, \gls{cid} Fonts und alle \gls{cie} fontbasierte Farbräume werden unterstützt \cite{schneeberger}.

\subsection{PDF 1.3}
1999 wurde PDF 1.3 mit fehlender Normierung auf den Markt gebracht und trug seinen Teil bei zur Standardisierung des \gls{iso} PDF/X-Standards (2001/2002). Es ist kompatibel mit PostScript 3 und bietet folgende Neuerungen an: 2-Byte-\gls{cid}-Fonttypen, \gls{opi} 2.0-Unterstützung, Farbraumerweiterung um Sonderfarben durch \gls{icc}-Profile, DeviceN-Farbraum, weiche Schatten und Farbübergänge bzw. Verläufe in einem auflösungsunabhängigen Modus (Smooth Shading), digitale Signaturen, RC4-Verschlüsselung (40 Bit in Acrobat 4 und 56 Bit in Acrobat 4.05) und JavaScript \cite{proj-consult, schneeberger}. Der DeviceN-Farbraum wird auch in PostScript 3 unterstützt und erlaubt die willkürliche Kombinationen von Farbkanälen beim Composite-Druck. Dokumente mit Schmuckfarben müssen auf einem Gerät mit physikalisch getrennten Kanälen für jede verwendete Schmuckfarbe ausgegeben werden. Schmuckfarben sind festgelegte Farbtöne, die nicht aus Prozessfarben gemischt werden. Sie können Kosten sparen und vermeiden Farbschwankungen. Besonders verbreitet sind HKS und Patone \cite{kompendium}, folglich kann kein CMYK- oder RGB-Gerät Dokumente mit Schmuckfarben farblich korrekt darstellen. Ohne den DeviceN-Farbraum können Bilder mit Kombinationen von, z.B. CMYK und 2 Schmuckfarben oder Schwarz und eine Schmuckfarbe, nicht im Composite-PostScript und Composite-PDF wiedergegeben werden, sondern höchstens mit CMYK als Näherung \cite{helios}. Dateianlagen jeglichen Typs können in Form von Streams im Body direkt eingebettet werden. 3 weitere Boxen werden dem PDF-Format hinzugefügt: TrimBox, BleedBox und ArtBox. Für Überfüllungen in der Druckvorstufe können Traps als form XObject gespeichert werden. Bilder können auf Basis von Image XObjects für komplexere Pfadresultate maskiert werden. Eine Maske kann Sektionen eines Bildes ausblenden, ohne diese zu löschen \cite{schneeberger}.

\subsection{PDF 1.5}
Im Jahr 2003 erschien PDF 1.5 auf dem Markt und hat sich nicht zur Norm entwickelt. In dieser Version wurden erstmals Ebenen und 16-Bit-Farbtiefe in eingebetteten Bildern implementiert. Des Weiteren wurden gesteigerte Kompressionstechniken, einschließlich Objekt-Streams und JPEG 2000-Kompression, sowie eine verbesserte Cross-Reference Tabelle und XRef-Streams implementiert. \cite{proj-consult, schneeberger}. 

\subsection{PDF 1.4}
Der erste PDF-\gls{iso}-Standard \gls{iso} 16612-1:2005 wurde zeitlich nach Version 1.5 verabschiedet. In diesem Format sind Transparenzen, JavaScript 1.5, bessere Integration von Datenbanken, JBIG2-Komprimierung und 128-Bit-RC4-Verschlüsselung erstmalig integriert worden \cite{proj-consult}. Außerdem wurde die Kennzeichnung der Ausgabeabsicht über den Output-Intent implementiert \cite{schneeberger}.

\subsection{PDF 1.6}
In diesem \gls{iso} 15930-8:2008 Standard sind erstmals folgende Technologien eingeführt worden: NChannel-Farbraum, welcher eine Erweiterung von NDevice mit Sonderfarben bzw. Schmuckfarben ist, JPEG 2000-Kompression, \gls{aes} Verschlüsselung, direkte Einbettung von OpenType-Schriften, 3D-Daten (U3D) und \gls{xml}-Formulare \cite{proj-consult}.

\subsection{PDF 1.7}
Mit der Veröffentlichung am 1. Juli 2008 ist PDF Version 1.7 (\gls{iso} 32000-1:2008) als Offener Standard definiert worden. Es wurden komplexe 3D-Objekte, Kontrolle über 3D-Animationen und Einbettung von Standard-Druckeinstellungen wie Papierauswahl, Anzahl der Kopien und Skalierung hinzugefügt\cite{proj-consult}.

\subsection{PDF 2.0}
Der PDF 2.0-Standard wurde erstmals im Juli 2017 veröffentlicht. Die überarbeitete Ausgabe der \gls{iso} 32000-2:2020 bestimmte, dass \gls{xfa} in PDF 2.0 als veraltet markiert werden sollte. Diesem Format sind mehr Einstellungsmöglichkeiten und Funktionen beigefügt: Erweiterte Definitionen der Halbtöne, z.B. für Flex- oder Tiefdruck, konsistente Transparenz, erweiterte Tagged PDF-Funktion für Barrierefreiheit, Definition von Sonderfarben über Spektralfarben, alternierende Reihenfolge der zu druckenden Farben, Steuerung der Schwarzpunktkompensation, \gls{aes}-256-Bit-Verschlüsselung, DPart-Metadaten und Einbettung von 3D-Messungen oder Querschnittsdaten \cite{proj-consult}.