\subsection{PDF Signature Spoofing}
Die PDF-Spezifikation ist sehr ungenau im Bezug auf digitale Signaturen bzw. auf welche Art und Weise sie validiert werden müssen. PDF-Viewer haben eine hohe Toleranz beim Öffnen, Validieren und Anzeigen von beschädigten PDF-Dateien. \\
Im Jahr 2019 wurden 3 Attacken zum Vortäuschen von validen PDF Signaturen erforscht: \gls{isa}, \gls{swa} und \gls{usf}. Das attack scenario sieht wie folgt aus: Der Angreifer besitzt das signierte PDF mit einer gültigen digitalen Signatur und manipuliert es. Das manipulierte signierte PDF wird an das Opfer geschickt und die Signatur bleibt gültig, obwohl der Inhalt geändert wurde.
\\
Die \gls{isa} nutzt das incremental update Feature von PDF aus, um bösartigen Inhalt in das PDF des Opfers zu schleusen. Im Prozess des Signierens wird ein incremental update verwendet, um die Signatur zu speichern. Am Ende des Originaltrailers wird ein neuer Katalog und ein neues Signatur-Objekt als Body Updates angehängt, was den Signaturwert und Informationen über den Ersteller der Signatur enthält. Danach kommt eine updated Xref section und ein updated Trailer. Beim inremental update können Objekte mit anderem Inhalt neu definiert werden. Bei einem Spezifikation-konformem incremental update wird der updated Body hinter dem Originaltrailer am Dateiende angehängt. Darunter zeigt die updated Xref section auf das neue Objekt und der updated Trailer wird ganz am Ende angefügt. Abbildung \ref{fig:incr-update} zeigt den Vorgang des Speicherns einer digitalen Signatur mittels incremental update. \\

\begin{figure}[!htb]
	\centering
	\includegraphics[width=0.8\textwidth]{"images/dig_sig_incr_up.png"}
	\caption{incremental update mit digitaler Signatur \cite{ccc-break-pdf-slides}}
	\label{fig:incr-update}
\end{figure}

Zuerst hat man bei der \gls{isa} geprüft, ob die PDF-Reader gegen eine Attacke anfällig waren, bei der hinter dem incremental update, was die digitale Signatur enthält, eine weitere Sektion mit Body Updates, Xref und Trailer angefügt wurde. Ausschließlich LibreOffice wurde mittels diese Vorgehensweise getäuscht. Alle weiteren Strategien produzieren beschädigte nicht Standard-konforme PDF-Dateien, jedoch PDF-Viewer sind sehr fehlertolerant. Als Nächstes haben die Forscher hinter dem updated Trailer des incremental updates der Signatur Body Updates hinzugefügt. Einige getestete PDF-Viewer haben lediglich überprüft, ob eine neue Xref und Trailer vorhanden sind. Da sie fehlten, blieb die Signatur gültig, die zusätzlichen Body Updates wurden ausgeführt und die Modifikation blieb ohne Warnung unbemerkt. Andere PDF-Viewer benötigten neben zusätzlichen Body Updates noch einen Trailer ohne Xref dazwischen, damit keine Warnung geworfen wurde. Die komplexeste Vorgehensweise, bei der neue Body Updates mit einer Kopie des Signatur-Objekts am Dateiende angebracht wurde, hat einige PDF-Viewer wie Foxit gezwungen die Signatur 2 Mal zu validieren und die Body Updates wurden eingefügt. In der Grafik \ref{fig:isa} werden die Techniken visualisiert, sowie die PDF-Viewer dargestellt, die anfällig für die \gls{isa}-Vorgehensweise waren.
\par

\begin{figure}[!htb]
	\centering
	\includegraphics[width=0.8\textwidth]{"images/isa.png"}
	\caption{\gls{isa}-Methoden mit anfälligen PDF-Viewern \cite{ccc-break-pdf-slides}}
	\label{fig:isa}
\end{figure}

Bei der \gls{swa} werden die Werte der signierten /ByteRange modifiziert und Platz wird für das Einschleusen von bösartigen Inhalten geschaffen. Das Signatur-Objekt besteht aus einem /Contents entry mit dem Signaturwert und einem /ByteRange entry mit 4 Werten, was den signierten Teil im Dokument anzeigt. Die ersten 2 Einträge der /ByteRange beziehen sich auf den Beginn des Dokuments bis zum Anfang des Signaturwerts. Hingegen definieren die letzten beiden ByteRange-Einträge den Bereich nach dem Signaturwert bis zu \%\%EOF. Diese beiden Bereiche wurden nicht angerührt. Die erfolgreiche Idee war eine zweite /ByteRange hinter dem Signaturwert mit einem angepassten dritten Wert, der Platz für bösartige Objekte, etwas Padding und eine bösartige Xref, die auf die Objekte des Angreifers zeigt, zu setzen. Lediglich die Position der neuen Xref ist im Trailer vorgegeben, der nicht verändert werden kann. Das Schaubild \ref{fig:swa} zeigt, wie die \gls{swa} arbeitet.
\par

\begin{figure}[!htb]
	\centering
	\includegraphics[width=0.8\textwidth]{"images/sig_wrap_attack.png"}
	\caption{\gls{swa}-Methodik \cite{ccc-break-pdf-slides}}
	\label{fig:swa}
\end{figure}

Mittels \gls{usf} wird die Signaturvalidierung außer Kraft gesetzt. Dennoch wird die Meldung „PDF is validly signed" dem Benutzer angezeigt. Vor allem die Adobe-Produkte waren anfällig für diese Attacke. Die Vorgehensweise gestaltete sich wie folgt: Die Forscher versuchten /Contents oder /ByteRange der Signatur entweder wegzulassen oder unüblichen Werten wie kein Wert oder null zuzuweisen. Zwei PDF-Viewer versagten, wenn man /Contents auf 0 Bytes setzte. Das Weglassen der /ByteRange oder das Gleichsetzen mit null brach die Sicherheitsmechanismen von Adobe. Die Grafik \ref{fig:usf} zeigt die \gls{usf}-Varianten, die von den Forschern getestet wurde.
\par

\begin{figure}[!htb]
	\centering
	\includegraphics[width=0.8\textwidth]{"images/univ_sig_forgery.png"}
	\caption{\gls{usf}-Varianten \cite{ccc-break-pdf-slides}}
	\label{fig:usf}
\end{figure}

Mittels dieser Attacken ist es den Forschern gelungen zu zeigen, dass 21 von 22 PDF-Viewer inklusive denen von Adobe und 6 von 8 Online-Anbieter anfällig waren. Der einzige PDF-Viewer, der gegen alle Attacken immun war, war die veraltete Version von Adobe Reader 9, die jedoch remote code execution enthält \cite{ccc-break-pdf}. 