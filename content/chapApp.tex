\chapter{Implementierung}
In diesem Kapitel werde ich zunächst auf die Bedienung und den Funktionsumfang der PDF Web App eingehen. Darauffolgend werde ich die Implementierung erläutern und auf welche Art und Weise ich wichtige Funktionen in JavaScript umgesetzt habe. Abschließend erfolgt ein Unterkapitel, in dem ich die PDF Web App in verschiedenen Bereichen getestet habe.

Mein öffentliches Github Repository der PDF Web App ist unter folgendem Link erreichbar: \\
\url{https://github.com/JaninaAlona/MultiPDFmin/}

Das Projekt heißt auf Github MultiPDFmin. Ich habe die PDF Web App auf Github Pages hochgeladen. Sie ist unter folgendem Link aufrufbar: \\
\url{https://janinaalona.github.io/MultiPDFmin/}

Der aktuell gültige Commit-Hash, der Stand des Projekts bezeichnet, in der git history lautet: \\
5ec70c3fb0dbbef7e90efded52549f2360aac99e

Das Github Repository ist öffentlich und enthält einen Ordner docs, in dem die PDF Web App mit ihren dependencies enthalten ist. Außerdem gibt es einen Ordner tutorials, der eine Tiddly Wiki HTML-Seite über die Bedienung der PDF Web App enthält. Sowohl die README.md und Tutorials als auch die PDF Web App sind in englischer Sprache gehalten.

\section{Bedienung und Funktionalität der PDF Web App}
Meine PDF Web App besteht aus den Modulen Reader, Creator, Splitter, Merger und Editor. Wenn man die PDF Web App zum ersten Mal öffnet, findet man die im Screenshot \ref{fig:start} abgebildete Startseite vor.

\begin{figure}[!htbp]
	\centering
	\includegraphics[width=1\textwidth]{"images/startseite.png"}
	\caption{Startseite der PDF Web App}
	\label{fig:start}
\end{figure}

Bei fast allen Modulen gibt es Möglichkeiten Benutzereingaben zu machen. Diese Benutzereingaben sind so programmiert, dass sie bei ungültigen Eingaben automatische korrigiert werden oder die darauf bezogene Operation nicht ausgeführt wird. Gibt man in ein input field, wo eine Zahl erwartet wird, einen String ein, so wird dessen Funktion nicht angewendet. Liegt eine Benutzereingabe als Zahl unter oder über dem Wertebereich des Eingabefeldes, so wird entsprechen auf den niedrigsten oder obersten Wert substituiert. Manche Eingabefelder erwarten Integers, anstatt Floats, z.B. das Eingabefeld für die Seitenzahl. In diesem Fall wird die Nachkommastelle automatisch von der Benutzereingabe entfernt. Haben sich beim Benutzer ein oder mehrere Leerzeichen in die Eingabe eingeschlichen, so werden diese Leerzeichen von der App automatisiert entfernt. 

\subsection{Bedienung des Readers}
Initial ist der Reader ausgewählt. Ausgewählte Funktionen im Hauptmenü sind dunkelgrau unterlegt mit grüner Schrift. Der Button Create führt zum Creator für leere PDFs, Split zum Splitter fürs Seiten Zerteilen, Merge zum Merger für PDFs zusammenfügen und Text, Draw, Shape bzw. Image öffnen den Editor. Bei Read, Text, Draw, Shape und Image erscheint zunächst der Choose file Button, damit man im Dateisystem ein PDF-Dokument auswählen kann zum Lesen oder Bearbeiten. Klickt man auf Choose file wird der Dateibrowser geöffnet und man kann ein einzelnes PDF auswählen zum Öffnen. Der Screenshot \ref{fig:reader} zeigt den Reader mit geöffneter PDF-Datei.

\begin{figure}[!htbp]
	\centering
	\includegraphics[width=1\textwidth]{"images/reader.png"}
	\caption{Geöffnetes PDF im Reader der PDF Web App}
	\label{fig:reader}
\end{figure}

Falls eine andere Dateiart in der PDF Web App geöffnet wurde, unabhängig vom Modul der App, erscheint die Fehlermeldung in Screenshot \ref{fig:errorfile}. 

\begin{figure}[!htbp]
	\centering
	\includegraphics[width=1\textwidth]{"images/errorfile.png"}
	\caption{Fehlermeldung bei einer nicht-PDF-Datei}
	\label{fig:errorfile}
\end{figure}

Auch bei einer verschlüsselten PDF-Datei wird eine in Screenshot \ref{fig:errorcrypt} dargestellte Fehlermeldung angezeigt.

\begin{figure}[!htbp]
	\centering
	\includegraphics[width=1\textwidth]{"images/errorcrypt.png"}
	\caption{Fehlermeldung bei einem verschlüsselten PDF}
	\label{fig:errorcrypt}
\end{figure}

Bei diesen gezeigten Fehlermeldungen kann man einfach erneut den Dateibrowser aufrufen mit Choose file oder einen anderen Menüpunkt wählen, damit die Fehlermeldung verschwindet. Hat man eine PDF-Datei im Reader geöffnet präsentieren sich einem 2 Zeilen mit Funktionsbuttons. Mittels Previous und Next kann der Benutzer zur vorherigen bzw. nächsten Seite blättern. Zwischen diesen Buttons wird die aktuelle Seite im Viewport im page counter input field und daneben die Anzahl an Seiten im Dokument angezeigt. Im input field kann man eine Seite eingeben und der Reader springt direkt zu dieser Seite. Alternativ kann man mit dem Scrollbar am linken Browserfensterrand oder dem Scrollrad der Maus durch die Seiten scrollen. Durch die Buttons Plus und Minus kann man in 20 \%-Schritten rein- bzw. rauszoomen. Der aktuelle Zoomfaktor wird angezeigt in Prozent und man kann den Zoomfaktor auch mittels Benutzerangabe manuell setzen mit und ohne Prozentzeichen. Außerdem wird der minimale und maximale Zoomwert in Prozent angezeigt. Spin CW und Spin CCW dreht die aktuelle Seite, die im input field angezeigt ist, um 90 Grad im Uhrzeigersinn (clockwise) und gegen den Uhrzeigersinn (counterclockwise). Durch die Funktion Drag kann man die aktuelle Seite verschieben im Viewport. Das ist beispielsweise nützlich, wenn man an das PDF nah rangezoomt hat und die Seite nur teilweise sehen kann. Dabei klickt man zuerst auf Drag, hält die Maustaste auf der aktuellen Seite, die im page counter input field angezeigt wird, gedrückt und bewegt sie in jegliche Richtungen. Dabei bekommt der Mauscursor das Aussehen eines weißen Kreuzes mit Pfeilen an den Enden. Der Save Button downloaded das aktuelle PDF im Downloads-Ordner des Benutzers. Alle Eingabefelder in der PDF Web App begrenzen Werte, die kleiner oder größer sind als der Minimal- oder Maximalwert automatisch auf den untersten bzw. obersten Schwellenwert. Außerdem werden Leerzeichen automatisch entfernt. Ungültige Eingaben werden ignoriert. Ein geöffnetes Dokument passt sich automatisch an das Browserfenster an, falls es über das Browserfenster hinausragen würde, sodass es fast formatfüllend ins Browserfenster passt. \\

\subsection{Bedienung des Creators}
Der Creator ist mittels Create Buttons im Hauptmenü aufrufbar. Im Screenshot \ref{fig:creator} ist die GUI vom Creator dargestellt. 

\begin{figure}[!htbp]
	\centering
	\includegraphics[width=1\textwidth]{"images/creator.png"}
	\caption{Creator GUI der PDF Web App}
	\label{fig:creator}
\end{figure}

Man gibt eine Anzahl an gewünschten Seiten des leeren PDFs ein und die Breite und Höhe in mm. Wahlweise kann man den Selector verwenden um ein DIN A-Preset zu verwenden. Mittels der Schnellauswahl kann man die Orientierung bestimmten: Portrait, Landscape oder Quadratisch. Der Save Button downloaded das leere PDF. \\

\subsection{Bedienung des Splitters}
Zum Splitter kann man mit dem Split Button gelangen, der vom Screenshot \ref{fig:splitter} gezeigt wird.

\begin{figure}[!htbp]
	\centering
	\includegraphics[width=1\textwidth]{"images/splitter.png"}
	\caption{Splitter GUI der PDF Web App}
	\label{fig:splitter}
\end{figure}

\begin{figure}[!htbp]
	\centering
	\includegraphics[width=0.7\textwidth]{"images/splitter2.png"}
	\caption{Splitter Selector der PDF Web App}
	\label{fig:splitter2}
\end{figure}

\begin{figure}[!htbp]
	\centering
	\includegraphics[width=0.7\textwidth]{"images/splitter3.png"}
	\caption{Splitter Download Datei Dialog der PDF Web App}
	\label{fig:splitter3}
\end{figure}

Im Auswählmenü kann man zwischen Zerteilen nach jeder, jeder geraden, jeder ungeraden und eine Liste von Seiten auwählen, was in Abbildung \ref{fig:splitter2} dargestellt ist. Wählt man list of pages aus, so kann man die einzelnen Seitennummern mit Komma separiert oder auch nur eine einzelne Seitennummer eintippen. Die Seitenzahlen müssen nicht in aufsteigender Reihenfolge eingegeben werden. Bewegt man die Maus über den Save Button wird ein Dateibenennungsdialog sichtbar mit dem default Dateinamen. Man kann dann den Dateinamen ändern, was Abbildung \ref{fig:splitter3} präsentiert. Bei jedem Save Button in der PDF Web App kann der default Dateiname geändert werden. Hat man eine Datei ausgewählt, so wird der Dateiname und die Anzahl an Seiten des Dokuments angezeigt. \\

\subsection{Bedienung des Mergers}
Der Merger ist mit dem Hauptmenüpunkt Merge zu öffnen. Abbildung \ref{fig:merger} zeigt die Startseite des Mergers. 
\begin{figure}[!htbp]
	\centering
	\includegraphics[width=1\textwidth]{"images/merger.png"}
	\caption{Merger Startseite der PDF Web App}
	\label{fig:merger}
\end{figure}
Hier kann man nacheinander mehrere Dateien im Dateidialog öffnen und sie erscheinen in einer Liste, je nach Auswahlreihenfolge untereinander. In der Liste kann man dann einzelne Blöcke von Dateien einzeln auswählen und in ihrer Position mit der Maus verschieben. Die ausgewählte Datei hat einen schwarzen Hintergrund, was Screenshot \ref{fig:mergelist} anzeigt.
\begin{figure}[!htbp]
	\centering
	\includegraphics[width=0.8\textwidth]{"images/mergelist.png"}
	\caption{Merger Dateiliste der PDF Web App mit selektierter Datei}
	\label{fig:mergelist}
\end{figure}
Mittels Remove Buttons kann eine Datei zum Mergen wieder aus der Liste entfernt werden, nachdem sie ausgewählt wurde. Der Save Button fügt alle Dateien zu einem PDF zusammen.



\subsection{Bedienung des Editors}
Der Editor ist über den Text, Draw, Shape oder Image Button erreichbar. Genau wie im Reader erscheint ein Button Choose file. Je nachdem ob man auf Text, Draw, Shape oder Image geklickt hat, wird als erstes der Text-, Zeichen-, Geometrie- oder Bildeditor geöffnet. Hat man eine Datei geöffnet, so befindet sich der Reader ohne die Operationen zum Seiten Drehen ebenfalls im Editor, sowie seine Drag und Save Funktionalitäten. Bei Save kann man wie bei anderen Modulen einen benutzerdefinierten Dateinamen vergeben und das aktuelle PDF wird in den Downloads-Ordner runtergeladen. Alle input fields im Editor sind mit dem gültigen Wertebereich für Benutzereingaben als Information Min: Max: versehen. Der Editor besteht aus einem grauen waagerechten Operations Bar, einem linken Ebenen Seitenmenü in Rosa und einem rechten grünen Tools Seitenmenü. Mit dem ganz linken grünen Button Layers im Operations Bar kann das Ebenen Seitenmenü aus- und eingeblendet werden. Daneben zeigt oder verbirgt der Button Tools das Tools Seitenmenü. 

\subsubsection{Textbearbeitung}
Hat man den Texteditor aufgerufen, so präsentiert sich einem der Editor in folgenden Abbildungen \ref{fig:texteditor} und \ref{fig:texteditor2}.

\begin{figure}[!htbp]
	\centering
	\includegraphics[width=1\textwidth]{"images/texteditor.png"}
	\caption{Startseite des Texteditors der PDF Web App}
	\label{fig:texteditor}
\end{figure}

\begin{figure}[!htbp]
	\centering
	\includegraphics[width=1\textwidth]{"images/texteditor2.png"}
	\caption{Mehr Tools der Startseite des Texteditors der PDF Web App}
	\label{fig:texteditor2}
\end{figure}

Mit dem Button Text in der grauen Leiste und nachfolgendem Klick aufs geöffnete Dokument kann man einen Text hinzufügen mit dem Platzhaltertext dummy. Unter dem Text erscheint eine dunkelrote control box, auf die man alle Operationen in der grauen Leiste und dem Tools Seitenmenü im Box Mode anwenden kann. Der Box Mode ist standardmäßig eingestellt. Alle Operationen im rechten Tools Seitenmenü beziehen sich jeweils auf das aktuelle Editor Element und sind nur auf diesem anwendbar. Ich werde zunächst alle Operationen im Box Mode beschreiben und später auf den Layer Mode eingehen. Man kann mehrere Texte ohne erneut Text drücken zu müssen dem PDF Dokument hinzufügen. Für jedes neu hinzugefügte Textelement wird eine Ebene mit einem  Element spezifischen Standardnamen erstellt, was im linken rosa Ebenenmenü zu sehen ist. Mit dem Delete Button und nachfolgendem Klick in eine oder mehrere control boxen im Box Mode können Texte wieder gelöscht werden. Move verschiebt einzelne Texte durch mit der Maus gedrückte control box. Wenn die Maus losgelassen, nachdem die control box verschoben wurde, springt der Text an die verschobene Stelle. Ganz oben im grünen Tools Seitenmenü werden dem Betrachter die x- und y-Koordinaten der Maus auf der PDF Seite angezeigt, wenn die Maus über eine Seite bewegt wird. Darunter kann man in der textarea den Text editieren. Es werden auch Zeilenumbrüche berücksichtigt. Nachdem man den dummy Text überschrieben hat, einem Klick auf den weißen Text Button und ein oder mehrere Klicks in control boxen, kann der Text angewendet werden. Alle Operationen in Tools werden genau gleich ausgeführt: Man tätigt seine Einstellung, drückt mit der linken Maustaste auf den weißen Button für die jeweilige Operation und klickt daraufhin auf ein oder mehrere Textelemente nacheinander. Darunter kann man den Zeilenabstand einstellen. Entweder verwendet man das selection menu mit voreingestellten Werten oder man gibt einen gewünschten Wert manuell in das input field ein. Alle selection Menüs und input fields in jedem Editor zeigen die default Werte, mit denen ein neu hinzugefügtes Element konfiguriert ist, an. Falls man zuletzt das selection Menü betätigt hat, überschreibt es den Wert im input field und umgekehrt. Maßgeblich ist, was man zuletzt betätigt hatte. Dieses Verhalten habe ich bei jeder selection menu und input field Kombination programmiert. Einen benutzerdefinierten Font kann man durch den dunkelgrauen Choose file Button auswählen und er erscheint in der Liste. Der zuletzt hochgeladene Font wird ausgewählt. Mittels clear kann man einen ausgewählten Font aus der Liste entfernen, was nicht heißt, dass er auch auf dem angewendeten Text entfernt wird. Abbildung \ref{fig:custom-font} zeigt 2 geöffnete .ttf Schriftdateien in der Liste.

\begin{figure}[!htbp]
	\centering
	\includegraphics[width=0.3\textwidth]{"images/custom-font.png"}
	\caption{Benutzerdefinierte Fontliste im Texteditor der PDF Web App}
	\label{fig:custom-font}
\end{figure}

Die Fontgröße kann man ebenfalls wie die Zeilenhöhe mit selection menu und input field justieren. Bei der Fontfarbe klickt man auf das initial schwarze Quadrat, was die aktuelle Farbe zeigt, und es öffnet sich ein color picker Menü. Hier kann man die Farbe und Transparenz einstellen. Die Werte kann man sich in RGBA, HSLA oder HEX formatieren lassen. Mit Klick auf die beiden kleinen senkrechten Pfeile im color picker wird jeweils das Format gewechselt. Das Fenster des color pickers für die Fontfarbe ist in Abbildung \ref{fig:fontcolor} abgebildet. 

\begin{figure}[!htbp]
	\centering
	\includegraphics[width=0.5\textwidth]{"images/fontcolor.png"}
	\caption{Color picker für die Fontfarbe des Texteditors der PDF Web App}
	\label{fig:fontcolor}
\end{figure}

Als vorletzte Option kann man den Text absolut drehen. Durch den weißen Button Rotation und der entsprechenden Benutzerinteraktion durch selection Menu oder input field wird das Textelement absolut gedreht. Das bedeutet, dass es eine feste Rotationsskala gibt mit denen das Element rotiert wird. Folglich passiert keine Veränderung, wenn man den gleichen Rotationswert 2 Mal hintereinander ausführt. Wählt man 0 Grad aus zum rotieren, wird das Element wieder zurück auf die Ausgangsrotation gedreht. Abschließend können alle Textelemente im Dokument mit dem Remove Button auf einen Schlag gelöscht werden. Beim Zeilenabstand und der Schriftgröße wird der Benutzer außerdem über den Wertebereich von Benutzereingaben informiert. Man kann generell nur Elemente hinzufügen und auf ihnen die Operationen anwenden. Man kann in der PDF Web App keine im PDF bereits bestehenden Elemente bearbeiten. Die Textoperationen werden in Abbildung \ref{fig:text} demonstriert.

\begin{figure}[!htbp]
	\centering
	\includegraphics[width=0.9\textwidth]{"images/text.png"}
	\caption{Bearbeiteter Text in der PDF Web App}
	\label{fig:text}
\end{figure}

\subsubsection{Zeichnungen erstellen}
Der Zeichneneditor präsentiert sich einem in Screenshot \ref{fig:drawer}. 

\begin{figure}[!htbp]
	\centering
	\includegraphics[width=1\textwidth]{"images/drawer.png"}
	\caption{Drawer der PDF Web App}
	\label{fig:drawer}
\end{figure}

Das Ebenenmenü und Tools Seitenmenü des Zeichneneditors erscheint selbst wenn man zuerst im Texteditor ein Dokument geöffnet hat. Nach einem geöffneten PDF kann man von jedem Editor in einen anderen wechseln ohne erneut ein PDF öffnen zu müssen. Mit dem Zeichnen kann man anfangen, wenn man auf Pencil klickt. Bei gedrückter Maustaste auf einer PDF Seite erscheint eine schwarze Linie dort wo die Maus sich bewegt hat. Zusätzlich wird dort wo man angefangen hat die Maus zu drücken eine magenta control box hinzugefügt. Das Zeichnen funktioniert auch mit einem Graphic Tablet. Es wird immer auf der zuletzt gezeichneten Ebene auf der Seite gemalt bzw. wenn man eine Ebene auswählt im linken Ebenenmenü wird auf der ausgewählten Ebene gezeichnet. Ein Klick auf New Layer und anschließender Zeichenmodus mit Pencil kreiert für die neue Zeichnung eine weitere Ebene. Die neue Zeichnung auf der Ebene erhält abermals eine magenta control box. Wurde auf einer Seite bisher noch nichts gezeichnet, so wird bei der ersten Zeichnung auf der Seite eine neue Ebene automatisch angelegt und man muss nicht New Layer drücken. Die Zeichenelemente sind die einzigen Objekte, bei denen der Nutzer selbst die Ebenen einer Seite zuweisen kann mit New Layer. Bei allen anderen Elementen, sei es Text, Geometrie oder Bilder, wird für jedes neue Element automatisch eine Ebene erstellt. Der Radierer ist  mit dem Eraser Button im grauen operations Menü anwendbar. Zuerst drückt man Eraser und geht dann mit gedrückter Maustaste über die Zeichnungen auf einer Seite, die man entfernen möchte. Dort wo bei gedrückter Maustaste die Maus die Linie berührt wird wegradiert. Zeichnen und Radieren bekommen jeweils ein neues Mauscursorsymbol: Beim Zeichnen hat man ein schwarzes dünnes Kreuz und beim Radieren ein weißes dickes Kreuz. Mit dem Delete Button kann man mehrere Zeichnungen nacheinander mit Klicks in die control boxen entfernen. Move und gedrückte Maustaste auf eine control box verschiebt diese. Delete und Move funktionieren analog zum Texteditor. In jedem Editor gibt es einen Delete und Move Button zum Löschen und Verschieben von Elementen. Genau wie alle Operationen im Tools Seitenmenü kann man mit Delete und Move nur die dem Editor zugehörigen Elemente editieren. Im Tools Seitenmenü kann man eine Zeichnung relativ skalieren, indem man einen Faktor eingibt. Der Faktor kann auch ein Float sein und multipliziert sich immer mit der aktuellen Größe, d.h. die aktuelle Größe ist 100 \%. Darunter kann man mit dem color picker menü die Farbe und Transparenz der Stiftfarbe definieren. Sie wird mit einem Klick auf Pencil Color auf die nächste Zeichenoperation angewendet. Außerdem definiert sie auch gleichzeitig die Radiererfarbe, was sich bei Transparenzen unter 1 bemerkbar macht. Sonst ist jede Radiererfarbe gleich, jedoch bei einer Transparenz von unter 1 radiert der Radierer weniger deckend, so als ob man einen Radiergrummi weniger stark auf das Papier drückt. Dann kann man die Größe des Stiftes bzw. Radierers einstellen. Sie greift auch ab der nächsten Zeichen- bzw. Radieroperationen. Ebenfalls kann man Zeichnung mit samt radierten Partien rotieren. Zum Schluss kann man mit Remove alle Zeichnungen im Dokument löschen. Teilweise transparente Zeichnungen werden im Bild \ref{fig:drawing} dargestellt. 

\begin{figure}[!htbp]
	\centering
	\includegraphics[width=1\textwidth]{"images/drawing.png"}
	\caption{Zeichnungen in der PDF Web App}
	\label{fig:drawing}
\end{figure}


\subsubsection{Geometrie hinzufügen}
Die Startseite des Geometrieeditors ist in den Screenshots \ref{fig:shaper} und \ref{fig:shaper2} abgebildet. 

\begin{figure}[!htbp]
	\centering
	\includegraphics[width=1\textwidth]{"images/shaper.png"}
	\caption{Geometrieeditor in der PDF Web App}
	\label{fig:shaper}
\end{figure}

\begin{figure}[!htbp]
	\centering
	\includegraphics[width=1\textwidth]{"images/shaper2.png"}
	\caption{Mehr Tools des Geometrieeditors in der PDF Web App}
	\label{fig:shaper2}
\end{figure}

Der Geometrietyp kann durch die Buttons Rectangle für Rechteck, Triangle für Dreieck oder Ellipse  und einem oder mehreren Klicks auf eine PDF-Seite bestimmt werden. Ebenfalls können alle Shapetypen durch Delete gelöscht und durch Move auf der Seite verschoben werden. Im Tools Seitenmenü des Geometrieeditors gibt es eine einzige Operation, die nur auf Dreiecke angewendet werden kann. Es handelt sich um die oberste Einstellung für die Position des dritten Punktes des Dreiecks. Hiermit kann der rechte Punkt der langen Spitze des default Dreiecks bearbeitet werden. Alle anderen Einstellmöglichkeiten können auf allen Geometrieelementen Rechteck, Dreieck und Ellipse arbeiten. Man hat 2 Möglichkeiten einen Shape zu skalieren. Zum einen kann man die Breite und Höhe unabhängig voneinander einstellen, was eine absolute Skalierung bedeutet, oder man verwendet den Skalierungsfaktor, der relativ vergrößert genauso wie beim Zeichneneditor. Für die Shape Umrandungslinien kann man auf der einen Seite die Farbe inklusive Deckkraft und auf der anderen Seite die Breite der Linie justieren. Die Strichfarbe muss mit der Checkbox Use Stroke in Grün eingeschaltet sein, was sie beim ersten Öffnen des Editors auch ist. Deaktiviert man die Use Stroke Checkbox schaltet sich automatisch die Use Fill Checkbox an. Man kann auch beide Checkboxen einschalten, aber nicht beide zusammen ausschalten. Ist Use Stroke rosa, d.h. deaktiviert, und man wendet die Strichbreite an, dann wird trotzdem ein Strich in der letzten Strichfarbe gesetzt in der gewünschten Breite. Use Fill muss Grün sein, um die Füllfarbe anzuwenden. Bei Strich- und Füllfarbe wird ein wie in dem Text- und Zeicheneditor der gleiche color picker verwendet. Alle Shapes können mit absoluter Rotation rotiert werden. Im ganzen Editor kann ausschließlich absolut rotiert werden. Zuunterst entfernt der Remove Button alle Geometrieelemente im geöffneten PDF. Der Screenshot \ref{fig:shaping} hebt mehrere bearbeitete Geometrieelemente hervor.

\begin{figure}[!htbp]
	\centering
	\includegraphics[width=1\textwidth]{"images/shaping.png"}
	\caption{Zeichnungen in der PDF Web App}
	\label{fig:shaping}
\end{figure}



\subsubsection{Bilder einfügen}

\subsubsection{Arbeiten im Layer Mode}

\subsubsection{Ebenensteuerung}




\section{Implementierung der PDF Web App}
\section{Testdurchführung}

\subsection{Funktionale User Tests}

\subsection{Performance Tests}
Hat man eine PDF-Datei im Reader oder Editor geöffnet und löscht diese, so gibt es keine Fehlermeldung, wie in vielen lokalen Programmen.
\subsubsection{Renderdauer}
Die renderPage Funktion wurde mit verschiedenen PDF-Testdateien auf dem USB-Stick getestet.

\begin{table}[ht]
	\centering
	\begin{tabular}{|p{4cm}|p{3cm}|p{3cm}|p{3cm}|}
		\hline
		\textbf{Datei}													& \textbf{Seitenanzahl} 	& \textbf{Dateigröße} 	& \textbf{Execution in ms}	\\ 
		\hline
		\parbox[t]{4cm}{vivaoptik\_Gutschein\_\\50euro}					& 1 						& 33,22 KB  			& 27						\\ 
		02-Sensoren														& 9 						& 1,17 MB  				& 182						\\ 
		l11manual\_en 													& 850 						& 91,8 MB  				& 99914						\\
		the-metamorphosis-franz-kafka 									& 88 						& 298,86 KB  			& 714						\\ 
		01. War and Peace author Leo Tolstoy 							& 2882 						& 7,21 MB  				& 29115						\\ 
		Animal Crossing Amiibo Card Art									& 50 						& 167,05 MB  			& 53545						\\  
		DevOps with Kubernetes											& 520 						& 13,7 MB  				& 9883						\\  
		02. The Critique of Pure Reason author Immanuel Kant			& 1277 						& 1,78 MB  				& 9428						\\  
		UNIX and Linux System Administration Handbook - Fifth Edition	& 1809						& 71,94 MB  			& 47366						\\ 
		\hline
	\end{tabular}
	\caption{Execution Times der renderPage Funktion für verschiedene PDF-Dateien}
	\label{table:render-dur}
\end{table}

\subsubsection{Downloadleistung}
\begin{table}[ht]
	\centering
	\begin{tabular}{|p{1.5cm}|p{2.5cm}|p{2cm}|p{2cm}|p{2cm}|p{2cm}|}
		\hline
		\textbf{Modul}		& \textbf{Output-datei}		& \textbf{Seiten-anzahl}		& \textbf{Download-größe} 		& \textbf{Execution in ms} 	& \textbf{Kompress-ionsfaktor}	\\ 
		\hline
		Creator	& blank\_pdf5000 & 5000 DIN A4 & 36,96 KB & 2180 &  \\
		Creator	& \parbox[t]{4cm}{blank\_pdf500\\p10000s} & 500 Größe: 10000 & 4,92 KB & 170 & \\
		\hline
	\end{tabular}
	\caption{Execution Times der Download Zip Funktion}
	\label{table:download-dur}
\end{table}

\subsection{Browserunterschiede}

\subsection{Testbewertung}