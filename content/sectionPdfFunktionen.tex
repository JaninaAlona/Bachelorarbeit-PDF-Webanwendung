\section{Wichtigste Features}
PDFs können Texte, Tabellen, Bilder, Pfade, Links, Buttons, Formulare, Audio-, Videoelemente und Funktionen enthalten. Rich Media-PDFs ermöglichen interaktive Inhalte, die eingebettet oder verlinkt werden können. Solche Elemente sind Bilder, Audio, Video oder Buttons \cite{wiki-pdf-engl}. In PDFs werden alle Informationen als nummerierte Objekte gespeichert. Objekte können zu Gruppen kombiniert werden. Der aktuelle Farbmodus im Dokument kann in andere Farbmodi konvertiert werden. Um die Navigation innerhalb eines PDF-Dokuments zu erleichtern, kann man anklickbare Inhaltsverzeichnisse und miniaturisierte Seitenvorschauen (Thumbnails) verwenden. Optional ist eine Gliederung als hierarchische Baumstruktur in Form von Lesezeichen möglich, mit der der Betrachter leichter durch das Dokument geführt werden kann. PDF-Dateien enthalten grundsätzlich Metadaten. PDF-Dateien können Dateianhänge enthalten, die geöffnet und im lokalen Dateisystem abgespeichert werden können \cite{wiki-pdf-engl}. \\
Bei der Erstellung einer PDF-Datei wird zwischen composite oder separiert unterschieden. Inhalte einer Composite-PDF-Datei können leicht verändert werden. Farbsimulationen und die Nutzung der Überdruckvorschau sind in einem separierten PDF-Dokument nicht mehr möglich. Separierte PDF-Dateien sind nicht medienneutral, da eine Farbverrechnung bei der Separation erfolgt. Überfüllung bleibt im Composite-PDF nicht erhalten, jedoch Überdrucken und Aussparen. In modernen Workflows der Druckindustrie hat man sich vom separierten Workflow abgewendet \cite{schneeberger}.

\subsection{WYSIWYG}
Ein PDF-Dokument hat ein festes Layout und eine fixe Anzahl von Seiten. Unabhängig von der Software, mit der das Dokument angezeigt oder mit welcher Hardware es ausgedruckt wird, bleiben alle Elemente nach dem Prinzip \gls{wysiwyg} auf den Seiten immer exakt an derselben Position. Alle Layout- und Formatierungsangaben stammen aus der Erstellungsanwendung. Bei der Konvertierung von Dokumenten mit variablem Layout zu PDF, wie z.B. .txt-Dateien oder HTML, muss der Inhalt auf die vorhandenen Seiten und den verfügbaren Platz verteilt werden. 

\subsection{Fonts}
Jedes Textzeichen ist ein abstraktes Symbol und ein Schriftzeichen beruht auf einer graphische Darstellung. Eine Schriftart ist in PDF als Objekt enthalten und kann mit Werkzeugen in Adobe Acrobat bearbeitet werden. Der Text muss ausgewählt sein und es können darauf folgende Operationen angewandt werden: Farbveränderung in RGB, Transparenzen, Verschiebung, Löschen, Skalierung, Verzerrung bzw. Scherung, Spiegelung, Drehung, Beschneidung oder Ersetzung. Der RGB-Farbraum eignet sich lediglich für die Bildschirmdarstellung und beschreibt die, für den Menschen 16,7 Millionen sichtbaren Farben, mit Hilfe von additiver Farbmischung (Rot, Grün und Blau). Ein Farbraum umfasst die mathematischen Parameter als Daten für die Gesamtzahl der Farben, die auf einem Monitor, Druckmaterial, usw. darstellbar sind \cite{farbraum}. \\
In Adobe Acrobat Pro kann der gesamte Text pro Seite in Pfade konvertiert werden. Pfade sind mathematisch berechnete Linien, die aus gekrümmten Segmenten bestehen. Der Anfang und das Ende jedes Segments werden als Ankerpunkte bezeichnet und Pfade können geschlossen oder geöffnet sein. Die Form des Pfads kann durch die Griffpunkte an den Ankerpunkten modifiziert werden und Pfadsegmente können somit verformt werden \cite{adobe-pfade}. PDF unterstützt folgende Fontformate: Type-1, Multiple-Master-Fonts, TrueType, OpenType, Dfonts, CID codierte Fonts und Composite-Fonts. Falls die Schriftart nicht im Dokument eingebettet wurde, wird sie aus der Ursprungsdatei möglicherweise durch eine Ersatzschrift des Benutzersystems im PDF-Programm substituiert \cite{schneeberger}. 

\subsection{Bilder}
Generell sollte für das Bearbeiten von Bildern ein externes Bildbearbeitungsprogramm verwendet werden, z.B. Adobe Photoshop oder das kostenlose Gimp. Vektorgrafiken als Pfadobjekte und Rasterbilder als Pixelobjekte (Bitmap) können nach Auswahl verschoben, gelöscht, skaliert, verzerrt, gespiegelt, gedreht, in der Deckkraft verändert, beschnitten oder ersetzt werden \cite{schneeberger}. Der Mehrgewinn an Vektorgrafiken liegt in dessen Eigenschaften, dass sie auflösungsunabhängig sind, da sie beliebig groß ohne Qualitätsverlust skaliert werden können und, dass sie wesentlich weniger Speicherplatz benötigen als Rasterbilder. Die Auflösung von Bildern kann in Acrobat neu berechnet werden. Niedrig aufgelöste Bilder behalten ihre Auflösung. Die bikubische Neuberechnung liefert qualitativ hochwertigere Ergebnisse. Bei Schwarzweißbildern kann eine Neuberechnung zu unschönen Artefakten führen \cite{buehler}. Generell führt eine Neuberechnung der Auflösung in Bildbearbeitungsprogrammen zu besseren Ergebnissen als in Acrobat. 

\subsection{3D-Daten}
PDFs mit 3D-Inhalten bestehen aus dem U3D-Flächenmodell oder BREP/Flächenmodell PRC. Diese Flächenmodelle werden vorwiegend bei der Visualisierung von \gls{cad} Daten verwendet. Beide Formate können im Adobe Reader angezeigt, animiert, geschnitten und gemessen werden. Viele Drittanbieter PDF-Reader und die PDF-Viewer im Browser können eingebettete 3D-Daten meist nicht darstellen \cite{wiki-pdf-de}. 

\subsection{Kommentare}
Ein Kommentarobjekt, das mit Dokumentenseiten verlinkt ist, besteht aus 2 technisch separaten Bausteinen. Zum einen werden Kommentare durch ein grafisches Element auf den zugehörigen Seiten symbolisiert, zum anderen wird der Kommentarinhalt in einem rechteckigen Kommentarbereich dargestellt. Ein Anwender kann die Darstellung des Kommentarobjekts je nach Geschmack modifizieren. Unüblicherweise kann ein Kommentar sogar als Video-Kommentar abgespielt werden. Die wichtigsten Kommentartypen sind Notizzettel, Textmarkierung, Stempel, Wasserzeichen, Textboxen, Formen, Freihand-Markierung, Audio, Video und 3D-Illustrationen. Kommentare können optional ausgedruckt werden \cite{softx}. 

\subsection{Verweise}
Technisch gesehen sind Verweise oder Hyperlinks spezialisierte Kommentare ohne Symboldarstellung. Auf der Seite wird ein Ausschnitt zur Platzierung des Verweises gewählt, der über einem Inhaltselement (Text oder Bild) liegt. Der Verweis zeigt auf eine Seite oder einen Seitenbereich im geöffneten Dokument, eine andere PDF-Datei, eine E-Mailadresse oder URL \cite{softx}. 

\subsection{Formulare}
In PDFs kann man Formularfelder vom Typ Textfeld, Kontrollkästchen, Auswahlknopf, Kombinationsfeld, Auswahlliste, Schaltfläche, Barcode- oder Unterschriftsfeld erstellen. Ein Formularfeld ist ein Objekt zum Befüllen und Speichern von Felddaten. Die unterschiedlichen Formularfeldtypen weisen verschiedene Eigenschaften in Bezug auf Interaktivität und Gestaltung auf. Jedes Feld hat einen eindeutigen Namen im gesamten Dokument. Mit diesem unikalen Namen können Namensgruppen realisiert werden. Durch eine hierarchische Struktur mittels Teilnamen, die mit einem Punkt voneinander getrennt sind, mit dem äußersten Gruppennamen zuerst, geschrieben werden, können Felddaten noch besser und logischer beschrieben bzw. strukturiert werden. Jedes Feldobjekt geht Hand in Hand mit einem Widget, welches ein spezielles Kommentarobjekt zur Steuerung darstellt. Diese Widgets stehen für Werte oder Zustände der Felder und sind dafür verantwortlich, dass man Formulare im PDF-Dokument mit dem Computer, Tablet oder Smartphone ausfüllen kann. Außerdem ist es möglich unsichtbare Feldobjekte, die ohne das Widget platziert werden können, zu erstellen, um die PDF-Software anzusprechen. Häufiger werden mehrere Widgets mit einem Feldobjekt gekoppelt \cite{softx}. \\
Um elektronisch ausfüllbare Formulare zu verwenden, müssen zusätzlich in Acrobat Formularfelder auf die entsprechenden Stellen platziert werden. Falls ein Listenfeld verwendet wird, sollte man eine Schrift für die Listeneinträge im PDF einbetten. Formulare können einen druckbaren und nicht druckbaren Teil enthalten. In der Druckvorstufe müssen vor dem Druck alle Formularfelder eliminiert werden, damit alle Schriften eingebettet werden können \cite{schneeberger}. Es gibt 2 verschiedene Möglichkeiten von PDF Formularen: AcroForms (Acrobat Forms) oder Adobes proprietäre XFA forms, welche mit Version 2.0 von der \gls{iso} als veraltet markiert wurden. \cite{schneeberger}. AcroForms unterstützen das Abschicken (submit), Zurücksetzen und Importieren von Daten. Die submit-Aktion transferiert die Namen und Werte eines ausgewählten interaktiven Formularfelds zu einer vordefinierten URL. In der Praxis werden Formulare in einem Grafik- oder Layoutprogramm gestaltet und als PDF exportiert.

\subsection{Incremental Update}
Die ursprüngliche Version einer PDF-Datei bleibt erhalten, während das incremental update die Änderungen im Dokument enthält. Professionelle PDF-Programme können ähnlich einer Versionsverwaltung jede geänderte Version des Dokuments laden. Bei einfacheren PDF-Programmen wird lediglich die letzte Version geladen. Bei Verwendung von incremental updates kann man digital unterschriebene Dokumente ändern ohne dass die Unterschrift ungültig wird, da die Dokumentversion mit der digitalen Unterschrift eine andere Version ist als die nachträgliche Änderung. Dabei muss die digitale Unterschrift als incremental update gespeichert werden, sonst würde sie bei nachträglicher Dokumentenmodifikation unabhängig von der Änderungsart verfallen. Folglich sollten mehrfach signierte Dokumente ebenfalls mit der Option incremental update gespeichert werden \cite{softx}. Pro incremental update erhöht sich der Speicherplatz einer PDF-Datei.

\subsection{Kompression}
PDF-Dateien sind komprimiert und haben üblicherweise einen Bruchteil der Größe des Ursprungsformats. Dies wird durch Vermeidung von Redundanzen, Erhöhung der Entropie (Zeichendichte) und Weglassen von Informationen bewerkstelligt. Im Allgemeinen gibt es verlustfreie und verlustbehaftete Kompression. Die Kompressionsalgorithmen RLE, die genauso effiziente LZW, Flate-Komprimierung, ZIP und CCITT gehören zur verlustfreien Kompression. Zur verlustbehafteten Kompression zählen JPEG, JBIG2 und JPEG2000 \cite{schneeberger}. Kompressionsalgorithmen sind nicht auf bestimmte Dateiformate beschränkt. In PDF können die folgenden Kompressionsalgorithmen für Bilder verwendet werden: IP, RLE, JPEG, JPEG 2000 und JBIG2. Eine hohe Bildqualität im PDF bedeutet eine größere Datei. Faktoren, die die Bildqualität beeinflussen, sind Breite x Höhe, Farbtiefe, Farbraum und die Kompressionsmethode \cite{softx}. \\
Außerdem ist es möglich, eine Datenreduktion durch Neuberechnung zu erzielen. Hierbei wird das verlustbehaftete Downsampling verwendet und führt häufig zu nicht befriedigenden Ergebnissen. Es gibt als Neuberechnungsmethoden die eher im Ergebnis mangelhafte Kurzberechnung, sowie die leistungsfähigere durchschnittliche und bikubische Neuberechnung \cite{schneeberger}. PDF-Dateien können zur Weboptimierung serialisiert (linearisiert) werden, sodass Teile des PDFs während des Ladevorgangs dargestellt werden. Liegen unkomprimierte Elemente im Dokument vor, werden diese beim Speichern durch die Flate-Komprimierung, die auch den ZIP-Algorithmus verwendet, komprimiert.

\subsection{Ebenen}
Ebenen werde auch als optional content layers bezeichnet und stellen quasi mehrere Inhaltsschichten auf einer einzelnen PDF-Seite dar, wobei jede Seite im Dokument beliebig viele Ebenen enthalten kann. Jede Ebene kann PDF-Objekte sozusagen logisch gruppieren. Die Bearbeitung von Objekten auf einer Ebene wirkt sich ausschließlich auf diese Ebene aus. Darüber hinaus können Objekte mehreren Ebenen oder keiner Ebene zugeordnet werden. Ebenen können ein- und ausgeblendet, ihre Reihenfolge verändert, gesperrt, zusammengeführt, aus anderen PDF-Dateien importiert und für unterstützende Dateiformate von Adobeprogrammen, z.B. Photoshop, Illustrator oder InDesign, exportiert werden. Zusätzlich kann man eine Ebenennavigation mit Hilfe von Links und Lesezeichen konstruieren, um Ebenensichtbarkeit für den Betrachter zu steuern \cite{adobe-ebenen}. 

\subsection{JavaScript}
In PDF kann man Ereignisse Aktionen zuordnen, d.h. bei Eintreffen eines Ereignisses wird automatisch eine Aktion ausgeführt. Ein Ereignis ist eine bestimmte Statusänderung von Objekten oder ein interaktives Anwenderereignis. Dabei lässt sich JavaScript-Code aufrufen, dessen Aktion mit Lesezeichen, Verweisen, Seiten, Formularfelder und Dokumentereignissen verknüpft ist \cite{softx}. Diese JavaScript-Erweiterung für Acrobat ist eine proprietäre Technologie von Adobe. Viele andere PDF-Programme bieten keine Unterstützung für JavaScript \cite{wiki-pdf-engl}. 