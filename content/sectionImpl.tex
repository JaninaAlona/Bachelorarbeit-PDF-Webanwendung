\section{Umsetzung in Code}
Im Folgenden werde ich auf die Implementierung der einzelnen Module eingehen und die Besonderheiten, Erfahrungen und Probleme während der Programmierung beschreiben. Um die PDF Web App auch offline nutzen zu können, habe ich die Library-Dateien im Ordner libs direkt eingebunden, anstatt \gls{cdn}-Links zu den jeweiligen Libraries zu verknüpfen. Zur Positionierung von HTML-Elementen habe ich hauptsächlich auf das CSS Flexible Box (Flex Box) Layout zurückgegriffen. 

\subsection{Werkzeuge}
\begin{itemize}
	\item Entwicklungsumgebung: Visual Studio Code
	\item ausführende Programme: Firefox Browser
	\item Sprachen: JavaScript, CSS, HTML
	\item Libraries: Vue JS 3 Version 3.0.2, PDF-LIB, Fontkit zur Einbettung von benutzerdefinierten Fonts, pdf.js, zip,js, Bootstrap Filestyle Version 2.1.0, Bootstrap Version 4.3.1, jQuery Version 3.6.0 dependency von Filestyle, Alwan für den Color Picker
	\item Tutorials: Tiddly Wiki
\end{itemize}

\subsection{Links der Resources}
\begin{itemize}
	\item \url{https://vuejs.org/}
	\item \url{https://pdf-lib.js.org/}
	\item \url{https://www.npmjs.com/package/@pdf-lib/fontkit}
	\item \url{https://github.com/mozilla/pdf.js}
	\item \url{https://github.com/gildas-lormeau/zip.js/}
	\item \url{https://github.com/markusslima/bootstrap-filestyle}
	\item \url{https://getbootstrap.com/}
	\item \url{https://jquery.com/}
	\item \url{https://github.com/SofianChouaib/alwan}
	\item \url{https://tiddlywiki.com/}
\end{itemize}

\subsection{Ordnerstruktur}
Abbildung \ref{fig:folders} stellt die Ordnerstruktur des Projekts dar. 

\begin{figure}[!htbp]
	\centering
	\includegraphics[width=0.3\textwidth]{"images/folders.png"}
	\caption{Ordnerstruktur der PDF Web App}
	\label{fig:folders}
\end{figure}

In den Ordnern js sind meine selbst geschriebenen JavaScript- und in css meine eigenen CSS-Dateien abgelegt. Genauso wurden alle HTML-Dateien von mir erstellt. Der Ordner libs enthält alle JavaScript- und CSS-Dateien von externen Libraries. Die Datei pdf.worker.min.js ist eine dependency von pdf.js und fontkit.umd.min.js gehört zur Library PDF-LIB, um benutzerdefinierte Fonts einzubinden. Im Prinzip hat jedes Modul seine eigene JavaScript-Datei. Das Script input\_control\_script.js, welches global implementiert wird, enthält Kontrollfunktionen für Benutzereingabefelder (Input Control) und die ZIP-Download-Funktionen. Der Editor teilt sich die Dateien pdf\_reader\_script.js und index\_reader\_style.css mit dem Reader. Die CSS-Datei widget\_style.css deckt den Creator, Merger und Splitter ab. Der Editor besteht aus mehreren Scripten: write\_pdf\_script.js für Text, draw\_pdf\_script.js für Drawings, geometry\_pdf\_script.js für Shapes, images\_pdf\_script.js für Images, layer\_buttons\_script.js fürs Layers Menu, select\_all\_script.js für die Selection and Deselection Filter Menus und stack\_layers\_script.js für Layers-Managementfunktionen. Reader, Splitter und Merger besitzen jeweils eine separaten HTML-Seite. Das Editormodul ist eine einzelne HTML-Seite, bei der je nach Funktion die entsprechenden Schaltflächen für die Elementoperationen mit display: flex eingeblendet und display: none ausgeblendet werden. 

\subsection{Layoutgestaltung}
Das Hauptmenü wollte ich minimalistisch, klar und übersichtlich gestalten. Daher habe ich für die Startseite lediglich einen weißen Hintergrund gewählt mit einem oberen Balken in der Farbe \#333 als dunkles Grau zur Hervorhebung der Hauptmenübuttons. Dasselbe dunkle Grau wiederholt sich in den beiden oberen Reader-Leisten und dem Reader-Hintergrund. Die Buttons haben einen vordefinierten Style von der Library Bootstrap. In der Bootstrap-Library kann der Style durch fixe Klassen bestimmt werden. Alle grünen Buttons und die selection menus im Creator und Splitter haben die Bootstrap-Klasse btn-success. Der dunkelgraue Button mit grüner Schrift und Umrandung des aktiven Hauptmenü-Buttons besitzt die Bootstrap-Klasse btn-outline-success. Die Beschriftung des Buttons für Choose file (INPUT type file HTML-Element) ist standardmäßig in Deutsch gehalten und lässt sich grundsätzlich nicht in englischer Sprache programmieren. Aus diesem Grund habe ich die Library Bootstrap Filestyle verwendet, um den Choose file-Button in Englisch zu beschriften. Um eine Designkonsistenz zu erzielen, habe ich die Bootstrap-Buttons flächendeckend in der GUI in verschiedenen Stylevarianten verwendet. Lediglich bei Tools im Editor habe ich einen eigenen Style für die Buttons der Elementoperationen programmiert. Diese weißen Buttons besitzen eine schwarze Schrift mit einer border-width von 2 Pixeln, einem border-radius von 5 Pixeln und einer border-color von \#333. Der inaktive, weiße Modus-Button und die ausgeschalteten Selection und Deselection Filter in Layers sind mit dem Bootstrap-Style btn-light versehen. Alle dunkelgrauen Buttons wurden mit dem Bootstrap-Style btn-dark implementiert. Jegliche rosa Flächen und ausgewählte Layers haben die Farbe \#dabdb6. Der hellgraue Operations Bar und das Namensfeld einer Layer sind in \#8c8c8c (Hellgrau) gefärbt. Tools und aktivierte checkboxes haben eine Hintergrundfarbe in \#5eb873 (Türkis). Der Hintergrund von Layers, Layers Menu, sowie die Seitengruppen und Tools haben einen Alphawert von 0.8. Abgewählte Layers, selection menus in Tools, der Hintergrund beim Output-Dateinamen und ausgewählte Dateinamen im Merger sind schwarz. Alle Horizontal-Leisten mit Operationen haben eine Höhe von 58 Pixeln und dessen Buttons einen horizontalen Abstand von 5 Pixeln. Layers und Tools sind jeweils 190 Pixel breit. In Dateinamen von Font- und Bilddateien wird nach 15 Zeichen ein Zeilenumbruch eingefügt. Der Dateiname des Output-PDFs ist auf 50 Zeichen beschränkt. Im Reader befinden sich zwischen den gerenderten Seiten 20 Pixel Abstand. Alle input fields, Scrollbars und INPUT type checkbox HTML-Elemente besitzen den default Style des Browsers. Die default Dateinamen für Reader und Editor sind wie folgt benannt: Source Dokument Dateiname, ein Unterstrich und edited. Der Creator benennt standardmäßig die Output-Datei mit blank\_pdf und der Merger mit merged\_pdf. Im Splitter wird als Dateibenennungsschema der source Dateiname, ein Unterstrich und split verwendet. 

\subsection{Input Control}
Eine automatisierte Eingabekontrolle von Benutzereingabefeldern findet in allen Modulen, außer dem Merger, statt. Alle input fields sind INPUT type text HTML-Elemente, anstatt type number – selbst wenn nur Zahlen als Eingabewerte gültig sind. Das hat folgenden Hintergrund: Es gibt eine Funktion namens restrictInputValues, die automatisch Benutzereingaben korrigiert. Die Funktion wird ausgelöst, sobald das input field verändert wurde (change event). Im Prozess wird der event handler zunächst entfernt, bevor er angeheftet wird. Würde der event handler nicht initial abgekoppelt werden, würde dieses Eingabefeld immer mehr event handler akkumulieren. RestrictInputValues entfernt jeglichen white space, wandelt die Benutzereingaben als String in Zahlen um und hält die eingegebenen Werte in einem gültigen Wertebereich. Außerdem habe ich eine Funktion convertInputToSuccess programmiert, die den Input-String in einen Zahlenwert parst und den geparsten Wert oder -1000 (invalid input) zurückgibt. Die return value von convertInputToSuccess bestimmt maßgeblich, ob Operationen, z.B. die Font Size im Writer, ausgeführt werden sollen. Für die Eingabe einer Seitenliste, existiert eine gesonderte Funktion convertPagelistToSuccess. An dieser Stelle wird bei invalid input die Benutzereingabe zurückgesetzt. Eine genauere Beschreibung des Verhaltens der Input Control werde ich im Unterkapitel Testdurchführung, Abschnitt Testauswertung, weiter ausführen.

\subsection{Einlesen einer PDF-Datei im Reader und Editor}
Die Funktion für den Choose file-Button ist in Codeabschnitt \ref{code:file} dargestellt.

\begin{lstlisting}[style=ES6, caption={Einlesen einer PDF-Datei}, label=code:file, breaklines=true]
	let inputFileButtons = document.getElementsByClassName('inputfile');
	for (let i = 0; i < inputFileButtons.length; i++) {
		inputFileButtons[i].addEventListener("change", function(e) {
			resetAllModes();
			// ...
			file = e.target.files[0];
			const fileReader = new FileReader(); 
			fileReader.onload = function() {
				const typedarray = new Uint8Array(this.result);
				const pdfBytes = typedarray;
				const loadingTask = pdfjsLib.getDocument(typedarray);
				loadingTask.promise.then(async (pdf) => {
					if (file.name.endsWith(".pdf")) {
						try {
							outputPDF = await PDFLib.PDFDocument.load(pdfBytes);
						} catch(encryptedErr) {
							encrypted = true;
						}
						if (!encrypted) {
							if (pdf._pdfInfo.numPages <= 5000) {
								pdfState.pdf = pdf;
								pdfState.originalPDFBytes = pdfBytes;
								pdfState.existingPDFBytes = pdfBytes;
								// ...
								pdfState.lastPage = pdf._pdfInfo.numPages;
								// ...
								adjustPDFToUserViewport();
								pdfState.renderedPage = 0;
								// ...
								let readerMode = true;
								let editorMode = false;
								const openPDFs = document.getElementsByClassName("open_pdf");
								for (let i = 0; i < openPDFs.length; i++) {
									if (openPDFs[i].classList.contains("btn-outline-success")) {
										readerMode = true;
										break;
									} else if (openPDFs[i].classList.contains("btn-success")) {
										readerMode = false;
									}
								}
								const editorBtns = document.getElementsByClassName("editor_btn");
								for (let i = 0; i < editorBtns.length; i++) {
									if (editorBtns[i].classList.contains("btn-outline-success")) {
										editorMode = true;
										break;
									} else if (editorBtns[i].classList.contains("btn-success")) {
										editorMode = false;
									}
								}
								if (!readerMode && editorMode) {
									onetimeSetup = true;
									// ...
									setTimeout(initEditor, 300);
								}
								startRender = performance.now();
								await renderPage(pageCounter, false);
							} else {
								pagesError = true;
								for (let i = 0; i < pagesErrorWidgets.length; i++) {
									pagesErrorWidgets[i].style.display = "flex";
								}
							}
						} else {
							encryptedError = true;
							for (let i = 0; i < encryptedErrorWidgets.length; i++) {
								encryptedErrorWidgets[i].style.display = "flex";
							}
						}
					}
				}).catch(unsupportedFileErr => {
					noPDFError = true;
					for (let i = 0; i < noPDFErrorWidgets.length; i++) {
						noPDFErrorWidgets[i].style.display = "flex";
					}
				});
			}
			if (file) {
				fileLoaded = true;
				fileReader.readAsArrayBuffer(file);
			}
		}, false);
	}
\end{lstlisting} 

Bei jedem Klick auf den Choose File-Button werden alle Modi für aktive Operation-Buttons des Editors und andere Steuerungsmodi in der Funktion resetAllModes auf false gesetzt. Die Datei wird mittels eines FileReader-Objekts eingelesen und dem load event wird eine anonyme Funktion zugewiesen. Die eingelesene PDF-Datei, bestehend aus einem Uint8Array, wird in den properties originalPDFBytes und existingPDFBytes des globalen Objekts pdfState gespeichert. Die PDF-LIB-Funktion für das Laden eines PDFs akzeptiert einen Parameter vom Typ Uint8Array. Die PDF-LIB-Speicherfunktion gibt ein als Uint8Array serialisiertes PDF zurück. Im Editor wird bei jedem Downloadvorgang des modifizierten PDFs das originale PDF aus Grundlage zur Einbettung verwendet (originalPDFBytes). Das Objekt pdfState, welches den aktuellen Anzeige- und Datenstatus des eingelesenen PDFs repräsentiert, ist in einem Codeschnipsel \ref{code:pdfstate} dargestellt:

\begin{lstlisting}[style=ES6, caption={Objekt für den Status eines geöffnetes PDF-Dokuments}, label=code:pdfstate, breaklines=true]
	let pdfState = {
		pdf: null,
		currentPage: 1,
		lastPage: 1,
		renderedPage: 0,
		zoom: 1,
		originalPDFBytes: null,
		existingPDFBytes: null,
		originalWidths: [],
		originalHeights: []
	}
\end{lstlisting} 

Eine loadingTask wird durch die pdf.js-Funktion getDocument erzeugt, die ein libraryinternes pdf-Objekt zurückliefert. Dieser Return-Wert wird in pdfState.pdf abgelegt. In pdfState.lastPage wird die Seitenzahl der letzten Seite gespeichert. Das pdf.js-eigene pdf-Objekt wird für das Rendern der PDF-Seiten im Reader und Editor benötigt. Weiter in der anonymen Funktion wird das PDF dem Viewport des Browsers angepasst. Anschließend wird pdfState.renderedPage initialisiert. Dieses property enthält die aktuell gerenderte Seite, die am Ende des Rendervorgangs der letzten Seite des PDFs entspricht. Ob es sich um den Reader oder Editor handelt, wird überprüft, indem die Bootstrap-Klassen der zugehörigen Hauptmenü-Buttons abgefragt werden. Falls der Editor geöffnet wurde, wird er mit einer Zeitverzögerung von 300 ms aufgebaut. Hintergrund dessen ist, dass die vorherige \gls{dom}-Version erst gerendert werden muss, sonst ergibt sich bei Zugriff auf HTML-Collections – ein arrayähnliches Objekt – eine Länge von Null. Am Ende wird der Rendervorgang durch die rekursive, asynchrone Funktion renderPage in Gang gesetzt. Das \gls{dom} ist eine Web API für Webdokumente wie HTML-Dokumente. Es repräsentiert die Webseite als nodes und Objekte, sodass Programmiersprachen die Dokumentenstruktur, -style und -inhalt modulieren können. Strukturell besteht das \gls{dom} aus dem \gls{dom} tree, dessen nodes den HTML-Inhalt repräsentieren. Im \gls{dom} sind alle properties, Methoden und events für die Manipulation und Erzeugung von Webseiten als Objekte organisiert \cite{mozilla-dom}. 

\subsection{Renderfunktion}
Im Codeauszug \ref{code:render} ist die Funktion renderPage abgebildet.

\begin{lstlisting}[style=ES6, caption={Renderfunktion}, label=code:render, breaklines=true]
	async function renderPage(num, renderSingle) {
		pdfState.pdf.getPage(num).then(function(page) {
			let viewport = page.getViewport({
				scale: pdfState.zoom
			});
			let viewportOriginal = page.getViewport({
				scale: 1
			});
			let canvas;
			let div;
			const pdfViewers = document.getElementsByClassName("pdf_viewer");
			for (let i = 0; i < pdfViewers.length; i++) {
				const pdfViewer = pdfViewers[i];
				// ...
				if (writeLayerStack.length < pdfState.pdf._pdfInfo.numPages) {
					div = document.createElement("div");
					div.style.display = "flex";
					div.width = viewport.width;
					div.height = viewport.height;
					div.style.width = viewport.width + "px";
					div.style.height = viewport.height + "px";
					div.style.marginBottom = "20px";
					pdfState.originalWidths.push(viewportOriginal.width);
					pdfState.originalHeights.push(viewportOriginal.height);
					div.setAttribute('data-write', pageCounter);
					div.classList.add("write_layer");
					canvas = document.createElement("canvas");
					canvas.style.display = "flex";
					canvas.width = viewport.width;
					canvas.height = viewport.height;
					canvas.style.width = viewport.width + "px";
					canvas.style.height = viewport.height + "px";
					canvas.setAttribute('data-page', pageCounter);
					canvas.classList.add("render_context");
					div.appendChild(canvas);
					pdfViewer.appendChild(div);
					writeLayerStack.push(div);
				} else if (writeLayerStack.length === pdfState.pdf._pdfInfo.numPages) {
					if (!renderSingle) {
						div = writeLayerStack[pageCounter-1];
						canvas = writeLayerStack[pageCounter-1].childNodes[0];
					} else {
						div = writeLayerStack[num-1];
						canvas = writeLayerStack[num-1].childNodes[0];
					}
					div.width = viewport.width;
					div.height = viewport.height;
					div.style.width = viewport.width + "px";
					div.style.height = viewport.height + "px";
					canvas.width = viewport.width;
					canvas.height = viewport.height;
					canvas.style.width = viewport.width + "px";
					canvas.style.height = viewport.height + "px";
				}
			}
			const context = canvas.getContext('2d');
			let renderTask = page.render({
				canvasContext: context,
				viewport: viewport
			});
			if (!renderSingle) {
				renderTask.promise.then(function() {
					pdfState.renderedPage = pageCounter;
					// ...
					pageCounter++;
					// ...
					if (pdfState.pdf != null && pageCounter <= pdfState.pdf._pdfInfo.numPages) {
						renderPage(pageCounter, false);
					}
				});
			}
		});
	}
\end{lstlisting} 

Die Funktion ruft sich rekursiv auf, falls noch nicht alle Seiten gerendert wurden. Sie kann den Rendervorgang für eine oder alle Seiten vornehmen. Eine einzelne Seite wird neu gerendert, sobald sie im Reader gedreht wurde. Pro Funktionsaufruf wird der globale pageCounter hochgezählt. Mittels des pdfState.pdf-Objekts wird die aktuelle Seite geholt und ein Seitenobjekt von pdf.js entgegengenommen. Aus diesem Seitenobjekt wird der Viewport der Seite extrahiert und eine Skalierung festgelegt. PdfState.zoom, welches die aktuelle Skalierung des gesamten PDF-Dokuments angibt, wird der Viewport-Skalierung zugewiesen. Viewport besitzt width und height properties. Zusätzlich wird der Viewport mit einer Skalierung von 1 gespeichert. Hintergrund dessen ist, dass im weiteren Verlauf von renderPage die Arrays pdfState.originalWidths und pdfState.originalHeights die Originalgrößen jeder Seite aufnehmen sollen. Der Container des gerenderten PDFs besteht aus mehreren verschachtelten DIV-HTML-Elementen, wobei das innere Element die Klasse pdf\_viewer besitzt. Dem inneren pdf\_viewer-Container werden pro Ausführung von renderPage ein DIV mit der Klasse write\_layer und ein data-Attribut data-write, was die zugehörige Seitenzahl speichert, hinzugefügt. Die write\_layer symbolisiert einen Seitencontainer, der die gerenderte PDF-Seite und beigefügten Elemente des Editors enthält. Die gerenderte PDF-Seite wird in einem CANVAS-HTML-Element als erstes child der write\_layer dargestellt. Diese CANVAS hat die Klasse render\_context und ein data-Attribut data-page für die identische Seitenzahl der write\_layer. Sowohl die write\_layer als auch der render\_context werden in der Größe des aktuell skalierten Viewports angelegt. Die write\_layers werden zusätzlich in einem Array writeLayerStack gesammelt. WriteLayerStack hält die write\_layers in der Reihenfolge der PDF-Seiten im Dokument. Während eines erneuten Rendervorgangs wird der render\_context wiederverwendet. Anschließend wird mittels der pdf.js-Library die Seite gerendert. Nach Beendigung der renderTask wird eine anonyme Funktion ausgeführt. Die pageCounter-Variable wird pdfState.renderedPage zugewiesen. Dieses property wird als Maximalwert für die Input Control bei der Navigation zu einer bestimmten Seite verwendet.

\subsection{Implementierung der Vue JS 3-Module}
Die Module Creator, Splitter und Merger verwenden das Framework Vue JS 3, zwar nicht in nativer Form als \gls{spa}, jedoch als stand-alone widgets. Beim Creator wird annähernd die eingestellte PDF-Seitengröße als Breite und Höhe der MediaBox durch die Berechnung in den Codezeilen \ref{code:mediabox}, die ich durch Ausprobieren herausgefunden habe, gesetzt.

\begin{lstlisting}[style=ES6, caption={Berechnung der PDF-Seitengröße}, label=code:mediabox, breaklines=true]
const pageWFactor = (blankPageWidth * 1000) / 352.8;
const pageHFactor = (blankPageHeight * 1000) / 352.8;
for (let i = 0; i < blankNumOfPagesCount; i++) {
	page = pdfDoc.addPage();
	page.setMediaBox(0, 0, pageWFactor, pageHFactor);
}
\end{lstlisting} 

Beim Anklicken der Portrait-Orientierung wird in den Width and Height input fields der kleinere Wert der Seitendimensionen für die Width gesetzt und der größere Wert für die Height. Im Falle von Landscape werden Width und Height vertauscht und Quadratic orientiert sich an der Width.
\par
Wenn man den Splitter öffnet, findet man zunächst einen gesperrten Save-Button vor. Der Save-Button wird erst benutzbar, wenn man eine PDF-Datei und Split-Methode im selection menu ausgewählt hat. Der Dateiname, begrenzt auf 50 Zeichen, wird unter dem Choose file-Button abgebildet. Die Methode list of pages kann man auch zum Splitten nach einer bestimmten Seite verwenden. Erst wenn man diesen Menüpunkt ausgewählt hat, wird das input field für die Seitenliste aktiviert. Die Split-Funktion fürs Splitten nach geraden und ungeraden Seiten heißt SplitAfter. Basierend auf den Rest Modulo 2 wird die Funktion nur ausgeführt, wenn bei der even pages-Split-Methode mehr als 3 Seiten im Input-PDF vorliegen bzw. bei der odd pages-Split-Methode mehr als 2 Seiten vorhanden sind. Andernfalls wird der Save-Button ebenfalls deaktiviert. Die Input-PDF Bestandteile als Dokumentobjekte von PDF-LIB, die nach jedem Split-Vorgang bei der PDF-LIB-Funktion PDFDocument.create entstehen, werden in einem globalen Array splittedPDFs gesichert. Die Arraybestandteile werden durch Klick auf Save einzeln zu einem Uint8Array serialisiert und in einem globalen Array pdfBytesList gespeichert. PdfBytesList wird dann der compressMultipleToZip-Funktion übergeben. Jedes gesplittete PDF enthält den Ursprungsdateinamen, einen Unterstrich und einen Index beginnend mit 0, der das Teil-PDF in der Split-Reihenfolge aufsteigend nummeriert. Alle Source-PDF-Bestandteile werden in einen ZIP-Ordner verpackt. Am Ende des Downloads werden splittedPDFs und pdfBytesList wieder als leeres Array gesetzt.
\par
Im Merger wird eine geöffnete Datei als Uint8Array einem globalen Array selectedPDFBytes hinzugefügt. Zusätzlich wird einem UL-HTML-Element, dem Dateilistencontainer, pro Datei ein LI-HTML-Element mit der Klasse file\_unselected, sowie dem Dateinamen, gekürzt auf 50 Characters und einem click event handler für die Dateimarkierung, hinzugefügt. Erst wenn man mehr als eine oder maximal 100 Dateien ausgewählt hat, werden der Save- und Remove-Button funktionstüchtig gemacht. Außerdem wird dem LI eine Reihe an drag events hinzugefügt: Dragstart, dragover und drop. Funktion der drag events ist, dass der User die Listeneinträge in ihrer Reihenfolge verschieben kann. Dabei spiegelt selectedPDFBytes die Reihenfolge der Listeneinträge wieder. Für das dragging muss der Listeneintrag nicht markiert sein, jedoch beim Entfernen von einzelnen Listeneinträgen schon. Ist ein Listeneintrag schwarz markiert, so wird die Klasse file\_unselected vom entsprechenden LI entfernt und stattdessen die Klasse file\_selected hinzugefügt. Hat man so viele Listeneinträge entfernt, dass nur noch 1 oder keine Datei übrig bleibt, so wird Save abermals gesperrt. Ebenso wird Remove bei 0 Elementen deaktiviert.

\subsection{Struktur der Editorelemente}
Die 4 Editorelemente Text, Drawing, Shape und Image werden bei Erzeugung auf der Seite in 4 zugehörigen globalen Arrays als Elementlisten in der Erstellungsreihenfolge abgelegt: userTextList für Text, drawLayerStack für Drawings, geometryPointsList für Shapes und userImageList für Images. Für die Erstellung der Editorelemente verwende ich zum einen Objekt, die die Nodestruktur der Elemente in der Webseite abbilden und zum anderen Objekte, die die elementspezifischen Eigenschaften repräsentieren. Diese Objekte werden nicht direkt verwendet, sondern werden als Prototypen nach dem Konzept der prototypischen Objektorientierung für die Erstellung konkreterer Objekte verwendet, die vom Prototyp erben. Im Codeabschnitt \ref{code:control-point} ist das strukturelle Prototyp-Objekt für Text, Drawing und Image abgebildet. Der darauffolgende Codeabschnitt \ref{code:shape-controller-point} zeigt das strukturelle Prototyp-Objekt für Shape.

\begin{lstlisting}[style=ES6, caption={Prototyp-Objekt für die Node-Struktur von Text, Drawing und Image}, label=code:control-point, breaklines=true]
	let controlPoint = {
		controlBox: null,
		editImg: null,
		elementToControl: null,
		type: '',
		layer: null,
		page: 1,
		x: 0,
		y: 0,
		index: 0,
		setControlPoint() {
			let div = document.createElement("div");
			div.style.position = "absolute";
			div.style.left = this.x + "px";
			div.style.top = this.y + "px";
			div.setAttribute('data-page', this.layer.getAttribute("data-write"));
			div.setAttribute('data-index', this.index);
			div.classList.add(this.type);
			div.classList.add("box");
			this.controlBox = div;
		}
	}
\end{lstlisting}

\begin{lstlisting}[style=ES6, caption={Prototyp-Objekt für die Node-Struktur von Shape}, label=code:shape-controller-point, breaklines=true]
	let shapeControllerPoint = {
		controlBox: null,
		editImg: null,
		elementToControl: null,
		layer: null,
		page: 1, 
		x: 0,
		y: 0,
		index: 0, 
		rotation: 0,
		originX: 0,
		originY: 0,
		setControlPoint() {
			let div = document.createElement("div");
			div.style.position = "absolute";
			div.style.left = this.x + "px";
			div.style.top = this.y + "px";
			div.setAttribute('data-page', this.layer.getAttribute("data-write"));
			div.setAttribute('data-index', this.index);
			div.classList.add("shape");
			div.classList.add("box");
			this.controlBox = div;
		},
		rotateControlPoint() {
			this.controlBox.style.marginLeft = this.originX + "px";
			this.controlBox.style.marginTop = this.originY + "px";
			this.controlBox.style.transform = "rotate(" + this.rotation + "deg)";
		}
	}
\end{lstlisting}  

Vom controlPoint-Prototyp leitet das Objekt controlP ab und von shapeControllerPoint leitet shapeControllerP ab. ShapeControllerPoint stellt einen spezialisierten controlPoint dar, der mit Rotationseigenschaften erweitert wurde. ControlP und shapeControllerP-Objekte werden direkt in userTextList, drawLayerStack, userImageList und geometryPointsList gespeichert. Die Prototypen besitzen ein Attribut elementToControl, um die Objekte mit elementspezifischen Eigenschaften zu speichern. Dessen Prototypen heißen userText für Text, drawLayer für Drawings, shape für Shapes und userImage für Images. Sie sind in den Codesegmenten \ref{code:user-text}, \ref{code;draw-layer}, \ref{code:shaoe} und \ref{code:user-image} aufgeführt.

\begin{lstlisting}[style=ES6, caption={Prototyp-Objekt für die textspezifischen Eigenschaften}, label=code:user-text, breaklines=true]
	let userText = {
		pdfDoc: null,
		pdfBytes: null,
		text: '',
		x: 1,
		y: 1,
		size: 1,
		fontKey: null,
		font: null,
		lineHeight: 34,
		color: rgb(0, 0, 0),
		page: 1,
		opacity: 1.0,
		rotation: degrees(0),
		setTextElem() {
			this.pdfDoc.getPages()[0].drawText(this.text, {
				x: this.x,
				y: this.y,
				size: this.size,
				font: this.font,
				color: this.color,
				lineHeight: this.lineHeight,
				rotate: this.rotation
			});
		}
	}
\end{lstlisting}

\begin{lstlisting}[style=ES6, caption={Prototyp-Objekt für die drawingspezifischen Eigenschaften}, label=code:draw-layer, breaklines=true]
	let drawLayer = {
		paths: [],
		currentPathIndex: 0, 
		rotation: 0,
		wasRotated: false
	}
\end{lstlisting}

\begin{lstlisting}[style=ES6, caption={Prototyp-Objekt für die shapespezifischen Eigenschaften}, label=code:shape, breaklines=true]
	let shape = {
		context: null,
		type: "",
		x: 0,
		y: 0,
		xp2: 50,
		yp2: 50,
		width: 100,
		height: 100,
		stroke: 'rgba(0,0,0,1.0)',
		strokeWidth: 3,
		fill: '',
		useFill: false,
		useStroke: false,
		rotation: 0,
		page: 1,
		drawShape() {
			if(this.type === "rectangle") {
				let rectCenterX = this.x + this.width / 2;
				let rectCenterY = this.y + this.height / 2;
				this.context.save();
				this.context.beginPath();
				this.context.translate(rectCenterX, rectCenterY);
				this.context.rotate(this.rotation * Math.PI / 180);
				this.context.translate(-rectCenterX, -rectCenterY);
				if (this.useFill) 
				this.context.fillStyle = this.fill;
				if (this.useStroke) {
					this.context.strokeStyle = this.stroke;
					this.context.lineWidth   = this.strokeWidth;
				}
				if (this.useFill) {
					this.context.fillRect(this.x, this.y, this.width, this.height);
				}
				if (this.useStroke) {
					this.context.strokeRect(this.x, this.y, this.width, this.height);
				}
				this.context.restore();
			} else if (this.type === "triangle") {
				let triCenterX = this.x + this.width / 2;
				let triCenterY = this.y + this.height / 2;
				this.context.save();
				this.context.beginPath();
				this.context.translate(triCenterX, triCenterY);
				this.context.rotate(this.rotation * Math.PI / 180);
				this.context.translate(-triCenterX, -triCenterY);
				this.context.moveTo(this.x, this.y);
				this.context.lineTo(this.x, this.y + this.height);
				this.context.lineTo(this.x + this.xp2 + this.width, this.y + this.yp2);
				this.context.closePath();
				
				if (this.useFill) 
				this.context.fillStyle = this.fill;
				
				if (this.useStroke) {
					this.context.strokeStyle = this.stroke;
					this.context.lineWidth   = this.strokeWidth;
				}
				
				if (this.useFill) 
				this.context.fill();
				
				if (this.useStroke)    
				this.context.stroke();
				
				this.context.restore();
			} else if (this.type === "circle") {
				this.context.beginPath();
				this.context.ellipse(this.x, this.y, this.width / 2, this.height / 2, this.rotation * Math.PI / 180, 0, 2 * Math.PI);
				
				if (this.useFill) {
					this.context.fillStyle = this.fill;
				}
				if (this.useStroke) {
					this.context.strokeStyle = this.stroke;
					this.context.lineWidth   = this.strokeWidth;
				}   
				this.context.closePath();
				if (this.useFill) {
					this.context.fill();
				}
				if (this.useStroke)    
				this.context.stroke();
			}
		}
	}
\end{lstlisting}

\begin{lstlisting}[style=ES6, caption={Prototyp-Objekt für die imagespezifischen Eigenschaften}, label=code:user-image, breaklines=true]
	let userImage = {
		pdfDoc: null,
		pdfBytes: null,
		image: null,
		base64String: "",
		type: "",
		x: 1,
		y: 1,
		width: 1,
		height: 1,
		page: 1,
		opacity: 1.0,
		rotation: degrees(0),
		setImageElem() {
			this.pdfDoc.getPages()[0].drawImage(this.image, {
				x: this.x,
				y: this.y,
				width: this.width,
				height: this.height,
				rotate: this.rotation,
			});
		}
	}
\end{lstlisting}

Die spezifischen Objekte zu ihren Prototypen heißen currentUserText für userText, drawingLayer für drawLayer, currentShape für shape und currentUserImage für userImage. Sowohl die Nodestrukturobjekte als auch die Eigenschaftsobjekte werden beim Hinzufügen von Elementen erstellt und mit Standardwerten initialisiert. 

\subsection{Aufbau eines PDF-Seitencontainers}
Die Wurzel eines PDF-Seitencontainers bildet ein DIV mit der Klasse write\_layer. Die write\_layer hat mindestens ein child, eine CANVAS mit der Klasse render\_context. Dieser Fall liegt vor, wenn man noch keine Elemente zum PDF hinzugefügt hat. Man kann sich den render\_context als Zeichenfläche vorstellen, die durch die Dimensionen der PDF-Seite begrenzt ist, mit dem gerenderten Seiteninhalt als Hintergrund. Sobald der User Elemente der Seite hinzufügt, werden weitere Schichten auf den render\_context gelegt, wobei ein Element aus einer CANVAS, die ich editimg genannt habe, für die Darstellung und einem 40 x 40 Pixel-großen DIV namens controlBox für den Kontrollpunkt des Elements besteht. Die CANVAS besitzt 3 Klassen: Editimg, visible und eine Elementtyp-Klasse. Die Klasse editimg bezeichnet die CANVAS, die das gerenderte, einzelne Element enthält. Visible definiert den Sichtbarkeitsstatus für das Ein- und Ausblenden von Elementen über die Ebenensteuerung. Die Typklasse kann entweder text, drawing, shape oder image sein. Jedes editimg enthält außerdem ein data-page Atribut für die Seitenzahl und data-index für die Position in der Elementliste. Die Größe des editimgs entspricht immer der Größe des zugehörigen render\_contextes. Auf der controlBox habe ich eine Reihe von event handler definiert, die die Operationen im Operations Bar und Tools ausführen. Sie besitzt eine Typklasse und die Klasse box, um alle controlBoxes im Dokument ansprechen zu können. Die controlBox besitzt ebenfalls die gleichen data-Attribute wie das zugehörige editimg. Die editimgs werden bei Erzeugung aufeinander gestapelt mit dem zuletzt erstellten editimg zuoberst. Alle editimgs werden in einem Gruppierungs-DIV mit der Klasse editimg\_group akkumuliert. Ebenfalls werden alle controlBoxes in einem gruppierenden DIV mit der Klasse control\_group vereint. Die editimg\_group liegt über dem render\_context. Ganz oben befindet sich die control\_group, damit controlBoxes nicht von gerenderten Elementen verdeckt werden. Die control\_group und editimg\_group werden beim Hinzufügen des ersten Elementes auf der Seite erstellt. Jede write\_layer hat jeweils eine editimg\_group und control\_group. Das Node-Tree Segment für den Aufbau einer write\_layer ist in Screenshot \ref{fig:write-layer} veranschaulicht. 

\begin{figure}[!htbp]
	\centering
	\includegraphics[width=1\textwidth]{"images/write-layer.png"}
	\caption{Node-Tree Segment für den Aufbau einer write\_layer mit 4 Elementen jedes Typs}
	\label{fig:write-layer}
\end{figure}

Für editimg und controlBox gibt es gleichnamige properties in den Nodestrukturobjekten. Dort wird die CANVAS und das DIV abgespeichert. Außerdem gibt es ein layer property, in dem die zugehörige write\_layer gesichert wird. In den properties type wird der Elementtyp als String hinterlegt, page beschreibt die Seitenzahl, x und y sind die Koordinaten auf der Seite und index ist die Position des Elements in der zugehörigen Elementliste. Die Reihenfolge der Elemente in userTextList, drawLayerStack, geometryPointsList und userImageList ändert sich nie, es sei denn, ein Element wird vom User gelöscht. Der index in den Nodestrukturobjekten muss immer mit data-index von editimg und controlBox übereinstimmen. Über diesen maßgeblichen index kann auf das Element in der jeweiligen Elementliste zugegriffen werden.

\subsection{Hinzufügen von Elementen}
Jedes Element hat seine separate Funktion, um es der Seite hinzuzufügen. Jede Funktion enthält die Funktion createStackLayer, um eine Layer für das Element zu erstellen. Am Ende jeder Funktion zum Hinzufügen wird das Nodestrukturobjekt der zugehörigen Elementliste hinzugefügt. Wenn Text hinzugefügt werden soll, wird der Text in Form des Eigenschaftsobjekts currentUserText erstellt. Alle Eigenschaftsobjekte werden im Nodestukturobjekt property elementToControl verknüpft. Am Ende jeder Methode zum Hinzufügen von Elementen wird das Nodestrukturobjekt in seiner Elementliste gespeichert. Aus Sicht der Implementierung wird bei der Texterstellung ein PDF-Dokument mit einer Seite mit PDF-LIB erstellt, worauf der Text an der Stelle, wo der User mit der Maus auf die Seite geklickt hat, platziert wird. Die einzelne PDF-Seite hat die Größe des render\_contextes bzw. der write\_layer, da der render\_context und die write\_layer die gleiche width und height haben. Dann wird diese Seite mit einem transparenten Hintergrund mit pdf.js in das editimg gerendert. Das click event ist mit der write\_layer verknüpft. Alle Funktionen zum Hinzufügen von Elementen sind an das click event auf der write\_layer gebunden. In die PDF-Seite wird der Standard-Font TimesRoman eingebettet. Im Eigenschaftsstrukturobjekt befindet sich die PDF-LIB-Funktion drawText und sie wird in addText mit Standardwerten gefüttert. Zu beachten ist, dass currentUserText.opacity nicht an drawText übergeben wird, da die opacity-Implementierung der PDF-LIB in der Library defekt zu sein scheint. Ich habe die opacity selbst implementiert. Außerdem wird das Dokument-Objekt von PDF-LIB in currentUserText.pdfDoc abgelegt und die Uint8Array-Repräsentation des Dokuments ebenfalls. Die Seitenzahl wird in currentUserText.page abgelegt und die currentUserText.rotation mit 0 initialisiert. Alle Editorelemente werde bei der Erzeugung mit einer Rotation von 0 erstellt und alle Eigenschaftsobjekte haben ein page property. Des Weiteren wird der Textinhalt als String in currentUserText.text festgehalten, das Font-Objekt, das beim Einbetten des Fonts entsteht, in currentUserText.font, sowie die ArrayBuffer-Variante des, mit dem Dateibrowser geöffneten Fonts, in currentUserText.fontKey gespeichert. Text-Eigenschaftsobjekte sind die einzigen Eigenschaftsobjekte mit einem size Attribut.
\par
Die Funktion addImage funktioniert ähnlich wie addText. Das Image wird ebenfalls auf eine PDF-Seite platziert und mit transparentem Hintergrund in editimg gerendert. Im Vorgang wird das eingelesene Image, was in der Bilddateiliste in Tools ausgewählt wurde, in Form von Base64 in das einseitige PDF eingebettet. Zurückgegeben wird ein Bildobjekt von PDF-LIB, was in currentUserImage.image abgelegt wird. Das erzeugte PDF-Objekt wird in currentUserImage.pdfDoc und das Base64-Image in currentUserImage.base64String gespeichert. Shape- und Image-Eigenschaftsobjekte haben width und height Attribute. In currentUserImage befindet sich die setImageElem-Methode mit der drawImage-PDF-LIB-Funktion.
\par
Shapes und Drawings benötigen keine PDF-Dokumentseite bei ihrer Entstehung. Der shapeControllerP besitzt noch properties rotation, originX, originY und eine Funktion rotateControlPoint, um die controlBox mit dem Shape gemeinsam im gleichen Winkel zu drehen. In der Funktion addShape wird der Shape direkt auf das editimg mit CANVAS-Funktionen für Rechtecke und Ellipsen gezeichnet. Dreiecke müssen durch einzelne Linien gezeichnet werden. In currentShape.context wird der CANVAS 2D-Rendering-Context gespeichert. Er wird für die Zeichenoperationen verwendet. CurrentShape.type speichert den Shapetyp als String. Die Typen heißen rectangle, triangle und circle. Die properties xp2 und yp2 stehen für den dritten Punkt des Dreiecks (Triangle's 3rd Point). Sie werden nur bei Erstellung von Triangles initialisiert und bei der Triangle's 3rd Point-Operation verwendet. CurrentShape besitzt stroke, strokeWidth, fill, useFill und useStroke Attribute. UseFill und useStroke werden für die Steuerung der Checkboxen in der GUI bei Stroke Color und Fill Color in Tools verwendet. Man kann entweder nur eine Stroke Color oder nur eine Fill Color anwenden oder beide gleichzeitig. 
\par 
Die Funktion zum Hinzufügen von Drawings heißt draw und ist mit den mousedown, mousemove und mouseup events auf den write\_layers verbunden. Im drawLayer.paths-Array werden die Pfadpunkte des Zeichenvorgangs angehängt. Zuerst wird geprüft, ob sich auf der Seite schon ein Drawing befindet, falls nicht wird ein controlP und eine drawingLayer erstellt. Sie wird ebenfalls erzeugt, sobald der User New Layer gedrückt hat. Gezeichnet wird auf der selektierten Layer. Wenn keine DrawingLayer selektiert ist wird standardmäßig auf der zuletzt gezeichneten Layer der Seite gemalt. Bei Erstellung der Pfadsegmente wird, wie bei Shape, direkt auf dem context des editimgs gezeichnet und in drawingLayer.paths wird ein Objekt mit den x, y, line, color und compositeOp properties hinzugefügt. DrawingLayer.currentPathIndex dient der Traversierung von paths. Gezeichnet wird mit dem Linienaussehen context.lineCap round und context.lineJoin round, d.h. die Linie ist abgerundet an ihren Enden und Verbindungsstücken. Die context.globalCompositeOperation ist source-over. Beim Radieren wäre sie destination-out. Diese globalCompositeOperation wird in compositeOp in paths vermerkt. Die erase-Funktion funktioniert ähnlich wie draw, nur dass keine Layers, controlPs und drawingLayers erstellt werden. Diese Objekte werden nur in draw erstellt.

\subsection{Erzeugung von Layers}
Die Layer zum zugehörigen Element wird in dessen Erstellungsmethode addText, addShape, addImage oder draw erzeugt. Die Funktion heißt createStackLayer. Anfangs ist die Layer nicht ausgewählt und unlocked. Dies wird durch die Klassen layer\_unselected und unlocked angezeigt. Die Klassen werden einem DIV mit der Klasse layercontainer hinzugefügt. Layercontainer hat außerdem ein data-page, data-index und data-type Attribut. Page entspricht der zugehörigen Seitenzahl, index dem Editorelementindex in der Elementliste und type kann entweder text, drawing, shape oder image sein. Alle weiteren HTML-Elemente im layercontainer sind mit den gleichen data-Attributen versehen. In createStackLayer werden event listener an HTML-Elemente von layercontainer gebunden. Die checkbox ist mit einer Funktion hideLayer und einem input event verknüpft und das INPUT-HTML-Element mit der Klasse layername für den editierbaren Layernamen hört auf die Funktion markLayer mit einem click event. Das DIV mit der Id layer\_stack\_con wird mit einer Funktion moveLayer verknüpft, die dragstart, dragover und drop events enthält, damit man die Layers in ihrer Reihenfolge verschieben kann, was bewirkt dass die Editorelemente Richtung Vordergrund oder Hintergrund in ihrer z-Achsenreihenfolge verschoben werden. Außerdem wird die neu erzeugte Layer mit dem aktuellen Element automatisch in Rosa markiert. Abbildung \ref{fig:layer:stack} zeigt den Aufbau der HTML-Elemente des layer\_stacks.

\begin{figure}[!htbp]
	\centering
	\includegraphics[width=1\textwidth]{"images/layer-stack.png"}
	\caption{Node-Tree Segment für den Aufbau der Layers}
	\label{fig:layer-stack}
\end{figure}

Beim Kopieren von Layers, was die Funktion dublicateLayerByElement hervorruft, wird zunächst die Layer per cloneNode(true) dubliziert und erhält einen inkrementierten Index in seinen data-index Attributen, d.h. wenn es bereits 3 Textelemente gibt und eins davon dubliziert wird, erhält das dublizierte Element und die Layer den Index von 3 (Index startet bei 0). Die dublizierte Layer wird hinter der source Layer im layer\_stack eingefügt. Folglich liegt das kopierte Element über dem source Element in der z-Achsenreihenfolge. Außerdem wird der kopierten Layer ein input event mit der hideLayer-Funktion und ein click event mit der markLayer-Funktion angeheftet. Zusätzlich wird erneut moveLayer auf dem layer\_stack\_con ausgeführt, damit die drag events auf der neuen Layer ausführbar sind. Die Funktion dublicateElement kopiert das Element der dublizierten Ebene. Dabei wird das Nodestrukturobjekt aus seiner Liste geholt und anhand dessen ein neues Nodestrukturobjekt, samt Eigenschaftsobjekt, erzeugt und dessen properties werden mit den Werten der source Objekte initialisiert. Bei kopierten drawingLayer-Objekten ist zu beachten, dass das Objekt mit der deepCopy-Funktion nicht nur die Referenzen kopiert, sondern tatsächlich das Objekt und alle Unterobjekte in paths. Das kopierte Objekt wird seiner Elementliste hinzugefügt und die source Layer bleibt ausgewählt, während die kopierte Layer nicht ausgewählt ist. Locked, kopierte Objekte bleiben locked.

\subsection{Box Mode Modi}
Bei jeder Operation auf einem Element im Operations Bar und Tools, außer bei den Operationen zum Hinzufügen von Elementen, wird zunächst geprüft, ob der User sich im Box Mode oder Layer Mode befindet und, ob die Layer locked ist. Alle Operationen im Box Mode besitzen einen Modus. Jeder Modus wird mit seinem Boolean-Wert in einem Modi-Array festgehalten. Die Modusliste für Operationen auf Text heißt userModes, für Drawings userModesDrawer, für Shapes userModesGeometry und für Images userModesImages. Bei den meisten Buttons im Reader und Editor wird eine Funktion resetAllModes ausgeführt. Sie setzt alle Modilistenbestandteile auf false. Dadurch kann der User einen Operationsmodus verlassen, wenn er eine andere Operation ausführt und in ihren Modus gelangen. Somit kann der User beispielsweise mehrere Textelemente mit einem Klick auf Font Color in Rot färben, ohne erneut auf Font Color klicken zu müssen. Der zugehörige Modus der Operation wird im event handler auf true gesetzt, falls der User sich im Box Mode befindet.

\subsection{Zoomfunktionalität}
Bei den Zoom Operationen ist eine Verzögerung von 300 ms eingebaut. Das liegt daran, dass man nicht zu schnell hintereinander Zoomen sollte, da sonst ein Fehler geworfen wird, dass man nicht gleichzeitige Renderoperationen auf dem gleichen CANVAS-HTML-Element ausführen kann. Bei den Funktionen zoomIn, zoomOut und enterZoomFactor wird erst eine Funktion namens placeEditorElements ausgeführt, die die controlBoxes und Editorelemente passend zum Zoomfaktor auf der Seite platziert. Alle Editorelemente werden neu und verkleinert bzw. vergrößert gezeichnet. ControlBoxes behalten beim Zoomen ihre Größe bei. Danach werden alle Seiten mit dem neuen Zoomfaktor gerendert.

\subsection{Downloadfunktion}
Die Downloadfunktion für Editorelemente wird in einer promise chain ausgeführt. Zunächst wird der aktuelle Zoomfaktor gespeichert und das PDF-Dokument wird auf 500 \% Prozent vergrößert. Das hat den Hintergrund, dass alle Editorelemente anschließend als DataURL in Form eines PNG-Bildes in das PDF mit originalPDFBytes eingebettet werden. Genauer gesagt werden alle CANVAS-HTML-Elemente editimg mit der Klasse visible als DataURL eingebettet. Editimgs erhalten die Klasse hidden, wenn Layers im Layers Menu über die grüne checkbox ausgeblendet werden. Nach der Einbettung der PNG-Bilder mit 500 \% Skalierung wird die Datei zu einem ZIP-Archiv komprimiert. Das Archiv wird zu einem ObjectURL umgewandelt und durch ein neu erstelltes Anchor-HTML-Element, was dem Body hinzugefügt wird, gedownloaded. Anschließend wird das Anchror-Element wieder vom Body entfernt und das PDF im Editor zoomt zurück auf seine vorherige Skalierung. Im nächsten Abschnitt werde ich einige Tests der PDF Web App durchführen, um Funktion und Performance zu erörtern. 