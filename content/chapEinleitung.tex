\chapter{Einleitung}
%
Das vorliegende Dokument kann als Muster und Anleitung für wissenschaftliche Abschlussarbeiten verwendet werden. Es beruht ursprünglich auf einem Leitfaden, den Prof.~Dr.~Stephan Freichel als Prüfungsausschussvorsitzender für die Studiengänge B.\,Sc.~Logistik und M.\,Sc.~\emph{Supply Chain and Operations Management} an der Fakultät für Fahrzeugsysteme und Produktion erstellt hat.
\par
Der Text in dieser Vorlage beschreibt allgemeine formale Anforderungen, insbesondere zum Inhaltsverzeichnis, zum Einfügen von Quellenverweisen und zum Erstellen eines Literaturverzeichnisses.
\par
Die Vorlage kann fachübergreifend als Musterdatei für Abschlussarbeiten an der TH Köln verwendet werden. Allerdings müssen Sie dann unbedingt klären, ob sie den Konventionen in ihrem Studienfach entspricht.
\par
Für die Möglichkeit eines fachunabhängigen Gebrauchs wurde das Dokument von Maria-Anna Worth (i.\,R.) und Susanne Neuzerling (Hochschulreferat Planung und Controlling) inhaltlich überarbeitet und modifiziert. Eine erneute Überarbeitung und Aktualisierung erfolgte durch Andreas Bissels (Schreibzentrum). Frau Katharina Bata hat die Dokumentvorlage in LaTeX erstellt. Zuletzt wurde diese Vorlage 2023 von André Ulrich und Jan Salmen überarbeitet und um diverse Hinweise speziell zur Gestaltung mit \LaTeX{} ergänzt (siehe \cref{chap:Textsatz}).
\par
Zur Vorbereitung auf Ihre Abschlussarbeit empfehlen wir Ihnen die Angebote des Schreibzentrums der Kompetenzwerkstatt\footnote{\href{https://www.th-koeln.de/schreibzentrum}{https://www.th-koeln.de/schreibzentrum}}; hierzu gehören sowohl eine Schreibberatung als auch Schreibkurse. Das Schreibzentrum ist Ihre Anlaufstelle an der TH Köln in Fragen rund um das wissenschaftliche Schreiben.
\par
Sichern Sie diese Dokumentvorlage bitte zweifach auf Ihrem Rechner: Einmal, um weiterhin auf den hier dargestellten Inhalt zugreifen zu können, und ein zweites Mal, um sie mit Ihrer eigenen Abschlussarbeit zu überschreiben.
\par
\emph{Bitte beachten Sie: Die Vorlage ersetzt nicht die spezifischen Vorgaben der jeweiligen Prüfungsausschüsse. Sollte es in Ihrem Fach besondere formale Vorgaben geben, so gelten diese.}\enlargethispage{\baselineskip}
%
\begin{flushright}
Köln, August 2023
\end{flushright}