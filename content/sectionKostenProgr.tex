\section{Kostenpflichtige PDF Programme und Onlinedienste}
Ich werde in den folgenden Abschnitten 7 Programme ausführlich behandeln und ihre interessantesten und wichtigsten Funktionen aufzeigen. 

\subsection{Adobe Acrobat Pro DC}
Produktseite: \url{https://www.adobe.com/de/acrobat/acrobat-pro.html} \\
Acrobat ist ein professionelles Werkzeug, um PDF-Dateien zu erstellen und bereits bestehende Inhalte zu bearbeiten. Text kann hinzugefügt, geändert, formatiert, gelöscht oder markiert werden. Ebenfalls kann der Text von Formularen bearbeitet werden. Bilder können gespiegelt, gedreht, zugeschnitten, ersetzt, ausgerichtet oder angeordnet werden. Beim Objekte Anordnen-Werkzeug können mehrere Objekte vor oder hinter anderen Objekten angeordnet werden. Das Arbeiten mit Ebenen ist möglich, jedoch eingeschränkter als bei Photoshop. PDF-Seiten können kopiert, ersetzt, gedreht, verschoben, gelöscht, extrahiert oder neu nummeriert werden. Eine PDF-Datei kann in mehrere Dokumente aufgeteilt werden und Kommentare können hinzugefügt werden. Webseiten können mit der Acrobat-Browsererweiterung in PDF konvertiert werden \cite{adobe-acrobat-um}. Eine PDF-Datei kann in eine PDF/A-Datei inklusive seiner Varianten, PDF/X, PDF/UA, PDF/VT oder PDF/E konvertiert werden. Außerdem kann die Kompatibilität mit diesen Formaten überprüft werden in Preflight-Profilen \cite{adobe-pdf-a}. Die Barrierefreiheit kann automatisch validiert werden oder ein neues Dokument kann direkt barrierefrei erstellt werden. Adobe Acrobat Pro kann andere Dokumentenformate wie HTML, DOC, DOCX, TXT und RTF in PDF konvertieren, PDF in andere Dateiformate wie Microsoft Word exportieren oder Dokumente unterschreiben \cite{adobe-formate}. PDF-Dokumente mit digitalen Signaturen unterzeichnet werden und die Signaturen können validiert werden \cite{adobe-acrobat}. Für Vertraulichkeit können PDF-Dokumente mit Passwörtern und Zertifikaten versehen werden. Interaktive Objekte, Audio und Video kann eingebettet werden. In der Pro-Variante gibt es Druckproduktionswerkzeuge u.a. für Druckermarken, Transparenz-Reduzierung, Farbkonvertierung oder Druckermanagement. Die Preflight-Option ist ebenfalls nur in Acrobat Pro verfügbar \cite{adobe-acrobat-um}. Mit dem Werkzeug Scan \& \gls{ocr} in Acrobat Pro kann man Pixelbilder als PDF und gescannte PDF-Dokumente in ein durchsuchbares PDF umwandeln \cite{adobe-search}. Lediglich Acrobat Pro kann PDFs vergleichen und schwärzen. Die Standard-Version ist nur unter Windows verfügbar. Hingegen ist Acrobat Pro für macOS und Windows erhältlich \cite{wondershare-acrobat}. Das Programm gibt es als kostenlosen Acrobat Reader, Acrobat Standard für 15,46 Euro pro Monat und Acrobat Pro für 23,79 Euro pro Monat. Hierbei handelt es sich um ein Jahres-Abo mit monatlicher Zahlung. Wählt man das Monats-Abo, so kostet die Standard-Version 27,36 Euro im Monat und die Pro-Variante 35,69 Euro im Monat. Es gibt auch die Möglichkeit das Jahres-Abo mit Vorauszahlung zu erwerben \cite{adobe-acrobat}. Mit den Adobe Acrobat Onlinetools kann man über den Browser verschiedene Dateitypen in PDF umwandeln, unter anderem PDF in JPEG oder andere Bildformate, PDF Dateien bearbeiten und Kompression anwenden. Die Onlinetools können ebenfalls PDF in Word umwandeln. \cite{adobe-search} Der Adobe Acrobat PDF-Converter der Onlinetools kann DOCX, DOC, XLSX, XLS, PPTX, PPT, TXT, RTF, JPEG, PNG, TIFF, BMP, sowie Adobe eigene AI-, INDD- und PSD-Dateien in PDF konvertieren. \cite{adobe-formate} Die kostenlose Version des PDF-Converters kann nur begrenzt oft genutzt werden.

\subsection{Wondershare PDFelement}
Produktseite: \url{https://pdf.wondershare.com/} \\
Wondershare PDFelement ist eine KI gestützte PDF-Alternative zu Acrobat. Das Programm wird für Windows, macOS, iOS und Android angeboten und die aktuellste Version ist PDFelement 10. Bereits bestehende Bilder oder Text des geöffneten PDFs können bearbeitet werden. Vom Autor eingebettete Objekte können in 90 Grad-Schritten gedreht, horizontal und vertikal gespiegelt oder ausgerichtet werden. PDF-Dokumente können mit Highlights versehen und Geometrie bzw. Freihandzeichnungen können hinzugefügt werden. Dabei handelt es sich um Kommentartypen, die auch gedreht, skaliert oder verschoben werden können. Das Radiererwerkzeug funktioniert nur für Freihandzeichnungen. Der Hintergrund einer PDF-Datei kann seitenweise mit Farbe gefüllt werden oder durch ein Bild verschönert werden. Ausfüllbare Formulare können sogar automatisch erstellt werden. Es gibt Sicherheitsfunktionen für Passwortschutz, Signaturerstellung und Validierung von Signaturen. PDFelement besitzt ebenfalls eine \gls{ocr}-Funktion. Im Bezug auf AI-Tools gibt es die Optionen Erklären von Inhalten und Code, Korrekturlesen, Zusammenfassen, Chat mit PDF, Übersetzungen und eine Funktion zur Erkennung von AI erstellten Inhalten, sowie eine automatische Lesezeichenfunktion basierend auf der PDF-Struktur und Überschriften. Der AI reading assistant nennt sich Lumi und beruht auf ChatGPT. Mehrere PDF-Dateien können gleichzeitig geöffnet sein und Screenshots, Screen Recording, Lineale, Hilfslinien und Hilfsgatter sind verfügbar. Eine Stapelverarbeitung von mehreren PDF-Dateien ist für Konvertierung in andere Dateiformate, Komprimierung, \gls{ocr}, einige Bearbeitungsoptionen, Drucken und Verschlüsselung verfügbar. Konvertierungen von PDF zu Word, Powerpoint, Excel und JPEG, PNG, TIFF, GIF, Epub, HTML, \gls{xml} oder PDF/A sind u.a. möglich \cite{wondershare-um}. In der kostenlosen 14-tägigen-Testversion mit Anmeldung, bei der dann mehrere Funktionen zur Verfügung stehen, wird ein PDF mit Wasserzeichen gespeichert. Für einen Einzelbenutzer kostet eine Dauerlizenz 119 Euro, ein Jahresabo 89 Euro und ein 2-Jahresabo 109 Euro. Darin sind PDFelement Updates enthalten, sowie 20 GB Document Cloud Speicher. Es gibt einen Rabatt für Schüler, Studenten und Lehrkräfte \cite{wondershare-preis}. 

\subsection{UPDF}
Produktseite: \url{https://updf.com/} \\

\subsection{Mathpix}
Produktseite: \url{https://mathpix.com/} \\


%%%%%%%%%%%%%%%%%%%%%%%%%%%%%%%%%%%%

\subsection{pdf-it}
Produktseite: \url{https://www.pdf-it.com/}

\subsection{Soda PDF}
Produktseite: \url{https://www.sodapdf.com/}

\subsection{Nitro PDF Pro}
Produktseite: \url{https://www.gonitro.com/}

\subsection{Ashampoo PDF Pro 3}
Produktseite: \url{https://www.ashampoo.com/de-ch/pdf-pro}

\subsection{Infix 7}
Produktseite: \url{https://www.iceni.com/infix.htm}

\subsection{PDF Director 2 Pro}
Produktseite: \url{https://pdfdirector.de/funktionen}

\subsection{Perfect PDF}
Produktseite: \url{https://soft-xpansion.de/products/perfect-pdf-12/}