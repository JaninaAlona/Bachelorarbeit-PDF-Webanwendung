\section{Rolle von PDF in der Druckvorstufe und Designbranche}
PDF hat sich zum Standard für den elektronischen Austausch von Dokumenten entwickelt und zunehmend zum Containerformat für alle grafischen Elemente. Vor allem in der grafischen Industrie wird PDF gerne verwendet, weil es eine plattformübergreifende Visualisierung auf allen Betriebssystemen bietet. PDF unterstützt beliebige Seitenformate. Heute ist es das meist eingesetzte Format bei der Produktion von Druckvorlagen in der digitalen Druckvorstufe. Das Dateiformat funktioniert mit allen Druckern und überwindet Kompatibilitätsprobleme. Da der wirtschaftliche Druck auf die produzierenden Unternehmen immer mehr zunimmt, müssen einzelne Produktionsschritte immer mehr zeitlich gestaucht werden und im Idealfall automatisch ablaufen. 
\par
PDF kann für die schnelle Webansicht optimiert werden. Statische PDF-Dateien sind nicht responsive, d.h. eine optimale Anpassung an verschiedene Endgeräte ist nicht garantiert und sie haben eine lange Ladezeit. Vor allem Bilder lassen sich ohne sichtbaren Qualitätsverlust und geringem Datenvolumen speichern. Im Vergleich zu PostScript-Dateien erzielen die kompaktere Codierung von Seiteninhalten, sowie das einmalige Speichern von identischen Bildern und die Verwendung von leistungsfähigen Kompressionsalgorithmen eine maßgeblich kleinere Dateigröße bei PDF. Downsampling und Kompression beschleunigen den Transport von der Agentur zum Dienstleistungsbüro enorm. Korrekturänderungen in PDF-Dateien sind kurz vor dem Druck noch möglich. Die Produktion von Druckerzeugnissen wird somit wesentlich flexibler und sicherer. Dank der Profile für unterschiedliche Geräte und Bedruckstoffe, kann eine Simulation des Ergebnisses am Monitor oder durch einen Prüfdruck bewerkstelligt werden. Dadurch steigt die Reproduktionssicherheit bei gleichbleibender Qualität und die Produktionszeiten verkürzen sich enorm. Die zeitintensive Optimierung der Druckdaten kann durch PDF auf einen späteren Zeitpunkt verlegt werden. 
\par
Für die Betrachtung von Druckvorstufen-PDF-Dateien sollte immer Acrobat Pro bzw. Adobe Reader verwendet werden, da viele Drittanbieter-PDF-Viewer druckvorstufenrelevante Informationen nicht fehlerlos darstellen können. In der Praxis werden PDF-Dateien mit Hilfe von Prüfprofilen überprüft. Agenturen entwickeln nach eigenen Kriterien vorgefertigte Prüfroutinen und einen automatisch generierten Prüfbericht (Report). Der Prüfbericht sollte auf schnelle Fehlersuche im überprüften Dokument optimiert sein. Notwendige Korrekturen können im Originaldokument, bei PDF-Erstellung oder im vorliegenden PDF-Dokument je nach Schweregrad des Fehlers ausgeübt werden. Korrekturen im Originaldokument sind vorzuziehen, da sie Folgefehler minimieren können, jedoch liegt der Agentur oftmals nicht das Originaldokument oder ausschließlich eine unvollständige PDF-Datei des Originaldokuments vor. Eine Erstellung von Korrekturprofilen für Änderungen an der gesamten PDF-Datei kann dabei hilfreich sein. Man kann zwischen einer Vielzahl an voreingestellten Prüfprofilen und einer wesentlich kleineren Menge an Korrekturprofilen im Preflight in Acrobat Pro wählen. Außerdem kann der Prüfbericht als Kommunikationsmittel zwischen Auftraggeber und Agentur dienen, falls der Agentur das Originaldokument nicht ausgehändigt wurde. 
\par
Das Preflight-Werkzeug in Acrobat Pro ist ein gängiges Werkzeug in der Druckvorstufe. Die Kontrolle von PDF-Dateien mit Preflight kann viel Geld sparen. Im tiefgreifenden Preflight-Check, der ein System vor dessen Einsatz überprüft, kann man die besten Ergebnisse erzielen, wenn das zum Job passende Prüfprofil verwendet wird. Beim Preflighting-Prozess werden fehlerhafte Objekte und Objekte mit speziellen Anforderungen gefunden, ihr Zustand abgefragt und dann dementsprechend ein Fehler, eine Warnung oder eine Information ausgegeben. Im Warnungsfall muss die Agentur entscheiden, ob das Objekt, was die Warnung verursacht hat, sich negative auf das Endprodukt auswirkt. Folglich können durch Preflight frühzeitig Mängel im Produktionszyklus beseitigt werden. Ein Report sollte für die automatisierte Auswertung in weiterführenden Workflow-Systemen im XML-Standard vorliegen. Erfolgreiche selbst erstellte Prüfprofile basieren auf qualitativ hochwertig gewählten Kriterien, wobei nicht zu viele verwendet werden sollten. Prüfprofile können als wichtigste Prüfkriterien das Dokument, Seiten, Bilder, Farbe, Zeichensätze, Rendering und Standard-Konformität enthalten.
\par
Textmodifikationen sind am meisten gefordert in der Druckvorstufe. Farbänderungen bzw. -konvertierungen sind ebenso erforderliche Eingriffe. Änderungen am Layout oder Neugestaltungen von Seiten müssen im Originalprogramm erledigt werden. Das modifizierte PDF muss in das Druck-PDF integriert werden. Ständige visuelle Kontrollen der PDF-Datei des Kunden mit der korrigierten Version werden durch eine Fachkraft im Betrieb durchgeführt. Am Ende jeder Korrektur steht die Prüfung auf PDF/X-Konformität, um als technisches Gütesiegel weitere Fehler für den Druck zu reduzieren. Der Druckvorstufenbetrieb gibt dem Auftraggeber den \gls{iso}-Standard der PDF-Datei für das Druckerzeugnis vor. Speziell bei PDF/X-Dateien müssen bestimmte Inhalte als Metadaten vorliegen, wie der Dokumententitel, das Änderungsdatum der Datei, das PDF-Erstellerprogramm und das Setzes des Trapping Keys. 
\par
Falls im Druck-PDF ein falscher Zielfarbraum eingestellt wurde, wirkt sich das auf den Gesamtfarbauftrag aus. Eine Farbraumkonvertierung in den, für das Druckergebnis gewünschten Farbraum, muss vorgenommen und der maximale Gesamtfarbauftrag angepasst werden. Alle RGB-Farbräume müssen nach CMYK umgewandelt werden. Schmuckfarben werden in CMYK oder in eine andere Schmuckfarbe konvertiert bzw. gemappt. Die hinterlegten Alternativfarbwerte für Schmuckfarben weichen in der Vielfalt der Programme voneinander ab. Ursächlich ist, dass unterschiedliche Pantone Libraries implementiert wurden. Häufig sind in PDF-Dateien verschiedene Farbwerte für dieselbe Farbe anzutreffen. Als Volltonfarben oder Schmuckfarben werden speziell vorgemischte Druckfarben bezeichnet, die die üblichen Prozessdruckfarben in CMYK ersetzen oder ergänzen. Bilder, die nicht im Zielfarbraum vorliegen, müssen erst in den Ausgabefarbraum überführt, separiert ausgegeben und Transparenzen reduziert werden. Jegliche Transparenzen in Objekten, vor allem Texte, Grafiken oder Bilder, werden beim Speichern in ein Dateiformat, welches keine Live-Transparenzen unterstützt, reduziert. Zu diesen Dateiformaten gehören u.a. \gls{eps}, JPEG, GIF, BMP, PDF 1.3 und niedriger. Nach der Transparenzreduzierung sollten für die Ausgabe Quellfarbprofile von den reduzierten Objekten entfernt werden, da sie oftmals zu unerwünschten Farbtransformationen im Ausgabegerät führen können. Das Anbringen von Transparenzen auf Objekte mit Volltonfarben ist, mit Ausnahme der Deckkraftänderung, verboten. Der Prozess der Transparenzreduzierung sollte auf den letztmöglichen Zeitpunkt in der Produktion von digitalen Daten verschoben werden.
\par
Schriften können bei Einbettung in PDF exakt wiedergegeben werden, unabhängig davon, ob es sich um eine Windows- oder macOS-Schrift handelt. Da allgemeine Schriftinformationen immer eingebettet sind und die Zeilenlängen im Prinzip immer stimmen, können Druckvorstufenbetriebe zumindest immer erkennen, welche Schrift bzw. welcher Schriftschnitt der Ersteller der PDF-Datei ursprünglich vorgesehen hatte, falls die Schrift nicht eingebettet wurde. Beim nachträglichen Einbetten von Schriften in letzter Minute, muss die richtige Schrift gesucht werden, denn der richtige Schriftname beschreibt nicht notwendigerweise dieselbe Schrift mit derselben Codierung bzw. Laufweite. Fehlende Zeichen, überlappende Buchstaben, unregelmäßige Abstände zwischen den Buchstaben, usw. sind gängige Probleme. Schriften können auch durch die PDF-Bearbeitung extrahiert (ausgebettet) werden. In der Druckvorstufe müssen Schriftbezeichnungen für die Ausgabe eindeutig benannt werden.
\par
Nach der Einbringung von Korrekturen im PDF sollte eine technische und visuelle Überprüfung der Druckdatei stattfinden. In einem korrigierten Kunden-PDF sind die, für die Druckerei wichtigen Druck-, Schneide- und Passermarken, sowie der Bereich des Anschnitts meistens unerwünscht. Um diesen Bereich und alle Druckmarken unsichtbar zu machen, wird die CropBox auf die Größe der TrimBox skaliert. PDF-Dateien, die nicht aus professionellen Anwendungen der Druckvorstufe erstellt worden sind, fehlen in den meisten Fällen die TrimBox bzw. die BleedBox. Hat der Datenersteller Schnittmarken sehr genau manuell platziert, so können fast alle PDF-Bearbeitungswerkzeuge die TrimBox bzw. BleedBox davon ableiten. 
\par
Für die Weiterverarbeitung sollten leere Seiten und nicht druckbare Objekte, die als solche gekennzeichnet sind oder sich außerhalb des druckbaren Bereichs befinden, entfernt werden. Große Dateien sind bei der Verarbeitung sehr zeitaufwendig und weniger ressourcenschonend. PDF-Dateien können sich durch Output-Intents, eingebettete Dateien, unzureichende Komprimierung von Bildern und Objekten, zu hohe Auflösung und die Präsenz nicht druckbarer Objekte in ihrer Dateigröße aufblähen. Falls viele Vektoren in einem Druckdokument enthalten sind, würde, trotz PDF-Optimierung, weiterhin eine hohe Auflösung in Vektorgrafiken bestehen bleiben und folglich nur eine geringe Dateigrößenreduktion erzielt werden. Das Rastern des Inhalts bietet einen praxisnahen Lösungsweg. Hierbei werden zu überdruckende Objekte in das Bild eingerechnet und Vektoren gerendert, wodurch eine maßgebliche Dateigrößenoptimierung erzielt werden kann. 
\par
In der Werbebranche werden PDF-Dateien vor allem für Korrekturabzüge verwendet. Ein Korrekturabzug ist eine Skizze bzw. Designvorlage des Werbeprodukts, welches die vom Kunden gewünschte Position und Größe des Druckmotivs auf dem Werbeartikel enthält. Als digitales Layout wird der Korrekturabzug mit dem Kunden abgestimmt und seine Änderungswünsche vor dem Druck entgegengenommen. Häufig wird in der Agentur ein Abzug-PDF zur Druckfreigabe an den Auftraggeber versandt. Die Seiteninhalte eines Abzug-PDFs sollten gerastert werden, da fehlerhafte Grundeinstellungen im Adobe Reader zu abweichenden Darstellungen des PDFs, im Vergleich zum Aussehen nach dem Druck, führen können. Abzug-PDFs sollten eine Qualität besitzen, die ausreichend ist für eine Ansicht, jedoch für die Produktion bei einem anderen Druckdienstleister nicht genügt. 
\par
Vor allem Office-Dokumente müssen für den Druck aufbereitet werden. Marketingabteilungen von Industrieunternehmen sind gezwungen, bei der Druck-PDF-Erzeugung auf die zur Verfügung stehenden Werkzeuge, den Microsoft-Office-Programmen bzw. OpenOffice-Programmen zurückzugreifen. Meistens liegen dann alle Objekte in RGB vor, es fehlt die TrimBox und der Anschnitt, sowie die Einzelseiten-PDFs für das Ausschießen sind nicht vorhanden. Jeglicher Schutz oder Sperren sollte an einer druckvorstufentauglichen PDF-Datei vermieden werden. Aktionen sind innerhalb druckfähiger Dateien untersagt, unabhängig von JavaScript oder Aktionen von Acrobat. Elemente, die nicht für den Druck vorgesehen werden, müssen aus dem druckbaren Bereich entfernt werden. Nur wenn sich die Ausgangsdatei mit der zum Druck verwendeten Enddatei in allen Einzelheiten deckt, kann man sich Reklamationen, Geld, Zeit und Diskussionen sparen \cite{schneeberger}. Das nächste Kapitel stellt die Anforderungen an meine PDF Web App vor.