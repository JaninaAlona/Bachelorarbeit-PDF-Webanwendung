\chapter{Fazit und Ausblick}
Der Markt an PDF Programmen ist zahlreich, aber es gibt wenige Open Source-PDF-Tools. Die meisten Firmen lassen sich nicht wenig für ihre PDF-Dienste bezahlen. Ist das nachvollziehbar, da das PDF-Dateiformat ein freies Format ist? Soll man sich ein kostenpflichtiges Programm anschaffen, was Sicherheitsfunktionen für ein unsicheres Dateiformat implementiert? Ich wollte mit meiner PDF Web App eine kostenlose Alternative zu der PDF-Programmlandschaft bietet, die nicht vor Funktionalität nur so strotzt, sondern sich auf die wichtigsten, gebräuchlichsten Operationen auf PDF-Dateien beschränkt. Der Benutzer soll nicht Seitenlange Tutorials lesen müssen, um das ganze Programm anwenden zu können. Es sollte eher leicht verständlich, sowie intuitiv sein und die Bedienung sollte man durch Ausprobieren erfassen können. Was mir bei der Analyse der PDF-Programme aufgefallen ist, ist, dass die Zeichenfunktionen nicht besonders fortschrittlich sind. Die meisten Programme implementieren Zeichnungen als Kommentarobjekte und bei vielen gibt es noch nicht einmal einen Radierer. Mir war bei meiner PDF Web App vor allem wichtig, dass die Zeichenfunktionen viel Möglichkeitsspielraum bieten, da mir dieser Malus bei den meisten PDF-Programmen schon vorher aufgefallen ist. Ebenfalls die Ebenenfunktionen, die ich implementiert habe, habe ich auf diese Art noch bei keinem anderen PDF-Programm gesehen. Meine Ebenen funktionieren nicht analog zu Photoshop, sondern ich habe mir eine eigene Funktionsweise ausgedacht. Genauso habe ich in keinem Programm Selection Filter entdecken können. Auch dieses Feature habe ich selbst erfunden, weil ich dachte, dass es sehr hilfreich sein kann, wenn man mehrere Elemente im Editor bearbeiten möchte. Anfangs, als ich damit begann, die PDF Web App zu programmieren, hatte ich Angst, dass sie sich kaum von den anderen PDF-Programmen unterscheiden wird. Ich denke diese Angst war unbegründet, wenn man das soweit finale Ergebnis dieser Arbeit ausprobiert.
\par
In puncto Dokumentensicherheit muss die PDF-Spezifikation noch um einiges nachlegen. Allein die Dateistruktur, die verschlüsselte und unverschlüsselte Inhalte erlaubt, ermöglicht dem Angreifer schadhafte Inhalte in das PDF-Dokument des Opfers einzuschleusen. Allein die Tatsache, dass die Spezifikation das Dateiformat zu ungenau spezifiziert, zu viel Toleranz in der Implementierung von PDF-Programmen zulässt und auch beschädigte Dateien in Readern oder Editoren repariert und geöffnet werden können, sollte sich in Zukunft ändern. Das incremental update-Feature muss auf eine andere Art und Weise implementiert werden, da es auf diese Art und Weise zu viel Schaden anrichten kann. Die 256-Bit-\gls{aes}-Verschlüsselung ist zwar ein guter Anfang, aber sollte mit einem \gls{mac} integrity check arbeiten. Eine überarbeitete PDF Version 3.0 ist längst überfällig und ein robusteres Dateiformat ist vonnöten. Im Jahr 2019 gab es einen Vortrag auf der Electronic Document Conference, in dem die Blockchain-Technologie für die Verschlüsselung von Dokumenten vorgeschlagen wurde. Die Blockchain-Technologie könnte verwendet werden, um digitale Signaturen in der Blockchain zu speichern, anstatt im PDF-Dokument selber. Dieser Ansatz würde schnellere Signaturprozesse anstoßen, Dokumentensicherheit unterstützen, sowie Echtzeitverfolgung von Signaturen und Versionskontrolle implementieren \cite{pdf-association-blockchain}. Generative AI ist die Zukunft und wird möglicherweise in den nächsten Jahren in fast allen PDF-Programmen Verwendung finden. Bisher gab es noch keine Generative AI-Implementierung, die komplette PDF-Dokumente auf Basis eines Prompts generieren kann. Jedoch die Funktion Text to Template von Adobe Firefly in Adobe Express geht schon in die richtige Richtung. 