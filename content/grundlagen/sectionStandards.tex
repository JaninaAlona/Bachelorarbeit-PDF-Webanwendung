\section{PDF Dateiformate}

\subsection{PDF-X}
Das PDF-X Dateiformat (\gls{iso} 15930) dient des simpleren Datenaustausches in der Druckvorstufe. Es beschreibt Eigenschaften von Druckvorlagen und vereinfacht die Datenübermittlung von der Druckvorstufe bis zum finalen Druck.

\subsection{PDF-VT}
Das PDF-VT Dateiformat stellt ein spezielles Austauschformat im variablen Datendruck und Transaktionsdruck dar.

\subsection{PDF-A}
Das PDF-A Dateiformat wurde zur gesetzteskonformen Langzeitarchivierung von digitalen Dokumenten entwickelt.
Langzeitarchivierung von PDF-Dateien (als PDF/A-1 in \gls{iso} 19005-1:2005) 

\subsection{PDF-E}
Das PDF-E Dateiformat wurde speziell für das Ingenieurwesen entworfen und kann interaktive 3D-Elemente darstellen. Im einzelnen können \gls{cad}-Dateien im 3D- und 2D-Format eingebettet werden.

\subsection{PDF-UA}
Das PDF-UA Dateiformat dient der Erstellung barrierefreier Dokumente.

\subsection{Durchsuchbares PDF}
Searchable PDFs können mit Suchfunktionalitäten eines PDF-Readers durchsucht werden. Es kann gezielt nach Zahlen oder Stichwörtern durchsucht werden. Diese PDF-Art wird üblicherweise durch die \gls{ocr} Technologie erstellt. Bei OCR handelt es sich um optische Zeichenerkennung, die Textzeichen und Dokumentstruktur analysiert. Auf diese Weise können gescannte Dokumente oder Pixelbilder als PDF abgespeichert und in ein Durchsuchbares PDF in Adobe Acrobat umgewandelt werden. In Acrobat ist es außerdem möglich mit der einfachen Suche innerhalb einer Datei nach Suchbegriffen zu suchen, mit der erweiterten Suche oder der Suchen-Werkzeugleiste mehrere PDF-Dokumente zu durchsuchen und speziell in der erweiterten Suche u.a. Objektdaten und Bildern zu lokalisieren. Textbasierte PDF-Dateien können grundsätzlich durchsucht werden und auch in andere Dateiformate wie Microsoft Word, Excel oder PowerPoint umgewandelt werden.
\cite{adobe-search}

\subsection{PAdES}
\gls{pades} ergänzt den Funktionsumfang um Werkzeuge, um elektronische Signaturen anzupassen.

\subsection{PDF-H}
Das PDF-H Dateiformat soll im Gesundheitswesen Patientendaten erfassen, austauschen und archivieren.
