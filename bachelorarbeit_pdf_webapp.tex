\documentclass[
	ngerman,
	%twoside,
	BCOR=8mm,
	headings=normal,
	parskip=half,
	headsepline,
	automark,
	listof=totoc,
	bibliography=totoc,
	%,captions=tableabove
	%draft
]{scrreprt}
%
%
%%%%%%%%%%%%%%%%%%%%%%%%%%%%%%%%%%%%%%%%%%%%%%%%%%%%%%%%%%%%%%%%%%%%%%%%%%%%%%%
% Pakete laden
%%%%%%%%%%%%%%%%%%%%%%%%%%%%%%%%%%%%%%%%%%%%%%%%%%%%%%%%%%%%%%%%%%%%%%%%%%%%%%%
%
\usepackage{ifluatex}
\usepackage{babel}
%
\ifluatex
 % LuaLaTeX
 \usepackage{fontspec}
 \usepackage{selnolig}
\else
 % PdfLaTeX
 \usepackage[T1]{fontenc}
 \usepackage{lmodern}
\fi
%
\usepackage{csquotes}
\usepackage{scrlayer-scrpage}
\usepackage{microtype}
%
%\usepackage{ziffer}% optional
%\usepackage[locale=DE]{siunitx}% optional
%
\usepackage{tikz}
\usepackage{pgfplots}
%
\usepackage[backend=biber]{biblatex}
%
\usepackage{amsmath}
\usepackage{amssymb}
%
\usepackage{hyperref}
%
%=== wichtig, dass folgende Pakete NACH hyperref geladen werden ===============
\usepackage{scrhack}% Um Warnung bzgl. \float@addtolists im listings-Paket (s.u.) zu vermeiden
\usepackage{listings}

%List enumerations in References part
\usepackage{enumitem}
%
\usepackage[nameinlink]{cleveref}
\usepackage[all]{hypcap}
\usepackage[
	automake,
	toc,
	symbols,
	acronyms,
]{glossaries}

%Sprache des Literaturverzeichnisses
%\usepackage[fixlanguage]{babelbib}
%
%%%%%%%%%%%%%%%%%%%%%%%%%%%%%%%%%%%%%%%%%%%%%%%%%%%%%%%%%%%%%%%%%%%%%%%%%%%%%%%
% Globale Definitionen
%%%%%%%%%%%%%%%%%%%%%%%%%%%%%%%%%%%%%%%%%%%%%%%%%%%%%%%%%%%%%%%%%%%%%%%%%%%%%%%

%=== pdf Metadaten ============================================================
\hypersetup{
	pdfauthor={Janina Schroeder},
	pdftitle={Entwicklung einer webbasierten Applikation zur Bearbeitung von PDF Dateien},
	pdfsubject={Bachelorarbeit},
	pdfkeywords={
		Bachelorarbeit,
	},
	bookmarksnumbered=true,
	pdfstartview=FitH,
	hidelinks,
}

%=== Vorwort vor Literatur ====================================================
\defbibnote{mynote}{%
	Wie in \cref{sec:bib-content} erläutert, werden im Literaturverzeichnis 
	ausschließlich die Quellen angegeben, auf die im Rahmen einer Arbeit 
	tatsächlich verwiesen wird. Bitte prüfen Sie also das Literaturverzeichnis 
	Ihrer Arbeit immer dahingehend, ob alle zitierten Quellen~--~und nur
	diese~--~erfasst wurden. Dies trifft auf das nun folgende Verzeichnis
	\emph{nicht} zu; die 
	meisten der hier aufgeführten Quellen werden in dieser Vorlage nicht 
	zitiert. Es handelt sich lediglich um ein Beispiel für ein 
	Literaturverzeichnis mit Literaturempfehlungen zum wissenschaftlichen 
	Schreiben.
}

%=== Kopf-/Fusszeile definieren ===============================================
\clearpairofpagestyles
\ohead[]{\headmark}
\ofoot[\pagemark]{\pagemark}

%=== Farben definieren ========================================================
\definecolor{THRed}{RGB}{207,24,32}
\definecolor{THOrange}{RGB}{236,101,37}
\definecolor{THPurple}{RGB}{175,54,140}

%=== Einstellungen für cref ===================================================
\newcommand{\crefpairconjunction}{ und~}
\newcommand{\crefrangeconjunction}{ bis~}
\crefname{figure}{Abbildung}{Abbildungen}

%=== Einstellungen für plots ==================================================
\pgfplotsset{
	compat=newest,
	/pgf/number format/.cd,
	dec sep={\text{,}},
	1000 sep={\,},
}

%table cell spacing
\bgroup
\def\arraystretch{2}

%=== paar convenience Sachen definieren =======================================
\DeclareMathOperator{\sgn}{sgn}
\newcommand{\vecW}{\ensuremath{\mathbf{w}}}%
\newcommand{\ci}{\ensuremath{\mathrm{i}}}%
%
\newcommand{\tb}{\textbackslash}
%
\newcommand{\comm}[1]{\enquote{\texttt{\tb #1}}}
%
\newcommand{\param}[1]{%
	$\langle$\textrm{\textit{#1}}$\rangle$%
}

%=== Arial als Hauptschriftart ================================================
%\setsansfont{Arial}
%\renewcommand{\familydefault}{\sfdefault}

%
%%%%%%%%%%%%%%%%%%%%%%%%%%%%%%%%%%%%%%%%%%%%%%%%%%%%%%%%%%%%%%%%%%%%%%%%%%%%%%%
% Begriffe für glossaries definieren
%%%%%%%%%%%%%%%%%%%%%%%%%%%%%%%%%%%%%%%%%%%%%%%%%%%%%%%%%%%%%%%%%%%%%%%%%%%%%%%

%=== Abkürzungen ==============================================================
\newacronym{pdf}{PDF}{Portable Document Format}
\newacronym{iso}{ISO}{International Organization for Standardization}
\newacronym{pdl}{PDL}{page description language}
\newacronym{rip}{RIP}{raster image processor}
\newacronym{gdi}{GDI}{graphics device interface}
\newacronym{pades}{PAdES}{PDF Advanced Electronic Signatures}
\newacronym{opi}{OPI}{Open Prepress Interface}
\newacronym{ceps}{CEPS}{Cisco Enterprise Print System}
\newacronym{cid}{CID}{character identifier}
\newacronym{icc}{ICC}{International Color Consortium}
\newacronym{pcs}{PCS}{profile connection space}
\newacronym{xfa}{XFA}{XML Forms Architecture}
\newacronym{xml}{XML}{Extensible Markup Language}
\newacronym{xmp}{XMP}{Extensible Metadata Platform}
\newacronym{cad}{CAD}{computer-aided design}
\newacronym{wysiwyg}{WYSIWYG}{what you see is what you get}
\newacronym{zsa}{ZSA}{Zeitstempelanbieter}
\newacronym{rle}{RLE}{run-length encoding}
\newacronym{ocr}{OCR}{Optical Character Recognition}
\newacronym{wcag}{WCAG}{Web Content Accessibility Guidelines}
\newacronym{bitv}{BITV}{Barrierefreie-Informationstechnik-Verordnung}
\newacronym{mime}{MIME}{Multipurpose Internet Mail Extensions}
\newacronym{etsi}{ETSI}{European Telecommunications Standards Institute}
\newacronym{cades}{CAdES}{CMS Advanced Electronic Signatures}
\newacronym{xades}{XAdES}{XML Advanced Electronic Signatures}
\newacronym{aes}{AES}{Advanced Encryption Standard}
\newacronym{ppd}{PPD}{PostScript Printer Description}
\newacronym{ppi}{ppi}{Pixels Per Inch}
\newacronym{cie}{CIE}{Commission internationale de l’éclairage}
\newacronym{eps}{EPS}{Encapsulated PostScript}
\newacronym{dcs}{DCS}{Desktop Color Separation}
\newacronym{ascii}{ASCII}{American Standard Code for Information Interchange}
\newacronym{isa}{ISA}{Incremental Saving Attack}
\newacronym{cbc}{CBC}{cipher block chaining}
\newacronym{mitm}{MITM}{man-in-the-middle}
\newacronym{mac}{MAC}{message authentication code}
\newacronym{swa}{SWA}{Signature Wrapping Attack}
\newacronym{usf}{USF}{Universal Signature Forgery}
\newacronym{ecb}{ECB}{electronic codebook}
\newacronym{dos}{DOS}{denial-of-service}
\newacronym{eea}{EEA}{Evil Annotation Attack}
\newacronym{ssa}{SSA}{Sneaky Signature Attack}
\newacronym{json}{JSON}{JavaScript Object Notation}
\newacronym{cdn}{CDN}{content delivery network}
\newacronym{gpt4}{GPT-4}{Generative Pre-trained Transformer 4}
\newacronym{ie}{IE}{information extraction}
\newacronym{re}{RE}{relationship extraction}
\newacronym{mlp}{MLP}{multilayer perception}
\newacronym{llm}{LLM}{large language model}
\newacronym{nlp}{NLP}{natural language processing}
\newacronym{ccaas}{CCaaS}{Contact Center as a Service}
\newacronym{aws}{AWS}{Amazon Web Services}
\newacronym{ffn}{FFN}{feedforward neural network}
\newacronym{dom}{DOM}{Document Object Model}
\newacronym{api}{API}{application programming interface}
\newacronym{spa}{SPA}{single-page application}
%
\makeglossaries
%
\graphicspath{{images/}}
%
\addbibresource{references.bib}

\nocite{*} % listet alle Quellen (auch die nicht zitierten!)
%
%=== für schnelleres kopilieren alle ungeänderten Dateien auskommentieren =====
\includeonly{
	content/chapDeclaration,
	content/chapAbstract,
	content/chapEinleitung,
	content/chapGrundlagen,
	content/chapMarktanalyse,
	content/chapApp,
	content/chapForm,
	content/chapGestaltung,
	content/chapLiteratur,
	content/grundlagen/sectionPdf,
	content/grundlagen/sectionPdfFunktionen,
	content/grundlagen/sectionStandards,
	content/grundlagen/sectionVersionen,
	content/grundlagen/sectionPdfImpl,
	content/grundlagen/sectionSicherheit,
	content/grundlagen/sectionAktuell,
	content/marktanalyse/sectionFreieProgr,
	content/marktanalyse/sectionKostenProgr,
}

%List enumerations in References part
\newlist{referenceslist}{itemize}{1}
\setlist[referenceslist]{label=\textbf{[1]}}
\newcommand\itema{\item[\textbf{[2]}]}
\newcommand\itemb{\item[\textbf{[3]}]}
\newcommand\itemc{\item[\textbf{[4]}]}
\newcommand\itemd{\item[\textbf{[5]}]}
\newcommand\iteme{\item[\textbf{[6]}]}
\newcommand\itemf{\item[\textbf{[7]}]}
\newcommand\itemg{\item[\textbf{[8]}]}
\newcommand\itemh{\item[\textbf{[9]}]}
\newcommand\itemi{\item[\textbf{[10]}]}
\newcommand\itemj{\item[\textbf{[11]}]}
\newcommand\itemk{\item[\textbf{[12]}]}
\newcommand\iteml{\item[\textbf{[13]}]}
\newcommand\itemm{\item[\textbf{[14]}]}
\newcommand\itemn{\item[\textbf{[15]}]}
\newcommand\itemo{\item[\textbf{[16]}]}
\newcommand\itemp{\item[\textbf{[17]}]}
\newcommand\itemq{\item[\textbf{[18]}]}
\newcommand\itemr{\item[\textbf{[19]}]}
\newcommand\items{\item[\textbf{[20]}]}
\newcommand\itemt{\item[\textbf{[21]}]}
\newcommand\itemu{\item[\textbf{[22]}]}
\newcommand\itemv{\item[\textbf{[23]}]}

%
%
%==============================================================================
%
\begin{document}
%
\pdfbookmark[0]{Titelseite}{titel}
\begin{titlepage}
%
%\sffamily% Umschalten auf serifenlose Schrift
%
\begin{center}
\begin{tikzpicture}
    \fill[THRed] (0, 0) rectangle (\textwidth/3, 3pt);
    \fill[THOrange] (\textwidth/3, 0) rectangle (2*\textwidth/3, 3pt);
    \fill[THPurple] (2*\textwidth/3, 0) rectangle (\textwidth, 3pt);
\end{tikzpicture}
\end{center}
%
\vfill
%
\begin{huge}
Entwicklung einer webbasierter Applikation zur Bearbeitung von PDF Dateien\\[10mm]
\end{huge}
%
Bachelorarbeit zur Erlangung des akademischen Grades\newline
\emph{Bachelor of Science}\newline
im Studiengang Technische Informatik\newline
an der Fakultät für Informations-, Medien- und Elektrotechnik\newline
der Technischen Hochschule Köln
%
\vfill
%
\begin{tabular}{@{}ll}
vorgelegt von: & Janina Schroeder\\
Matrikel-Nr.:  & 11132206\\
Adresse:       & Laurentiusweg 10\\
                & 50321 Brühl\\
                & janina\_jessika\_jelena.schroeder@smail.th-koeln.de\\[5mm]
eingereicht bei:   & Prof. Dr. Chunrong Yuan\\
Zweitgutachter*in: & Prof. Dr. René Wörzberger
\end{tabular}	
%
\\[10mm]
%
Köln, 04.03.2024%
%
\rmfamily% Umschalten auf Standard-Schrift mit Serifen
%
\end{titlepage}
\cleardoublepage
\pagenumbering{Roman}
\chapter*{Bachelorarbeit}
\label{chap:zusammenfassung}
%
\textbf{Titel:} Entwicklung einer webbasierter Applikation zur Bearbeitung von PDF-Dateien

\textbf{Gutachter:}
\par
- Prof. Dr. Chunrong Yuan
\par
- Prof. Dr. René Wörzberger

\textbf{Zusammenfassung:} Für meine Bachelorarbeit habe ich eine Open-Source-Webseite zur Bearbeitung von PDF-Dateien im Firefox-Browser programmiert. Seit Adobe den PDF-Standard entwickelt hat, sind zahlreiche kostenpflichtige PDF-Anwendungen auf dem Markt erschienen. In meiner Arbeit erläutere ich zunächst das Hintergrundwissen zum PDF-Dateiformat und zeige anschließend die damit verbundenen Sicherheitsprobleme auf. Ich beleuchte den aktuellen Stand der Technik und analysiere sieben PDF-Programme auf dem Markt. Später erkläre ich die Implementierung meiner PDF Web App und führe GUI- und Performance-Tests durch. Die JavaScript-Bibliotheken pdf.js und PDF-LIB bilden das tragende Fundament meiner Anwendung. Sie vereinen alle Funktionen, die für gängige PDF-Bearbeitungen benötigt werden. PDFs können gelesen, gesplittet, gemergt und erstellt werden. Außerdem können sie mit Texten, Bildern, Geometrie und Zeichnungen versehen werden. Abschließend werden Anregungen für Erweiterungen und zukünftige Features diskutiert.

\textbf{Stichtwörter:} PDF-Editor, Adobe, JavaScript, Vue JS 3, Splitten, Mergen, pdf.js, PDF-LIB

\textbf{Datum:} 04. März 2024


\newpage
\chapter*{Bachelor Thesis}
\label{chap:abstract}
%
\textbf{Title:} Development of a web-based application for editing PDF files

\textbf{Reviewers:}
\par
- Prof. Dr. Chunrong Yuan
\par
- Prof. Dr. René Wörzberger

\textbf{Abstract:} As part of my bachelor thesis, I created an open-source website for editing PDF files in the Firefox browser.  While there are many PDF applications available for editing PDF files, most of them require a fee. In my thesis, I provided background knowledge about the PDF file format and discussed security issues related to this format. Additionally, I presented the latest technology and analysed seven PDF programs. In the later part of my work, I explain the implementation of my PDF Web App and conduct GUI and performance tests. The JavaScript libraries pdf.js and PDF-LIB form the foundation of my PDF Web App, providing all the necessary functionalities for common PDF editing, such as reading, splitting, merging, creating PDF files, and adding text, images, geometric shapes, and drawings. Finally, I discuss the functions that are still evolving and propose features planned for the future.

\textbf{Keywords:} PDF editing, Adobe, JavaScript, Vue JS 3, splitting, merging, pdf.js, PDF-LIB

\textbf{Date:} 04 March 2024

\cleardoublepage
\pdfbookmark[0]{Inhaltsverzeichnis}{toc}
\tableofcontents
\listoftables
\listoffigures
\printglossary
\printglossary[type=\acronymtype, title={Abkürzungsverzeichnis}]
\printglossary[type=symbols, title={Symbolverzeichnis}]
%
\cleardoublepage
\KOMAoptions{open=right}
\pagenumbering{arabic}
\chapter{Einleitung}

\section{Motivation}
Zu Hause benutze ich 4 PDF-Programme, um alle für mich zufriedenstellenden, häufigen Anforderungen der PDF-Bearbeitung zu erledigen: Adobe Reader zum Anzeigen von PDF, Drawboard PDF zum Zeichnen, Foxit Reader zum Schreiben und PDF Sam Basic zum Splitten und Mergen. Ich habe dann später herausgefunden, dass man mit Foxit Reader auch Zeichnen kann, jedoch finde ich dieses Programm sehr unintuitiv und unübersichtlich. PDF Sam Basic ist da einfacher, jedoch ist es sehr störend, dass man bei seiner Installation auch eine andere Werbeversion zusätzlich installiert. Ich habe mir gedacht, dass es nicht sein kann, dass mir kein kostenloses PDF-Programm so wirklich gefällt und deshalb beschlossen, meine eigene PDF Web App zu entwickeln. Weiter habe ich überlegt: Es wäre sinnvoll eine Webseite zu programmieren, die man auch offline nutzen kann. Alles was man benötigen soll ist ein Browser und keine Installationen mit zusätzlichen Werbeprogrammen, die man eigentlich nicht installieren wollte. Wie wäre es mit einer Open Source-Web App, die auch andere Studenten mit wenig Geld benutzen können? Folglich habe diese PDF Web App für diese Bachelorarbeit programmiert und sie ist neben ein paar Ecken und Kanten besser geworden, als ich es je erwartet hätte. Ich habe während der Programmierung viel gelernt über JavaScript, asynchrone Programmierung, event handler, uvm. und bin generell sicherer geworden im Programmieren. Vor allem habe ich Wert darauf gelegt, dass die PDF Web App einfach und intuitiv zu bedienen ist. Einige Features waren nicht geplant, aber mir hat es Spaß gemacht sie zu implementieren, da ich die Motivation hatte ein für andere und mich gutes Tool zu entwickeln. Natürlich gibt es noch eine lange Liste an Features, die ich in Zukunft noch implementieren möchte. Die PDF Web App ist kein abgeschlossenes Projekt. Dieses Hobby-Projekt möchte ich gerne nach der Bachelorarbeit weiter auf meiner Github-Seite maintainen. Vielleicht gibt es andere Entwickler, die mein Repository forken oder bug fixes hinzusteuern wollen. Vielleicht kann ich andere Entwickler dafür begeistern an der PDF Web App mitzuarbeiten und sie noch besser zu machen. 

\section{Aufbau der Arbeit}
Die Arbeit besteht im Prinzip aus 5 Kapiteln ohne Einleitung und Fazit. Ich beginne mit einem längeren Kapitel über die Grundlagen des PDF-Dateiformats. Zuerst nenne ich die wichtigsten Features, beleuchte die Dateiformate und -versionen, wobei ich diese benenne, die von der \gls{iso} standardisiert wurden. Danach erkläre ich die Implementierung des Dateiformats und die kritischen Sicherheitsaspekte. Im nächsten Kapitel Stand der Technik zeige ich die neusten Generative AI-Entwicklungen auf, wobei ich mir einige AI assistants herausgesucht habe, um sie näher zu beschreiben. Einige PDF-Programme auf dem Markt implementieren AI-Funktionalitäten. Ich habe mir 7 interessante PDF-Programme auf dem Markt ausgesucht und erkläre dessen Funktionsumfang. Danach beleuchte ich die Rolle von PDF in allgemeinen Marktbranchen und speziell in der Druckvorstufe bzw. Designbranche. Im 4. Kapitel Anforderungen und Konzept stelle ich die Anforderungen inklusive Qualitätsanforderungen an meine programmierte PDF Web App vor. Das anschließende Kapitel Implementierung startet mit der Bedienung der PDF Web App, inklusive Screenshots der Ergebnisse. In dem Umsetzung in Code-Abschnitt gehe ich detailliert auf die Programmierung ein und erkläre, wie ich Funktionalitäten realisiert habe. Anschließend zeige ich die Performance in der Testdurchführung. Im letzten Kapitel Diskussion und Kritik erörtere ich, was man bei der Programmierung noch verbessern könnte, welche Anforderungen ich nicht umgesetzt habe und welche Features ich mir für zukünftige Releases der PDF Web App vorstellen kann. 
\chapter{Grundlagen}

\section{PDF Einführung}
Die Popularität von PDF (Portable Document Format) Dateien ist seit 2008 rasant angestiegen in der globalen Informationsübertragung. Täglich werden weltweit 2,5 Milliarden PDF-Dokumente erzeugt. Seine Beliebtheit verdankt PDF vor allem an der plattformübergreifenden Kompatibilität (Desktop-Computer, Tablets und Smartphones), denn PDF Dokumente ist auf mehr als 1,5 Milliarden Geräten ohne zusätzliche Software lesbar. Über 80\% der geschäftlichen Dokumente werden als PDF Datei weitergegeben. [1] 90 \% der Büroangestellten wollen auf das PDF Dateiformat nicht mehr verzichten. Drei Viertel aller archivierten Dokumente sind PDF Dokumente. [3] Das PDF Dateiformat steht für Plattformunabhängigkeit, Konsistenz in Formatierung und Layout und soll ein möglichst originalgetreues Druckergebnis liefern.
\par
PDF wurde 1993 von Adobe veröffentlicht und ging aus dem 1991 von Adobe-Mitbegründer John Warnock gestarteten „Project Camelot" hervor. Ziel dieses Projektes war, ein Dateiformat für elektronische Dokumente zu kreieren, sodass diese Anwendungsprogramm, Betriebssystem und Hardware unabhängig originalgetreu wiedergegeben werden können. Anfangs war der Adobe Reader kostenpflichtig und PDF war für einen langen Zeitraum ein proprietäres Dateiformat, welches offengelegt im PDF Reference Manual von Adobe dokumentiert ist. Die \acrshort{iso} übernahm PDF 2007 in den Standardisierungsprozess und seit der Veröffentlichung von PDF Version 1.7 am 1. Juli 2008 gilt PDF als Offener Standard. [2] Der Begriff Offener Standard bezeichnet einen Standard, der für alle Teilhaber am Markt besonders leicht zugänglich, weiterentwickelbar und einsetzbar ist. Das bedeutet, dass der Standard von einer gemeinnützigen Organisation eingeführt, veröffentlicht, weiterentwickelt und gleichmäßige Einflussnahme aller interessierten Parteien ermöglicht. [4]

\section{PDF Funktionsumfang}
PDFs können Texte, Tabellen, Bilder, Links, Buttons, Formulare, Audio-, Videoelemente und Funktionen enthalten.
Um die Navigation innerhalb eines PDF Dokuments zu erleichtern können PDFs anklickbare Inhaltsverzeichnisse und miniaturisierte Seitenvorschauen enthalten.
\section{PDF Dateiformate}


\subsection{PDF/X}
Speziell für den simpleren Datenaustausch in der Druckvorstufe wurde PDF/X (Exchange) als (\gls{iso} 15930) entwickelt. Dieser 2001 entwickelte Standard beschreibt die speziellen Eigenschaften von Druckvorlagen und vereinfacht die Datenübermittlung von der Druckvorstufe bis zum finalen Druck. Besonderen Wert wurde darauf gelegt, dass in offenen Dateiformaten aus Layoutprogrammen keine Informationen über Farbe und Schrift verloren gehen und einer Verfälschung im Druckergebnis vorgebeugt werden kann. \cite{adobe-pdf-e}


\subsection{PDF/VT}
Basierend auf PDF/X wurde PDF/VT als spezielles Austauschformat im variablen Datendruck (V) und Transaktionsdruck (T) im Jahr 2010 auf den Markt gebracht. Wiederkehrende Elemente wie Texte, Grafiken oder Bilder sollen effizienter verarbeitet werden können. \cite{adobe-pdf-e}


\subsection{PDF/A}
Das PDF/A Dateiformat (Archivable) wurde zur gesetzeskonformen Langzeitarchivierung von digitalen Dokumenten entwickelt und solche Dokumente sind zunächst schreibgeschützt. Der \gls{iso}-Standard definiert die Konformität der Form von Elementen wie Schriften oder Layout für eine Langzeitarchivierung. Dadurch ist die Lesbarkeit der Dokumente über lange Zeiträume gesichert und die Bedingungen einer revisionssicheren Archivierung gewährleistet. \cite{adobe-pdf-a} Revisionssichere Archivierung bedeutet, dass gespeicherte Daten vor nachträglichen Modifikationen, Fälschung oder Manipulation geschützt sind. \cite{adobe-revisions} Der Fokus in diesem Dateiformat liegt auf langfristige und einfache Speicherung der PDF-Dateien. Folglich ist die Einbettung von Audio und Video nicht implementiert, aktive Komponenten wie Links, sowie externe Ressourcen, wie Grafiken und Schriftarten werden nicht unterstützt, sondern müssen direkt eingebettet werden. Ebenso können Dokumente nicht verschlüsselt werden. Die Einbettung von Metadaten als \gls{xmp} wird unterstützt, was die Identifizierung und Suche von Dokumenten erleichtert. Es gibt einige Nachteile von PDF/A. Nicht alle Dokumente können problemlos in dieses Dateiformat umgewandelt werden, wie beispielsweise Dokumente mit Audio, Video oder JavaScript. Nach der Konvertierung zu PDF/A kann es zu Fehlern in der visuellen Darstellung kommen und die Dateigröße kann enorm werden, da alle Elemente direkt eingebettet werden müssen. \cite{adobe-pdf-a}

\subsubsection{PDF/A-1}
Seit der ersten Version von PDF/A wurde es in die \gls{iso}-Norm übernommen worden als PDF/A-1 in \gls{iso} 19005-1:2005. \cite{proj-consult} Die Originalversion stellt sicher, dass alle externen Quellen wie Schriften oder Bilder eingebettet sind, unterstützt digitale Signaturen und Hyperlinks. PDF/A-1 ist abwärtskompatibel. Es gibt 2 Qualitätsebenen von PDF/A-1: PDF/A-1b (Basic) und PDF/A-1a (Accessible). Die Basic Variante legt Wert darauf, dass Dokumente eindeutig visuell reproduzierbar sind und Accessible ist zusätzlich für Barrierefreiheit optimiert. Bei Accessible können Text und inhaltliche Struktur von einem Screenreader vorgelesen werden. \cite{adobe-pdf-a} Des weiteren werden Tagged PDFs, Sprach-Angabe und Unicode Mappings unterstützt. \cite{proj-consult}

\subsubsection{PDF/A-2}
Im Jahr 2011 wurde die PDF/A-2 Version als \gls{iso} 19005-2:2011 auf den Markt gebracht. Sie ermöglicht die Kompression von Grafikformaten mit JPEG-2000, Transparenzen, PDF-Ebenen, Portfolios, Object Level \gls{xmp} Metadaten, Kommentartypen und Annotationen und digitale Signaturen. \cite{proj-consult} PDF/A-1-Dateien können in PDF/A-2-Dateien eingebunden werden. Es gibt 3 Varianten von PDF/A-2: PDF/A-2b (Basic), PDF/A-2u (Unicode-Textsemantik) und PDF/A-2a (Accessible). Basic gewährleistet das unveränderte Erscheinungsbild eines Dokuments und definiert die Mindestanforderungen. Die Unicode-Version ergänzt um Unicode-Unterstützung und Indexierung. Accessible setzt alle Anforderungen der \gls{iso}-Norm 19005-2 um. \cite{adobe-pdf-a}

\subsubsection{PDF/A-3}
Ein Jahr später wurde PDF/A-3 im Standard \gls{iso}-19005-3:2012 veröffentlicht. Er basiert auf PDF 1.7 und ermöglicht die Einbettung dynamischer, zur Laufzeit interpretierbare Komponenten und Dateiformate. Gleichfalls definiert PDF/A-3 die Konformitätsebenen 3b, 3u und 3a. Die u-Variante bietet eine Vereinfachung in der Durchsuchbarkeit von Texten und das Kopieren von Unicode-Text. \cite{proj-consult}

\subsubsection{PDF/A-4}
Viel später im Jahr 2020 wurde PDF/A-4 als \gls{iso} 19005-4:2020 herausgebracht. Dieser Standard basiert auf der PDF 2.0 Dateiversion. Sie spezifiziert 2 neue Konformitätsebenen PDF/A-4f für nicht-PDF/A konforme Dateianhänge und PDF/A-4e für Einbindung von 3D-Inhalten in den Formaten U3D oder PRC für den Engineering-Bereich. \cite{proj-consult}


\subsection{PDF/E}
PDF/E (Engineering) gilt als international standardisiertes Austauschformat als Norm \gls{iso} 24517 für technische Dokumente und wird im Maschinenbau, in der Fertigung und im Baugewerbe für Fertigungspläne, Konstruktionszeichnungen oder technische Dokumentationen verwendet. Das PDF/E Dateiformat von 2008 ist speziell für das Ingenieurwesen entworfen und kann interaktive 3D-Elemente und Animationen darstellen. Im einzelnen können \gls{cad}-Dateien im 3D- und 2D-Format eingebettet werden. Die 3D-Elemente können im Dokument ausgeklappt oder gedreht werden. Metadaten und interaktive Funktionen wie Lesezeichen, Formulare oder Hyperlinks werden unterstützt. \cite{adobe-pdf-e}


\subsection{PDF/UA}
Das PDF/UA (Universal Accessibility) Dateiformat dient der Erstellung barrierefreier Dokumente. Die PDF/UA-Kennzeichnung stellt eine Klassifizierung für barrierefreie Dokumente dar und orientiert sich an den Anforderungen der \gls{wcag} 2.0 des World Wide Web Consortiums. Als Rechtliche Grundlage dient die \gls{bitv} 2.0. Um die Anforderungen der PDF/UA-Kennzeichnung zu erfüllen, müssen Dokumente bestimmte technische und inhaltliche Vorgaben erfüllen. Auf Basis des Matterhorn-Protokolls, welches aus 31 Prüfpunkten und 136 Konformitätskriterien besteht, müssen Aufbau von Texten, Bildern, Listen, Tabellen und Formularefeldern festgelegte Einstellungen haben. Diese Konformitätskriterien können auf der einen Seite nur von einer Software geprüft werden und auf der anderen Seite nur von Menschen. Menschen mit Einschränkungen sollen das Dokument optimal nutzen. Zur Erleichterung des Verständnisses sollten Überschriften, alternative Texte für Bilder, Beschreibungen für Tabellen, Tags und eine klare Lesereihenfolge verwendet werden. Mittels der Tags können Screenreader Inhalt und Struktur des Dokuments erfassen. \cite{adobe-pdf-ua} Es gibt die PDF/UA-1 Version aus \gls{iso} 14289-1:2012 und die überarbeitete Version PDF/UA-1 aus \gls{iso} 14289-1:2014, die erst 2020 als gültig erklärt wurde. \cite{proj-consult}


\subsection{Durchsuchbares PDF}
Searchable PDFs können mit Suchfunktionalitäten eines PDF-Readers durchsucht werden. Es kann gezielt nach Zahlen oder Stichwörtern durchsucht und Inhalte können zur Bearbeitung in anderen Programmen kopiert werden. Man erkennt durchsuchbare PDFs daran, dass man den Text markieren kann. Diese PDF-Art wird üblicherweise durch die \gls{ocr} Technologie erstellt. Bei \gls{ocr} handelt es sich um optische Zeichenerkennung, die Textzeichen und Dokumentstruktur analysiert. Auf diese Weise können gescannte Dokumente oder Pixelbilder als PDF abgespeichert und in ein Durchsuchbares PDF in Adobe Acrobat umgewandelt werden. Während der Umwandlung wird dem Dokument eine zusätzliche unsichtbare Textebene, die unter der Bildebene liegt und durchsuchbar ist, auf der Seite hinzugefügt. In Acrobat ist es außerdem möglich mit der einfachen Suche innerhalb einer Datei nach Suchbegriffen zu suchen, mit der erweiterten Suche oder der Suchen-Werkzeugleiste mehrere PDF-Dokumente zu durchsuchen und speziell in der erweiterten Suche u.a. Objektdaten und Bildern zu lokalisieren. Textbasierte PDF-Dateien können grundsätzlich durchsucht werden und auch in andere Dateiformate wie Microsoft Word, Excel oder PowerPoint umgewandelt werden. Durchsuchbare PDFs ermöglichen Barrierefreiheit. Sie können von Bildschirmleseprogrammen für Sehbehinderte vorgelesen oder vergrößert werden. 
\cite{adobe-search}


\subsection{PAdES}
\gls{pades} ergänzt den Funktionsumfang um Werkzeuge, um elektronische Signaturen anzupassen.


\subsection{PDF/H}
Das PDF/H (Healthcare) Dateiformat soll im Gesundheitswesen Patientendaten erfassen, austauschen und archivieren. Hierbei wird besonders Wert auf die Anforderungen des Datenschutzes in gesundheitsspezifischen Ämtern, Institutionen und Arztpraxen gelegt. PDF/H wurde 2008 kreiert, jedoch wurde es nicht in den Normierungsprozess der \gls{iso} eingebunden. \cite{proj-consult} Es handelt sich eher um eine Best-Practice für die Umstellung von Papiergesundheitsakten auf PDF als e-Akte. Die Digitalisierung soll gesundheitsbezogene Daten strukturieren, verwalten und so präsentieren, dass Forscher*innen und Beschäftigte im Gesundheitswesen effizienter auf sie zugreifen können.

\section{PDF Dateiversionen}

\subsection{PDF 1.0}
PDF 1.0 wurde 1992/1993 entwickelt und ist keine Norm. 1992 wurde die Spezifikation als Buch verkauft und 1993 wurde das der Spezifikation entsprechende digitale Format entwickelt, welches ausschließlich den RGB Farbraum darstellen konnte. Medien, die einen anderen Farbraum besaßen wurden in RGB umkonvertiert. Der RGB Farbraum ist nur für die Bildschirmdarstellung geeignet und beschreibt die für den Menschen 16,7 Mio. sichtbaren Farben mit Hilfe von additiver Farbmischung. In der Druckindustrie ist jedoch der CMYK Farbraum von Bedeutung und daher war PDF 1.0 nicht für den Printbereich ausgelegt. Damals war Adoba Acrobat 1.0 das einzige Programm, um mit dieser Dateiversion zu arbeiten. \cite{proj-consult}

\subsection{PDF 1.1}
Genauso ist das 1994 kreierte PDF 1.1 keine Norm und implementiert weiterhin nur den RGB Farbraum, jedoch geräteunabhängig. Zusätzlich benötigte man ein Update von Adobe Acrobat auf Version 2.0. Erstmals sind in diesem Format das Einbetten von externen Links, mehrseitige Artikel und Threads, Passwortverschlüsselung und Notizen und Anmerkungen erschienen. \cite{proj-consult}

\subsection{PDF 1.2}
Das ebenfalls 1996 erschienene PDF 1.2 wurde keine Norm, jedoch ermöglichte es erstmals den druckbaren CMYK Farbraum und Sonderfarben zu verwenden. Des weiteren wurden interaktive Formularfunktionen, Unicode Unterstützung, Multimedia Kompatibilität, Unterstützung der \gls{opi} 1.3 Spezifikationen und eine Druckrasterfunktion implementiert. \cite{proj-consult} In PDF 1.2 wurden erstmal AcroForms (Acrobat forms) vorgestellt.

\subsection{PDF 1.3}
1999 wurde PDF 1.3 auf den Markt gebracht und trug seinen Teil 2001 und 2002 bei zur Standardisierung des \gls{iso} PDF/X Standards bei. Es ist kompatibel mit PostScript 3 und bietet die Neuerungen der 2-Byte \gls{cid} Schrifttypen, OPI 2.0 Unterstützung, Farbraumerweiterung für Sonderfarben durch \gls{icc}-Profile, DeviceN Farbraum, weiche Schatten und Farbübergänge (Smooth Shading), digitale Signaturen, RC4-Verschlüsselung (40 Bit in Acrobat 4 und 56 Bit in Acrobat 4.05) und JavaScript. \cite{proj-consult}


\subsection{PDF 1.5}
2003 kam PDF 1.5 auf den Markt und hat sich nicht zur Norm entwickelt. In dieser Version wurden erstmals Ebenen implementiert, die erlauben dass man mehrere Elemente wie eine Gruppe auf einer Ebene speichern kann und diese Elemente auf einmal nach Bedarf ein- und ausblenden kann, sperren oder Operationen anwenden kann. Diese Funktionalität enthalten auch die Adobe Programme Photoshop, InDesign und Illustrator. Des weiteren wurden gesteigerte Kompressionstechniken einschließlich Objekt-Streams und JPEG 2000-Kompression, sowie eine verbesserte XRef-Tabelle und XRef-Streams implementiert.
Die XRef-Tabelle enthält die Positionen der indirekten Objekte innerhalb der Datei. Streams binden Dateien ein. 12 weitere Seitenübergänge für Präsentationen, verbesserte Unterstützung für Tagged PDF und die Adobe proprietäre Technologie \gls{xfa} wurden außerdem hinzugefügt. \cite{proj-consult} \gls{xfa}s Haupterweiterung zu \gls{xml} sind rechnergestützte, aktive Tags uns sein Datenformat ist kompatibel mit anderen Systemen, Anwendungen und Technologiestandards. \cite{wiki-xfa}

\subsection{PDF 1.4}
Der erste PDF \gls{iso}-Standard \gls{iso} 16612-1:2005 wurde endlich verabschiedet.

\subsection{PDF 1.6}

\subsection{PDF 1.7}
Veröffentlichung am 1. Juli 2008 ist PDF in Version 1.7 als \gls{iso} 32000-1:2008 ein Offener Standard

\subsection{PDF 2.0}
\gls{xfa} ist in PDF 2.0 vom \gls{iso} Gremium als veraltet markiert.
\section{\gls{pdf} Implementierung}
\gls{pdf} ist eine vektorbasierte \gls{pdl} (Seitenbeschreibungssprache) und basiert auf dem PostScript-Format. Eine \gls{pdl} beschreibt den Seitenaufbau, wie die Seite in einem Ausgabeprogramm bzw. Ausgabegerät, z.B. einem Drucker, aussehen soll. \gls{pdl}s können Seiten mit Vektoren beschreiben. Das Ausgabeformat ist normalerweise nicht zur weiteren Bearbeitung vorgesehen. An den Drucker wird durch die \gls{pdl} ein Datenstrom der zu druckenden Aufgabe erzeugt und an den Drucker gesendet. Dabei müssen Drucker die \gls{pdl} nicht selbst verarbeiten. Im Common Unix Printing System, der Standard-Druckersteuerung von Linux hat der PostScript und der \gls{pdf}-Interpreter ghostscript die Aufgabe eines \gls{rip}, d.h. er ist für die Umwandlung in die gerasterte Druckausgabe auf dem Drucker zuständig. Viele APIs der Hardwareabstraktionsschicht im Computer wie \gls{gdi} oder OpenGL können in \gls{pdl} ausgeben.

\section{\gls{pdf} Sicherheitsaspekte}
Etwa 40 \% der Unternehmen setzen \gls{pdf}s für geschützte Inhalte ein. In den letzten 2 Jahren ist die Nutzung der elektronischen Signaturfunktion in \gls{pdf}s um mehr als 150 \% gestiegen. [1]
\par
\gls{pdf} unterstützt Verschlüsselung und die Vergabe von 2 Passworttypen. Eventuell kann beim Öffnen einer Datei ein Passwort gefordert werden oder das Kopieren von Teilinhalten, jeglichem Inhalt oder das Ausdrucken ist gesperrt.
\section{Aktueller Stand von Forschung und Technik von \gls{pdf}}
\chapter{PDF Programme auf dem Markt}
Bis 2025 werden über 3 Millarden Dollar jährlich für PDF Editoren ausgegeben werden. \cite{kofax}

\section{Aktueller Stand von Forschung und Technik}
Artificial Intelligence (AI) Tools können können das Arbeiten mit PDF erleichtern. \gls{ie} ist eine Unterkategorie von \gls{nlp} und ist relevant für die Identifizierung von relevanten Informationen in Text und die Extrahierung dieser Informationen zu spezifischen Output-Formaten. Bei der \gls{re} werden relevante Einheiten im Text in Beziehung zueinander gesetzt. 


Claude 2 von Anthropic ist ein AI assistant, der PDFs parsen und die Dokumentstruktur mittels Machine Learning (ML) Techniken erfassen kann. Es hat eine eingebaute PDF analyser Komponente. Die maximale Dateigröße ist 10 MB. Der Benutzer kann Claude Fragen stellen, die im PDF-Dokument behandelt werden und Claude liefert die Antworten. Direkte Zitate können aus der PDF-Datei gefunden werden und Claude zeigt die Seitenzahl an, wo das Zitat vorkommt. Unglücklicherweise ist Claude zurzeit nicht für Deutschland verfügba \cite{hackernoon-claude}r.

Perplexity AI ist ein Werkzeug für \gls{nlp} mit Dokumenten. Eine PDF-Datei als Dateiformat, plaintext oder Code kann hochgeladen werden und Perplexity verwendet dessen Dateiinhalte, um Antworten auf Fragen zum PDF inklusive Zitate mit Quellenangaben zu formulieren. Bei kurzen Dateien wird das gesamte Dokument beim language model analysiert. Umfangreiche PDFs können manuell in Themenbereiche unterteilt werden und als input für \gls{gpt4} für Kreatives Schreiben verwendet werden. Wissenschaftliche Artikel können verglichen, ihre Unterschiede herausgearbeitet, themenverwandte Dokumente durch eine query gefunden, Daten analysiert, Überblicke von verschiedenen Quellen generiert, Daten visualisiert, Grafiken von Quellen erstellt und Text in eine andere Sprache übersetzt werden. Bei der kostenlosen Version ist der Benutzer auf eine bestimmte Anzahl an Anfragen begrenzt \cite{hackernoon-claude}.

ChatGPT Plus Mitglieder können seit Oktober 2023 ein PDF-Analysefeature als Beta-Version genießen. Benutzer können PDF-Dateien und andere Dokumente hochgeladen werden und der chatbot kann Zusammenfassungen erstellen und Graphen oder Tabellen basierend auf die Dokument-Daten erstellen \cite{hackernoon-claude}.

Amazon Textract ist ein ML-Service, der automatisch Text, Handschrift, Layoutelemente und Daten von gescannten Dokumenten, Bildern oder PDF-Dateien ohne manuelle Konfiguration extrahieren kann. Bei Textract handelt es sich um einen Cloud-Dienst von Amazon Web Services. Im Gegensatz zu traditionellen \gls{ocr}-Anwendungen kann Textract außerdem strukturierte Informationen aus Tabellen oder Formularen erfassen. Nebst Zeichenerkennung werden auch Formatierungen, sowie die Struktur des Text herausgearbeitet. Der Service kann per Konsole, Command Line Interface (CLI) oder Application Programming Interface (API) verwendet werden. Mittels der Detect Document Text API wird eine \gls{ocr}-Schnittstelle bereitgestellt, um handschriftliche oder gedruckte Texte aus Dokumenten zu entnehmen. Zusätzlich hat die Analyze Document API das Aufgabenfeld, strukturierte Daten einzulesen und Beziehungen bzw. Schlüsselwertpaare aus Tabellen- oder Formularfelder zu erstellen. Extrahierte Informationen sind mit einem Confidence-Score versehen, um dem Benutzer mitzuteilen, wie exakt und verlässlich die Daten sind. Handschriftlicher und gedruckter Text kann mit hoher Genauigkeit und Zuverlässigkeit erkannt werden. Die Extraktion der Daten geschieht sehr performant in kurzer Zeit. Das Pricing des Produkts ist nutzungsabhängig \cite{textract}.
\section{Freie PDF Programme und Onlinedienste}
PDF Dateien lassen sich in vielen Programmen einfach über den Druckdialog erstellen. Apple hat das Lesen von PDF Dokumenten in seiner Apples Vorschau integriert. Viele Webbrowser stellen PDF Viewer bereit, so Google Chrome seit 2010 \cite{wiki-pdf-de}.

\subsection{foxit}
Produktseite: \url{https://www.foxit.com/}
Die Firma foxit hat einen kostenlosen PDF Editor auf den Markt gebracht, der sehr umfangreiche Funktionen bietet.

\subsection{PDF24}
Produktseite: \url{https://www.pdf24.org/en/}

\subsection{PDFsam}
Produktseite: \url{https://pdfsam.org/}

\subsection{DrawboardPDF}
Produktseite: \url{https://www.drawboard.com/pdf/pdf}


\section{Kostenpflichtige PDF Programme und Onlinedienste}
PDF Dateien lassen sich in vielen Programmen einfach über den Druckdialog erstellen. Apple hat das Lesen von PDF Dokumenten in seiner Apples Vorschau integriert. Viele Webbrowser stellen PDF Viewer bereit, so Google Chrome seit 2010 \cite{wiki-pdf-de}. Ich werde in den folgenden Abschnitten 7 kostenlose und kostenpflichtige Programme ausführlich behandeln und ihre interessantesten und wichtigsten Funktionen aufzeigen. Die Liste an PDF-Programmen lässt sich unendlich fortführen, deshalb habe ich mir die interessantesten herausgepickt. Andere kostenpflichtige PDF-Programme heißen pdf-it, Soda PDF, Nitro PDF Pro, Ashampoo PDF Pro 3, Infix 7, PDF Director 2 Pro und Perfect PDF.

\subsection{Adobe Acrobat Pro}
Produktseite: \url{https://www.adobe.com/de/acrobat/acrobat-pro.html} \\
Acrobat ist ein professionelles Werkzeug, um PDF-Dateien zu erstellen und bereits bestehende Inhalte zu bearbeiten. Text kann hinzugefügt, geändert, formatiert, gelöscht oder markiert werden. Ebenfalls kann der Text von Formularen bearbeitet werden. Bilder können gespiegelt, gedreht, zugeschnitten, ersetzt, ausgerichtet oder angeordnet werden. Beim Objekte Anordnen-Werkzeug können mehrere Objekte vor oder hinter anderen Objekten angeordnet werden. Das Arbeiten mit Ebenen ist möglich, jedoch eingeschränkter als bei Photoshop. PDF-Seiten können kopiert, ersetzt, gedreht, verschoben, gelöscht, extrahiert oder neu nummeriert werden. Eine PDF-Datei kann in mehrere Dokumente aufgeteilt und Kommentare können hinzugefügt werden. Webseiten können mit der Acrobat-Browsererweiterung in PDF konvertiert werden \cite{adobe-acrobat-um}. Eine PDF-Datei kann in eine PDF/A-Datei inklusive seiner Varianten, PDF/X, PDF/UA, PDF/VT oder PDF/E konvertiert werden. Außerdem kann die Kompatibilität mit diesen Formaten in Preflight-Profilen überprüft werden \cite{adobe-pdf-a}. Die Barrierefreiheit kann automatisch validiert oder ein neues Dokument kann direkt barrierefrei erstellt werden. Adobe Acrobat Pro kann andere Dokumentenformate wie HTML, DOC, DOCX, TXT und RTF in PDF konvertieren, PDF in andere Dateiformate wie Word exportieren oder Dokumente unterschreiben \cite{adobe-formate}. Der Benutzer kann PDF-Dokumente mit digitalen Signaturen unterzeichnen und das Programm kann die Signaturen validieren \cite{adobe-acrobat}. Für Vertraulichkeit können PDF-Dokumente mit Passwörtern und Zertifikaten versehen werden. Interaktive Objekte, Audio und Video können eingebettet werden. In der Pro-Variante gibt es Druckproduktionswerkzeuge u.a. für Druckermarken, Transparenz-Reduzierung, Farbkonvertierung oder Druckermanagement. Die Preflight-Option ist ebenfalls nur in Acrobat Pro verfügbar \cite{adobe-acrobat-um}. Mit dem Werkzeug Scan \& \gls{ocr} in Acrobat Pro kann man Pixelbilder als PDF und gescannte PDF-Dokumente in ein durchsuchbares PDF umwandeln \cite{adobe-search}. Lediglich Acrobat Pro kann PDFs vergleichen und schwärzen. Die Standard-Version ist nur unter Windows verfügbar. Hingegen ist Acrobat Pro für macOS und Windows erhältlich \cite{wondershare-acrobat}. Das Programm gibt es als kostenlosen Acrobat Reader mit sehr eingeschränkten Funktionen, Acrobat Standard für 15,46 Euro pro Monat und Acrobat Pro für 23,79 Euro pro Monat. Hierbei handelt es sich um ein Jahres-Abo mit monatlicher Zahlung. Wählt man das Monats-Abo, so kostet die Standard-Version 27,36 Euro im Monat und die Pro-Variante 35,69 Euro im Monat. Es gibt auch die Möglichkeit das Jahres-Abo mit Vorauszahlung zu erwerben \cite{adobe-acrobat}. Die Testversion von Acrobat ist 7 Tage nutzbar. Mit den Adobe Acrobat Onlinetools kann man über den Browser verschiedene Dateitypen in PDF umwandeln, unter anderem PDF in JPEG oder andere Bildformate, PDF Dateien bearbeiten und Kompression anwenden. Die Onlinetools können ebenfalls PDF in Word umwandeln. \cite{adobe-search} Der Adobe Acrobat PDF-Converter der Onlinetools kann DOCX, DOC, XLSX, XLS, PPTX, PPT, TXT, RTF, JPEG, PNG, TIFF, BMP, sowie Adobe eigene AI-, INDD- und PSD-Dateien in PDF konvertieren. \cite{adobe-formate} Die kostenlose Version des PDF-Converters kann nur begrenzt oft genutzt werden.

\subsection{Wondershare PDFelement}
Produktseite: \url{https://pdf.wondershare.com/} \\
Wondershare PDFelement ist eine KI gestützte PDF-Alternative zu Acrobat. Das Programm wird für Windows, macOS, iOS und Android angeboten und die aktuellste Version ist PDFelement 10. Bereits bestehende Bilder oder Text des geöffneten PDFs können bearbeitet werden. Vom Autor eingebettete Objekte können in 90 Grad-Schritten gedreht, horizontal und vertikal gespiegelt oder ausgerichtet werden. PDF-Dokumente können mit Highlights versehen und Geometrie bzw. Freihandzeichnungen können hinzugefügt werden. Dabei handelt es sich um Kommentartypen, die auch gedreht, skaliert oder verschoben werden können. Das Radiererwerkzeug funktioniert nur für Freihandzeichnungen. Der Hintergrund einer PDF-Datei kann seitenweise mit Farbe gefüllt werden oder durch ein Bild verschönert und leere PDFs können erstellt werden. Ausfüllbare Formulare können sogar automatisch erstellt werden. Es gibt Sicherheitsfunktionen für Passwortschutz, Signaturerstellung und Validierung von Signaturen. PDFelement besitzt ebenfalls eine \gls{ocr}-Funktion. Im Bezug auf AI-Tools gibt es die Optionen Erklären von Inhalten und Code, Korrekturlesen, Zusammenfassen, Chat mit PDF, Übersetzungen und eine Funktion zur Erkennung von AI erstellten Inhalten, sowie eine automatische Lesezeichenfunktion basierend auf der PDF-Struktur und Überschriften. Der AI reading assistant nennt sich Lumi und beruht auf ChatGPT. Mehrere PDF-Dateien können gleichzeitig geöffnet sein und Screenshots, Screen Recording, Lineale, Hilfslinien und Hilfsgatter sind verfügbar. Eine Stapelverarbeitung von mehreren PDF-Dateien ist für Konvertierung in andere Dateiformate, Komprimierung, \gls{ocr}, einige Bearbeitungsoptionen, Drucken und Verschlüsselung verfügbar. Konvertierungen von PDF zu Word, Powerpoint, Excel und JPEG, PNG, TIFF, GIF, Epub, HTML, \gls{xml} oder PDF/A sind u.a. möglich \cite{wondershare-um}. In der kostenlosen 14-tägigen-Testversion mit Anmeldung, bei der dann mehrere Funktionen zur Verfügung stehen, wird ein PDF mit Wasserzeichen gespeichert. Für einen Einzelbenutzer kostet eine Dauerlizenz 119 Euro, ein Jahresabo 89 Euro und ein 2-Jahresabo 109 Euro. Darin sind PDFelement Updates enthalten, sowie 20 GB Document Cloud Speicher. Es gibt einen Rabatt für Schüler, Studenten und Lehrkräfte \cite{wondershare-preis}. 

\subsection{UPDF Pro}
Produktseite: \url{https://updf.com/} \\
UPDF bietet eine kostenlose Testversion mit 1 GB Cloud-Speicher an, die fast alle Features der Pro-Variante freigeschaltet hat. Die kostenlose Testversion enthält mehr Funktionen, wenn man sich bei UPDF registriert. Es gibt bei UPDF verschiedene Anzeigemodi. Man kann auch 2 PDFs nebeneinander anzeigen lassen und sogar scrollen. Mehrere PDFs können in Tabs geöffnet werden. Ein Dokument kann als Slide Show mit dunklem Hintergrund für Präsentationszwecke angezeigt werden. Dabei kann man mit dem Eraser, Pen und Laser Pointer bestimmte Stellen im PDF während des Präsentationsmoduses hervorheben. Lesezeichen und Kommentare können hinzugefügt werden. Als Kommentarvarianten stehen Highlights, Text, Stift und Radierer, Geometrie, Sticker, Stempel und Dateianhänge zur Verfügung. Bei UPDF kann man Seiten neu ordnen, drehen, zerteilen, beschneiden, ersetzen, extrahieren und löschen. Eine \gls{ocr}-Funktionalität ist vorhanden. PDF-Formulare können erstellt und ausgefüllt werden. Eine PDF-Datei kann zu PDF/A, Word, Excel, PowerPoint, CSV, TXT, Bilder oder \gls{xml} konvertiert werden. Aus folgenden Dateiformaten kann ein PDF erstellt werden: CAJ, Word, PowerPoint, Excel, Visio und Bilddateien (.png, .jpeg, .bmp und .gif). Wenn man ein PDF-Dokument in der kostenlosen Testversion speichert wird es mit einem Wasserzeichen versehen. Jedes Mal, wenn man ein Dokument in der kostenlosen Testversion mit Login geöffnet hat, fragt UPDF nach, ob man sich die Pro-Version kaufen möchte, was sehr störend ist. Verschlüsselung ist durch eine Passworterstellung fürs Öffnen oder Rechte gewährleistet. Elektronische und digitale Signaturen mit Zertifikaten werden unterstützt. Man kann ein PDF in die UPDF Cloud hochladen und dem AI assistant Fragen über das hochgeladene PDF stellen. Basierend auf das PDF in der Cloud kann der AI assistant das PDF zusammenfassen, übersetzen und Inhalte erläutern. Es gibt auch einen Chatbot, bei dem man Text in den Prompt eingeben kann und der AI assistant kann vom Input Text übersetzen, zusammenfassen, erläutern, Rechtschreibung prüfen und Artikel schreiben. Bereits in der Testversion kann man source Dokument-Inhalte in Text- und Bildform bearbeiten. Bilder können um jeweils 90 Grad rotiert werden, extrahiert, zugeschnitten, ersetzt, verschoben, gelöscht und skaliert werden. Texte können skaliert, die Schriftart verändert, fett und kursiv gemacht, verschoben, gelöscht und die Farbe verändert werden. Text und Bilder können außerdem neu hinzugefügt werden. Beim Verschieben von Text, was durch das Ziehen der Textbox in verschiedene Richtungen möglich ist, bricht der Text automatisch um und passt sich der Textbox an. Hyperlinks und der Hintergrund können hinzugefügt, editiert und gelöscht werden. Stapelprozesse sind in der Pro-Version möglich. Als Prozesse kann man zwischen Konvertieren, Kombinieren, Einfügen, Drucken und Verschlüsseln wählen. PDFs können in maximum, high, medium oder low Stärken komprimiert werden. UPDF ist in 11 Sprachen inklusive Deutsch verfügbar \cite{updf-um}. UPDF kann auf Windows, macOS, iOS und Android installiert werden. Für eine jährliche Pauschale von 32,99 US Dollar ist UPDF Pro zu erwerben. Eine Dauerlizenz kostet 52,99 Dollar. Ausschließlich das AI Add-on kann für 59 Dollar pro Jahr gebucht werden \cite{updf-preis}. 

\subsection{Mathpix Snip}
Produktseite: \url{https://mathpix.com/snip} \\
Mathpix Snip ist ein PDF Editor mit Konvertierungsfeatures, dessen Zielgruppe Forscher, Lehrer und Studenten sind, d.h. Mathpix ist auf Forschungsdokumente und Research Papers ausgerichtet. Snip ist als Web App, im Apple Store, Google App Store und in der Huawei AppGallery verfügbar. Mittels \gls{ocr}-Technologie können sogar Formeln, Tabellen und zweispaltige PDFs in folgende Formate konvertiert werden: Word, LaTex, HTML und Markdown. Beim Online-Editor wird eine Mathpix Markdown Sprache verwendet und man kann auch Markdown, LaTeX und HTML verwenden um eine PDF-Datei zu erstellen. Für Desktop ist Snip ebenfalls für die Plattformen MacOS, Windows und Linux verfügbar. Es gibt außerdem eine Google Chrome Extension. Bilder können u.a. zu LaTeX, HTML, MathML, AsciiMath oder CSV konvertiert werden. Text und Matheformeln können direkt vom source PDF kopiert werden. Mit Search AI kann man Fragen über das Dokument stellen und erhält Antworten. Collaborative editing ermöglicht dem Ersteller, andere Benutzer zum Editiere hinzuzufügen und man kann live sehen wie bei Google Docs, wer gerade editiert. Handschriftliche Notizen können digitalisiert werden. Außerdem gibt es einen PDF Reader, der ebenfalls PDFs durch \gls{ocr} digitalisieren kann. Der Reader ist als Web App oder Mobil-Version verfügbar \cite{snip-um}. Die Snip App kostet 4,99 US Dollar pro Monat oder 49,90 Dollar pro Jahr. Die Gebühren für Organisationen, wie Schulen oder Firmen, kosten doppelt so viel. Es gibt eine kostenlose Version mit sehr begrenzter Anzahl an zu bearbeitenden PDF-Dateien. Sie reicht bei seltener Nutzung \cite{snip-price}.
\section{PDF zu Word Programme und Onlinedienste}
In PDF ist keine automatische Anpassung des Seiteninhalt-Layouts, wie z.B. in Microsoft Word, möglich. Daher kann ein PDF-Dokument nicht sinnvoll in das Word-Format umgewandelt werden ohne möglicherweise das ursprüngliche PDF-Layout zu beeinflussen und zu ändern, sowie die maximalen Bearbeitungsmöglichkeiten von Word ausschöpfen zu können.


\subsection{UPDF}


\subsection{WPS}


\subsection{LightPDF}
\section{PDF zu Latex Konvertierung}
\section{Relevanz von PDF in verschiedenen Marktbranchen}
Die PDF-Datei ersetzt als elektronisches Informationsaustauschformat in privaten Organisationen, Behörden und Bildungswesen papierbasierte Arbeitsprozesse. Vor allem in Behörden wird der E-Mail-Verkehr noch frequent verwendet. Starke Kompressionsalgorithmen ermöglichen es, speicherintensive PDF-Dateien auf leichtem Wege per E-Mail zu verschicken. Die PDF Spezifikation ist offen und sehr detailliert über sämtliche Aspekte des Dateiformats. Dadurch können Softwareunternehmen eigene Programme schreiben, die PDF Dokumente erzeugen, lesen und bearbeiten können. Searchable PDFs werden in Verträgen, Rechnungen und Geschäftsunterlagen verwendet, damit Mitarbeiter*innen Informationen gezielter suchen und Daten abteilungsübergreifend effizienter verwaltet werden können. In Forschungsarbeiten und wissenschaftlichen Artikeln werden searchable PDFs hauptsächlich bei der Überprüfung von Quellen oder dem Extrahieren von Zitaten verwendet. Behörden, Bibliotheken und Unternehmen digitalisieren Dokumente zur Archivierung und wandeln sie in ein searchable PDF um, was den langzeitigen Bestand der Dokumente sichert \cite{adobe-search}.
\par
Das PDF/A-Dateiformat wird in Bibliotheken und Archiven zur digitalen Archivierung von Büchern, Zeitschriften und historischen Dokumenten verwendet. Auch im Behördenzweig und Verwaltungssektor hat PDF/A für die Aufbewahrung von Verwaltungsakten und rechtlichen Dokumenten seine Existenzberechtigung. Im Gesundheitswegen wird es außerdem zur Speicherung von Patientenakten und medizinischen Unterlagen verwendet. Hingegen im Finanzwesen werden mit diesem Format Geschäftsunterlagen und Finanzdokumente verwahrt. Unternehmen und Organisationen können mit PDF/A gesetzliche und Compliance-Vorschriften einhalten \cite{adobe-pdf-a}. PDF/VT-Dateien werden im Direktmarketing verwendet. Personalisierte Werbematerialien erhöhen die Wahrscheinlichkeit einer positiven Reaktion bei den Kundi*nnen auf die Werbebotschaft und verbessert die Bindung von Unternehmen und Kund*innen. Der Transaktionsdruck findet bei Finanzdienstleistungen, Versicherungen und E-Commerce besonderen Anklang. Beliebte Transaktionsdokumente sind Rechnungen, Kontoauszüge, Versicherungspolicen oder Bestellbestätigungen \cite{adobe-pdf-vt}. PDF-Dokumente mit der PDF/UA-Kennzeichnung stärken den Ruf und die Reputation eines Unternehmens oder einer Organisation durch Engagement für Inklusion \cite{adobe-pdf-ua}. Digitale Signaturen werden bei digitalen Freigabe-, Abnahme-, Genehmigungs- und Vertragsprozesse verwendet. \gls{pades} wird in Rechtssystemen, Finanzwesen und Regierungssektor eingesetzt\cite{adobe-pdf-pades}. Im nächsten Kapitelabschnitt beleuchte ich die Verwendung von PDF-Dateien in der Druck- und Designindustrie genauer und beschreibe die Rolle von PDF im Agenturprozess.  
\section{Rolle von PDF in der Druckvorstufe und Designbranche}
Vor allem in der grafischen Industrie wird PDF gerne verwendet, weil es eine plattformübergreifende Visualisierung bietet auf allen Betriebssystemen. Schriften können bei Einbettung exakt wiedergegeben werden, unabhängig ob es sich um eine Windows oder MacOS Schrift handelt. Im Vergleich zu PostScript-Dateien erzielen die kompaktere Codierung von Seiteninhalten, dem einmaligen Speichern von identischen Bildern und die Verwendung von Kompressionsalgorithmen eine maßgeblich kleinere Dateigröße bei PDF. Korrekturänderungen in PDF-Dateien sind kurz vor dem Druck noch möglich und PDF entwickelte sich zunehmend zum Containerformat für alle grafischen Elemente. Die Produktion von Druckerzeugnissen wird somit wesentlich flexibler und sicherer. Downsampling und Kompression beschleunigt den Transport von der Agentur zum Dienstleistungsbüro enorm. In der Ausgabe ist die effektive Auflösung maßgeblich. Effektive Auflösung ist die Bildauflösung, die aus der Anzahl der Bildpunkte und der Fläche resultieren, auf der das Bild platziert wurde. Downsampling beeinflusst diese effektive Auflösung. Starke Artefakte fallen im Offsetdruck weniger auf als im Digitaldruck. \\ \cite{schneeberger}
Für die Betrachtung von Druckvorstufen-PDF-Dateien sollte immer Acrobat Pro bzw. Adobe Reader verwendet werden, da viele Drittanbieter-PDF-Viewer druckvorstufenrelevante Informationen nicht fehlerlos darstellen können. \cite{schneeberger}

In der Werbebranche werden PDF-Dateien vor allem für Korrekturabzüge verwendet. Ein Korrekturabzug ist eine Skizze bzw. Designvorlage des Werbeprodukts in vom Kunden gewünschten Position und Größe des Druckmotivs auf dem Werbeartikel. Als digitales Layout wird der Korrekturabzug mit dem Kunden abgestimmt und seine Änderungswünsche vor dem Druck entgegengenommen. \cite{korrektur}

Dank der Profile für unterschiedliche Geräte und Bedruckstoffe, kann eine Simulation des Ergebnisses am Monitor oder durch einen Prüfdruck bewerkstelligt werden. Dadurch steigt die Reproduktionssicherheit bei gleichbleibender Qualität und die Produktionszeiten verkürzen sich enorm. In den meisten Druckprojekten steht zu Beginn noch nicht fest, wann, wo und auf welchem Bedruckstoff gedruckt werden soll. Die zeitintensive Optimierung der Druckdaten kann durch PDF auf einen späteren Zeitpunkt verlegt werden.

Jeglicher Schutz sollte an einer druckvorstufentauglichen PDF-Datei vermieden werden, selbst wenn lediglich das Editieren gesperrt ist. Aktionen sind innerhalb druckfähiger Dateien untersagt, unabhängig von JavaScript oder Aktionen von Acrobat. \cite{schneeberger}

Da allgemeine Schriftinformationen immer eingebettet sind und die Zeilenlängen im Prinzip immer stimmen, können Druckvorstufenbetriebe zumindest immer erkennen, welche Schrift bzw. Schriftschnitt der Ersteller der PDF-Datei ursprünglich vorgesehen hatte, falls die Schrift nicht eingebettet wurde. \cite{schneeberger}

In der Praxis werden PDF-Dateien mit Hilfe von Prüfprofilen überprüft, Agenturen entwickeln nach eigenen Kriterien vorgefertigte Prüfroutinen und einen automatisch generierten Prüfbericht (Report). Der Prüfbericht sollte auf schnelle Fehlersuche im überprüften Dokument optimiert sein. Notwendige Korrekturen können im Originaldokument, bei PDF-Erstellung oder im vorliegenden PDF-Dokument ausgeübt werden je nach Schweregrad des Fehlers. Eine Erstellung von Korrekturprofilen kann dabei hilfreich sein. Man kann zwischen einer Vielzahl an voreingestellten Prüfprofilen und einer wesentlich kleineren Menge an Korrekturprofilen in Adobe Acrobat wählen. Außerdem kann der Prüfbericht als Kommunikationsmittel zwischen Auftraggeber und Agentur dienen, falls der Agentur das Originaldokument nicht ausgehändigt wurde. Das Preflight-Werkzeug in Acrobat Pro ist ein gängiges Werkzeug in der Druckvorstufe. Die Kontrolle von PDF-Dateien mit Preflight kann viel Geld sparen. Im tiefgreifenden Preflight Check, der ein System vor dessen Einsatz überprüft, kann man die besten Ergebnisse erzielen, wenn das zum Job passende Prüfprofil verwendet wird. Beim Preflighting-Prozess werden fehlerhafte und Objekte mit speziellen Anforderungen gefunden, ihr Zustand abgefragt und dann dementsprechende ein Fehler, eine Warnung oder eine Information ausgegeben. Im Warnungsfall muss die Agentur entscheiden, ob das Objekt, was die Warnung verursacht hat, sich negative auf das Endprodukt auswirkt. Folglich können durch Preflight frühzeitig Mängel im Produktionszyklus beseitigt werden. Ein Report sollte für die automatisierte Auswertung in weiterführenden Workflow-Systemen im \gls{xml}-Standard vorliegen. Erfolgreiche selbst erstellte Prüfprofile basieren auf qualitativ hochwertig gewählte Kriterien, wobei nicht zu viele verwendet werden sollten. Prüfprofile können als wichtigste Prüfkriterien das Dokument, Seiten, Bilder, Farbe, Zeichensätze, Rendering und Standard-Konformität enthalten. 
\cite{schneeberger}





\chapter{Implementierung}
In diesem Kapitel werde ich zunächst auf die Bedienung und den Funktionsumfang der PDF Web App eingehen. Darauffolgend werde ich die Implementierung genau erklären und wie ich Funktionen in JavaScript umgesetzt habe. Abschließend gibt es ein Unterkapitel, in dem ich die PDF Web App in verschiedenen Bereichen getestet habe und diese dokumentieren.

\section{Bedienung und Funktionalität der PDF Web App}
Meine PDF Web App besteht aus den Modulen Reader, Creator, Splitter, Merger und Editor. Wenn man die PDF Web App zum ersten Mal öffnet, findet man die im Screenshot \ref{fig:start} abgebildete Startseite vor.

\begin{figure}[!htbp]
	\centering
	\includegraphics[width=1\textwidth]{"images/startseite.png"}
	\caption{Startseite der PDF Web App}
	\label{fig:start}
\end{figure}

Bei fast allen Modulen gibt es Möglichkeiten Benutzereingaben zu machen. Diese Benutzereingaben sind so programmiert, dass sie bei ungültigen Eingaben automatische korrigiert werden oder die darauf bezogene Operation nicht ausgeführt wird. Gibt man in ein input field, wo eine Zahl erwartet wird, einen String ein, so wird dessen Funktion nicht angewendet. Liegt eine Benutzereingabe als Zahl unter oder über dem Wertebereich des Eingabefeldes, so wird entsprechen auf den niedrigsten oder obersten Wert substituiert. Manche Eingabefelder erwarten Integers, anstatt Floats, z.B. das Eingabefeld für die Seitenzahl. In diesem Fall wird die Nachkommastelle automatisch von der Benutzereingabe entfernt. Haben sich beim Benutzer ein oder mehrere Leerzeichen in die Eingabe eingeschlichen, so werden diese Leerzeichen von der App automatisiert entfernt. 

\subsection{Bedienung des Readers}
Initial ist der Reader ausgewählt. Ausgewählte Funktionen im Hauptmenü sind dunkelgrau unterlegt mit grüner Schrift. Der Button Create führt zum Creator für leere PDFs, Split zum Splitter fürs Seiten Zerteilen, Merge zum Merger für PDFs zusammenfügen und Text, Draw, Shape bzw. Image öffnen den Editor. Bei Read, Text, Draw, Shape und Image erscheint zunächst der Choose file Button, damit man im Dateisystem ein PDF-Dokument auswählen kann zum Lesen oder Bearbeiten. Klickt man auf Choose file wird der Dateibrowser geöffnet und man kann ein einzelnes PDF auswählen zum Öffnen. Der Screenshot \ref{fig:reader} zeigt den Reader mit geöffneter PDF-Datei.

\begin{figure}[!htbp]
	\centering
	\includegraphics[width=1\textwidth]{"images/reader.png"}
	\caption{Geöffnetes PDF im Reader der PDF Web App}
	\label{fig:reader}
\end{figure}

Falls eine andere Dateiart in der PDF Web App geöffnet wurde, unabhängig vom Modul der App, erscheint die Fehlermeldung in Screenshot \ref{fig:errorfile}. 

\begin{figure}[!htbp]
	\centering
	\includegraphics[width=1\textwidth]{"images/errorfile.png"}
	\caption{Fehlermeldung bei einer nicht-PDF-Datei}
	\label{fig:errorfile}
\end{figure}

Auch bei einer verschlüsselten PDF-Datei wird eine in Screenshot \ref{fig:errorcrypt} dargestellte Fehlermeldung angezeigt.

\begin{figure}[!htbp]
	\centering
	\includegraphics[width=1\textwidth]{"images/errorcrypt.png"}
	\caption{Fehlermeldung bei einem verschlüsselten PDF}
	\label{fig:errorcrypt}
\end{figure}

Bei diesen gezeigten Fehlermeldungen kann man einfach erneut den Dateibrowser aufrufen mit Choose file oder einen anderen Menüpunkt wählen, damit die Fehlermeldung verschwindet. Hat man eine PDF-Datei im Reader geöffnet präsentieren sich einem 2 Zeilen mit Funktionsbuttons. Mittels Previous und Next kann der Benutzer zur vorherigen bzw. nächsten Seite blättern. Zwischen diesen Buttons wird die aktuelle Seite im Viewport im page counter input field und daneben die Anzahl an Seiten im Dokument angezeigt. Im input field kann man eine Seite eingeben und der Reader springt direkt zu dieser Seite. Alternativ kann man mit dem Scrollbar am linken Browserfensterrand oder dem Scrollrad der Maus durch die Seiten scrollen. Durch die Buttons Plus und Minus kann man in 20 \%-Schritten rein- bzw. rauszoomen. Der aktuelle Zoomfaktor wird angezeigt in Prozent und man kann den Zoomfaktor auch mittels Benutzerangabe manuell setzen mit und ohne Prozentzeichen. Außerdem wird der minimale und maximale Zoomwert in Prozent angezeigt. Spin CW und Spin CCW dreht die aktuelle Seite, die im input field angezeigt ist, um 90 Grad im Uhrzeigersinn (clockwise) und gegen den Uhrzeigersinn (counterclockwise). Durch die Funktion Drag kann man die aktuelle Seite verschieben im Viewport. Das ist beispielsweise nützlich, wenn man an das PDF nah rangezoomt hat und die Seite nur teilweise sehen kann. Dabei klickt man zuerst auf Drag, hält die Maustaste auf der aktuellen Seite, die im page counter input field angezeigt wird, gedrückt und bewegt sie in jegliche Richtungen. Dabei bekommt der Mauscursor das Aussehen eines weißen Kreuzes mit Pfeilen an den Enden. Der Save Button downloaded das aktuelle PDF im Downloads-Ordner des Benutzers. Alle Eingabefelder in der PDF Web App begrenzen Werte, die kleiner oder größer sind als der Minimal- oder Maximalwert automatisch auf den untersten bzw. obersten Schwellenwert. Außerdem werden Leerzeichen automatisch entfernt. Ungültige Eingaben werden ignoriert. Ein geöffnetes Dokument passt sich automatisch an das Browserfenster an, falls es über das Browserfenster hinausragen würde, sodass es fast formatfüllend ins Browserfenster passt. \\

\subsection{Bedienung des Creators}
Der Creator ist mittels Create Buttons im Hauptmenü aufrufbar. Im Screenshot \ref{fig:creator} ist die GUI vom Creator dargestellt. 

\begin{figure}[!htbp]
	\centering
	\includegraphics[width=1\textwidth]{"images/creator.png"}
	\caption{Creator GUI der PDF Web App}
	\label{fig:creator}
\end{figure}

Man gibt eine Anzahl an gewünschten Seiten des leeren PDFs ein und die Breite und Höhe in mm. Wahlweise kann man den Selector verwenden um ein DIN A-Preset zu verwenden. Mittels der Schnellauswahl kann man die Orientierung bestimmten: Portrait, Landscape oder Quadratisch. Der Save Button downloaded das leere PDF. \\

\subsection{Bedienung des Splitters}
Zum Splitter kann man mit dem Split Button gelangen, der vom Screenshot \ref{fig:splitter} gezeigt wird.

\begin{figure}[!htbp]
	\centering
	\includegraphics[width=1\textwidth]{"images/splitter.png"}
	\caption{Splitter GUI der PDF Web App}
	\label{fig:splitter}
\end{figure}

\begin{figure}[!htbp]
	\centering
	\includegraphics[width=0.7\textwidth]{"images/splitter2.png"}
	\caption{Splitter Selector der PDF Web App}
	\label{fig:splitter2}
\end{figure}

\begin{figure}[!htbp]
	\centering
	\includegraphics[width=0.7\textwidth]{"images/splitter3.png"}
	\caption{Splitter Download Datei Dialog der PDF Web App}
	\label{fig:splitter3}
\end{figure}

Im Auswählmenü kann man zwischen Zerteilen nach jeder, jeder geraden, jeder ungeraden und eine Liste von Seiten auwählen, was in Abbildung \ref{fig:splitter2} dargestellt ist. Wählt man list of pages aus, so kann man die einzelnen Seitennummern mit Komma separiert oder auch nur eine einzelne Seitennummer eintippen. Die Seitenzahlen müssen nicht in aufsteigender Reihenfolge eingegeben werden. Bewegt man die Maus über den Save Button wird ein Dateibenennungsdialog sichtbar mit dem default Dateinamen. Man kann dann den Dateinamen ändern, was Abbildung \ref{fig:splitter3} präsentiert. Bei jedem Save Button in der PDF Web App kann der default Dateiname geändert werden. Hat man eine Datei ausgewählt, so wird der Dateiname und die Anzahl an Seiten des Dokuments angezeigt. \\

\subsection{Bedienung des Mergers}
Der Merger ist mit dem Hauptmenüpunkt Merge zu öffnen. Abbildung \ref{fig:merger} zeigt die Startseite des Mergers. 
\begin{figure}[!htbp]
	\centering
	\includegraphics[width=1\textwidth]{"images/merger.png"}
	\caption{Merger Startseite der PDF Web App}
	\label{fig:merger}
\end{figure}
Hier kann man nacheinander mehrere Dateien im Dateidialog öffnen und sie erscheinen in einer Liste, je nach Auswahlreihenfolge untereinander. In der Liste kann man dann einzelne Blöcke von Dateien einzeln auswählen und in ihrer Position mit der Maus verschieben. Die ausgewählte Datei hat einen schwarzen Hintergrund, was Screenshot \ref{fig:mergelist} anzeigt.
\begin{figure}[!htbp]
	\centering
	\includegraphics[width=0.8\textwidth]{"images/mergelist.png"}
	\caption{Merger Dateiliste der PDF Web App mit selektierter Datei}
	\label{fig:mergelist}
\end{figure}
Mittels Remove Buttons kann eine Datei zum Mergen wieder aus der Liste entfernt werden, nachdem sie ausgewählt wurde. Der Save Button fügt alle Dateien zu einem PDF zusammen.



\subsection{Bedienung des Editors}
Der Editor ist über den Text, Draw, Shape oder Image Button erreichbar. Genau wie im Reader erscheint ein Button Choose file. Je nachdem ob man auf Text, Draw, Shape oder Image geklickt hat, wird als erstes der Text-, Zeichen-, Geometrie- oder Bildeditor geöffnet. Hat man eine Datei geöffnet, so befindet sich der Reader ohne die Operationen zum Seiten Drehen ebenfalls im Editor, sowie seine Drag und Save Funktionalitäten. Bei Save kann man wie bei anderen Modulen einen benutzerdefinierten Dateinamen vergeben und das aktuelle PDF wird in den Downloads-Ordner runtergeladen. Alle input fields im Editor sind mit dem gültigen Wertebereich für Benutzereingaben als Information Min: Max: versehen. Der Editor besteht aus einem grauen waagerechten Operations Bar, einem linken Ebenen Seitenmenü in Rosa und einem rechten grünen Tools Seitenmenü. Mit dem ganz linken grünen Button Layers im Operations Bar kann das Ebenen Seitenmenü aus- und eingeblendet werden. Daneben zeigt oder verbirgt der Button Tools das Tools Seitenmenü. 

\subsubsection{Textbearbeitung}
Hat man den Texteditor aufgerufen, so präsentiert sich einem der Editor in folgenden Abbildungen \ref{fig:texteditor} und \ref{fig:texteditor2}.

\begin{figure}[!htbp]
	\centering
	\includegraphics[width=1\textwidth]{"images/texteditor.png"}
	\caption{Startseite des Texteditors der PDF Web App}
	\label{fig:texteditor}
\end{figure}

\begin{figure}[!htbp]
	\centering
	\includegraphics[width=1\textwidth]{"images/texteditor2.png"}
	\caption{Mehr Tools der Startseite des Texteditors der PDF Web App}
	\label{fig:texteditor2}
\end{figure}

Mit dem Button Text in der grauen Leiste und nachfolgendem Klick aufs geöffnete Dokument kann man einen Text hinzufügen mit dem Platzhaltertext dummy. Unter dem Text erscheint eine dunkelrote control box, auf die man alle Operationen in der grauen Leiste und dem Tools Seitenmenü im Box Mode anwenden kann. Der Box Mode ist standardmäßig eingestellt. Alle Operationen im rechten Tools Seitenmenü beziehen sich jeweils auf das aktuelle Editor Element und sind nur auf diesem anwendbar. Ich werde zunächst alle Operationen im Box Mode beschreiben und später auf den Layer Mode eingehen. Man kann mehrere Texte ohne erneut Text drücken zu müssen dem PDF Dokument hinzufügen. Für jedes neu hinzugefügte Textelement wird eine Ebene mit einem  Element spezifischen Standardnamen erstellt, was im linken rosa Ebenenmenü zu sehen ist. Mit dem Delete Button und nachfolgendem Klick in eine oder mehrere control boxen im Box Mode können Texte wieder gelöscht werden. Move verschiebt einzelne Texte durch mit der Maus gedrückte control box. Wenn die Maus losgelassen, nachdem die control box verschoben wurde, springt der Text an die verschobene Stelle. Ganz oben im grünen Tools Seitenmenü werden dem Betrachter die x- und y-Koordinaten der Maus auf der PDF Seite angezeigt, wenn die Maus über eine Seite bewegt wird. Darunter kann man in der textarea den Text editieren. Es werden auch Zeilenumbrüche berücksichtigt. Nachdem man den dummy Text überschrieben hat, einem Klick auf den weißen Text Button und ein oder mehrere Klicks in control boxen, kann der Text angewendet werden. Alle Operationen in Tools werden genau gleich ausgeführt: Man tätigt seine Einstellung, drückt mit der linken Maustaste auf den weißen Button für die jeweilige Operation und klickt daraufhin auf ein oder mehrere Textelemente nacheinander. Darunter kann man den Zeilenabstand einstellen. Entweder verwendet man das selection menu mit voreingestellten Werten oder man gibt einen gewünschten Wert manuell in das input field ein. Alle selection Menüs und input fields in jedem Editor zeigen die default Werte, mit denen ein neu hinzugefügtes Element konfiguriert ist, an. Falls man zuletzt das selection Menü betätigt hat, überschreibt es den Wert im input field und umgekehrt. Maßgeblich ist, was man zuletzt betätigt hatte. Dieses Verhalten habe ich bei jeder selection menu und input field Kombination programmiert. Einen benutzerdefinierten Font kann man durch den dunkelgrauen Choose file Button auswählen und er erscheint in der Liste. Der zuletzt hochgeladene Font wird ausgewählt. Mittels clear kann man einen ausgewählten Font aus der Liste entfernen, was nicht heißt, dass er auch auf dem angewendeten Text entfernt wird. Abbildung \ref{fig:custom-font} zeigt 2 geöffnete .ttf Schriftdateien in der Liste.

\begin{figure}[!htbp]
	\centering
	\includegraphics[width=0.3\textwidth]{"images/custom-font.png"}
	\caption{Benutzerdefinierte Fontliste im Texteditor der PDF Web App}
	\label{fig:custom-font}
\end{figure}

Die Fontgröße kann man ebenfalls wie die Zeilenhöhe mit selection menu und input field justieren. Bei der Fontfarbe klickt man auf das initial schwarze Quadrat, was die aktuelle Farbe zeigt, und es öffnet sich ein color picker Menü. Hier kann man die Farbe und Transparenz einstellen. Die Werte kann man sich in RGBA, HSLA oder HEX formatieren lassen. Mit Klick auf die beiden kleinen senkrechten Pfeile im color picker wird jeweils das Format gewechselt. Das Fenster des color pickers für die Fontfarbe ist in Abbildung \ref{fig:fontcolor} abgebildet. 

\begin{figure}[!htbp]
	\centering
	\includegraphics[width=0.5\textwidth]{"images/fontcolor.png"}
	\caption{Color picker für die Fontfarbe des Texteditors der PDF Web App}
	\label{fig:fontcolor}
\end{figure}

Als vorletzte Option kann man den Text absolut drehen. Durch den weißen Button Rotation und der entsprechenden Benutzerinteraktion durch selection Menu oder input field wird das Textelement absolut gedreht. Das bedeutet, dass es eine feste Rotationsskala gibt mit denen das Element rotiert wird. Folglich passiert keine Veränderung, wenn man den gleichen Rotationswert 2 Mal hintereinander ausführt. Wählt man 0 Grad aus zum rotieren, wird das Element wieder zurück auf die Ausgangsrotation gedreht. Abschließend können alle Textelemente im Dokument mit dem Remove Button auf einen Schlag gelöscht werden. Beim Zeilenabstand und der Schriftgröße wird der Benutzer außerdem über den Wertebereich von Benutzereingaben informiert. Man kann generell nur Elemente hinzufügen und auf ihnen die Operationen anwenden. Man kann in der PDF Web App keine im PDF bereits bestehenden Elemente bearbeiten. Die Textoperationen werden in Abbildung \ref{fig:text} demonstriert.

\begin{figure}[!htbp]
	\centering
	\includegraphics[width=0.9\textwidth]{"images/text.png"}
	\caption{Bearbeiteter Text in der PDF Web App}
	\label{fig:text}
\end{figure}

\subsubsection{Zeichnungen erstellen}
Der Zeichneneditor präsentiert sich einem in Screenshot \ref{fig:drawer}. 

\begin{figure}[!htbp]
	\centering
	\includegraphics[width=1\textwidth]{"images/drawer.png"}
	\caption{Drawer der PDF Web App}
	\label{fig:drawer}
\end{figure}

Das Ebenenmenü und Tools Seitenmenü des Zeichneneditors erscheint selbst wenn man zuerst im Texteditor ein Dokument geöffnet hat. Nach einem geöffneten PDF kann man von jedem Editor in einen anderen wechseln ohne erneut ein PDF öffnen zu müssen. Mit dem Zeichnen kann man anfangen, wenn man auf Pencil klickt. Bei gedrückter Maustaste auf einer PDF Seite erscheint eine schwarze Linie dort wo die Maus sich bewegt hat. Zusätzlich wird dort wo man angefangen hat die Maus zu drücken eine magenta control box hinzugefügt. Das Zeichnen funktioniert auch mit einem Graphic Tablet. Es wird immer auf der zuletzt gezeichneten Ebene auf der Seite gemalt bzw. wenn man eine Ebene auswählt im linken Ebenenmenü wird auf der ausgewählten Ebene gezeichnet. Ein Klick auf New Layer und anschließender Zeichenmodus mit Pencil kreiert für die neue Zeichnung eine weitere Ebene. Die neue Zeichnung auf der Ebene erhält abermals eine magenta control box. Wurde auf einer Seite bisher noch nichts gezeichnet, so wird bei der ersten Zeichnung auf der Seite eine neue Ebene automatisch angelegt und man muss nicht New Layer drücken. Die Zeichenelemente sind die einzigen Objekte, bei denen der Nutzer selbst die Ebenen einer Seite zuweisen kann mit New Layer. Bei allen anderen Elementen, sei es Text, Geometrie oder Bilder, wird für jedes neue Element automatisch eine Ebene erstellt. Der Radierer ist  mit dem Eraser Button im grauen operations Menü anwendbar. Zuerst drückt man Eraser und geht dann mit gedrückter Maustaste über die Zeichnungen auf einer Seite, die man entfernen möchte. Dort wo bei gedrückter Maustaste die Maus die Linie berührt wird wegradiert. Zeichnen und Radieren bekommen jeweils ein neues Mauscursorsymbol: Beim Zeichnen hat man ein schwarzes dünnes Kreuz und beim Radieren ein weißes dickes Kreuz. Mit dem Delete Button kann man mehrere Zeichnungen nacheinander mit Klicks in die control boxen entfernen. Move und gedrückte Maustaste auf eine control box verschiebt diese. Delete und Move funktionieren analog zum Texteditor. In jedem Editor gibt es einen Delete und Move Button zum Löschen und Verschieben von Elementen. Genau wie alle Operationen im Tools Seitenmenü kann man mit Delete und Move nur die dem Editor zugehörigen Elemente editieren. Im Tools Seitenmenü kann man eine Zeichnung relativ skalieren, indem man einen Faktor eingibt. Der Faktor kann auch ein Float sein und multipliziert sich immer mit der aktuellen Größe, d.h. die aktuelle Größe ist 100 \%. Darunter kann man mit dem color picker menü die Farbe und Transparenz der Stiftfarbe definieren. Sie wird mit einem Klick auf Pencil Color auf die nächste Zeichenoperation angewendet. Außerdem definiert sie auch gleichzeitig die Radiererfarbe, was sich bei Transparenzen unter 1 bemerkbar macht. Sonst ist jede Radiererfarbe gleich, jedoch bei einer Transparenz von unter 1 radiert der Radierer weniger deckend, so als ob man einen Radiergrummi weniger stark auf das Papier drückt. Dann kann man die Größe des Stiftes bzw. Radierers einstellen. Sie greift auch ab der nächsten Zeichen- bzw. Radieroperationen. Ebenfalls kann man Zeichnung mit samt radierten Partien rotieren. Zum Schluss kann man mit Remove alle Zeichnungen im Dokument löschen. Teilweise transparente Zeichnungen werden im Bild \ref{fig:drawing} dargestellt. 

\begin{figure}[!htbp]
	\centering
	\includegraphics[width=1\textwidth]{"images/drawing.png"}
	\caption{Zeichnungen in der PDF Web App}
	\label{fig:drawing}
\end{figure}


\subsubsection{Geometrie hinzufügen}
Die Startseite des Geometrieeditors ist in den Screenshots \ref{fig:shaper} und \ref{fig:shaper2} abgebildet. 

\begin{figure}[!htbp]
	\centering
	\includegraphics[width=1\textwidth]{"images/shaper.png"}
	\caption{Geometrieeditor in der PDF Web App}
	\label{fig:shaper}
\end{figure}

\begin{figure}[!htbp]
	\centering
	\includegraphics[width=1\textwidth]{"images/shaper2.png"}
	\caption{Mehr Tools des Geometrieeditors in der PDF Web App}
	\label{fig:shaper2}
\end{figure}

Der Geometrietyp kann durch die Buttons Rectangle für Rechteck, Triangle für Dreieck oder Ellipse  und einem oder mehreren Klicks auf eine PDF-Seite bestimmt werden. Ebenfalls können alle Shapetypen durch Delete gelöscht und durch Move auf der Seite verschoben werden. Im Tools Seitenmenü des Geometrieeditors gibt es eine einzige Operation, die nur auf Dreiecke angewendet werden kann. Es handelt sich um die oberste Einstellung für die Position des dritten Punktes des Dreiecks. Hiermit kann der rechte Punkt der langen Spitze des default Dreiecks bearbeitet werden. Alle anderen Einstellmöglichkeiten können auf allen Geometrieelementen Rechteck, Dreieck und Ellipse arbeiten. Man hat 2 Möglichkeiten einen Shape zu skalieren. Zum einen kann man die Breite und Höhe unabhängig voneinander einstellen, was eine absolute Skalierung bedeutet, oder man verwendet den Skalierungsfaktor, der relativ vergrößert genauso wie beim Zeichneneditor. Für die Shape Umrandungslinien kann man auf der einen Seite die Farbe inklusive Deckkraft und auf der anderen Seite die Breite der Linie justieren. Die Strichfarbe muss mit der Checkbox Use Stroke in Grün eingeschaltet sein, was sie beim ersten Öffnen des Editors auch ist. Deaktiviert man die Use Stroke Checkbox schaltet sich automatisch die Use Fill Checkbox an. Man kann auch beide Checkboxen einschalten, aber nicht beide zusammen ausschalten. Ist Use Stroke rosa, d.h. deaktiviert, und man wendet die Strichbreite an, dann wird trotzdem ein Strich in der letzten Strichfarbe gesetzt in der gewünschten Breite. Use Fill muss Grün sein, um die Füllfarbe anzuwenden. Bei Strich- und Füllfarbe wird ein wie in dem Text- und Zeicheneditor der gleiche color picker verwendet. Alle Shapes können mit absoluter Rotation rotiert werden. Im ganzen Editor kann ausschließlich absolut rotiert werden. Zuunterst entfernt der Remove Button alle Geometrieelemente im geöffneten PDF. Der Screenshot \ref{fig:shaping} hebt mehrere bearbeitete Geometrieelemente hervor.

\begin{figure}[!htbp]
	\centering
	\includegraphics[width=1\textwidth]{"images/shaping.png"}
	\caption{Zeichnungen in der PDF Web App}
	\label{fig:shaping}
\end{figure}



\subsubsection{Bilder einfügen}

\subsubsection{Arbeiten im Layer Mode}

\subsubsection{Ebenensteuerung}




\section{Implementierung der PDF Web App}
\section{Testdurchführung}

\subsection{Funktionale User Tests}

\subsection{Performance Tests}
Hat man eine PDF-Datei im Reader oder Editor geöffnet und löscht diese, so gibt es keine Fehlermeldung, wie in vielen lokalen Programmen.
\subsubsection{Renderdauer}
Die renderPage Funktion wurde mit verschiedenen PDF-Testdateien auf dem USB-Stick getestet.

\begin{table}[ht]
	\centering
	\begin{tabular}{|p{4cm}|p{3cm}|p{3cm}|p{3cm}|}
		\hline
		\textbf{Datei}													& \textbf{Seitenanzahl} 	& \textbf{Dateigröße} 	& \textbf{Execution in ms}	\\ 
		\hline
		\parbox[t]{4cm}{vivaoptik\_Gutschein\_\\50euro}					& 1 						& 33,22 KB  			& 27						\\ 
		02-Sensoren														& 9 						& 1,17 MB  				& 182						\\ 
		l11manual\_en 													& 850 						& 91,8 MB  				& 99914						\\
		the-metamorphosis-franz-kafka 									& 88 						& 298,86 KB  			& 714						\\ 
		01. War and Peace author Leo Tolstoy 							& 2882 						& 7,21 MB  				& 29115						\\ 
		Animal Crossing Amiibo Card Art									& 50 						& 167,05 MB  			& 53545						\\  
		DevOps with Kubernetes											& 520 						& 13,7 MB  				& 9883						\\  
		02. The Critique of Pure Reason author Immanuel Kant			& 1277 						& 1,78 MB  				& 9428						\\  
		UNIX and Linux System Administration Handbook - Fifth Edition	& 1809						& 71,94 MB  			& 47366						\\ 
		\hline
	\end{tabular}
	\caption{Execution Times der renderPage Funktion für verschiedene PDF-Dateien}
	\label{table:render-dur}
\end{table}

\subsubsection{Downloadleistung}
\begin{table}[ht]
	\centering
	\begin{tabular}{|p{1.5cm}|p{2.5cm}|p{2cm}|p{2cm}|p{2cm}|p{2cm}|}
		\hline
		\textbf{Modul}		& \textbf{Output-datei}		& \textbf{Seiten-anzahl}		& \textbf{Download-größe} 		& \textbf{Execution in ms} 	& \textbf{Kompress-ionsfaktor}	\\ 
		\hline
		Creator	& blank\_pdf5000 & 5000 DIN A4 & 36,96 KB & 2180 &  \\
		Creator	& \parbox[t]{4cm}{blank\_pdf500\\p10000s} & 500 Größe: 10000 & 4,92 KB & 170 & \\
		\hline
	\end{tabular}
	\caption{Execution Times der Download Zip Funktion}
	\label{table:download-dur}
\end{table}

\subsection{Browserunterschiede}

\subsection{Testbewertung}
\include{content/chapForm}
\include{content/chapGestaltung}
\chapter*{Literatur}
\label{chap:literature}
%
\begin{referenceslist}
	%[1]
	\item Mehmet Bayram, formilo, \emph{Popularität und Statistiken der PDF}. Adresse: \url{https://www.formilo.com/pdf-formulare/einfuehrung/popularitaet-statistiken/} (besucht am 19.12.2023).
	
	%[2]
	\itema Wikipedia, \emph{Portable Document Format}, 2023. Adresse: \url{https://de.wikipedia.org/wiki/Portable_Document_Format} (besucht am 19.12.2023).
	
	%[3]
	\itemb Oliver Helfrich, KOFAX, \emph{30 Jahre PDF - Ein Geschenk, das uns immer wieder neu überrascht}, Blogeintrag, 2023. Adresse: \url{https://www.kofax.de/learn/blog/30-years-of-pdf} (besucht am 19.12.2023).
	
	%[4]
	\itemc Wikipedia, \emph{Offener Standard}, 2023. Adresse: \url{https://de.wikipedia.org/wiki/Offener_Standard} (besucht am 20.12.2023).
	
	%[5]
	\itemd Wikipedia, \emph{Seitenbeschreibungssprache}, 2021. Adresse: \url{https://de.wikipedia.org/wiki/Seitenbeschreibungssprache} (besucht am 20.12.2023). 
	
	%[6]
	\iteme Adobe Systems Incorporated, \emph{PostScript LANGUAGE REFERENCE third edition}, 1999. Adresse: \url{https://web.archive.org/web/20090419181826/http://www.adobe.com/devnet/postscript/pdfs/PLRM.pdf} (besucht am 19.12.2023).
	
	%[7]
	\iteme Wikipedia, \emph{PostScript}, 2023. Adresse: \url{https://de.wikipedia.org/wiki/PostScript} (besucht am 19.12.2023).
	
	\itemf Oliver Helfrich, KOFAX, \emph{30 Jahre PDF - Ein Geschenk, das uns immer wieder neu überrascht}, Blogeintrag, 2023. Adresse: \url{https://www.kofax.de/learn/blog/30-years-of-pdf} (besucht am 19.12.2023).
	
	\itemg Oliver Helfrich, KOFAX, \emph{30 Jahre PDF - Ein Geschenk, das uns immer wieder neu überrascht}, Blogeintrag, 2023. Adresse: \url{https://www.kofax.de/learn/blog/30-years-of-pdf} (besucht am 19.12.2023).
	
	\itemh Oliver Helfrich, KOFAX, \emph{30 Jahre PDF - Ein Geschenk, das uns immer wieder neu überrascht}, Blogeintrag, 2023. Adresse: \url{https://www.kofax.de/learn/blog/30-years-of-pdf} (besucht am 19.12.2023).
	
	\itemi GitHub, \emph{Understanding GitHub Actions}. [Online]. Available: \\
	https://docs.github.com/en/actions/learn-github-actions/understanding-github-actions  (accessed: June 11 2023).
	
	\itemj w3schools, \emph{JavaScript Promises}. [Online]. Available: \\ 
	https://www.w3schools.com/js/js\_promise.asp (accessed: June 8 2023).
	
	\itemk Imperva, \emph{Cross site request forgery (CSRF) attack}. [Online]. Available: \\
	https://www.imperva.com/learn/application-security/csrf-cross-site-request-forgery/ (accessed: June 10 2023).
	
	\iteml PortSwigger, \emph{HTTP Host header attacks}. [Online]. Available: \\
	https://portswigger.net/web-security/host-header (accessed: June 10 2023).
	
	\itemm Mozilla, \emph{Cross-Origin Resource Sharing (CORS)}. [Online]. Available: \\
	https://developer.mozilla.org/en-US/docs/Web/HTTP/CORS (accessed: June 10 2023).
	
	\itemn Rowan Merewood, \emph{SameSite cookies explained}. [Online]. Available: \\
	https://web.dev/samesite-cookies-explained/ (accessed: June 10 2023).
	
	\itemo django, \emph{HTTP Strict Transport Security}. [Online]. Available: \\
	https://docs.djangoproject.com/en/4.2/ref/middleware/\#http-strict-transport-security (accessed: June 10 2023).
	
	\itemp mohanpedala, \emph{set -e, -u, -o, -x pipefail explanation }. [Online]. Available: \\
	https://gist.github.com/mohanpedala/1e2ff5661761d3abd0385e8223e16425? \\ permalink\_comment\_id=3935570 (accessed: June 10 2023).
	
	\itemq Django, \emph{Migrations}. [Online]. Available: \\
	https://docs.djangoproject.com/en/4.2/topics/migrations/  (accessed: June 11 2023).
	
	\itemr Docker, \emph{Docker run reference}. [Online]. Available: \\
	https://docs.docker.com/engine/reference/run/  (accessed: June 11 2023).
	
	\items Docker, \emph{Overlay network driver}. [Online]. Available: \\
	https://docs.docker.com/network/drivers/overlay/  (accessed: June 11 2023).
	
	\itemt Docker, \emph{Use a volume with Docker Compose}. [Online]. Available: \\
	https://docs.docker.com/storage/volumes/  (accessed: June 11 2023).
	
	\itemu Docker, \emph{Performance tuning for volume mounts (shared filesystems)}. [Online]. Available: \\
	https://docs.docker.com.zh.xy2401.com/v17.12/docker-for-mac/osxfs-caching/  (accessed: June 11 2023).
	
	\itemv Traefik, \emph{Configuration Introduction}. [Online]. Available: \\
	https://doc.traefik.io/traefik/v2.0/getting-started/configuration-overview/  (accessed: June 12 2023).
\end{referenceslist}

%
\printbibliography[prenote=mynote]
%
\appendix
\addchap{Anhang}
\KOMAoptions{open=any}
%
\include{content/chapDeclaration}
%
%
\end{document}