\chapter{Einleitung}

\section{Motivation}
Zu Hause benutze ich 4 PDF-Programme, um alle für mich zufriedenstellenden, häufigen Anforderungen der PDF-Bearbeitung zu erledigen: Adobe Reader zum Anzeigen von PDF, Drawboard PDF zum Zeichnen, Foxit Reader zum Schreiben und PDF Sam Basic zum Splitten und Mergen. Ich habe dann später herausgefunden, dass man mit Foxit Reader auch Zeichnen kann, jedoch finde ich dieses Programm sehr unintuitiv und unübersichtlich. PDF Sam Basic ist da einfacher, jedoch ist es sehr störend, dass man bei seiner Installation auch eine andere Werbeversion zusätzlich installiert. Ich habe mir gedacht, dass es nicht sein kann, dass mir kein kostenloses PDF-Programm so wirklich gefällt und deshalb beschlossen, meine eigene PDF Web App zu entwickeln. Weiter habe ich überlegt: Es wäre sinnvoll eine Webseite zu programmieren, die man auch offline nutzen kann. Alles was man benötigen soll ist ein Browser und keine Installationen mit zusätzlichen Werbeprogrammen, die man eigentlich nicht installieren wollte. Wie wäre es mit einer Open Source-Web App, die auch andere Studenten mit wenig Geld benutzen können? Folglich habe diese PDF Web App für diese Bachelorarbeit programmiert und sie ist neben ein paar Ecken und Kanten besser geworden, als ich es je erwartet hätte. Ich habe während der Programmierung viel gelernt über JavaScript, asynchrone Programmierung, event handler, uvm. und bin generell sicherer geworden im Programmieren. Vor allem habe ich Wert darauf gelegt, dass die PDF Web App einfach und intuitiv zu bedienen ist. Einige Features waren nicht geplant, aber mir hat es Spaß gemacht sie zu implementieren, da ich die Motivation hatte ein für andere und mich gutes Tool zu entwickeln. Natürlich gibt es noch eine lange Liste an Features, die ich in Zukunft noch implementieren möchte. Die PDF Web App ist kein abgeschlossenes Projekt. Dieses Hobby-Projekt möchte ich gerne nach der Bachelorarbeit weiter auf meiner Github-Seite maintainen. Vielleicht gibt es andere Entwickler, die mein Repository forken oder bug fixes hinzusteuern wollen. Vielleicht kann ich andere Entwickler dafür begeistern an der PDF Web App mitzuarbeiten und sie noch besser zu machen. 

\section{Aufbau der Arbeit}
Die Arbeit besteht im Prinzip aus 5 Kapiteln ohne Einleitung und Fazit. Ich beginne mit einem längeren Kapitel über die Grundlagen des PDF-Dateiformats. Zuerst nenne ich die wichtigsten Features, beleuchte die Dateiformate und -versionen, wobei ich diese benenne, die von der \gls{iso} standardisiert wurden. Danach erkläre ich die Implementierung des Dateiformats und die kritischen Sicherheitsaspekte. Im nächsten Kapitel Stand der Technik zeige ich die neusten Generative AI-Entwicklungen auf, wobei ich mir einige AI assistants herausgesucht habe, um sie näher zu beschreiben. Einige PDF-Programme auf dem Markt implementieren AI-Funktionalitäten. Ich habe mir 7 interessante PDF-Programme auf dem Markt ausgesucht und erkläre dessen Funktionsumfang. Danach beleuchte ich die Rolle von PDF in allgemeinen Marktbranchen und speziell in der Druckvorstufe bzw. Designbranche. Im 4. Kapitel Anforderungen und Konzept stelle ich die Anforderungen inklusive Qualitätsanforderungen an meine programmierte PDF Web App vor. Das anschließende Kapitel Implementierung startet mit der Bedienung der PDF Web App, inklusive Screenshots der Ergebnisse. In dem Umsetzung in Code-Abschnitt gehe ich detailliert auf die Programmierung ein und erkläre, wie ich Funktionalitäten realisiert habe. Anschließend zeige ich die Performance in der Testdurchführung. Im letzten Kapitel Diskussion und Kritik erörtere ich, was man bei der Programmierung noch verbessern könnte, welche Anforderungen ich nicht umgesetzt habe und welche Features ich mir für zukünftige Releases der PDF Web App vorstellen kann. 