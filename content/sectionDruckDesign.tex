\section{Rolle von PDF in der Druckvorstufe und Designbranche}
Vor allem in der grafischen Industrie wird PDF gerne verwendet, weil es eine plattformübergreifende Visualisierung bietet auf allen Betriebssystemen. Schriften können bei Einbettung exakt wiedergegeben werden, unabhängig ob es sich um eine Windows oder MacOS Schrift handelt. Im Vergleich zu PostScript-Dateien erzielen die kompaktere Codierung von Seiteninhalten, dem einmaligen Speichern von identischen Bildern und die Verwendung von Kompressionsalgorithmen eine maßgeblich kleinere Dateigröße bei PDF. Korrekturänderungen in PDF-Dateien sind kurz vor dem Druck noch möglich und PDF entwickelte sich zunehmend zum Containerformat für alle grafischen Elemente. Die Produktion von Druckerzeugnissen wird somit wesentlich flexibler und sicherer. Downsampling und Kompression beschleunigt den Transport von der Agentur zum Dienstleistungsbüro enorm. In der Ausgabe ist die effektive Auflösung maßgeblich. Effektive Auflösung ist die Bildauflösung, die aus der Anzahl der Bildpunkte und der Fläche resultieren, auf der das Bild platziert wurde. Downsampling beeinflusst diese effektive Auflösung. Starke Artefakte fallen im Offsetdruck weniger auf als im Digitaldruck. \\ \cite{schneeberger}
Für die Betrachtung von Druckvorstufen-PDF-Dateien sollte immer Acrobat Pro bzw. Adobe Reader verwendet werden, da viele Drittanbieter-PDF-Viewer druckvorstufenrelevante Informationen nicht fehlerlos darstellen können. \cite{schneeberger}

In der Werbebranche werden PDF-Dateien vor allem für Korrekturabzüge verwendet. Ein Korrekturabzug ist eine Skizze bzw. Designvorlage des Werbeprodukts in vom Kunden gewünschten Position und Größe des Druckmotivs auf dem Werbeartikel. Als digitales Layout wird der Korrekturabzug mit dem Kunden abgestimmt und seine Änderungswünsche vor dem Druck entgegengenommen. \cite{korrektur}

Dank der Profile für unterschiedliche Geräte und Bedruckstoffe, kann eine Simulation des Ergebnisses am Monitor oder durch einen Prüfdruck bewerkstelligt werden. Dadurch steigt die Reproduktionssicherheit bei gleichbleibender Qualität und die Produktionszeiten verkürzen sich enorm. In den meisten Druckprojekten steht zu Beginn noch nicht fest, wann, wo und auf welchem Bedruckstoff gedruckt werden soll. Die zeitintensive Optimierung der Druckdaten kann durch PDF auf einen späteren Zeitpunkt verlegt werden.

Jeglicher Schutz oder Sperren sollte an einer druckvorstufentauglichen PDF-Datei vermieden werden, selbst wenn lediglich das Editieren gesperrt ist. Aktionen sind innerhalb druckfähiger Dateien untersagt, unabhängig von JavaScript oder Aktionen von Acrobat. \cite{schneeberger}

Da allgemeine Schriftinformationen immer eingebettet sind und die Zeilenlängen im Prinzip immer stimmen, können Druckvorstufenbetriebe zumindest immer erkennen, welche Schrift bzw. Schriftschnitt der Ersteller der PDF-Datei ursprünglich vorgesehen hatte, falls die Schrift nicht eingebettet wurde. \cite{schneeberger}

In der Praxis werden PDF-Dateien mit Hilfe von Prüfprofilen überprüft, Agenturen entwickeln nach eigenen Kriterien vorgefertigte Prüfroutinen und einen automatisch generierten Prüfbericht (Report). Der Prüfbericht sollte auf schnelle Fehlersuche im überprüften Dokument optimiert sein. Notwendige Korrekturen können im Originaldokument, bei PDF-Erstellung oder im vorliegenden PDF-Dokument ausgeübt werden je nach Schweregrad des Fehlers. Korrekturen im Originaldokument sind vorzuziehen, da sie Folgefehler minimieren können, jedoch liegt der Agentur oftmals nicht das Originaldokument oder ausschließlich eine unvollständige PDF-Datei des Originaldokuments vor. Eine Erstellung von Korrekturprofilen für Änderungen an der gesamten PDF-Datei kann dabei hilfreich sein. Man kann zwischen einer Vielzahl an voreingestellten Prüfprofilen und einer wesentlich kleineren Menge an Korrekturprofilen im Preflight in Adobe Acrobat Pro wählen. Außerdem kann der Prüfbericht als Kommunikationsmittel zwischen Auftraggeber und Agentur dienen, falls der Agentur das Originaldokument nicht ausgehändigt wurde. Das Preflight-Werkzeug in Acrobat Pro ist ein gängiges Werkzeug in der Druckvorstufe. Die Kontrolle von PDF-Dateien mit Preflight kann viel Geld sparen. Im tiefgreifenden Preflight Check, der ein System vor dessen Einsatz überprüft, kann man die besten Ergebnisse erzielen, wenn das zum Job passende Prüfprofil verwendet wird. Beim Preflighting-Prozess werden fehlerhafte und Objekte mit speziellen Anforderungen gefunden, ihr Zustand abgefragt und dann dementsprechende ein Fehler, eine Warnung oder eine Information ausgegeben. Im Warnungsfall muss die Agentur entscheiden, ob das Objekt, was die Warnung verursacht hat, sich negative auf das Endprodukt auswirkt. Folglich können durch Preflight frühzeitig Mängel im Produktionszyklus beseitigt werden. Ein Report sollte für die automatisierte Auswertung in weiterführenden Workflow-Systemen im \gls{xml}-Standard vorliegen. Erfolgreiche selbst erstellte Prüfprofile basieren auf qualitativ hochwertig gewählte Kriterien, wobei nicht zu viele verwendet werden sollten. Prüfprofile können als wichtigste Prüfkriterien das Dokument, Seiten, Bilder, Farbe, Zeichensätze, Rendering und Standard-Konformität enthalten. Nach der Einbringung von Korrekturen im PDF sollte eine technische und visuelle Überprüfung der Druckdatei stattfinden. In einem korrigierten Kunden-PDF sind die für die Druckerei wichtigen Druck-, Schneide- und Passermarken sowie der Bereich des Anschnitts meistens unerwünscht. Um diesen Bereich und alle Druckmarken unsichtbar zu machen wird die CropBox auf die Größe der TrimBox skaliert. PDF-Dateien, die nicht aus professionellen Anwendungen der Druckvorstufe erstellt worden sind, fehlen in den meisten Fällen die TrimBox bzw. BleedBox. Hat der Datenersteller Schnittmarken sehr genau manuell platziert, so können fast alle PDF-Bearbeitungswerkzeuge davon die TrimBox bzw. BleedBox ableiten. Umschläge und Rücken werden häufig als Druckbogen als einseitiges PDF ausgegeben. Falls der Umschlag versehentlich als PDF-Datei aus Einzelseiten erstellt wurde, müssen diese Seiten zu einem Druckbogen zusammengeführt werden. Für die Weiterverarbeitung sollten leere Seiten und nicht druckbare Objekte, die als solche gekennzeichnet sind oder sich außerhalb des druckbaren Bereichs befinden, entfernt werden. Größe Dateien sind bei der Verarbeitung sehr zeitaufwendig und weniger ressourcenschonend. PDF-Dateien können sich durch Output-Intents, eingebettete Dateien, unzureichende Komprimierung von Bildern und Objekten, zu hohe Auflösung und die Präsenz nicht druckbarer Objekte in ihrer Dateigröße aufblähen. Häufig wird in der Agentur ein Abzug-PDF zur Druckfreigabe an den Auftraggeber versandt. Die Seiteninhalte eines Abzug-PDFs sollten gerastert werden, da fehlerhafte Grundeinstellungen im Adobe Reader zu einer abweichenden Darstellung des PDFs im Vergleich zum Aussehen nach dem Druck führen kann. Abzug-PDFs sollten eine Qualität besitzen, die ausreichend ist für eine Ansicht, jedoch für die Produktion bei einem anderen Druckdienstleister nicht genügt und der Output-Intent sollte zwecks Dateigrößenoptimierung entfernt werden. Falls viele Vektoren in einem Druckdokument enthalten sind, würde trotz PDF-Optimierung weiterhin eine hohe Auflösung in Vektorgrafiken bestehen bleiben und folglich nur eine geringe Dateigrößenreduktion erzielt werden. Das Rastern des Inhalts bietet einen praxisnahen Lösungsweg. Hierbei werden überdruckende Objekte in das Bild eingerechnet und Vektoren gerendert, wodurch eine maßgebliche Dateigrößenoptimierung erzielt werden kann. Speziell bei PDF/X-Dateien müssen bestimmte Inhalte als Metadaten vorliegen, wie der Dokumententitel, das Änderungsdatum der Datei, das PDF-Erstellerprogramm und das Setzes des Trapping Keys. 
\cite{schneeberger}

Textmodifikationen sind am meisten gefordert in der Druckvorstufe. Farbänderungen bzw. Farbkonvertierungen sind ebenso erforderliche Eingriffe. Änderungen am Layout oder Neugestaltungen von Seiten müssen im Originalprogramm erledigt werden und das modifizierte PDF muss in das Druck-PDF integriert werden. Ständige visuelle Kontrollen der PDF-Datei des Kunden mit der korrigierten Version werden durch eine Fachkraft im Betrieb durchgeführt. Am Ende jeder Korrektur steht die Prüfung auf PDF/X-Konformität, um als technisches Gütesiegel weitere Fehler für den Druck zu reduzieren. \cite{schneeberger}

Fehlende Schriften in PDF-Dateien sind in der Druckvorstufe katastrophal. Beim nachträglichen Einbetten von Schriften in letzter Minute muss die richtige Schrift gesucht werden und der richtige Schriftname beschreibt nicht notwendigerweise dieselbe Schrift mit derselben Codierung bzw. Laufweite. Fehlende Zeichen, überlappende Buchstaben, unregelmäßige Abstände zwischen den Buchstaben, usw. sind gängige Probleme. Schriften können auch durch die PDF-Bearbeitung extrahiert (ausgebettet) werden. In der Druckvorstufe müssen Schriftenbezeichnungen für die Ausgabe eindeutig gemacht werden. Bilder sollten sollten nach der PDF-Bearbeitung anschließend geschärft werden. Das wird vom Datenersteller oftmals vergessen. Falls im Druck-PDF ein falscher Zielfarbraum eingestellt wurde, wirkt sich das auf den Gesamtfarbauftrag aus. Eine Farbraumkonvertierung in den für das Druckergebnis gewünschten Farbraum muss vorgenommen und der maximale Gesamtfarbauftrag muss angepasst werden. 
\cite{schneeberger}

Vor allem Office-Dokumente müssen für den Druck aufbereitet werden. Marketingabteilungen von Industrieunternehmen sind gezwungen auf die zur Verfügung stehenden Werkzeuge, den Microsoft-Office-Programmen bzw. OpenOffice-Programmen zurückzugreifen bei der Druck-PDF-Erzeugung. Meistens liegen dann alle Objekte in RGB vor, es fehlt die TrimBox und der Anschnitt, sowie keine Einzelseiten-PDFs liegen zum Ausschießen vor. \cite{schneeberger}
Ausschießen bezeichnet den Prozess, dass die einzelnen Seiten des Druckerzeugnisses in der Seitenmontage seitenverkehrt so angeordnet werden, dass sie nach dem Druck und der Weiterverarbeitung (Falzen) die korrekte Reihenfolge vorweisen. \cite{kompendium}
Alle RGB-Farbräume müssen nach CMYK umgewandelt werden. Im Besonderen müssen beim Offsetdruck schwarze Texte auf den Farbaufzug K in CMYK gebracht werden. Schmuckfarben werden in CMYK oder eine andere Schmuckfarbe konvertiert bzw. gemappt. Die hinterlegten Alternativ-Farbwerte für Schmuckfarben weichen in der Vielfalt der Programme von einander ab. Ursächlich ist, dass unterschiedliche Pantone Libraries implementiert wurden. Häufig sind in PDF-Dateien unterschiedliche Farbwerte für dieselbe Farbe anzutreffen. Das entfernen von Druckmarken ist für den Datenersteller und die Druckvorstufe, die mehrere Seiten auf einem Druckbogen ausschießen muss, ein gängigere Arbeitsschritt. Die letzte Arbeitsphase im PDF-Workflow ist das Überfüllen bzw. Unterfüllen, um weiße Blitzer zu vermeiden. Spezialisten in der Druckvorstufe sollten das Hinzufügen von Traps übernehmen bzw. über dessen Notwendigkeit entscheiden. Der Druckvorstufenbetrieb gibt dem Auftraggeber den \gls{iso}-Standard der PDF-Datei für das Druckerzeugnis vor. Nur wenn sich die Ausgangsdatei mit der zum Druck verwendeten Enddatei in allen Einzelheiten deckt, kann man sich Reklamationen, Geld, Zeit und Diskussionen sparen. Sind die Druckdaten erstellt, müssen die enthaltenen Druckseiten noch in der Druckvorstufe für die Ausgabe auf die Druckplatte bzw. für die Bebilderung von Digitaldrucksystemen in eine korrekte Anordnung gebracht werden. Dieser Prozess, genannt Ausschießen, wird üblicherweise von der Druckvorstufe digital ausgeführt. In Produktionsumgebungen, wo auf Digitaldrucksystemen direkt ausgegeben wird, muss das Ausschießen auch vom Datenersteller selbst vor Ort vorgenommen werden. Ziel des Ausschusses ist eine schnellere und kostengünstigere Produktion durch die bestmöglichste Ausnutzung des Papiers bzw. Materials und nachfolgenden Schritte (Zusammentragen, Binden und Schneiden) der Endverarbeitung zu ermöglichen und vereinfachen. In der Praxis wird am häufigsten der Ausschuss durch eine Softwarelösung nach dem Designprozess praktiziert. Vor allem in der Magazin- und Zeitungsproduktion, wo Personen zu gleicher Zeit auf unterschiedlichen Seiten arbeiten und diese zu schwankenden Zeitpunkten fertigstellen, wird nach dem Designprozess ausgeschossen. Es gibt jedoch auch die Möglichkeiten im Erstellungsprogramm, beim Drucken auf dem Drucker oder im \gls{rip} auszuschießen.
\cite{schneeberger}

Zum Erstellen von PDF-Dateien mit geringer Dateigröße für den Online-Bereich muss der Output-Intent mit dem angehängten \gls{icc}-Profil entfernt werden, da \gls{icc}-Profile eine eher größere Dateigröße aufweisen. Für eine korrekte farbliche Simulation am Monitor in der Druckvorstufe ist es in einigen Fällen wünschenswert, dass ein Output-Intent gesetzt wird. Bei der Erzeugung einer \gls{iso}-konformen PDF-Datei, z.B. bei PDF/X, PDF/A oder PDF/E wird immer ein Output-Intent angelegt. Je nach Standard wird auch das Zielprofil eingebettet. Die direkte Ausgabe von PDF-Dateien über den Druckdialog in Acrobat Pro ist in der Praxis nicht mehr gängig, da alle gängigen Workflow-Systeme auf einem PostScript-3-\gls{rip} basieren. Intern konvertiert der PostScript-\gls{rip} die PDF-Datei in eine PostScript-Datei automatisch. Die Ausgabe von PostScript aus Acrobat Pro heraus spielt noch für Laserdrucker, Farbkopierer und Broschüren auf Farbdruckern eine Rolle.
\cite{schneeberger}

Bilder, die nicht im Zielfarbraum vorliegen müssen erst in den Ausgabefarbraum überführt, separiert ausgegeben und Transparenzen reduziert werden. Jegliche Transparenzen in Objekten, vor allem Texte, Grafiken oder Bilder, werden beim speichern in ein Dateiformat, welches keine Live-Transparenzen unterstützt, reduziert. Zu diesen Dateiformaten gehören u.a. \gls{eps}, JPEG, GIF, BMP, PDF 1.3 und niedriger. Nach der Transparenzreduzierung sollten für die Ausgabe Quellfarbprofile von den reduzierten Objekten entfernt werden, da sie oftmals zu unerwünschten Farbtransformationen im Ausgabegerät führen können. Das Anbringen von Transparenzen auf Objekte mit Volltonfarben ist mit Ausnahme der Deckkraftänderung verboten. Bilder können unscharf werden, wenn in den meisten Fällen eine zu niedrige Reduzierungsauflösung global eingestellt ist.
\cite{schneeberger}

Da der wirtschaftliche Druck auf die produzierenden Unternehmen immer mehr zunimmt, müssen einzelne Produktionsschritte immer mehr zeitlich gestaucht werden und im Idealfall automatisch ablaufen. 

Von einer medienneutralen Druckproduktion spricht man, wenn die Farbraumkonvertierung der neutral abgespeicherten Daten ausschließlich in RGB oder in RGB und CMYK zu einem möglichst späten Zeitpunkt stattfindet. Bei Logos werden medienneutrale PDF-Dokumente nicht bevorzugt. Durch das medienneutrale Konzept gewinnt man an Flexibilität und reduziert Mehrarbeit im Produktionsprozess, z.B. wenn ein späterer Wechsel des Druckverfahrens und der Papierklasse erfolgt. Der Gegensatz zur medienneutralen Arbeitsweise ist die verfahrensangepasste Arbeitsweise.
\cite{schneeberger}

Der Begriff Binding bezeichnet im Prozess der Druckdatenerstellung die Farbkonvertierung in die gewünschten Druckbedingungen. Beim Eearly Binding liegen alle verfahrensangepasste Daten schon im gewünschten Zielfarbraum vor. Hingegen bei medienneutralen Daten kann bei der PDF-Erzeugung eine Konvertierung in den Zielfarbraum erfolgen. Dies wird als Intermediate Binding bezeichnet. Im Prozess des Late Bindings wird die Konvertierung in den Zielfarbraum bei medienneutralen Daten erst in der Ausgabe vorgenommen. 
\cite{schneeberger}