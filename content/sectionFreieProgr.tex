\section{Freie PDF Programme und Onlinedienste}
PDF Dateien lassen sich in vielen Programmen einfach über den Druckdialog erstellen. Apple hat das Lesen von PDF Dokumenten in seiner Apples Vorschau integriert. Viele Webbrowser stellen PDF Viewer bereit, so Google Chrome seit 2010 \cite{wiki-pdf-de}.

\subsection{foxit PDF Reader}
Produktseite: \url{https://www.foxit.com/}
Die Firma foxit hat einen kostenlosen PDF Reader auf den Markt gebracht, der sehr umfangreiche Funktionen bietet.

\subsection{PDF24 Tools}
Produktseite: \url{https://www.pdf24.org/en/} \\
Bei den PDF24 Tools handelt es sich um eine Online-Lösung für PDF-Bearbeitung.

\subsection{Aspose.PDF}
Produktseite: \url{https://products.aspose.app/pdf/family} \\

%%%%%%%%%%%%%%%%%%%%%%%%%%%%%%%

\subsection{PDFsam}
Produktseite: \url{https://pdfsam.org/}

\subsection{DrawboardPDF}
Produktseite: \url{https://www.drawboard.com/pdf/pdf}

